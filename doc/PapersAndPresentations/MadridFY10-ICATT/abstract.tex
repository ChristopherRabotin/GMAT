

\begin{abstract}

The General Mission Analysis Tool (GMAT) is a software system for trajectory
optimization, mission analysis, trajectory estimation, and prediction developed
by NASA, the Air Force Research Lab, and private industry. GMAT's design and
implementation are based on four basic principles: open source visibility for
both the source code and design documentation; platform independence; modular
design; and user extensibility. The system, released under the NASA Open Source
Agreement, runs on Windows, Mac and Linux. User extensions, loaded at run time,
have been built for optimization, trajectory visualization, force model
extension, and estimation, by parties outside of GMAT's development group. The
system has been used to optimize maneuvers for the Lunar Crater Observation and
Sensing Satellite (LCROSS) and ARTEMIS missions and is being used for formation
design and analysis for the Magnetospheric Multiscale Mission (MMS).

In this paper, we discuss two primary topics: GMAT's current feature set; and
how to write plug-in libraries written outside of the main development code. The
existing feature set is broken down into two principal categories, called
resources and commands. GMAT's resources consist of models of the components
used to build a mission timeline: celestial objects, spacecraft and ground
stations, hardware components, propagators, numerical solvers, variables and
arrays, and output components. The commands are used to tie these resources
together in a time ordered sequence, and are used to describe how the resources
interact.

We present several examples of extensions to GMAT that have been built to
support mission specific goals using custom plug-in libraries. The key elements
required for a GMAT plug-in are presented, along with an overview of the class
structure for the system that makes these elements work.

\end{abstract}
