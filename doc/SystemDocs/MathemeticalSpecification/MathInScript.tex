\chapter{Mathematics in GMAT Scripting} \label{Ch:MathInScript} \index{Math in script}

\section{Basic Operators}

\section{Math Functions}

\subsection{max}

\st{ [maxX] = max(X) }

\st{X} is an $n$x$m$ array.  \st{maxX} is a $1$x$m$ row vector
containing the maximum value in each column of \st{X}.

\subsection{min}

\st{ [minX] = min(X) }

\st{X} is an $n$x$m$ array.  \st{minX} is a $1$x$m$ array containing
the minumum value contained in each row of \st{X}.

\subsection{abs}

\st{ [absX] = abs(X) }

\st{X} is an $n$x$m$ array.  \st{absX} is a $n$x$m$ array where each
component is the absolute value of the corresponding component of
\st{X}.

\subsection{mean}

\st{ [meanX] = mean(X) }

\st{X} is an $n$x$m$ array.  \st{meanX} is a $1$x$m$ row vector
containing the mean of each column of \st{X}.

\subsection{dot}

\st{[dotp] = dot(vec1,vec2)}

The \st{dot} function calculates the dot (scalar) product of two
vectors. \st{vec1} and \st{vec2} must both be vectors with the same
length.  \st{dotp} is the scalar product.


\subsection{cross}

\st{[crossp] = cross(vec1,vec2)}

The \st{cross} function calculates the cross product of two vectors.
\st{vec1} and \st{vec2} must both be vectors with the same length.
\st{crossp} is the cross product.

\subsection{norm}

\st{[normv] = norm(vec)}

The \st{norm} function calculates the 2-norm of a vector. \st{vec}
must both be a vector. \st{normv} is the root-sum-square of the
components of \st{vec}.

\subsection{det}

\st{[detX] = norm(X)}

The \st{det} function calculates the determinant of a matrix. \st{X}
is an $n$x$n$ array.  \st{detX} is the determinant of \st{X}.

\subsection{inv}

\st{[invX] = inv(X)}

The \st{inv} function returns the inverse of a matrix.  \st{X} must
be a square matrix.

\subsection{eig}

\subsection{sin, cos, tan}

\subsection{asin, acos, atan, atan2}

\subsection{sinh, cosh, tanh}

\subsection{asinh, acosh, atanh}

\subsection{transpose}

\subsection{DegToRad}

\subsection{RadToDeg}

\subsection{log}

\subsection{log10}

\subsection{exp}

\subsection{sqrt}
