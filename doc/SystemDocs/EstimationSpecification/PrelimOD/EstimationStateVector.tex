\section{Estimation State Vector}

Depending upon the application, GMAT may maintain  two types of
state vectors:  the propagation state vector and the estimation
state vector.  Here we define the propagation state vector to be the
set of components  the propagator component advances in time via
numerical integration or analytic propagation.  The estimation state
vector is defined as the vector of solve-for and consider parameters
associated with a particular estimator.

Elements in the propagation state vector may not be contained in the
estimation vector and vice-versa depending upon the particular
problem.  Managing and organizing state data presents a difficult
bookkeeping problem for several reasons

\begin{compactenum}
     \item Different state components may require different types of
     propagation methods.  (numerical integration vs. analytic
     or ephemeris interpolation)
     \item In GMAT, different spacecraft can be propagated by
     simulateously using different numerical integrators and
     force models.
     \item Propagation state data is not specified in one location.
     \item GMAT must maintain synchronous propagation if requested
     by the user.
     \item All components of propagation may not be known until
     runtime for a propagation command.
\end{compactenum}



\subsection{Orbit State}

\subsection{Dynamics Properties}

\subsection{Biases}

\subsection{Process Noise}
