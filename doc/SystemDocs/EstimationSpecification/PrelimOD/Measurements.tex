\section{Measurements}

%When observed values are not available, a user can configure a
%Measurement Model to simulate the desired measurements.  In this
%case, GMAT will compute both the observed and expected values for a
%measurement, or by using the SimulateData command, GMAT will simply
%write the requirested simulation data to a measurement file.    A
%script snippet that contains a sample measurement model is shown
%below.  Each field for the Measurement Model object is discussed in
%detail in Table 1.  Note that for complex, such as GPS psuedorange,
%there will be many more fields.  These fields are not included here.
%
%Below is a script example of a simple measurement model.  The
%participants in the measurement are ODSat and Canberra, which are
%assumed to have been configured in a previous script segment.   The
%measurement types are Azimuth, Elevation, and Range.  The DataSource
%field allows the user to tell GMAT whether to read the observations
%from a file, or to simulate the data. The FileFormat and Filename
%fields allow the user to specify the observations data file if it is
%available.    The fields  LightTimeModel, IonosphericModel, and
%TroposphereModel,allow the user to specify the model for these error
%corrections, where None is an option if the error source is to be
%neglected.   All fields with the prefix field ".Sim" are for
%configuration of the data simulator for this measurement.

\subsection{Ground Station}

\subsection{GPS}

\subsection{TDRSS}

\subsection{Measurement Editing}
