\documentclass[10pt,letterpaper]{article}
\usepackage[latin1]{inputenc}
\usepackage{amsmath}
\usepackage{amsfonts}
\usepackage{amssymb}
\usepackage[left=2cm,right=2cm,top=2cm,bottom=2cm]{geometry}
\usepackage{tabularx}
\author{Darrel J. Conway\\Thinking Systems, Inc.}
\title{Comparison of Options for GMAT's C-Interface  }
\begin{document}
\maketitle
\abstract{This document compares approaches to building GMAT's C-Interface.}

\section{Introduction}

Whatever.  The goal here is the table below.

\section{Notes}

Below there is a table stating what I know so far.  This is basically notes while looking at the options.  Other notes:

\begin{itemize}
\item Another group looking to use C++ in MATLAB and Python produced this report:\\ http://verdandi.gforge.inria.fr/doc/high\_level\_interface.pdf
\item This URL looks interesting and pertinent:  http://undocumentedmatlab.com/
\end{itemize}

\noindent Open questions/comments:

\begin{itemize}
\item Is this project misnamed?  Most of the approaches we discuss are really Java interface issues, not C interface.  It seems to me that C only enters as an intermediary.  (I think we do want a C interface.  But if the goal is really to talk to MATLAB, Java seems the most natural approach.)
\item Java calls in MATLAB are straightforward.  You add path data to the MATLAB Java path, and then call into the .jar or .class files.  JNI/JNA calls are a bit more convoluted because MATLAB needs to be able to find the associated shared libraries.
\item For what it's worth, I'm becoming more and more convinced that we do want Python support along with MATLAB support.  That seems to point to a native C interface to me, and postprocessing (swig or something like it) to get the further hooks.
\end{itemize}

\newcolumntype{Y}{>{\small\raggedright\arraybackslash}X}
\newcolumntype{Z}{>{\small\raggedright\arraybackslash}X}
\begin{tabularx}{\linewidth}{|l|>{\setlength\hsize{0.8\hsize}}Y|>{\setlength\hsize{1.2\hsize}}Z|}
\hline 
\textbf{Tool} & \textbf{URL} & \textbf{Notes} \\ 
\hline 
loadlibrary &  &\begin{itemize}
\item Method prototyped last year for propagation proof of principle
\item Requires C wrappers for GMAT classes
\item Interface changes inside of GMAT will ripple through the interface code, and need hand coding to adapt
\end{itemize}\\
\hline 
MEX & www.mathworks.com & \begin{itemize}
\item Requires specific MEX function access to each exposed piece of code.
\item Interface changes inside of GMAT will ripple through the MEXFunction code, and need hand coding to adapt
\end{itemize}\\
\hline 
JNI & docs.oracle.com/javase/7/docs/ technotes/guides/jni/index.html & \begin{itemize}
\item Supports classes
\item Has the reputation of being cumbersome to use and error prone.
\end{itemize} \\ 
\hline 
swig & www.swig.org & \begin{itemize}\item There is a lot of chatter on SWIG-MATLAB interconnections, but not a lot of information about how well/if it works.  One interesting project is SwigMatlabPlus
\item SwigMatlabPlus doesn't seem to be available anymore -- or will take some work to track down.  The link (http://alumni.media.mit.edu/ {\textasciitilde}sbasu/code/swigmatlabplus/) does not include the source, and other links on the developer's web site at MIT are broken.  It is Windows/Visual C++ specific.
\item Requires a custom .i file to specify what the interface exposes
\item swig builds JNI files and java files used to interface into Java, along with (at least one) .c file to link into the shared library.
\item A note from the swig manual:  
``If you are going to use optimisations turned on with gcc (for example -O2), ensure you also compile with -fno-strict-aliasing. The GCC optimisations have become more aggressive from gcc-4.0 onwards and will result in code that fails with strict aliasing optimisations turned on. See the C/C++ to Java typemaps section for more details.''
\end{itemize}\\ 
\hline 
\end{tabularx} 

\begin{tabularx}{\linewidth}{|l|>{\setlength\hsize{0.8\hsize}}Y|>{\setlength\hsize{1.2\hsize}}Z|}
\hline 
\textbf{Tool} & \textbf{URL} & \textbf{Notes} \\ 
\hline 
JNA & jna.java.net/ & \begin{itemize}
\item Provides C access, so no direct class support.
\item Allows for -- and requires -- an independent API definition through Java classes
\end{itemize} \\ 
\hline 
%JMI & undocumentedmatlab.com/blog/ jmi-java-to-matlab-interface/ & \begin{itemize}
%\item The Java to MATLAB intefcae -- calling MATLAB from Java
%\end{itemize} \\ 
%\hline 
BridJ / JNAerator & //code.google.com/p/bridj/ & \begin{itemize}
\item Appears to be pretty young.  Will it survive?
\item Notes from the website:

\textbf{Key features}
\begin{itemize}
\item Dynamic C / C++ / COM interop : call C++ methods, create C++ objects (and subclass C++ classes from Java !)
\item You never need to compile any native code : we deal with the cross-compilation hassle for you once and for all in BridJ ! (works on Windows, Linux, MacOS X, Solaris, Android...)
\item Full JNAerator support : stay away from C / C++ headers !
\item Small library size (~ 600 kB all included)
\item Straightforward type mappings with good use of generics 
\end{itemize}
\item Untested
\end{itemize} \\ 
\hline 
.NET &  &\begin{itemize}
\item Basically windows specific
\item May be feasible via mono (www.mono-project.com); downloaded and installed on Linux in AZ, but so far untested.
\end{itemize} \\
\hline 
COM &  & \begin{itemize}
\item Windows specific
\item COM interfaces are a bit old school, and have been replaced in large part by .NET.
\end{itemize}\\ 
\hline 
TCP/IP &  &  \\ 
\hline 
\end{tabularx} 

\end{document}