\chapter{Dynamics Modelling}

One of the fundamental capabilities of GMAT is to model the motion
of spacecraft in many different flight regimes.  The flight regimes,
such as low Earth, or Libration Points, are determined by which
forces and perturbations dominate the dynamics.  In this chapter we
present how GMAT models the dynamics of spacecraft in motion.  We
discuss how GMAT calculates many different types of forces including
multiple non-spherical gravity perturbations, third-body effects,
and atmospheric drag among others. We being by looking at the
general form of the equations of motion.

\section{Orbit Dynamics}


\subsection{Orbit State Equations}

 Let's begin by defining the position and velocity of a
spacecraft with respect to the central body of integration as \br
and \bv. From Newton's Second Law we know that
%
\begin{equation}
    m \frac{d^2 \mathbf{r}}{d^2 t} = \sum F
\end{equation}
%
which says that the mass, times the acceleration, is equal to the
sum of the forces.  Solving for the acceleration gives us the
second order differential equation
%
\begin{equation}
   \frac{d^2 \mathbf{r}}{d^2 t} = \sum \frac{F}{m}
   \label{Eq:SCAcc}
\end{equation}
%
GMAT has the capability to model many different types of
accelerations experienced by spacecraft in orbit.  If we include
all of the possible forces GMAT can model in the summation on the
left hand side of Eq.~(\ref{Eq:SCAcc}), then we would have

\begin{equation} \begin{split}
    \frac{d^2\mathbf{r}}{dt^2}
    =  &-\frac{\mu}{r^3}\mathbf{r} +  \nabla \phi_{sj}^o +
    %
    G\sum_{\stackrel{k=1}{k \neq j}}^{n_b}m_k \left(\frac{\mathbf{r}_{ks}}{r_{ks}^3} -
     \frac{\mathbf{r}_{kj} }{r_{kj}^3}   \right)\\
     %
   & + \sum_{\stackrel{k=1}{k \neq j}}^{n_b}\left( \nabla
    \phi_{ks}^o +
     \nabla\phi_{kj}^o
     \right)+ \frac{\dot{m}_s }{m}\frac{d\mathbf{r}}{dt}\\ & - \frac{1}{2}\rho v_{rel}^2 \frac{C_d A}{m_s}\hat{\mathbf{v}}_{rel}
     +\frac{   P_{SR}C_R A_{\odot}   }{m_s}\hat{\mathbf{r}}_{s\odot}\\
     \end{split} \label{Eq:CompleteEOM}
\end{equation}
%
\begin{table}[h]
\centering
\begin{tabular}{p{2.0 in} p{1.0 in} }
  \hline\hline
  % after \\: \hline or \cline{col1-col2} \cline{col3-col4} ...
  Description & Term \\
  \hline
  Central Body Point Mass & $ -\displaystyle\frac{\mu}{r^3}\mathbf{r}$ \\
  & \\
  Central Body Direct Nonspherical &  $\nabla \phi_{sj}^o$  \\
  & \\
  Direct Third Body Point Mass & $G\displaystyle\sum_{\stackrel{k=1}{k \neq j}}^{n_b}m_k \left(\frac{\mathbf{r}_{ks}}{r_{ks}^3}\right)$ \\
   & \\
  Indirect Third Body Point Mass &  $G\displaystyle\sum_{\stackrel{k=1}{k \neq j}}^{n_b}m_k \left( -
     \frac{\mathbf{r}_{kj} }{r_{kj}^3}   \right)$ \\
      & \\
   Third Body Direct Nonspherical & $\displaystyle\sum_{\stackrel{k=1}{k \neq j}}^{n_b}\left( \nabla
    \phi_{ks}^o\right)$  \\
    &\\
    Third Body Indirect Nonspherical &  $\sum_{\stackrel{k=1}{k \neq j}}^{n_b}\left(
     \nabla\phi_{kj}^o \right)$\\
    & \\
   Spacecraft Thrust &$\displaystyle\frac{\dot{m}_s }{m}\frac{d\mathbf{r}}{dt}$\\
   &\\
   Atmospheric Drag & $- \displaystyle\frac{1}{2}\rho v_{rel}^2 \displaystyle\frac{C_d
   A}{m_s}\hat{\mathbf{v}}_{rel}$\\
   &\\
   Solar Radiation Pressure & $\displaystyle\frac{   P_{SR}C_R A_{\odot}   }{m_s}\hat{\mathbf{r}}_{s\odot}$\\
  \hline\hline
\end{tabular}
\end{table}
%


In general, Eq.~(\ref{Eq:CompleteEOM}) does not have an analytic
solution so GMAT uses numerical integration to find approximate,
although very accurate, solutions.  GMAT uses first order
numerical integrators, so we must take the three second order
differential equations in Eq.~(\ref{Eq:CompleteEOM}) and convert
them to six first order equations.  So, we define a new variable
$\mathbf{x}$ such that
%
\begin{equation}
    \mathbf{x} = \left[\mbr^T \hspace{.2 in} \mbv^T  \right]^T = \left[x \hspace{.2 in} y \hspace{.2 in} z \hspace{.2 in} \dot{x} \hspace{.2 in}
    \dot{y} \hspace{.2 in} \dot{z}  \right]^T
\end{equation}
%
then taking the derivative we arrive at
%
\begin{equation}
    \dot{\mathbf{x}} = \left[\dot{\mbr}^T \hspace{.2 in} \dot{\mbv}^T  \right]^T = \left[\dot{x} \hspace{.2 in}
    \dot{y}
    \hspace{.2 in} \dot{z} \hspace{.2 in} \ddot{x} \hspace{.2 in}
    \ddot{y} \hspace{.2 in} \ddot{z}  \right]^T
\end{equation}

\subsection{State Transition Matrix Equations}

\begin{equation}
     \dot{\boldsymbol{\Phi}}(t,t_o) = \tilde{\mathbf{A}}\boldsymbol{\Phi}(t,t_o)
\end{equation}
%
where
%
\begin{equation}
     \tilde{\mathbf{A}} = \frac{\partial \dot{\mathbf{x}}}{\partial \mathbf{x}}
\end{equation}
%
subject to the initial conditions
%
\begin{equation}
     \boldsymbol{\Phi}(t_o,t_o) = \mathbf{I}_{6\times6}
\end{equation}


If we define $\mathbf{x}$ as
%
\begin{equation}
     \mathbf{x} = \left(\begin{array}{cc}
                 \mathbf{r}\\
                 \mathbf{v}
          \end{array}\right)
\end{equation}
%
then
%
%
\begin{equation}
     \dot{\mathbf{x}} = \left(\begin{array}{cc}
                 \mathbf{v}\\
                 \mathbf{a}
          \end{array}\right)
\end{equation}
%
Now we can write $\tilde{\mathbf{A}}$ as
%
\begin{equation}
     \tilde{\mathbf{A}} = \frac{\partial \dot{\mathbf{x}}}{\partial
     \mathbf{x}}=
     \left(\begin{array}{ccc}
              \displaystyle\frac{\partial \mathbf{v}}{\partial \mathbf{r}} & \displaystyle\frac{\partial \mathbf{v}}{\partial
              \mathbf{v}}\vspace{.1 in}\\
              %
              \displaystyle\frac{\partial \mathbf{a}}{\partial \mathbf{r}} & \displaystyle\frac{\partial \mathbf{a}}{\partial
              \mathbf{v}}
     \end{array}\right)
\end{equation}
%
For convenience, lets use the following notation
%
\begin{equation}
    \mathbf{A} = \displaystyle\frac{\partial \mathbf{v}}{\partial \mathbf{r}}
\end{equation}
%
\begin{equation}
    \mathbf{B} = \displaystyle\frac{\partial \mathbf{v}}{\partial
              \mathbf{v}}
\end{equation}
%
\begin{equation}
    \mathbf{C} = \displaystyle\frac{\partial \mathbf{a}}{\partial \mathbf{r}}
\end{equation}
%
\begin{equation}
    \mathbf{D} = \displaystyle\frac{\partial \mathbf{a}}{\partial \mathbf{v}}
\end{equation}

\subsection{Multiple Spacecraft Propgation and Coupled Propagation of the Equations of Motion}

\section{Force Modelling}

\subsection{Non-Spherical Gravity }

GMAT integrates all spacecraft equations of motion using the
Earth's Mean J2000 axis system. However, the user can choose
central bodies other than the Earth as the origin of the
coordinate system of integration.  Gravitational forces are
conservative and only a function of position.  To calculate the
gravitational force due to a non-spherical body, we need to
determine the position of the spacecraft in the body fixed frame
$\mathcal{F}_F$.  However, the equations of motion are expressed
in terms of the position of the spacecraft in the inertial frame.



We know from dynamics that the acceleration in an inertial frame
can be calculated using
%
\begin{equation}
   \mathbf{a}_{cb} = \nabla U \label{Eq:a_cb}
\end{equation}
%
where $U$ is the gravitational potential.  The potential for a
nonspherical body comes from the solution to Laplace's equation:
%
\begin{equation}
     \nabla^2 U= 0
\end{equation}
%
The solution to this equation is most easily expressed in spherical,
body-fixed coordinates because it allows for a convenient separation
of variables.

In spherical coordinates the gradient of the gravitational potential
is
%
\begin{equation}
   \nabla U = \frac{\partial U}{\partial r}\mathbf{u}_r +
   \frac{1}{r}\frac{\partial U}{\partial \phi} \mathbf{u}_\phi
   + \frac{1}{r \cos{\phi}}\frac{\partial U}{\partial
   \lambda}\mathbf{u}_\lambda \label{Eq:SphericalGradient}
\end{equation}
%
We see that there are two singularities in
Eq.~(\ref{Eq:SphericalGradient}).  The first is when $r = 0$, which
is a nonphysical case and we will not discuss it further.  The
second singularity occurs when $ \phi = \pm 90^\circ$.  Pines
\cite{Pines:73} developed a uniform expression of the gravitational
potential that avoids the singularity at the poles:
%
\begin{equation}\begin{split}
    U = & \frac{\mu}{r}  \biggl[   1 + \sum_{n=1}^\infty \left(\frac{R_{\otimes}}{r}\right)^n
    \sum_{m=0}^{n} A_{nm}(u)[C_{nm} cos{(m \lambda)} \cos^m{\phi} \\
    %
    & +S_{nm}sin{(m \lambda)} \cos^m{\phi} ]    \biggr]\end{split}
    \label{Eq:Pines1}
\end{equation}
%
Examining this form of the potential it is easy to see that there is
not a singularity at the poles when taking the gradient in spherical
coordinates.  Pines rewrites Eq.~(\ref{Eq:Pines1}) as
%
\begin{equation}\begin{split}
    U = & \frac{\mu}{r}  \biggl[   1 + \sum_{n=1}^\infty \left(\frac{R_{\otimes}}{r}\right)^n
    \sum_{m=0}^{n} A_{nm}(u)[C_{nm} r_m(s,t) \\
    %
    & +S_{nm} i_m(s,t) ]    \biggr]\end{split}
    \label{Eq:PinesPotential}
\end{equation}
%
where $C_{nm}$ and $S_{nm}$ are the gravitational coefficients, $s$,
$t$, and $u$ are given by
%
\begin{equation}
     s = x/r, \hspace{.2 in} t = y/r, \hspace{.2 in} u = z/r =
     \sin{\phi} \nonumber
\end{equation}
%
and $r_m(s,t)$ and $i_m(s,t)$ are calculated using the recursive
relationships
%
\begin{equation}
    \begin{split}
        r_0 = 1, \hspace{.2 in} r_1 = s, \hspace{.2 in} i_0 = 0,
        \hspace{.2 in} i_1 = t \nonumber\\
        %
        r_m = sr_{m-1} - ti_{m-1}, \hspace{ .2 in} i_m = si_{m-1} +
        tr_{m-1} \nonumber
    \end{split}
\end{equation}


The coefficients $A_{nm}(u)$ are called ``derived" Legendre
functions and are given by
%
\begin{equation}
    A_{nm}(u) = \frac{d^m}{du^m}(P_n(u))
\end{equation}
%
where we know from Rodrigues' \cite{Lundberg:88} formula that
%
\begin{equation}
    P_{n0}(u) = P_n(u) = \frac{1}{2^n n!}\frac{d^n}{du^n}(u^2 - 1)^n
\end{equation}
%
and
%
\begin{equation}
    P_{nm}(u) = (1 - u^2)^{m/2}\frac{d^m}{du^m}P_n(u)
\end{equation}
%

For numerical reasons it is useful to normalize some of the terms in
the potential function, $U$.  By normalizing the spherical
coefficients and the derived Legendre polynomials we can improve the
stability of recursive algorithms used to calculate the Legendre
polynomials and improved numerical problems.  We use the
nondimensionalization approach and described by
Lundberg\cite{Lundberg:88}.  Lundberg chooses the normalization
factor so that the normalized spherical harmonics $\bar{C}_{nm}$ and
$\bar{S}_{nm}$ will have a mean square value of one on the unit
sphere.  The normalized Legendre functions, $\bar{P}_{nm}$, are
defined so that the product of the spherical harmonic coefficients
and the corresponding Legendre functions remain constant, or
%
\begin{equation}
   \bar{P}_{nm}\bar{C}_{nm} = P_{nm}C_{nm}  \hspace{.5 in}  \bar{P}_{nm}\bar{S}_{nm} = P_{nm}S_{nm}
\end{equation}
%
GMAT uses the normalization factor $N_{nm}$ given by
%
\begin{equation}
    N_{nm} =  \left[ \frac{(n-m)!(2n+1)!}{(n+m)!} \right]^{1/2}
\end{equation}
%
The non-dimensional spherical harmonic coefficients and Legendre
functions are
%
\begin{equation}
     \bar{P}_{nm} = N_{nm}P_{nm} \hspace{.25 in} \bar{C}_{nm} =
     \frac{C_{nm}}{N_{nm}} \hspace{.25 in} \bar{S}_{nm} =
     \frac{S_{nm}}{N_{nm}}
\end{equation}
%
The derived Legendre polynomials are normalized using
%
\begin{equation}
    \bar{A}_{nm} = N_{nm} A_{nm} \label{Eq:LegendreNormalization}
\end{equation}
%
where $\bar{A}_{nm}$ are the normalized Legendre polynomials.
Lundberg\cite{Lundberg:88} showed that there are several recursive
algorithms to compute $\bar{A}_{nm}$  but that only two are stable.
GMAT uses the following algorithm to recursively calculate the
derived Legendre polynomicals
%
\begin{equation}
     \begin{split}
     \bar{A}_{nm} = & u\left[ \frac{(2n+1)(2n-1)}{(n-m)(n+m)}
     \right]^{1/2}\bar{A}_{n-1,m}  \\
     %
     & - \left[ \frac{(2n+1)(n-m-1)(n+m-1)}{(2n-3)(n+m)(n-m)}
     \right]^{1/2}\bar{A}_{n-2,m}
     \end{split}
\end{equation}
%
The recursive algorithm is started using
%
\begin{eqnarray}
     \bar{A}_{11} & = & \sqrt{3} \cos{\phi}\\
     \bar{A}_{nn} & = & \cos{\phi}\sqrt{\frac{2n+1}{2n}}\bar{A}_{n-1,n-1}
\end{eqnarray}
%
The above equations are normalized using
Eq.~(\ref{Eq:LegendreNormalization}) and used in


The acceleration due to nonspherical gravity can be written as
%
\begin{equation}
   \begin{split}
   \mathbf{a}_g = & \left( \frac{\partial U}{\partial r} - \frac{s}{r}\frac{\partial U}{\partial s}
    -\frac{t}{r}\frac{\partial U}{\partial t} -\frac{u}{r}\frac{\partial U}{\partial u}
    \right) \hat{\mathbf{r}}\\
    %
    & + \left( \frac{1}{r}\frac{\partial U}{\partial s} \hspace{.1
    in} \frac{1}{r}\frac{\partial U}{\partial t}  \hspace{.1
    in} \frac{1}{r}\frac{\partial U}{\partial u}\right)^T
    \end{split} \label{Eq:NonSphericalAcc}
\end{equation}
%


To simplify the partial derivatives in
Eq.~(\ref{Eq:NonSphericalAcc}), Pines defines some intermediate
variables as follows
%
\begin{eqnarray}
     \rho   & = & a/r \\
     \rho_0 & = & \mu/r \nonumber \\
     \rho_1 & = & \rho \rho_0 \\
     \rho_n & = & \rho \rho_{n-1} \hspace{.2 in} \mbox{for $n>1$}\nonumber
\end{eqnarray}
%
Using Lundberg's nondimensionalization approach, we can write
%
\begin{eqnarray}
     \bar{D}_{nm}(s,t) & = & \bar{C}_{nm}r_m(s,t) + \bar{S}_{nm}i_m(s,t) \nonumber\\
     \bar{E}_{nm}(s,t) & = & \bar{C}_{nm}r_{m-1}(s,t) + \bar{S}_{nm}i_{m-1}(s,t) \nonumber\\
     \bar{F}_{nm}(s,t) & = & \bar{S}_{nm}r_{m-1}(s,t) - \bar{C}_{nm}i_{m-1}(s,t) \nonumber\\
     \bar{G}_{nm}(s,t) & = & \bar{C}_{nm}r_{m-2}(s,t) + \bar{S}_{nm}i_{m-2}(s,t) \nonumber\\
     \bar{H}_{nm}(s,t) & = & \bar{S}_{nm}r_{m-2}(s,t) - \bar{C}_{nm}i_{m-2}(s,t) \nonumber\\
\end{eqnarray}
%
The partial derivatives in Eq.~(\ref{Eq:NonSphericalAcc}) can be
written as
%
\begin{equation}
    \begin{split}
   &\frac{\partial U}{\partial r} - \frac{s}{r}\frac{\partial U}{\partial s}
    -\frac{t}{r}\frac{\partial U}{\partial t} -\frac{u}{r}\frac{\partial U}{\partial u}
     =\\
     -\sum_{n=0}^{\infty}&\frac{\rho_{n+1}}{R_\otimes}\sum_{m=0}^{n}c_{n+1,m+1}\bar{A}_{n+1,m+1}\bar{D}_{nm}
    \end{split} \label{Eq:a4}
\end{equation}
%
\begin{equation}
    \frac{1}{r}\frac{\partial U}{\partial s} =
    \sum_{n=0}^{\infty}\frac{\rho_{n+1}}{R_\otimes}\sum_{m=0}^{n}\bar{A}_{nm}(u) m \bar{E}_{nm}
\end{equation}
%
\begin{equation}
    \frac{1}{r}\frac{\partial U}{\partial t} =
    \sum_{n=0}^{\infty}\frac{\rho_{n+1}}{R_\otimes}\sum_{m=0}^{n}\bar{A}_{nm}(u) m \bar{F}_{nm}
\end{equation}
%
\begin{equation}
    \frac{1}{r}\frac{\partial U}{\partial s} =
    \sum_{n=0}^{\infty}\frac{\rho_{n+1}}{R_\otimes}\sum_{m=0}^{n}c_{n,m+1}\bar{A}_{n,m+1}(u)\bar{D}_{nm}
\end{equation}
%
where
%
\begin{eqnarray}
    c_{n,m+1}   & = & \left[(n-m)(n+m+1) \right]^{1/2}  \nonumber\\
    c_{n+1,m+1} & = & \left[\frac{( n + m + 2) (n + m + 1 )}{(2n+3)(2n+2)}
    \right]^{1/2}  \nonumber
\end{eqnarray}


To calculate the nonzero portion of the sensitivity matrix, we begin
by calcluting the following 9 terms:
%
\begin{eqnarray}
    a_{11} & = & \sum_{n=0}^{\infty}\frac{\rho_{n+2}}{R_\otimes^2}\sum_{m=0}^{n}m(m-1)\bar{A}_{nm}\bar{G}_{nm}\\
    %
    a_{12} & = & \sum_{n=0}^{\infty}\frac{\rho_{n+2}}{R_\otimes^2}\sum_{m=0}^{n}m(m-1)\bar{A}_{nm}\bar{H}_{nm}\\
    %
    a_{13} & = & \sum_{n=0}^{\infty}\frac{\rho_{n+2}}{R_\otimes^2}\sum_{m=0}^{n}mc_{n,m+1}\bar{A}_{n,m+1}\bar{E}_{nm}\\
    %
    a_{14} & = & -\sum_{n=0}^{\infty}\frac{\rho_{n+2}}{R_\otimes^2}\sum_{m=0}^{n}mc_{n+1,m+1}\bar{A}_{n+1,m+1}\bar{E}_{nm}\\
    %
    a_{23} & = & \sum_{n=0}^{\infty}\frac{\rho_{n+2}}{R_\otimes^2}\sum_{m=0}^{n}mc_{n,m+1}\bar{A}_{n,m+1}\bar{F}_{nm}\\
    %
    a_{24} & = & -\sum_{n=0}^{\infty}\frac{\rho_{n+2}}{R_\otimes^2}\sum_{m=0}^{n}mc_{n+1,m+1}\bar{A}_{n+1,m+1}\bar{F}_{nm}\\
    %
    a_{33} & = & \sum_{n=0}^{\infty}\frac{\rho_{n+2}}{R_\otimes^2}\sum_{m=0}^{n}c_{n,m+2}\bar{A}_{n,m+2}\bar{D}_{nm}\\
    %
    a_{34} & = & -\sum_{n=0}^{\infty}\frac{\rho_{n+2}}{R_\otimes^2}\sum_{m=0}^{n}c_{n+1,m+2}\bar{A}_{n+1,m+2}\bar{D}_{nm}\\
    %
    a_{44} & = &
    \sum_{n=0}^{\infty}\frac{\rho_{n+2}}{R_\otimes^2}\sum_{m=0}^{n}c_{n+2,m+2}\bar{A}_{n+2,m+2}\bar{D}_{nm}
\end{eqnarray}
%
where
%
\begin{eqnarray}
    c_{n+1,m+2} &=& c_{n+1,m+1}  \left[(n-m)( n + m + 3)
    \right]^{1/2} \nonumber\\
    %
    c_{n,m+2} &=& c_{n,m+1} \left[( n - m - 1)(n + m +2 )
    \right]^{1/2} \nonumber\\
    %
    c_{n+2,m+2} &=&
    c_{n+1,m+1}\left[\frac{(n+m+4)(n+m+3)}{(2n+5)(2n+4)}\right]^{1/2} \nonumber
\end{eqnarray}

Finally,
%
\begin{equation}
   \mathbf{C}_g = \frac{\partial \mathbf{a}_g}{\partial \mathbf{r}}
\end{equation}
%
where $\mathbf{C}_g$ is a symmetric matrix with components given by
%
\begin{eqnarray}
   c_{11} & = & a_{11} + s^2a_{44} + a_4/r + 2sa_{14} \\
%
   c_{12} & = & c_{21} = a_{12}+ sta_{44} + sa_{24} + ta_{14}\\
%
   c_{13} & = & c_{31} = a_{13}+ sua_{44} + sa_{34} + ua_{14}\\
%
   c_{22} & = & -a_{11} +t^2a_{44} + a_4/r +2ta_{24}\\
%
   c_{23} & = & c_{32} = a_{23} + tua_{44} + ua_{24} + ta_{34}\\
%
   c_{33} & = & a_{33} + u^2a_{44} + a_4/r + 2*u*a_{34}
\end{eqnarray}
%
Note that
%
\begin{equation}
    a_4 = \frac{\partial U}{\partial r} - \frac{s}{r}\frac{\partial U}{\partial s}
    -\frac{t}{r}\frac{\partial U}{\partial t} -\frac{u}{r}\frac{\partial U}{\partial u}
\end{equation}
%
and is given in Eq.~(\ref{Eq:a4}).

\subsection{$n$-Body Point Mass Gravity}

The gravitational perturbation due to $n$ point masses is well know.
However, we will derive the governing differential equation here, as
well as the componenents of the sensitivity matrix.
 Let's begin by defining some notation referring to
Fig.\ref{fig:NBody}. Assume the $j^{\mbox{th}}$ body is the central
body of the integration.
%
\begin{figure}[h!]
\centerline{
\begin{picture}(100,500)
\special{psfile= Images/NBodyDiagram.eps hoffset= -135 voffset= -45
hscale=85 vscale=85} \makebox(-20,585){$\hat{\mathbf{x}}_{I}$}
\makebox(270,700){$\hat{\mathbf{y}}_{I}$} \makebox(-330,770){$
\tilde{\mathbf{r}}_{s}$} \makebox(-330,877){$\mathbf{r}$}
\makebox(-270,900){$\mathbf{r}_{sk}$}
\makebox(-390,964){$\mathbf{r}_{k}$}
\makebox(-500,814){$\tilde{\mathbf{r}}_{j}$}
\makebox(-500,980){Central Body} \makebox(-320,995){$k^{th}$ Body}
\end{picture}}\vskip -4.0 in  \caption{ N-Body Illustration} \label{fig:NBody}
\end{figure}
%
%
\begin{itemize}
   %
   \item  $\tilde{\mathbf{r}}_s$ is the position of the spacecraft with respect
   a hypothesized inertial frame.
   %
   \item  $\tilde{\mathbf{r}}_j$ is the position of the central body with respect
   a hypothesized inertial frame.
   %
   \item  $\tilde{\mathbf{r}}_k$ is the position of the $k^{th}$ gravitational body with respect
   a hypothesized inertial frame.
   %
   \item  $\mathbf{r}$ is the position of the spacecraft with respect
   to the central body of integration ($j^{th}$ body).
   %
   \item  $\mathbf{r}_k$ is the position of the $k^{th}$ gravitational body with respect
   to the central body.
   %
\end{itemize}

We need the governing differential equation that describes the
motion of the spacecraft with respect to the central body.  However,
we know that we must apply Newton's 2nd Law in an inertial frame.
So, we begin by defining the relative position of the spacecraft
with respect to the central body.  From inspection of
Fig.\ref{fig:NBody} we see that
%
\begin{equation}
     \tilde{\mathbf{r}}_j +  \mathbf{r} = \tilde{\mathbf{r}}_s
\end{equation}
%
By reordering and taking the second derivative with respect to time
we obtain
%
\begin{equation}
     \ddot{\mathbf{r}} = \ddot{\tilde{\mathbf{r}}}_s - \ddot{\tilde{\mathbf{r}}}_j
     \label{Eq:SCRelativeODE}
\end{equation}
%
We can apply Newton's 2nd Law to the spacecraft and obtain
%
\begin{equation}
     m_s \ddot{\tilde{\mathbf{r}}}_s = \sum_{k=1}^n F_k =
     G\sum_{k=1}^n \frac{m_s m_k}{\| \mathbf{r}_{k} - \mathbf{r}\|^3} \left(\mathbf{r}_{k} -
     \mathbf{r}\right)
\end{equation}
%
where $\left(\mathbf{r}_{k} - \mathbf{r}\right)$ is a vector from
the spacecraft to the $k^{th}$ body, $m_s$ is the mass of the
spacecraft, and $m_k$ is the mass of the $k^{th}$ body.  We can
write $\ddot{\tilde{\mathbf{r}}}_s$ as simply
%
\begin{equation}
    \ddot{\tilde{\mathbf{r}}}_s =
     G\sum_{k=1}^n \frac{m_k}{\| \mathbf{r}_{k} - \mathbf{r}\|^3} \left(\mathbf{r}_{k} -
     \mathbf{r}\right) \label{Eq:SCInertialODE}
\end{equation}
%
We can apply Newton's 2nd Law to the $j^{th}$ body and obtain
%
\begin{equation}
     m_j \ddot{\tilde{\mathbf{r}}}_j = \frac{G m_s
     m_j}{r^3}\mathbf{r} +
     G\sum_{\stackrel{k=1}{k \neq j}}^{n} \frac{m_j m_k}{\| \mathbf{r}_{k}\|^3}\mathbf{r}_{k}
\end{equation}
%
where the first term is the influence of the spacecraft on the
central body, and the second term is the influence of the $k$ point
mass gravitational bodies.  We can write
$\ddot{\tilde{\mathbf{r}}}_j$ as simply
%
\begin{equation}
     \ddot{\tilde{\mathbf{r}}}_j = \frac{G m_s
     }{r^3}\mathbf{r} +
     G\sum_{\stackrel{k=1}{k \neq j}}^{n} \frac{ m_k}{\|
     \mathbf{r}_{k}\|^3}\mathbf{r}_{k} \label{Eq:CentalBodyInertialODE}
\end{equation}
%
Substituting Eq.~(\ref{Eq:SCInertialODE}) and
(\ref{Eq:CentalBodyInertialODE}) into (\ref{Eq:SCRelativeODE}) we
get
%
\begin{equation}
     \ddot{\mathbf{r}} =      G\sum_{k=1}^n \frac{m_k}{\| \mathbf{r}_{k} - \mathbf{r}\|^3} \left(\mathbf{r}_{k} -
     \mathbf{r}\right) - \frac{G m_s
     }{r^3}\mathbf{r} -
     G\sum_{\stackrel{k=1}{k \neq j}}^{n} \frac{ m_k}{\|
     \mathbf{r}_{k}\|^3}\mathbf{r}_{k}
\end{equation}
%
Finally, collecting terms yields
%
\begin{equation}
     \mathbf{a}_{pm} = \ddot{\mathbf{r}} =   \underbrace{- \frac{\mu_j
     }{r^3}\mathbf{r}}_1  +  G \sum_{\stackrel{k=1}{k \neq j}}^{n} m_k\left( \underbrace{\frac{\mathbf{r}_{k} -
     \mathbf{r}}{\| \mathbf{r}_{k} - \mathbf{r}\|^3}}_2  -
     \underbrace{
      \frac{ \mathbf{r}_{k}}{\|
     \mathbf{r}_{k}\|^3}}_3\right)
\end{equation}
%
We can break down the acceleration in the equation above into three
physical categories.   The first term is the acceleration on the
spacecraft due to a point mass central body.   The second type of
terms are called direct terms.  They account for the force of the
$k^{th}$ body on the spacecraft.  The third type of terms are called
indirect.  They account for the force of the $k^{th}$ body on the
central body.

Let's look at the contributions to the sensitivity matrix due to
point mass perturbations.  We notice that $\mathbf{a}_{pm}$ is not a
function of velocity.  So,
%
\begin{equation}
    \mathbf{A}_{pm} = \mathbf{D}_{pm}  = \mathbf{0}_{3\times3}
\end{equation}
%
We also know that
%
\begin{equation}
    \mathbf{B}_{pm} = \mathbf{I}_{3\times3}
\end{equation}
%
This leaves $\mathbf{C}_{pm}$ as the only non-trivial term for point
mass gravitational effects.  Let's look first at the derivatives of
the point mass term.  We can use the vector identity in
Eq.~(\ref{Eq:vecIDaveca3}) to arrive at
%
\begin{equation}
     \frac{\partial }{\partial \mathbf{r}} \left(- \frac{\mu_j
     }{r^3}\mathbf{r}\right) =  -\frac{\mu_j}{r^3} \mathbf{I}_3
     + 3\mu_j\frac{\mathbf{r}\mathbf{r}^T}{r^5}
\end{equation}

Similarly, applying Eq.~(\ref{Eq:vecIDaveca3}) to the direct terms
we see that
%
\begin{equation}\begin{split}
     \frac{\partial }{\partial \mathbf{r}} \left( \sum_{\stackrel{k=1}{k \neq j}}^{n} \mu_k \frac{\mathbf{r}_{k} -
     \mathbf{r}}{\| \mathbf{r}_{k} - \mathbf{r}\|^3}\right) = &\\ -\sum_{\stackrel{k=1}{k \neq j}}^{n}
     \frac{\mu_k}{\| \mathbf{r}_{k} - \mathbf{r}\|^3}\mathbf{I}_3 + 3\sum_{\stackrel{k=1}{k \neq
     j}}^{n}\mu_k &\left( \frac{\left( \mathbf{r}_{k} - \mathbf{r} \right)\left( \mathbf{r}_{k} -
     \mathbf{r} \right)^T}{\left( \|\mathbf{r}_{k} - \mathbf{r} \right)\|^5}  \right)
     \end{split}
\end{equation}
%
Finally, the derivative of the indirect terms are zero and we have
%
\begin{equation} \begin{split}
   &\mathbf{C}_{pm} =  \underbrace{-\frac{\mu_j}{r^3} \mathbf{I}_3
     + 3\mu_j\frac{\mathbf{r}\mathbf{r}^T}{r^5}}_{ 1 }\\
     %
      &\underbrace{
     %
     -  \sum_{\stackrel{k=1}{k \neq j}}^{n}
     \frac{\mu_k}{\| \mathbf{r}_{k} - \mathbf{r}\|^3}\mathbf{I}_3 + 3\sum_{\stackrel{k=1}{k \neq
     j}}^{n} \mu_k \left( \frac{\left( \mathbf{r}_{k} - \mathbf{r} \right)\left( \mathbf{r}_{k} -
     \mathbf{r} \right)^T}{\left( \|\mathbf{r}_{k} - \mathbf{r} \right)\|^5}  \right)
       }_{2}
     %
     \end{split}
\end{equation}
%
Combining similar terms we can express the result as
%
\begin{equation} \begin{split}
   &\mathbf{C}_{pm} =  -  \left( \frac{\mu_j}{r^3} + \sum_{\stackrel{k=1}{k \neq j}}^{n}
     \frac{\mu_k}{\| \mathbf{r}_{k} - \mathbf{r}\|^3} \right)\mathbf{I}_3 \\
          %
      &
     %
     + 3 \left( \mu_j\frac{\mathbf{r}\mathbf{r}^T}{r^5}
       + \sum_{\stackrel{k=1}{k \neq
     j}}^{n} \mu_k \left( \frac{\left( \mathbf{r}_{k} - \mathbf{r} \right)\left( \mathbf{r}_{k} -
     \mathbf{r} \right)^T}{\left( \|\mathbf{r}_{k} - \mathbf{r} \right)\|^5}
     \right) \right)
     %
     \end{split}
\end{equation}


\subsection{Atmospheric Drag}

\begin{equation}
  \mathbf{a}_d = - \displaystyle\frac{1}{2}\rho v_{rel}^2 \displaystyle\frac{C_d
   A}{m_s}\hat{\mathbf{v}}_{rel}
\end{equation}
%
where
%
\begin{equation}
    \mathbf{v}_{rel} = \mathbf{v} - \boldsymbol{\omega}_\otimes \times
    \mathbf{r}
\end{equation}
%
where $\boldsymbol{\omega}_\otimes$ is the Earth's angular velocity
vector in the FK5 system.

GMAT does not currently support calculating the STM using drag.  The
components of the sensitivity matrix  $\tilde{\mathbf A}$ contain
derivatives of the atmospheric density with respect to position.
These derivatives are non trivial for most density models and are
not currently included in GMAT.

\subsection{Solar Radiation Pressure}

\begin{equation}
     \mathbf{a}_s = -P_{SR}\displaystyle\frac{   C_R A    }{m_s}\hat{\mathbf{s}}
\end{equation}
%
where $\hat{\mathbf{s}}$ is a unitized vector pointing from the
spacecraft to the sun
%
\begin{equation}
    \mathbf{s} = \mathbf{r}_s - \mathbf{r}
\end{equation}
%
where $\mathbf{r}_s$ is the Sun's position vector and $\mathbf{r}$
is the spacecrafts position vector.



\begin{equation}
    \mathbf{A}_{s} = \mathbf{D}_{s}  = \mathbf{0}_{3\times3}
\end{equation}
%
\begin{equation}
    \mathbf{B}_{s} = \mathbf{I}_{3\times3}
\end{equation}
%
\begin{equation}
    \mathbf{C}_{s} = P_{SR}\displaystyle\frac{   C_R A
    }{m_s}\left( \frac{1}{s^3}\mathbf{I}_3 - 3 \frac{ \mathbf{s}\mathbf{s}^T}{s^5}\right)
\end{equation}
%
where
%
\begin{equation}
    s = \| \mathbf{s} \|
\end{equation}

\subsection{Spacecraft Thrust}

\begin{footnotesize}

\begin{equation}
\begin{split}
    F_T(T,P) = C_1 + C_2 P +  (C_3 + C_4 P + C_5 P^2 + C_6 P^{C_7}\\ +  C_8P^{C_9}
     + C_{10}P^{C_{11}}
    + C_{12}(C_{13})^{C_{14}P})\left(\frac{T}{T_{ref}}\right)^{1 + C_{15} + C_{16}P }
    \label{Eq:ThrustPolynomial}
     \end{split}
\end{equation}

%
\begin{equation}
\begin{split}
    I_{sp}(T,P) = K_1 + K_2 P + (K_3 + K_4 P + K_5 P^2 + K_6 P^{K_7} \\ + K_8P^{K_9} + K_{10}P^{K_{11}}
    + K_{12}(K_{13})^{K_{14}P})\left(\frac{T}{T_{ref}}\right)^{1 + K_{15} + K_{16}P }
         \end{split}
\end{equation}
\end{footnotesize}

\begin{equation}
    \dot{m} =  f_d\frac{F_T(T,P)}{I_{sp}(T,P)g}
\end{equation}

\begin{equation}
    \mathbf{T} = f_s f_d F_T(T,P) \mathbf{R}_{iT}\hat{\mathbf{T}}_d
\end{equation}
%
where $F_T(T,P)$ is given in equation \ref{Eq:ThrustPolynomial},
$f_s$ and $f_d$ are the thrust scale factor and duty cycle
respectively, $\mathbf{R}_{iT}$ is the rotation matrix from the
thruster coordinate system to the EarthMJ2000 equatorial system, and
$\hat{\mathbf{T}_d}$ is the unitized thrust direction in the
thruster coordinate system.

\begin{table}[h!]
\centering \caption{ Thrust and Isp Coefficient Units }
      \begin{tabular}{llll}
      \hline\hline
         Coeff. & Unit \\
         \hline
         %---New Row---%
         $C_1$ & N & $K_1$ & s \\
         %---New Row---%
         $C_2$ & N/Pa & $K_2$ & s/Pa\\
         %---New Row---%
         $C_3$ & N  & $K_3$ & s  \\
         %---New Row---%
         $C_4$ & N/Pa & $K_4$ & s/Pa \\
         %---New Row---%
         $C_5$ & N/Pa$^2$ & $K_5$ & s/Pa$^2$ \\
         %---New Row---%
         $C_6$ & N/Pa$^{C_7}$ & $K_6$ & s/Pa$^{C_7}$ \\
         %---New Row---%
         $C_7$ & None & $K_7$ & None  \\
         %---New Row---%
         $C_8$ & N/Pa$^{C_9}$ & $K_8$ & s/Pa$^{C_9}$ \\
         %---New Row---%
         $C_9$ & None & $K_9$ & None \\
         %---New Row---%
         $C_{10}$ & N/Pa$^{C_{11}}$  & $K_{10}$ & s/Pa$^{C_{11}}$  \\
         %---New Row---%
         $C_{11}$ & None & $K_{11}$ & None \\
         %---New Row---%
         $C_{12}$ & N & $K_{12}$ & s\\
         %---New Row---%
         $C_{13}$ & None & $K_{13}$ & None \\
         %---New Row---%
         $C_{14}$ & 1/Pa  & $K_{14}$ & 1/Pa  \\
         %---New Row---%
         $C_{15}$ & None & $K_{15}$ & None \\
         %---New Row---%
         $C_{16}$ & 1/Pa & $K_{16}$ & 1/Pa  \\
      \hline\hline
      \label{Table:OE_RigorouslyJ2Inv}
\end{tabular} \normalsize
\end{table}

\textbf{VNB Thruster System}

It is possible to specify thrust with respect to rotating
coordinates systems.  Then, during integration of the equations of
motion, GMAT uses the coordinate system definition to determine the
thrust in the inertial system being used for numerical integration
of the equations of motion.    One of the coordinate systems useful
in mission analysis is the Velocity-Normal-Binormal (VNB) system
based on the motion of a spacecraft with respect to a reference
origin. One way to configure a thruster to use a local VNB system is
to set \st{CoordinateSystem} to local.  Internally, GMAT creates a
coordinate system based on the \st{Axes} and \st{Origin} specified
chosen for the thruster. We illustrate this below by example. The
thruster named \st{Thruster1} is configured to use a local VNB
coordinate system based on the motion of the owner-spacecraft and
the Earth.

\noindent\st{Create Thruster Thruster1}\\
\st{Thruster1.CoordinateSystem = Local;}\\
\st{Thruster1.Origin = Earth;}\\
\st{Thruster1.Axes   = VNB;}\\

To convert the thrust from the requested local VNB system to the
inertial system, GMAT creates a coordinate system configured as
shown below. The \st{Origin} field on the thruster is used in two
places on the coordinate system: as both the \st{Origin} and the
\st{Primary}. The axes are set to ObjectReferenced, and the $x$-axis
and $y$-axis are respectively set to ``V'' and ``N''.

\noindent\st{ Create CoordinateSystem SATVNB;}\\
\st{ GMAT SATVNB.Origin = DefaultSC; }\\
\st{ GMAT SATVNB.Axes = ObjectReferenced;}\\
\st{ GMAT SATVNB.Primary = Earth;}\\
\st{ GMAT SATVNB.Secondary = DefaultSC;}\\
\st{ GMAT SATVNB.XAxis = V; }\\
\st{ GMAT SATVNB.YAxis = N; }\\

Using this system, the $x$-axis is in the velocity direction, the
$n$-axis is in the velocity direction, and the $z$-axis completes
the right-handed set. Note, the secondary body is not set until the
thruster is assigned to a spacecraft, and then, the secondary is set
to be the owner- spacecraft.  The script snippet above shows the
configuration after the thruster has been attached to spacecraft
``DefaultSC'', using, for example, the script line
%
\st{DefaultSC.Thrusters = \{Thruster1\}};

\noindent \textbf{LVLH Thruster System}\\

The LVLH system, similarly to the VNB system, is a local system that
is constructed based on the motion of the owner spacecraft with
respect to an origin and axes system chosen by the user.  As an
example, below the thruster named \st{Thruster1} is configured to
use a local LVLH coordinate system based on the motion of the owner
spacecraft ``MySat'' and the moon.

\noindent\st{Create Thruster Thruster1}\\
\st{Thruster1.CoordinateSystem = Local;}\\
\st{Thruster1.Origin = Luna;}\\
\st{Thruster1.Axes   = LVLH;}\\

Internally, GMAT creates a coordinate system configured as shown
below.  The \st{Origin} field on the thruster is used in two places
on the coordinate system, as both the \st{Origin} and the
\st{Primary}.  The axes are set to  \st{ObjectReferenced}.

\noindent\st{ Create CoordinateSystem SATVNB;}\\
\st{ GMAT SATVNB.Origin = MySat; }\\
\st{ GMAT SATVNB.Axes = ObjectReferenced;}\\
\st{ GMAT SATVNB.Primary = Luna;}\\
\st{ GMAT SATVNB.Secondary = MySat;}\\
\st{ GMAT SATVNB.XAxis = -R; }\\
\st{ GMAT SATVNB.YAxis = -N; }\\

Using this system, the $y$-axis is opposite of orbit normal, the
$z$-axis points towards the origin, and the $x$-axis completes the
right-handed set.


\section{Attitude} \label{Ch:Attitude}

The attitude of a spacecraft can be defined qualitatively as how the
spacecraft is oriented in inertial space, and how that orientation
changes in time.  GMAT has the ability to model the orientation and
rate of rotation of a spacecraft using several different
mathematical models. Currently, GMAT assumes that a spacecraft is a
rigid body.

There are many ways to quantitatively describe the orientation and
rate of rotation of a spacecraft, just like there are many ways we
can quantitatively describe an orbit state.  Let's define any set of
numbers that can uniquely define the spacecraft attitude as an
attitude parameterization. GMAT allows the users to use several
common attitude parameterizations including quaternions, Euler
angles, the Direction Cosine Matrix (DCM), Euler angle rates, and
the angular velocity vector. Given an initial attitude state, GMAT
can propagate the attitude using one of several kinematic attitude
propagation models.

In this chapter, we discuss the attitude parameterizations supported
in GMAT, and how to convert between the different types.  We discuss
the internal state parameterization that GMAT uses.   Next we
investigate the types of attitude modes in GMAT and discuss in
detail how GMAT propagates the spacecraft attitude in all of the
Kinematic attitude modes.  We conclude the chapter with a discussion
of how GMAT converts between different attitude parameterizations.

\subsection{Attitude Propagation}

Given a set of initial conditions that define the attitude, GMAT can
propagate the attitude using several methods.  Currently, GMAT only
supports kinematic attitude propagation.  In Kinematic mode, the
attitude is defined by describing the desired orientation with
respect to other objects such as spacecraft or celestial bodies.
With this information, GMAT can calculate the required attitude to
satisfy the desired geometrical configuration.  This section
presents the different Kinematic attitude modes, and how GMAT
calculates the attitude state in each mode.  Let's begin by looking
at the internal attitude state representation and how the user can
define initial conditions.

\vspace{- .1 in} \subsubsection{Internal State Representation and
Attitude Initial Conditions}

Certain attitude parameterizations are more useful for attitude
propagation, while other attitude parameterizations are more
intuitive for providing attitude initial conditions or output. GMAT
uses different internal parameterizations of the attitude
orientation depending upon the attitude mode.  The type of
parameterization is chosen to make the attitude propagation
algorithms natural and convenient.  For the kinematic modes, GMAT
uses the DCM that represents the rotation from the inertial system
to the body axes as the attitude orientation parameterization. In
the future, when 6 degree of freedom attitude modelling is
implemented, GMAT will use the quaternion that represents the
rotation from the inertial system to the body axes. GMAT uses the
angular velocity of the body with respect to the inertial frame,
expressed in the body frame, $\{\boldsymbol\omega_{IB}\}_B$, as the
rate portion of the state vector.

For convenience, the user can choose a coordinate system in which to
define the initial attitude state.  Let's call this system
$\mathcal{F}_i$.  The user can define the initial attitude
orientation with respect to $\mathcal{F}_i$ using Euler angles, the
DCM, or quaternions.  The user can define the body rate with respect
to $\mathcal{F}_i$ by defining the angular velocity in
$\mathcal{F}_i$, $\{\boldsymbol\omega_{IB}\}_i$, or by defining the
Euler angle rates.  Note that not all attitude modes require these
three pieces of information.  The specific inputs for each attitude
mode are discussed below, along with details about how attitude
propagation is performed in each mode.

\subsubsection{Kinematic Attitude Propagation}

The Kinematic attitude mode allows a user to define a geometrical
configuration based on the relative position of a spacecraft with
respect to other spacecraft or celestial bodies, and with respect to
different coordinate systems.  In Kinematic mode, GMAT does not
integrate the attitude equations of motion, but rather calculates
the attitude based on the geometrical definition provided by the
user.  There are several Kinematic modes to choose from.  The
different modes allow the user to conveniently define the spacecraft
attitude depending on the type of attitude profile needed for a
specific mission.  To begin, let's look at how GMAT calculates the
attitude state in the Coordinate System Fixed attitude mode
(CSFixed).

\subsubsection{Coordinate System Fixed Mode}

In the CSFixed attitude mode, the user supplies two pieces of
information. They first specify a coordinate  system in which to fix
the attitude, $\mathcal{F}_i$.  $\mathcal{F}_i$ can be any of the
default coordinate systems or any user defined coordinate system.
Secondly, the user specifies how the body axis system,
$\mathcal{F}_B$ is oriented with respect to $\mathcal{F}_i$ by
defining $\mathbf{R}_{Bi}$ or an equivalent parameterization. With
this information, GMAT calculates the rotation from the inertial to
the body axes and the angular velocity of the body with respect to
the inertial frame, expressed in the body frame,
$\{\mathbf{\boldsymbol\omega}_{IB}\}_B$.

GMAT calculates the rotation matrix from $\mathcal{F}_i$  to
$\mathcal{F}_B$, $\mathbf{R}_{Bi}$, from the initial conditions
provided by the user.  For CSFixed mode, $\mathbf{R}_{Bi}$ is
constant and is stored for use in the equations below.  Knowing
$\mathbf{R}_{Bi}$, we can calculate the rotation matrix from the
inertial frame to the body frame, $\mathbf{R}_{BI}$, using the
following equation
%
\begin{equation}
     \mathbf{R}_{BI} = \mathbf{R}_{Bi}\mathbf{R}_{iI}
     \label{Eq:CSFixedRotationMatrix}
\end{equation}
%
$\mathbf{R}_{iI}$ is the rotation matrix from $\mathcal{F}_I$ to
$\mathcal{F}_i$ and GMAT knows how to calculate this matrix for all
allowable $\mathcal{F}_i$.  For details on the calculation of this
matrix for all coordinate systems in GMAT  see
Ch.~\ref{Ch:CoordinateSystems}.

To calculate $\{\mathbf{\boldsymbol\omega}_{IB}\}_B$, we start from
Eulers' equation:
%
\begin{equation}
   \dot{\mathbf{R}}_{BI} =
   -\{\mathbf{\boldsymbol\omega^\times}_{IB}\}_B\mathbf{R}_{BI}
   \label{Eq:CSFixedKinematics}
\end{equation}
%
where
%
\begin{equation}
       \{ \mathbf{\boldsymbol\omega^\times}_{IB}\}_B = \begin{pmatrix}
     0 & -\omega_3 & \omega_2\\
     \omega_3 & 0 & -\omega_1\\
     -\omega_2 & \omega_1 & 0\\
     \end{pmatrix}
\end{equation}
%
and $\{\mathbf{\boldsymbol\omega}_{IB}\}_B$ is the rotation of
$\mathcal{F_B}$ with respect to $\mathcal{F}_I$, expressed in
$\mathcal{F}_B$.
%
Solving Eq.~\ref{Eq:CSFixedKinematics} for
$\{\mathbf{\boldsymbol\omega}_{IB}^\times\}_B$ we obtain
%
\begin{equation}
   \{\mathbf{\boldsymbol\omega^\times}_{IB}\}_B =  -\dot{\mathbf{R}}_{BI}
   \mathbf{R}_{BI}^{T}
   \label{Eq:CSFixedKinematics2}
\end{equation}
%
Taking the derivative of Eq.~(\ref{Eq:CSFixedRotationMatrix}) with
respect to time yields
%
\begin{equation}
     {\mathbf{\dot R}_{BI}} = \mathbf{R}_{Bi}{\mathbf{\dot
     R}_{iI}}\label{Eq:CSFixedTimeDerivative}
\end{equation}
%
because by definition, for the CSFixed mode,  $ \mathbf{\dot R}_{Bi}
= \mathbf{0}$. Substituting Eq.~(\ref{Eq:CSFixedTimeDerivative})
into Eq.~(\ref{Eq:CSFixedKinematics2}) we obtain
%
%
\begin{equation}
   \{\mathbf{\boldsymbol\omega^\times}_{IB}\}_B =  -\mathbf{R}_{Bi}{\mathbf{\dot
     R}_{iI}} \mathbf{R}_{BI}^{T}
   \label{Eq:CSFixedKinematics3}
\end{equation}
%
where $\mathbf{R}_{Bi}$ is known from user input, and
$\mathbf{R}_{BI}$ is known from Eq.~(\ref{Eq:CSFixedRotationMatrix})
. GMAT knows how to calculate $\mathbf{\dot{R}}_{iI}$ for all
allowable $\mathcal{F}_i$ and details are contained in
Ch.~\ref{Ch:CoordinateSystems}.

In summary, in CSFixed mode, Eq.(\ref{Eq:CSFixedRotationMatrix}) is
used to calculate $\mathbf{R}_{BI}$, and
Eq.~(\ref{Eq:CSFixedKinematics3}) is used to calculate
$\{\mathbf{\boldsymbol\omega}_{IB}^x\}_B$.  If another attitude
parameterization is required, GMAT uses the algorithms in
Sec.~\ref{Sec:AttitudeParameterizations} to transform from
$\mathbf{R}_{BI}$ and $\{\mathbf{\boldsymbol\omega}_{IB}\}_B$ to the
required parameterization.  Now let's look at the spinning
spacecraft mode.

\subsubsection{Spinning Spacecraft Mode}

In spinning spacecraft mode, GMAT propagates the attitude by
assuming the spin axis direction is fixed in inertial space.  The
spacecraft attitude at some time, $t$, is determined from the
attitude initial conditions, the angular velocity vector, and the
elapsed time from the initial spacecraft epoch.  Let's take a closer
look at the calculations.

In the spinning spacecraft mode, the user provides three pieces of
information.  They first choose a coordinate system,
$\mathcal{F}_i$, in which to define the initial conditions.
Secondly, they define the initial orientation with respect to
$\mathcal{F}_i$ by providing $\mathbf{R}_{Bi}$ or an equivalent
parameterzation that is then converted to the DCM.   The user also
provides the angular velocity of the body axes with respect to the
inertial axes expressed in $\mathcal{F}_i$, $\{
\boldsymbol\omega_{IB}\}_i$.

To calculate $\mathbf{R}_{BI}(t)$ where $t$ is an arbitrary epoch,
we begin by calculating $\mathbf{R}_{B_{o}I}$ where
$\mathbf{R}_{B_{o}I} = \mathbf{R}_{BI}(t_o)$.  We calculate
$\mathbf{R}_{B_{o}I}$ using
%
\begin{equation}
     \mathbf{R}_{B_{o}I} =  \mathbf{R}_{Bi}\mathbf{R}_{iI}(t_o)
\end{equation}
%
where $\mathbf{R}_{Bi}$ comes from user provided data, and
$\mathbf{R}_{iI}(t_o)$ is calculated by GMAT and is dependent upon
$\mathcal{F}_i$.  See Ch.~\ref{Ch:CoordinateSystems} for details on
how GMAT calculates $\mathbf{R}_{iI}$ for all allowable coordinate
systems in GMAT.

Before calculating $\mathbf{R}_{BI}(t)$ we must determine the spin
axis in the body frame, $\{\boldsymbol\omega_{IB}\}_B$.  The user
provides $\{\boldsymbol\omega_{IB}\}_i$.  In spinning mode we assume
the spin axis direction is constant in inertial space and in the
body frame so $\{ \boldsymbol\omega_{IB} \}_B (t)$ $ = \{
\boldsymbol\omega_{IB} \}_B (t_o) = \{ \boldsymbol\omega_{IB} \}_B
$.  We can find the spin axis in the body frame using
$\mathbf{R}_{Bi}$ as follows
%
\begin{equation}
      \{ \boldsymbol\omega_{IB}\}_B = \mathbf{R}_{Bi} \{ \boldsymbol\omega_{IB}\}_i
\end{equation}
%
Once calculated, GMAT saves $\mathbf{R}_{B_{o}I}$ and $\{
\boldsymbol\omega_{IB}\}_B$ for use in calculating the attitude
orientation and rate at other epochs.

GMAT calculates $\mathbf{R}_{BI}(t)$ using the Euler axis/angle
rotation algorithm in Sec. \ref{Sec:AttitudeParameterizations}. The
Euler axis is simply the unitized angular velocity vector or,
%
\begin{equation}
     \mathbf{a} =   \frac{  \{ \boldsymbol\omega_{IB} \}_B  }{\omega_{IB} }
\end{equation}
%
where
%
\begin{equation}
     \omega_{IB} = \| \{\boldsymbol{\omega}_{IB} \}_B \|
\end{equation}
%
The Euler angle $\phi$ is calculated using
%
\begin{equation}
    \phi(t) = \omega_{IB}(t -t_o)
\end{equation}
%
where $t$ is the current epoch, and $t_o$ is the spacecraft's
initial epoch.  Let's define the rotation matrix that results from
the Euler axis/angle rotation using $\mathbf{a}$ and $\phi(t)$, as
$\mathbf{R}_{BB_{o}}(t)$.  We can calculate $\mathbf{R}_{BI}(t)$
using
%
\begin{equation}
     \mathbf{R}_{BI}(t) =
     \mathbf{R}_{BB_{o}}(t)\mathbf{R}_{B_{o}I}
\end{equation}
%

To summarize, in spinning mode the user provides $\mathbf{R}_{Bi}$
and  $\{ \boldsymbol\omega_{IB}\}_i$.  GMAT assumes that that the
spin axis direction is constant, and uses the Euler axis/angle
method to propagate the attitude to find $\mathbf{R}_{BI}$.

Now let's look at how GMAT performs conversions between the
different attitude parameterizations.

\subsection{6 DOF Modelling}

\begin{equation}
   \dot{\mathbf{X}} = \left[ \hspace{.05 in} \dot{\mathbf{r}}^T \hspace{.1 in} \dot{\mathbf{v}}^T \hspace{.1 in}  \dot{\mathbf{h}}^T \hspace{.1 in}  \dot{\mathbf{q}}^T  \hspace{.05 in} \right]^T
\end{equation}
%
\begin{equation}
   \dot{\mathbf{h}} = \mathbf{T} - \mathbf{\boldsymbol\omega}\times\mathbf{h}
\end{equation}
%
\begin{equation}
   \dot{\mathbf{q}} = \frac{1}{2}\mathbf{\Omega}\mathbf{q}
\end{equation}
%
where
%
\begin{equation}
    \mathbf{\Omega} = \left(\begin{array}{ccccc}
     0 & \omega_z  & -\omega_y & \omega_x\\
     -\omega_z & 0 & \omega_x & \omega_y\\
     \omega_y & -\omega_x & 0 & \omega_z\\
     -\omega_x & -\omega_y & -\omega_z & 0\\
  \end{array}\right)
\end{equation}
%
\begin{equation}
    \mathbf{\boldsymbol\omega} = \mathbf{I}^{-1}\mathbf{h}
\end{equation}

\subsection{Attitude Parameterizations \\ and Conversions}
\label{Sec:AttitudeParameterizations}

This section details how GMAT converts between different attitude
parameterizations.  For each conversion type, any singularities that
may occur are addressed.  The orientation parameterizations in GMAT
include the DCM, Euler Angles, quaternions, and Euler axis/angle.
The body rate parameterizations include Euler angle rates and
angular velocity.  We begin with the algorithm to transform from the
quaternions to the DCM.

\subsubsection{Conversion:  Quaternions to DCM}\label{sec:AttQuattoR}
\index{Attitude Parameterization!Quaternions to DCM}

Given:  $\mathbf{q}$, $q_4$

\noindent Find:  $\mathbf{R}$

\noindent Name:  \emph{ToCosineMatrix}

\begin{equation}
    \mathbf{q} = \left( q_1 \hspace{.1 in} q_2 \hspace{.1 in} q_3 \right)^T
\end{equation}
%
\begin{equation}
     \mathbf{q}^{\times} = \begin{pmatrix}
     0 & -q3 & q2\\
     q3 & 0 & -q1\\
     -q2 & q1 & 0\\
     \end{pmatrix}
\end{equation}
%
\begin{equation}
    c = \frac{1}{q_1^2 + q_2^2 + q_3^2 + q_4^2}
\end{equation}
%
\begin{equation}
     \mathbf{R} = c\left[ (q_4^2 - \mathbf{q}^T\mathbf{q})\mathbf{I}_3 +
      2\mathbf{q}\mathbf{q}^T -2q_4\mathbf{q}^{\times}\right]
\end{equation}


\subsubsection{Conversion:  DCM to Quaternions} \label{sec:DCMtoQuat}
\index{Attitude Parameterization!DCM to Quaternions}

Given:  $\mathbf{R}$

\noindent Find: $\mathbf{q}$, $q_4$

Define following vector
%
\begin{equation}
   \mathbf{v} = [ \hspace{.02 in} R_{11} \hspace{.1 in} R_{22}\hspace{.1 in}
   R_{33} \hspace{.1 in}  \mbox{trace}(\mathbf{R}) \hspace{.02 in}]
\end{equation}
%
Define $i_m$ as the index of the maximum component of $\mathbf{v}$
%

\noindent if $i_m = 1$
%
\begin{equation}
     \mathbf{q}''  = \begin{pmatrix}
     2v_{i_m} + 1 - \mbox{trace}(\mathbf{R})\\
     R_{12} + R_{21}\\
     R_{13} + R_{31}\\
     R_{23} - R_{32}\\
     \end{pmatrix}
\end{equation}
%
if $i_m = 2$
%
\begin{equation}
     \mathbf{q}''  = \begin{pmatrix}
     R_{21} + R_{12}\\
     2v_{i_m} + 1 - \mbox{trace}(\mathbf{R})\\
     R_{23} + R_{32}\\
     R_{31} - R_{13}\\
     \end{pmatrix}
\end{equation}
%
if $i_m = 3$
%
\begin{equation}
     \mathbf{q}''  = \begin{pmatrix}
     R_{31} + R_{13}\\
     R_{32} + R_{23}\\
     2v_{i_m} + 1 - \mbox{trace}(\mathbf{R})\\
     R_{12} - R_{21}\\
     \end{pmatrix}
\end{equation}
%
if $i_m = 4$
%
\begin{equation}
     \mathbf{q}''  = \begin{pmatrix}
     R_{23} - R_{32}\\
     R_{31} - R_{13}\\
     R_{12} - R_{21}\\
     1 + \mbox{trace}(\mathbf{R})\\
     \end{pmatrix}
\end{equation}
%
We normalize $\mathbf{q}''$ using
%
\begin{equation}
    \mathbf{q}' = \frac{\mathbf{q}''}{\| \mathbf{q}'' \|}
\end{equation}
%
Finally,
%
\begin{equation}
   \mathbf{q} = [\hspace{.05 in} q_{1}' \hspace{.2 in} q_{2}' \hspace{.2 in} q_3'
   \hspace{.05 in}]^T
\end{equation}
%
and
%
\begin{equation}
     q_4 = q_4'
\end{equation}
%Note: There is not a unique quaternion for a given DCM.  GMAT
%assumes that the ``+" sign is used in Eq.~(\ref{Eq:q_4}).

\subsubsection{Conversion:  DCM to Euler\\ Axis/Angle}

\index{Attitude Parameterization!DCM to Axis/Angle}

Given:  $\mathbf{R}$

\noindent Find: $\mathbf{a}$, $\phi$

\begin{equation}
     \mathbf{R}  = \begin{pmatrix}
     R_{11} & R_{12} & R_{13}\\
     R_{21} & R_{22} & R_{23}\\
     R_{31} & R_{32} & R_{33}\\
     \end{pmatrix}
\end{equation}
%
\begin{equation}
   \phi = \cos^{-1}\left( \frac{1}{2}\left(\mbox{trace}(\mathbf{R}) -
   1 \right)\right)
\end{equation}
%
\begin{equation}
    \mathbf{a} = \frac{1}{2\sin{\phi}}\begin{pmatrix}
     R_{23} - R_{32}\\
     R_{31} - R_{13}\\
     R_{12} - R_{21}\\
     \end{pmatrix}
\end{equation}
%
If $\|\sin{\phi} \| < 10^{-14}$, then we assume
%
\begin{equation}
    \mathbf{a} = \left[\hspace{.05 in} 1 \hspace{.1 in} 0 \hspace{.1 in}
    0 \hspace{.05 in} \right]^T
\end{equation}
%
Note that if $\|\sin{\phi} \| < 10^{-14}$ then $\cos{\phi} \approx
1 $ and we arrive at a DCM of $\mathbf{I}_3$.


\subsubsection{Conversion:  Euler Axis/Angle to DCM} \index{Attitude
Parameterization!Axis/Angle to DCM} \label{Sec:EulerAxis/AngletoDCM}

Given:  $\mathbf{a}$, $\phi$

\noindent Find: $\mathbf{R}$

\begin{equation}
     \mathbf{a}^{\times} = \begin{pmatrix}
     0 & -a_3 & a_2\\
     a_3 & 0 & -a_1\\
     -a_2 & a_1 & 0\\
     \end{pmatrix}
\end{equation}
%
\begin{equation}
    \mathbf{R} = \cos{\phi}\mathbf{I}_3 + (1 -
    \cos{\phi})\mathbf{a}\mathbf{a}^T -
    \sin{\phi}\mathbf{a}^{\times}
\end{equation}

\subsubsection{Conversion:  Euler Angles to DCM}
\label{sec:AttEulerAnglestoDCM}

Given:  Sequence order  ( i.e. 123, 121, .... 313),  $\theta_1$,
$\theta_2$, $\theta_3$

\noindent Find: $\mathbf{R}$

We'll give an example for a 321 rotation, and then present results
for the remaining 11 Euler angle sequences.  First, let's define
$\mathbf{R}_3(\theta_1)$, $\mathbf{R}_2(\theta_2)$, and
$\mathbf{R}_1(\theta_3)$.
%
\begin{equation}
    \mathbf{R}_3(\theta_1) =  \begin{pmatrix}
      \cos{\theta_1}  & \sin{\theta_1} & 0    \\
      -\sin{\theta_1} & \cos{\theta_1} & 0    \\
      0               & 0              & 1
     \end{pmatrix}
\end{equation}
%
\begin{equation}
    \mathbf{R}_2(\theta_2) =  \begin{pmatrix}
      \cos{\theta_2}  & 0              & -\sin{\theta_2}    \\
      0               & 1              & 0    \\
      \sin{\theta_2}  & 0              & \cos{\theta_2}
     \end{pmatrix}
\end{equation}
%
\begin{equation}
    \mathbf{R}_1(\theta_3) =  \begin{pmatrix}
      1               & 0                & 0    \\
      0               & \cos{\theta_3}   & \sin{\theta_3}    \\
      0               & -\sin{\theta_3}  & \cos{\theta_3}
     \end{pmatrix}
\end{equation}
%
Now we can write
%
\begin{eqnarray}
       &\mathbf{R}_{321} =  \mathbf{R}_1(\theta_3)\mathbf{R}_2(\theta_2)\mathbf{R}_3(\theta_1) = \nonumber\\
      &=         \begin{pmatrix}
      1               & 0                & 0    \\
      0               & c_3   & s_3    \\
      0               & -s_3  & c_3
     \end{pmatrix}
     %
      \begin{pmatrix}
      c_2  & 0              & -s_2    \\
      0               & 1              & 0    \\
      s_2  & 0              & c_2
     \end{pmatrix}
     %
     \begin{pmatrix}
      c_1   & s_1 & 0    \\
      -s_1  & c_1 & 0    \\
      0               & 0              & 1
     \end{pmatrix}\nonumber\\
\end{eqnarray}
%
where $c_1 =\cos{\theta_1}$, $s_1 = \sin{\theta_1}$ etc.  We can
rewrite $\mathbf{R}_{321} $ as
%
\begin{equation}
       \mathbf{R}_{321}  =\begin{pmatrix}
      c_2c_1              &  c_2s_1             & -s_2 \\
     -c_3s_1 + s_3s_2c_1  &  c_3c_1 + s_3s_2s_1 & s_3c_2 \\
     s_3s_1 + c_3s_2c_1   &  -s_3c_1 +c_3s_2s_1 & c_3c_2
     \end{pmatrix} \label{Eq:R321}
\end{equation}
%

The approach is similar for the remaining 11 Euler angle sequences.
Rather than derive the DCM matrices for the remaining 11 sequences,
we present them in Table \ref{table:EulerAnglestoDCM}.

\subsubsection{Conversion:  DCM to Euler Angles}
\label{sec:AttDCMtoEulerAngles}

Given:  Sequence order  ( i.e. 123, 121, .... 313), $\mathbf{R}$

\noindent Find:  $\theta_1$, $\theta_2$, $\theta_3$

We'll give an example for a 321 rotation, and then present results
for the remaining 11 Euler angle sequences.  Examining,
Eq.~(\ref{Eq:R321}), we see that
%
\begin{equation}
     \frac{ R_{21} }  { R_{11}  } = \frac{  \cos{\theta_2}\sin{\theta_1}     }
                                 {  \cos{\theta_2}\cos{\theta_1}    }
\end{equation}
%
From this we can see that
%
\begin{equation}
    \theta_1 =  \tan^{-1}{\frac{ R_{21} }  { R_{11}  }}
\end{equation}
%
Further inspection of Eq.~(\ref{Eq:R321}) shows us that
%
\begin{equation}
    \theta_2 = \sin^{-1}{R_{13}}
\end{equation}
%
At first glance, we may choose to calculate $\theta_3$ using
$\theta_3 = \tan^{-1}{(R_{23}/R_{33})}$.  However, in the case that
$\theta_2 = 90^\circ$, this would result in the indeterminate case,
$\theta_3 =$ $\tan^{-1}(R_{23}/R_{33})$ $= \tan^{-1}(0/0)$.  An
improved method, found in the ADEAS mathematical specifications
document, is to determine $\theta_3$ using
%
\begin{equation}
    \theta_3 = \tan^{-1} \left(\frac{ R_{31} \sin{\theta_1} - R_{32} \cos{\theta_1} }
    { -R_{21} \sin{\theta_1} + R_{22} \cos{\theta_1}} \right)
    \label{Eq:Rtotheta3}
\end{equation}
%
Substituting values from Eq.~(\ref{Eq:R321}) into
Eq.~(\ref{Eq:Rtotheta3}), and using abbreviated notation, we see
that
%
\begin{equation}
     \theta_3 = \tan^{-1} \left(  \frac{ s_1( s_3s_1 + c_3s_2c_1) - c_1(-s_3c_1 + c_3s_2s_1 )}
    { s_1(c_3s_1 - s_3s_2c_1  ) + c_1( c_3c_1 + s_3s_2s_1 ) }  \right)
\end{equation}
%
Now, if $\theta_2 = 90^\circ$, and we substitute $c_2 = 0$ and $s_2
= 1$ into the above equation, we see we get a determinate form.
Results for all twelve Euler Sequences are shown in Table
\ref{table:DCMtoEulerAngles}.

\noindent Note:  All $\tan^{-1}$ use a quadrant check ( equaivalent
to atan2 ) to make sure the the correct quadrant is chosen.

\subsubsection{Conversion:  Angular Velocity to \\ Euler Angles
Rates}

Given:  Sequence ( i.e. 123, 121, .... 313), $\theta_2$,
$\theta_3$ $\boldsymbol\omega$

\noindent Find: $\dot\theta_1$, $\dot\theta_2$, $\dot\theta_3$
%
\begin{equation}
    \begin{pmatrix}
         \dot\theta_1\\
         \dot\theta_2\\
         \dot\theta_3
    \end{pmatrix}
    %
    = \mathbf{S}^{-1}(\theta_2,\theta_3)\boldsymbol\omega
\end{equation}
%
$\mathbf{S}^{-1}(\theta_2,\theta_3)$ is dependent upon the Euler
sequence.  Table \ref{table:EulerAngleKinematics} contains the
different expressions for $\mathbf{S}^{-1}(\theta_2,\theta_3)$ for
each of the 12 unique Euler sequences.

Note:  Each of the forms of $\mathbf{S}^{-1}$ have a possible
singularity due to the appearance of either $\sin{\theta_2}$ or
$\cos{\theta_2}$ in the denominator.  If GMAT encounters a
singularity, an error message is thrown, and the zero vector is
returned.

\subsubsection{Conversion:  Euler Angles Rates to Angular Velocity}

\noindent Given: Sequence ( i.e. 123, 121, .... 313), $\theta_2$,
$\theta_3$, $\dot\theta_1$, $\dot\theta_2$, $\dot\theta_3$

\noindent Find: $\boldsymbol\omega$
%
\begin{equation}
    \boldsymbol\omega = \mathbf{S}(\theta_2,\theta_3)
    %
        \begin{pmatrix}
         \dot\theta_1\\
         \dot\theta_2\\
         \dot\theta_3
    \end{pmatrix}
\end{equation}
%
$\mathbf{S}(\theta_2,\theta_3)$ is dependent upon the Euler
sequence.  Table \ref{table:EulerAngleKinematics} contains the
different expressions for $\mathbf{S}^{-1}(\theta_2,\theta_3)$ for
each of the 12 unique Euler sequences.

\subsubsection{Conversion:  Quaternions to Euler Angles}

\noindent Given: $\mathbf{q}$, $q_4$, Euler Sequence

\noindent Find: $\theta_1$, $\theta_2$, and $\theta_3$

There is not a direct transformation to convert from the quaternions
to the Euler Angles.  GMAT first converts from the quaternion to the
DCM using the algorithm in Sec. \ref{sec:AttQuattoR}.  The DCM is
then used to calculate the Euler Angles for the given Euler angle
sequence using the algorithm in Sec. \ref{sec:AttDCMtoEulerAngles}.
%\section{Attitude Initial Conditions}

\subsubsection{Conversion:  Euler Angles to Quaternions}

\noindent Given: $\theta_1$, $\theta_2$, and $\theta_3$, Euler
Sequence

\noindent Find: $\mathbf{q}$, $q_4$

There is not a direct transformation to convert from Euler Angles to
quaternions.  GMAT first converts from the Euler Angles to the DCM
using the algorithm in Sec. \ref{sec:AttEulerAnglestoDCM}.  The DCM
is then used to calculate the quaternions using the algorithm in
Sec. \ref{sec:DCMtoQuat}.
%\section{Attitude Initial Conditions}



\subsubsection{Euler Angles to $\mathbf{R}$ Matrix}



\onecolumn
 \begin{table}[h]
        \centering
        \vspace{0 pt}
        \caption{Rotation Matrices for 12 Unique Euler Angle Rotation Sequences}
        \begin{tabular}{clccccccc}  \hline \hline
        \\
     %------------------------------------
     %------------------------------------
     $\mathbf{R}_3(\theta_3)\mathbf{R}_2(\theta_2)\mathbf{R}_1(\theta_1)$
     &
     $=\begin{pmatrix}
     c_3c_2    & c_3 s_2 s_1 + s_3 c_1  & -c_3 s_2 c_1 + s_1 s_3\\
     -s_3 c_2  & -s_3 s_2 s_1 + c_3 c_1 & s_3 s_2 c_1 + c_3 s_1\\
     s_2       & -c_2 s_1               & c_2 c_1\\
     \end{pmatrix}  \vspace{.1 in}$\\
     %
     %------------------------------------
     %------------------------------------
     $\mathbf{R}_2(\theta_3)\mathbf{R}_3(\theta_2)\mathbf{R}_1(\theta_1)$
     &
     $=\begin{pmatrix}
     c_3c_2   &   c_3 s_2 c_1 + s_1 s_3  &  c_3s_2s_1-s_3c_1 \\
      -s_2    &   c_2c_1                 &  c_2s_1\\
      s_3c_2  &   s_3s_2c_1 - c_3s_1     &  s_3s_2s_1 + c_3c_1 \\
     \end{pmatrix}   \vspace{.1 in}$\\
     %
     $\mathbf{R}_1(\theta_3)\mathbf{R}_3(\theta_2)\mathbf{R}_2(\theta_1)$
     &
     $=\begin{pmatrix}
     c_2c_1               &    s_2     &   -c_2s_1\\
     -c_3s_2c_1 + s_3 s_1 & c_3c_2     & c_3s_2s_1 + s_3c_1     \\
     s_3s_2c_1 +c_3s_1    &  -s_3c_2   &  -s_3s_2s_1 + c_3 c_1  \\
    \end{pmatrix}   \vspace{.1 in}$\\
     %
     %------------------------------------
     %------------------------------------
     $\mathbf{R}_3(\theta_3)\mathbf{R}_1(\theta_2)\mathbf{R}_2(\theta_1)$
     &
     $=\begin{pmatrix}
      c_3c_1 + s_3s_2s_1     &  s_3c_2    & -c_3s_1 + s_3s_2c_1    \\
      -s_3c_1 + c_3 s_2s_1   &  c_3c_2    & s_3s_1 + c_3s_2c_1    \\
      c_2s_1                 &  -s_2      & c_2c_1
     \end{pmatrix}  \vspace{.1 in}$\\
     %
     $\mathbf{R}_2(\theta_3)\mathbf{R}_1(\theta_2)\mathbf{R}_3(\theta_1)$
     &
     $=\begin{pmatrix}
      c_3c_1 - s_3s_2s_1     & c_3s_1 + s_3s_2c_1 & -s_3c_2 \\
      -c_2s_1                & c_2c_1             &  s_2  \\
      s_3c_1 + c_3s_2s_1     & s_3s_1 - c_3s_2c_1 & c_3c_2
     \end{pmatrix}  \vspace{.1 in}$\\
     %
     %------------------------------------
     %------------------------------------
     $\mathbf{R}_1(\theta_3)\mathbf{R}_2(\theta_2)\mathbf{R}_3(\theta_1)$
     &
     $=\begin{pmatrix}
      c_2c_1              &  c_2s_1             & -s_2 \\
     -c_3s_1 + s_3s_2c_1  &  c_3c_1 + s_3s_2s_1 & s_3c_2 \\
     s_3s_1 + c_3s_2c_1   &  -s_3c_1 +c_3s_2s_1 & c_3c_2
     \end{pmatrix}  \vspace{.1 in}$\\
     %
     $\mathbf{R}_1(\theta_3)\mathbf{R}_2(\theta_2)\mathbf{R}_1(\theta_1)$
     &
     $=\begin{pmatrix}
     c_2                  & s_2s_1              & -s_2c_1 \\
     s_3s_2               & c_3c_1 - s_3c_2s_1  & c_3s_1 + s_3c_2c_1\\
     c_3s_2               & -s_3c_1 - c_3c_2s_1 & -s_3s_1 +c_3c_2c_1
     \end{pmatrix}  \vspace{.1 in}$\\
          %
     %------------------------------------
     %------------------------------------
     $\mathbf{R}_1(\theta_3)\mathbf{R}_3(\theta_2)\mathbf{R}_1(\theta_1)$
     &
     $=\begin{pmatrix}
      c_2                  & s_2c_1              & s_2s_1 \\
     -c_3s_2              & c_3c_2c_1 - s_3s_1  & c_3c_2s_1 + s_3c_1\\
     s_3s_2               & -s_3c_2c_1 - c_3s_1 & -s_3c_2s_1 +   c_3c_1
     \end{pmatrix}  \vspace{.1 in}$\\
               %
     $\mathbf{R}_2(\theta_3)\mathbf{R}_1(\theta_2)\mathbf{R}_2(\theta_1)$
     &
     $=\begin{pmatrix}
     c_3c_1 - s_3c_2s_1   &  s_3s_2    &  -c_3s_1 -s_3c_2c_1\\
     s_2s_1               &  c_2       &  s_2c_1            \\
     s_3c_1 + c_3c_2s_1   &  -c_3s_2   &  -s_3s_1 + c_3c_2c_1
     \end{pmatrix}  \vspace{.1 in}$\\
                    %
     %------------------------------------
     %------------------------------------
     $\mathbf{R}_2(\theta_3)\mathbf{R}_3(\theta_2)\mathbf{R}_2(\theta_1)$
     &
     $=\begin{pmatrix}
     c_3c_2c_1 - s_3s_1   &  c_3s_2  & -c_3c_2s_1 - s_3c_1\\
     -s_2c_1              &  c_2     & s_2s_1             \\
     s_3c_2c_1 + c_3s_1   &  s_3s_2  & -s_3c_2s_1 + c_3c_1
     \end{pmatrix}  \vspace{.1 in}$\\
                         %
     $\mathbf{R}_3(\theta_3)\mathbf{R}_1(\theta_2)\mathbf{R}_3(\theta_1)$
     &
     $=\begin{pmatrix}
      c_3c_1 - s_3c_2s_1 & c_3s_1 + s_3c_2c_1  &  s_3s_2\\
     -s_3c_1 - c_3c_2s_1 & -s_3s_1 + c_3c_2c_1 &  c_3s_2\\
      s_2s_1             &  -s_2c_1            &  c_2
     \end{pmatrix}  \vspace{.1 in}$\\
                              %
     %------------------------------------
     %------------------------------------
     $\mathbf{R}_3(\theta_3)\mathbf{R}_2(\theta_2)\mathbf{R}_3(\theta_1)$
     &
     $=\begin{pmatrix}
      c_3c_2c_1-s_3s_1     & c_3c_2s_1 + s_3c_1  &  -c_3s_2\\
      -s_3c_2c_1 - c_3s_1  & -s_3c_2s_1 + c_3c_1 &  s_3s_2\\
      s_2c_1               & s_2s_1              &  c_2
     \end{pmatrix}  \vspace{.1 in}$\\
         \hline \hline
        \end{tabular}
        \label{table:EulerAnglestoDCM}
\end{table}
\twocolumn

\subsubsection{Euler Rates and Angular Velocity} \onecolumn
 \begin{table}[h]
        \centering
        \vspace{0 pt}
        \caption{Kinematics of Euler Angle Rotation Sequences}
        \begin{tabular}{cllcccccc}  \hline \hline
        Euler Sequence & $\mathbf{S}(\theta_2,\theta_3)$ &
        $\mathbf{S}^{-1}(\theta_2,\theta_3)$\\
        \hline
        \\
        %------------------
        %------------------
     $\mathbf{R}_3(\theta_3)\mathbf{R}_2(\theta_2)\mathbf{R}_1(\theta_1)$
     &
     $\begin{pmatrix}
        c3c2&          s3&                0\\
       -s3c2&          c3&                0\\
          s2&                0&                1\\
     \end{pmatrix}  \vspace{.1 in}$
     &
     $\begin{pmatrix}
                  c3/c2&         -s3/c2&                        0\\
                  s3&                  c3&                        0\\
                  -s2c3/c2&  s3s2/c2& 1\\
     \end{pmatrix}  \vspace{.1 in}$\\
     %
        %------------------
        %------------------
     $\mathbf{R}_2(\theta_3)\mathbf{R}_3(\theta_2)\mathbf{R}_1(\theta_1)$
     &
     $\begin{pmatrix}
         c3c2&        -s3&               0\\
        -s2&               0&               1\\
         s3c2&         c3&               0\\
     \end{pmatrix}  \vspace{.1 in}$
     &
     $\begin{pmatrix}
         c3/c2&                       0&         s3/c2\\
           -s3&                       0&                 c3\\
          s2c3/c2&                       1& s3s2/c2\\
     \end{pmatrix}  \vspace{.1 in}$\\
     %
     %------------------
     %------------------
     $\mathbf{R}_1(\theta_3)\mathbf{R}_3(\theta_2)\mathbf{R}_2(\theta_1)$
     &
     $\begin{pmatrix}
        s2&                0&                1\\
       c3c2&          s3&                0\\
      -s3c2&          c3&                0\\
     \end{pmatrix}  \vspace{.1 in}$
     &
     $\begin{pmatrix}
                        0&          c3/c2&         -s3/c2\\
                        0&                  s3&                  c3\\
                        1& -s2c3/c2&  s3s2/c2\\
     \end{pmatrix}  \vspace{.1 in}$\\
     %
     %------------------
     %------------------
     $\mathbf{R}_3(\theta_3)\mathbf{R}_1(\theta_2)\mathbf{R}_2(\theta_1)$
     &
     $\begin{pmatrix}
         s3c2&         c3&               0\\
         c3c2&        -s3&               0\\
        -s2&               0&               1\\
     \end{pmatrix}  \vspace{.1 in}$
     &
     $\begin{pmatrix}
         s3/c2&         c3/c2&                       0\\
                 c3&                -s3&                       0\\
 s3s2/c2& s2c3/c2&                       1\\
     \end{pmatrix}  \vspace{.1 in}$\\
     %
     %------------------
     %------------------
     $\mathbf{R}_2(\theta_3)\mathbf{R}_1(\theta_2)\mathbf{R}_3(\theta_1)$
     &
     $\begin{pmatrix}
       -s3c2&          c3&                0\\
          s2&                0&                1\\
        c3c2&          s3&                0\\
     \end{pmatrix}  \vspace{.1 in}$
     &
     $\begin{pmatrix}
         -s3/c2&                        0&          c3/c2\\
                  c3&                        0&                  s3\\
         s3s2/c2&                        1& -s2c3/c2\\
     \end{pmatrix}  \vspace{.1 in}$\\
     %------------------
     %------------------
     $\mathbf{R}_1(\theta_3)\mathbf{R}_2(\theta_2)\mathbf{R}_3(\theta_1)$
     &
     $\begin{pmatrix}
                -s2&               0&               1\\
       s3c2&         c3&               0\\
       c3c2&        -s3&               0\\
     \end{pmatrix}  \vspace{.1 in}$
     &
     $\begin{pmatrix}
                       0&         s3/c2&         c3/c2\\
                       0&                 c3&                -s3\\
                       1& s3s2/c2& s2c3/c2\\
     \end{pmatrix}  \vspace{.1 in}$\\
          %
     %------------------
     %------------------
     $\mathbf{R}_1(\theta_3)\mathbf{R}_2(\theta_2)\mathbf{R}_1(\theta_1)$
     &
     $\begin{pmatrix}
                c2&               0&               1\\
       s3s2&         c3&               0\\
      c3s2&        -s3&               0\\
     \end{pmatrix}  \vspace{.1 in}$
     &
     $\begin{pmatrix}
                        0&          s3/s2&          c3/s2\\
                        0&                  c3&                 -s3\\
                        1& -s3c2/s2& -c3c2/s2\\
     \end{pmatrix}  \vspace{.1 in}$\\
                    %
     %------------------
     %------------------
     $\mathbf{R}_1(\theta_3)\mathbf{R}_3(\theta_2)\mathbf{R}_1(\theta_1)$
     &
     $\begin{pmatrix}
          c2&                0&                1\\
       -c3s2&          s3&                0\\
        s3s2&          c3&                0\\
     \end{pmatrix}  \vspace{.1 in}$
     &
     $\begin{pmatrix}
                        0&         -c3/s2&          s3/s2\\
                        0&                  s3&                  c3\\
                        1&  c3c2/s2& -s3c2/s2\\
     \end{pmatrix}  \vspace{.1 in}$\\
                    %
     %------------------
     %------------------
     $\mathbf{R}_2(\theta_3)\mathbf{R}_1(\theta_2)\mathbf{R}_2(\theta_1)$
     &
     $\begin{pmatrix}
        s3s2&          c3&                0\\
          c2&                0&                1\\
       -c3s2&          s3&                0\\
     \end{pmatrix}  \vspace{.1 in}$
     &
     $\begin{pmatrix}
          s3/s2&                        0&         -c3/s2\\
                  c3&                        0&                  s3\\
         -s3c2/s2&                        1&  c3c2/s2\\
     \end{pmatrix}  \vspace{.1 in}$\\
                              %
     %------------------
     %------------------
     $\mathbf{R}_2(\theta_3)\mathbf{R}_3(\theta_2)\mathbf{R}_2(\theta_1)$
     &
     $\begin{pmatrix}
       c3s2&        -s3&               0\\
         c2&               0&               1\\
       s3s2&         c3&               0\\
     \end{pmatrix}  \vspace{.1 in}$
     &
     $\begin{pmatrix}
          c3/s2&                        0&          s3/s2\\
                 -s3&                        0&                  c3\\
        -c3c2/s2&                        1& -s3c2/s2\\
     \end{pmatrix}  \vspace{.1 in}$\\
                              %
     %------------------
     %------------------
     $\mathbf{R}_3(\theta_3)\mathbf{R}_1(\theta_2)\mathbf{R}_3(\theta_1)$
     &
     $\begin{pmatrix}
       s3s2&         c3&               0\\
       c3s2&        -s3&               0\\
         c2&               0&               1\\
     \end{pmatrix}  \vspace{.1 in}$
     &
     $\begin{pmatrix}
          s3/s2&          c3/s2&                        0\\
                  c3&                 -s3&                        0\\
      -s3c2/s2& -c3c2/s2&                        1\\
     \end{pmatrix}  \vspace{.1 in}$\\
     %
          %------------------
     %------------------
     $\mathbf{R}_3(\theta_3)\mathbf{R}_2(\theta_2)\mathbf{R}_3(\theta_1)$
     &
     $\begin{pmatrix}
        -c3s2&          s3&                0\\
         s3s2&          c3&                0\\
           c2&                0&                1\\
     \end{pmatrix}  \vspace{.1 in}$
     &
     $\begin{pmatrix}
         -c3/s2&          s3/s2&                        0\\
                  s3&                  c3&                        0\\
        c3c2/s2& -s3c2/s2&                        1\\
     \end{pmatrix}  \vspace{.1 in}$\\
         \hline \hline
        \end{tabular}
        \label{table:EulerAngleKinematics}
\end{table}

\begin{table}[h]
        \centering
        \vspace{0 pt}
        \caption{ Computation of Euler Angles from DCM}
        \begin{tabular}{llllllll}  \hline \hline \\
        Euler Sequence & Euler Angle Computations \\
        \hline \\
     $\mathbf{R}_3(\theta_3)\mathbf{R}_2(\theta_2)\mathbf{R}_1(\theta_1)$
     \hspace{.1 in}
     &
      $\theta_1 =  \tan^{-1}(-R_{32}/R_{33})$ &
      $\theta_2 =  \sin^{-1}(R_{31})$ &
      $\theta_3  = \tan^{-1}\left(\displaystyle\frac{R_{13}\sin{\theta_1}+
      R_{12}\cos{\theta_1}}{R_{23}\sin{\theta_1}+R_{22}\cos{\theta_1}}\right )$\vspace{.15 in}
     \\
     %
     $\mathbf{R}_2(\theta_3)\mathbf{R}_3(\theta_2)\mathbf{R}_1(\theta_1)$ \hspace{.1 in}
     &
     $\theta_1 =  \tan^{-1}(R_{23}/R_{22})$ &
         $\theta_2 =  \sin^{-1}(-R_{21})$ &
         $\theta_3  = \tan^{-1}\left(\displaystyle\frac{R_{12}\sin{\theta_1}- R_{13}\cos{\theta_1}}{-R_{32}\sin{\theta_1}+
         R_{33}\cos{\theta_1}}\right)$ \vspace{.15 in}
     \\
     $\mathbf{R}_1(\theta_3)\mathbf{R}_3(\theta_2)\mathbf{R}_2(\theta_1)$
     &
    $\theta_1 =  \tan^{-1}(-R_{13}/R_{11})$ &
         $\theta_2 =  \sin^{-1}(R_{12})$ &
         $\theta_3  = \tan^{-1}\left(\displaystyle\frac{R_{21}\sin{\theta_1}+ R_{23}\cos{\theta_1}}{R_{31}\sin{\theta_1}+
         R_{33}\cos{\theta_1}}\right)$ \vspace{.15 in}
     \\
     %
     $\mathbf{R}_3(\theta_3)\mathbf{R}_1(\theta_2)\mathbf{R}_2(\theta_1)$
     &
         $\theta_1 =  \tan^{-1}(R_{31}/R_{33})$ &
         $\theta_2 =  \sin^{-1}(-R_{32})$ &
         $\theta_3  = \tan^{-1}\left(\displaystyle\frac{R_{23}\sin{\theta_1}- R_{21}\cos{\theta_1}}{-R_{13}\sin{\theta_1}+
         R_{11}\cos{\theta_1}}\right)$ \vspace{.15 in}
     \\
     $\mathbf{R}_2(\theta_3)\mathbf{R}_1(\theta_2)\mathbf{R}_3(\theta_1)$
     &
     $\theta_1 =  \tan^{-1}(-R_{21}/R_{22})$ &
     $\theta_2 =  \sin^{-1}(R_{23})$ &
     $\theta_3  = \tan^{-1}\left(\displaystyle\frac{R_{32}\sin{\theta_1}+ R_{31}\cos{\theta_1}}{R_{12}\sin{\theta_1}+
     R_{11}\cos{\theta_1}}\right)$ \vspace{.15 in}
     \\
     %
     $\mathbf{R}_1(\theta_3)\mathbf{R}_2(\theta_2)\mathbf{R}_3(\theta_1)$
     &
     $\theta_1 =  \tan^{-1}(R_{12}/R_{11})$ &
     $\theta_2 =  \sin^{-1}(-R_{13})$ &
     $\theta_3  = \tan^{-1}\left(\displaystyle\frac{R_{31}\sin{\theta_1}- R_{32}\cos{\theta_1}}{-R_{21}\sin{\theta_1}+
     R_{22}\cos{\theta_1}}\right)$ \vspace{.15 in}\\
     %
     $\mathbf{R}_1(\theta_3)\mathbf{R}_2(\theta_2)\mathbf{R}_1(\theta_1)$
     &
     $\theta_1 =  \tan^{-1}(R_{12}/(-R_{13}))$ &
     $\theta_2 =  \cos^{-1}(R_{11})$ &
     $\theta_3  = \tan^{-1}\left(\displaystyle\frac{-R_{33}\sin{\theta_1}- R_{32}\cos{\theta_1}}{R_{23}\sin{\theta_1}+
     R_{22}\cos{\theta_1}}\right)$ \vspace{.15 in}\\
     %
     $\mathbf{R}_1(\theta_3)\mathbf{R}_3(\theta_2)\mathbf{R}_1(\theta_1)$
     &
     $\theta_1 =  \tan^{-1}(R_{13}/(R_{12}))$ &
     $\theta_2 =  \cos^{-1}(R_{11})$ &
     $\theta_3  = \tan^{-1}\left(\displaystyle\frac{-R_{22}\sin{\theta_1} + R_{23}\cos{\theta_1}}{-R_{32}\sin{\theta_1}+
     R_{33}\cos{\theta_1}}\right)$ \vspace{.15 in}\\
     %
     $\mathbf{R}_2(\theta_3)\mathbf{R}_1(\theta_2)\mathbf{R}_2(\theta_1)$
     &
     $\theta_1 =  \tan^{-1}(R_{21}/(R_{23}))$ &
     $\theta_2 =  \cos^{-1}(R_{22})$ &
     $\theta_3  = \tan^{-1}\left(\displaystyle\frac{-R_{33}\sin{\theta_1} + R_{31}\cos{\theta_1}}{-R_{13}\sin{\theta_1}+
     R_{11}\cos{\theta_1}}\right)$ \vspace{.15 in}\\
     %
     $\mathbf{R}_2(\theta_3)\mathbf{R}_3(\theta_2)\mathbf{R}_2(\theta_1)$
     &
     $\theta_1 =  \tan^{-1}(R_{23}/(-R_{21}))$ &
     $\theta_2 =  \cos^{-1}(R_{22})$ &
     $\theta_3  = \tan^{-1}\left(\displaystyle\frac{-R_{11}\sin{\theta_1} - R_{13}\cos{\theta_1}}{R_{31}\sin{\theta_1}+
     R_{33}\cos{\theta_1}}\right)$ \vspace{.15 in}\\
     %
     $\mathbf{R}_3(\theta_3)\mathbf{R}_1(\theta_2)\mathbf{R}_3(\theta_1)$
     &
     $\theta_1 =  \tan^{-1}(R_{31}/(-R_{32}))$ &
     $\theta_2 =  \cos^{-1}(R_{33})$ &
     $\theta_3  = \tan^{-1}\left(\displaystyle\frac{-R_{22}\sin{\theta_1} - R_{21}\cos{\theta_1}}{R_{12}\sin{\theta_1}+
     R_{11}\cos{\theta_1}}\right)$ \vspace{.15 in}\\
     %
     $\mathbf{R}_3(\theta_3)\mathbf{R}_2(\theta_2)\mathbf{R}_3(\theta_1)$
     &
     $\theta_1 =  \tan^{-1}(R_{32}/(R_{31}))$ &
     $\theta_2 =  \cos^{-1}(R_{33})$ &
     $\theta_3  = \tan^{-1}\left(\displaystyle\frac{-R_{11}\sin{\theta_1} + R_{12}\cos{\theta_1}}{-R_{21}\sin{\theta_1}+
     R_{22}\cos{\theta_1}} \right)$ \vspace{.15 in}\\
         \hline \hline
        \end{tabular}
        \label{table:DCMtoEulerAngles}
\end{table}

\section{Spacecraft Model}

\subsection{Thruster Models}

GMAT supports several thruster models.  The thruster models employ
physics and empirical data provided by the thruster manufacturer to
model thrust and mass flow rate used in orbit and attitude equations
of motion.   The thrust magnitude and $I_{sp}$ are assumed to be
functions of thruster inlet flow conditions including  pressure,
temperature, and for bi-propellant thrusters, the oxidizer to fuel
ratio.

In the following subsections we present models for thrust magnitude
and mass flow rates for several thruster types.  All thrusters have
a location and orientation.  The location is described in the
spacecraft body system.  The orientation can be described with
respect to any coordinate system known to GMAT. Let's define the
rotation matrix from the thruster frame $\mathcal{F}_T$ to Earth's
MJ2000 Equator as $\mathbf{R}_T$. Then, the thrust used in the orbit
equations of motion is
%
\begin{equation}
    \mathbf{F}_T = F_T \mathbf{R}_T \hat{\mathbf{T}}
\end{equation}
%
where $F_T$ is the thrust magnitude and is thruster dependent, and
$\hat{\mathbf{T}}$

Now let's look at how to calculate the thrust magnitude for a
mono-propellant chemical thruster.

\subsubsection{Mono-Propellant Chemical Thruster}

temperature.  The specific form of Eqs.~(\ref{Eq:MonoPropThrust})
and (\ref{Eq:MonoPropIsp}) are determined by fitting test data to
approximate thrust and $I_{sp}$ as function $T_i$ and $P_i$.  The
user can supply this relationship via a scripte
%
\begin{figure}[h!]
\centerline{
\begin{picture}(100,480)
\special{psfile= ./Images/MonoPropThruster.eps hoffset= -20 voffset= 250
hscale=25 vscale=25}
\makebox(100,540){$\dot{m}_e$,$P_e$,$v_e$}
%
\makebox(-105,710){$\dot{m}_c$,$T_c$,$P_c$}
%
\makebox(-105,920){$m_f$,$T_f$,$P_f$}
%
\makebox(-255,880){Catalyst Bed}
\end{picture}}\vskip -3.75 in  \caption{ Mono-Prop Thruster Diagram} \label{fig:MonoPropThruster}
\end{figure}

We assume thruster data is given as a function of thruster inlet
properties (as opposed to thrust chamber properties), and thrust
magnitude and $I_{sp}$ are modelled using
%
\begin{equation}
    F_T = f(P_i,T_i)\label{Eq:MonoPropThrust}
\end{equation}
%
\begin{equation}
    I_{sp} = f(P_i,T_i)\label{Eq:MonoPropIsp}
\end{equation}
%
where $P_i$ and $T_i$ are the thruster inlet pressure and
temperature.  The specific form of Eqs.~(\ref{Eq:MonoPropThrust})
and (\ref{Eq:MonoPropIsp}) are determined by fitting test data to
approximate thrust and $I_{sp}$ as function $T_i$ and $P_i$.  The
user can supply this relationship via a scripted equation or by
providing a function name.  After calculating $F_T$ and $I_{sp}$, we
calculate the mass flow rate using
%
\begin{equation}
   \dot{m}_e = \frac{F_T}{I_{sp}}
\end{equation}

\subsubsection{Bi-Propellant Chemical Thruster}

\begin{figure}[h!]
\centerline{
    \begin{picture}(100,470)
    \special{psfile= ./Images/BiPropThruster.eps hoffset= -20 voffset= 260
    hscale=25 vscale=25}
    \makebox(80,545){$\dot{m}_e$,}\makebox(-47,545){$P_e$,}
    \makebox(-30,545){$v_e$}
    \makebox(-70,730){$\dot{m}_c$,$m_r$,$T_c$,$P_c$}
    \makebox(-30,938){$m_f$,$T_f$,$P_f$}
    \makebox(-140,938){$m_o$,$T_o$,$P_o$}
    \end{picture}}\vskip -3.75 in  \caption{ Bi-Prop Thruster Diagram} \label{fig:BiPropThruster}
\end{figure}

\begin{equation}
    m_r = \frac{\dot{m}_o}{\dot{m}_f}
\end{equation}
%
\begin{equation}
    \dot{m}_c = \dot{m}_o + \dot{m}_f
\end{equation}
%
\begin{equation}
    T'=  \frac{\dot{m}_o T_o + \dot{m}_f T_f}{\dot{m}_o + \dot{m}_f}
\end{equation}
%
\begin{equation}
    P' =  \frac{\dot{m}_o P_o + \dot{m}_f P_f}{\dot{m}_o + \dot{m}_f}
\end{equation}
%
\begin{equation}
   F = f(P_c,T_c,of)
\end{equation}
%
\begin{equation}
   I_{sp} = f(P_c,T_c,of)
\end{equation}
%
\begin{equation}
   \dot{m}_e = \frac{\mathbf{F}_T}{I_{sp}}
\end{equation}

\subsubsection{Thruster Pulse Modelling}

\begin{equation}
T'=\frac{T(t)}{T_{max}} = \left\{\begin{array}{ll}
             \displaystyle\frac{t^2(t - t_{si})^2}{t_{si}^4} &
             \mbox{$t \leq t_{si}$}\\
             1 & \mbox{$t_{si}<t<t_{sf}$}\\
             \displaystyle\frac{(t - 2 t_{sf} + t_{f})^2(t - t_f)^2}{(t_f - t_{sf})^4}
             & \mbox{$t_{sf} \leq t \leq t_f$}\\
             \end{array}\label{eq:quartic}
      \right.
\end{equation}
%
\begin{figure}[ht]
\centerline{
    \begin{picture}(100,385)
    \special{psfile= ./Images/ThrustPulseProfile.eps hoffset= -150 voffset= 60
    hscale=65 vscale=65}
    \end{picture}}\vskip -3.75 in  \caption{ Sample Thrust Pulse Profile } \label{fig:ThrustPulseProfile}
\end{figure}
%
%
The time to the thrust centroid, $t_c$, is calculated using
%
\begin{equation}
     t_c = \displaystyle\frac{\displaystyle\int_0^{t_{f}} t \mbox{ }T'(t) dt}{\displaystyle\int_0^{t_{f}} T'(t)dt}
\end{equation}
%
performing the integral yields
%
\begin{equation}
    t_c = \frac{-4 t_{si}^2  + 4 t_{sf}^2  + 6 t_{sf} t_f + 5 t_f^2 }{-14 t_{si} + 14 t_{sf} + 16 t_f}
\end{equation}

\subsubsection{Thruster Hot Fire Test Data \\ and Thruster Models}

Thruster hot fire test data is used to develop empirical models that
describe thruster performance as a function of inlet conditions such
as fuel pressure and temperature.  In this section we'll discuss how
the empirical models are developed and discuss how the empirical
models are consistent with the physical models.  First we present
the physics model  for a thruster test stand experiment.  Next we
present what is measured during a thrust stand test, and show how
the measurements are used in combination with the physics model to
generate a model of thruster performance over  a given range of
thruster inlet conditions.

In Fig.~\ref{fig:ThrustStand} we see an illustration of a simple
thrust test setup.  The thruster is mounted to a rigid surface.
%
\begin{figure}[h!]
\centerline{
\begin{picture}(100,400)
\special{psfile= ./Images/ThrusterOnStand.eps hoffset= -35 voffset= 220
hscale=25 vscale=25}
\makebox(-120,700){$\dot{m}$}\makebox(-120,675){$T_i$}
\makebox(-125,650){$P_i$}\makebox(225,705){$\dot{m}$}\makebox(-190,680){$c^*=
I_{sp}g$}\makebox(-225,655){$P_e$}\makebox(-225,530){$F_T$}
\end{picture}}\vskip -3.5 in  \caption{ Thrust Stand Illustration} \label{fig:ThrustStand}
\end{figure}
%
The force due to thrust, $F_T$, can be written as
%
\begin{equation}
    F_T = \dot{m}v_e - (P_e - P_a)A_e
\end{equation}
%
where
\begin{tabbing}
    12345678 \= Reynolds number based on length $s$ \kill
    $\dot{m}$    \>  mass flow rate, kg/s \\
    $T_i$        \>  fuel inlet temperature, K$^{\circ}$\\
    $P_i$        \>  fuel inlet pressure, Pa\\
    $c^*$        \>  characteristic velcocity, m/s \\
    $I_sp$       \>  specific impulse, s \\
    $g_o$        \>  9.801 m/s$^2$ \\
    $P_e$        \>  nozzle exit pressure, Pa\\
    $A_e$        \>  nozzle exit area, m$^2$\\
    $F_T$        \>  force due to thrust, N\\
\end{tabbing}
%
we can rewrite this as
\begin{equation}
    F_T = \dot{m}\left(v_e - \frac{(P_e - P_a)A_e}{\dot{m}}\right)  \label{Eq:Fvsmv}
\end{equation}
%
From this equation we define the characteristic velocity, $c^*$,
using
%
\begin{equation}
    c^* = \left(v_e - \frac{(P_e - P_a)A_e}{\dot{m}}\right)
\end{equation}
%
In practice, $v_e$ or $P_e$ are not measured.  We'll assume the
tests are performed in a vacuum so $P_a = 0$.  To understand how we
relate the measurements to physical model, let's define a new
quantity $I_{sp}$, where
%
\begin{equation}
    I_{sp} = \frac{F_T}{\dot{m} g_o}
\end{equation}
%
we can rewrite this as
%
\begin{equation}
    F_T = \dot{m} I_{sp} g_o \label{Eq:IspVer2}
\end{equation}
%
Comparing Eq.s~(\ref{Eq:Fvsmv}) and (\ref{Eq:IspVer2}) we see that
%
\begin{equation}
     c^* = I_{sp} g_o = v_e - \frac{(P_e - P_a)A_e}{\dot{m}}
     \label{Eq:CstarIsp}
\end{equation}
%
Equation (\ref{Eq:CstarIsp}) shows that $I_{sp}$ is a measure of the
effective (characteristic) exhaust velocity.  $I_{sp}$ contains
information on the energy stored in the fuel and how that energy
translates to exit velocity.   When $I_{sp}$ is calculated from
measured thrust data, the $I_{sp}$ contains a correction for exhaust
velocity for force due to pressure $(P_e - P_a)A_e$.

The characteristic velocity, and hence $I_{sp}$, depend on the type
of fuel and the inlet temperature and pressure of the fuel.
Experimental data determines how
%
\begin{equation}
    I_{sp}(T_i,P_i) = \left.\frac{F_T}{\dot{m}g_o}\right|_{(T_i,P_i)}
\end{equation}
%
\begin{tabbing}
    12345678     \= Reynolds number based on length $s$ \kill
    $\dot{m}$    \>  Known      \\
    $T_i$        \>  Known      \\
    $P_i$        \>  Known      \\
    $g_o$        \>  Known      \\
    $F_T$        \>  Measured   \\
    $I_sp$       \>  Calculated \\
\end{tabbing}



\subsection{Tank Models}

Accurately modelling tank pressure changes is essential for accurate
maneuver modelling and reconstruction.  The following sections
discuss three types of tank models:  pressurant tank, regulated fuel
tank, and blowdown fuel tank.  For each tank there are three models:
isothermal, heat transfer, and adiabatic.

The models used in GMAT are based on work by Estey\cite{Estey:83} ,
Hearn\cite{Hearn:01} and Moran\cite{Moran}.  For each tank, we
select a set of state variables that when defined, allow us to
determine all remaining properties of the tank.  For the state
variables, we provide differential equations that describe how the
state variables change with respect to time.  The number of state
variables varies between different tanks, and with the model type
such as isothermal and heat transfer.

For each of the three tanks, we develop a heat transfer model, an
adiabatic model, and an isothermal model.  The heat transfer model
is derived using the laws of conservation of energy and the
conservation of mass. An adiabatic model is provided by setting the
heat transfer rates to zero in the heat transfer model. The
isothermal model for each tank is developed separately. Each ot
these models is useful for certain classes of maneuvers.  Isothermal
models are useful for maneuvers with low mass flow rates, adiabatic
models are useful for short maneuvers with high mass flow rates.
Heat transfer models are useful for long maneuvers with large mass
flow rates.

When developing heat transfer models, we'll assume that specific
internal energy is given by
%
\begin{equation}
    u = c T
\end{equation}
%
specific enthalpy for a liquid is given by
%
\begin{equation}
    h_\ell = c_\ell T_\ell
\end{equation}
%
and specific enthalpy  for a gas is given by
%
\begin{equation}
    h_g = T_g( c_g + R_g)
\end{equation}
%
The notation used in tank model development is shown below.  After
presenting notation, we present the dynamics model for a pressurant
tank.

 \noindent\textit{Nomenclature}\vspace{-.1 in}
\begin{tabbing}
    -----------------------     \= Reynolds number based on length $s$ \kill
    $A_g,A_\ell,A_w$    \> = Heat transfer area     \\
    $c_v, c_g$        \>  = Specific heat at constant volume     \\
    $D$        \>  = Tank diameter     \\
    $d$        \>  = Liquid surface diameter    \\
    $Gr$        \>  = Grashof number  \\
    $h_\ell, h_v$        \>  = Enthalpy     \\
    $m_g,m_\ell,m_w,m_v$    \> = Mass   \\
    $P_g,P_v,P_t$    \> = Pressure   \\
    $R_v,R_g$    \> = Gas constant   \\
    $T_g,T_\ell,T_w,T_v,T_a$    \> = Temperature   \\
    $u_g,u_\ell,u_w,u_v$    \> = Specific internal energy   \\
    $V_g,V_\ell,V_t$    \> = Volume   \\
    $\dot{W}$    \> = Work rate   \\
    $\dot{Q}_g,\dot{Q}_v,\dot{Q}_l,\dot{Q}_w$    \> = Heat transfer rate     \\
    $\nu_l,\nu_g,\nu_v$    \> = Specific volume  \\
\end{tabbing}
%
\noindent\textit{Subscripts}\vspace{-.1 in}
\begin{tabbing}
    -----------------------     \= Reynolds number based on length $s$ \kill
    $a$    \> = Ambient    \\
    $g$    \> = Pressurant gas    \\
    $\ell$    \> = Propellant liquid  \\
    $t$    \> = Total    \\
    $v$    \> = Propellant vapor \\
    $w$    \> = Tank wall      \\
    $e$    \> = Exit-flow      \\
    $i$    \> = In-flow
\end{tabbing}


\subsubsection{Pressurant Tank}

The pressurant tank model is the simplest of the tank models
primarily due to the fact that there is only one substance, the
pressurant gas, contained in the tank.  In this section, we develop
a state space model for pressurant tank dynamics.  We choose the
state variables to be pressurant gas mass and temperature, $m_g$ and
$T_g$ respectively, and tank wall temperature $T_w$.


In Fig.\ref{fig:PressurantTank} we see an illustration of a
pressurant tank.  We divide the tank into two control volumes: the
gas region and the tank wall region.  The only mass flow in the
system occurs where pressurant gas exits the tank.  Heat transfer
occurs between the gas and the wall, and the wall and the ambient
surroundings.
%
\begin{figure}[ht]
\centerline{
    \begin{picture}(110,440)
    \special{psfile= ./Images/PressurantTank.eps hoffset= -120 voffset= 115
    hscale=55 vscale=55}
    \makebox(90,525){$\dot{m}_e$,$h_g$}
    \makebox(-105,790){1.Gas}
    \makebox(-105,765){$m_g$, $P_g$, $T_g$}
    \makebox(-165,680){$\dot{Q}_g$}
    \makebox(-295,760){$\dot{Q}_w$}
    \makebox(-260,865){2.Tank Wall}
    \makebox(-270,840){$m_w$,  $T_w$}
    \end{picture}}\vskip -3.65 in  \caption{ Pressurant Tank Diagram} \label{fig:PressurantTank}
\end{figure}
%

Knowing the volume of the tank and the state variables  $m_g$,
$T_g$, and $T_w$, we calculate pressure from one of the following
two equations of state:
%
\begin{equation}
   P_g = \frac{m_g R_g T_g}{V_g}
\end{equation}
%
or from the Beattie-Bridgeman Eq.
%
\begin{equation}
   P_g = \frac{R_g T_g}{V_g} + \frac{a_g}{V_g^2} + \frac{b_g}{V_g^3}
\end{equation}
%

The state variables $m_g$, $T_g$, and $T_w$ are described by
ordinary differential equations found by applying the first law of
thermodynamics and the conservation of mass.  The 1st Law applied to
the gas control volume yields
%
\begin{equation}
   \frac{d}{dt}\left(m_g u_g \right) =  \dot{Q}_g - \dot{m}_e h_g
  \label{Eq:PressurantGas1stLaw}
\end{equation}
%
The 1st Law applied to the wall control volume  yields
%
\begin{equation}
     \frac{d}{dt}\left( m_w u_w \right) = \dot{Q}_w -  \dot{Q}_g \label{Eq:PressurantWall1stLaw}
\end{equation}
%
and finally from conservation of mass we obtain
%
\begin{equation}
    \dot{m}_g = -\dot{m}_e \label{Eq:PressurantMassCon}
\end{equation}
%
For these equations to be useful for numerical integration, we need
to expand the derivatives, and if necessary, decouple the equations
(as we'll see, for the pressurant tank, the equations are not
coupled).

Expanding the terms in Eq.~(\ref{Eq:PressurantGas1stLaw}) we have
%
\begin{equation}
    \dot{m}_g c_g T_g + m_g c_g \dot{T}_g = \dot{Q}_g - \dot{m}_e T_g \left( c_g + R_g\right)
\end{equation}
%
Similarly, expanding Eq.~(\ref{Eq:PressurantWall1stLaw}) we obtain
%
\begin{equation}
    m_w c_w \dot{T}_w = \dot{Q}_w -  \dot{Q}_g
\end{equation}
%
Solving the system of equations yields the following differential
equations of state for the pressurant tank heat transfer model.
%
\begin{eqnarray}
    \dot{m}_g &=& -\dot{m}_e\\
    %
    \dot{T}_g &=& \frac{1}{m_g c_g} \left( \dot{Q}_g - T_g R_g  \dot{m}_e  \right)\\
    %
    \dot{T}_w &=& \frac{1}{m_w c_w} \left( \dot{Q}_w - \dot{Q}_g  \right)
\end{eqnarray}
%
The adiabatic model is obtained by setting the terms $\dot{Q}_g$ and
$\dot{Q}_w$ to zero in the above equations.  (Note for the adiabatic
model there are only two state variables, $m_g$ and $T_g$, as the
wall temperature $T_w$ is removed from the system of equations.)
Similarly, the isothermal model is obtained by setting $\dot{T}_g$
and $\dot{T}_w$ to zero.  So, for the isothermal model there is only
one state variable $m_g$.

In summary, for the pressurant tank, all models calculate the tank
pressure using
%
\[
   P_g = \frac{m_g R_g T_g}{V_g}
\]
%
 then the specific equations for the heat transfer, adiabatic, and
isothermal models, are as follows

\noindent\textit{Pressurant Tank:  Heat Transfer}

\noindent State Variables:  $m_g$, $T_g$, $T_w$
%
\begin{eqnarray}
    \dot{m}_g &=& -\dot{m}_e \nonumber\\
    %
    \dot{T}_g &=& \frac{1}{m_g c_g} \left( \dot{Q}_g - T_g R_g  \dot{m}_e  \right)\nonumber\\
    %
    \dot{T}_w &=& \frac{1}{m_w c_w} \left( \dot{Q}_w - \dot{Q}_w  \right)\nonumber
\end{eqnarray}
%
\textit{Pressurant Tank:  Adiabtic}

\noindent State Variables:  $m_g$, $T_g$
%
\begin{eqnarray}
    \dot{m}_g &=& -\dot{m}_e \nonumber\\
    %
    \dot{T}_g &=& \dot{m}_e\frac{T_g R_g   }{m_g c_g} \nonumber
\end{eqnarray}
%
\noindent\textit{Pressurant Tank:  Isothermal}

\noindent State Variables:  $m_g$
%
\begin{equation}
    \dot{m}_g = -\dot{m}_e
\end{equation}

Now let's look at a model for a fuel tank operating in blow down
mode.

%\subsubsection{Blow-Down Tank w/o Vapor Pressure}
%
%
%\begin{figure}[ht]
%\centerline{
%    \begin{picture}(100,510)
%    \special{psfile= ./Images/BlowDownTank.eps hoffset= -120 voffset= 170
%    hscale=55 vscale=55}
%    \makebox(80,525){$\dot{m}_e$,$h_l$}
%    \makebox(-80,970){1.Gas}
%    \makebox(-90,940){$m_g, P_g, T_g$}
%    \makebox(-190,905){$\dot{Q}_v$}
%    %\makebox(-140,905){$\dot{m}_v, h_v$}
%    \makebox(10,905){$\dot{Q}_g$}
%    \makebox(-135,820){2.Liquid}
%    \makebox(-145,790){$m_\ell, T_\ell$}
%    \makebox(-235,740){$\dot{Q}_\ell$}
%    \makebox(-335,990){3.Tank Wall}
%    \makebox(-335,970){$m_w, T_w$}
%    \makebox(-277,1020){$\dot{Q}_w$}
%    \end{picture}}\vskip -3.65 in  \caption{ Bi-Prop Thruster Diagram} \label{fig:BlowDownTankWOVap}
%\end{figure}
%
%\textit{Assumptions}
%\begin{itemize}
%    \item Vapor pressure is zero.
%    \item Liquid density is constant.
%    \item Gas mass is constant.
%\end{itemize}
%%
%Assume we are given $m_g$, the tank diameter $D$, and hence know the
%total tank volume $V_t$, and we know the physical constants
%associated with the liquid and gas ($R_g$,$c_g$,$\nu_g$,$c_\ell$,
%$\nu_\ell$). We choose the state variables $m_\ell$, $T_\ell$,
%$T_g$, and $T_w$, all other tank properties can be calculated from
%these state variables using the following equations:
%%
%\begin{eqnarray}
%    V_\ell & = &  \nu_\ell m_\ell\\
%   %
%    V_g & = &  V_t - V_\ell\\
%   %
%   P_g & = &  \frac{m_g R_g T_g}{V_g}\\
%\end{eqnarray}
%%
%
%We require differential equations that describe the time rate of
%change of the state variables $m_\ell$, $T_\ell$, $T_g$, and $T_w$.
%The differential equations are found by applying the 1st law of
%thermodynamics and conservation of mass to the three control volumes
%illustrated in Fig. \ref{fig:BlowDownTankWOVap}. The 1st Law applied
%to the gas control volume yields
%%
%\begin{equation}
%   \frac{d}{dt}\left(m_g u_g \right) = \dot{Q}_v + \dot{Q}_g - P_g \dot{V}_g
%  \label{Eq:Blowdown1stLawWOVap}
%\end{equation}
%%
%The 1st Law applied to the liquid control volume yields
%%
%\begin{equation}
%   \frac{d}{dt}\left( m_\ell u_\ell \right) = \dot{Q}_\ell - \dot{Q}_v +
%   P_g \dot{V}_g   - \dot{m}_e h_\ell \label{Eq:BlowdownLiquid1stLawWOVap}
%\end{equation}
%%
%The 1st Law applied to the wall control volume  yields
%%
%\begin{equation}
%     \frac{d}{dt}\left( m_w u_w \right) = \dot{Q}_w - \dot{Q}_\ell - \dot{Q}_g
%\end{equation}
%%
%and finally from conservation of mass we obtain
%%
%\begin{equation}
%    \dot{m}_\ell = -\dot{m}_e \label{Eq:BlowdownMassConWOVap}
%\end{equation}
%
%
%The equations above give us four ordinary differential equations
%that allow us to solve for the tank states as a function of time.
%For numerical integration, we need to decouple these equations.
%
%Let's continue with Eq.~(\ref{Eq:Blowdown1stLawWOVap}).   Taking the
%derivative assuming $\dot{m}_g = 0$ and noting that $\dot{V}_g = -
%\nu_\ell \dot{m}_\ell $ yields
%%
%\begin{equation}
%     m_g c_g \dot{T}_g = \dot{Q}_v + \dot{Q}_g + P_g\nu_\ell \dot{m}_\ell
%\end{equation}
%%
%Gathering all state terms on the left hand side yields
%%
%\begin{equation}
%       -P_g\nu_\ell \dot{m}_\ell  + m_g c_g \dot{T}_g = \dot{Q}_v + \dot{Q}_g \label{Eq:CV1}
%\end{equation}
%
%
%Continuing with Eq.~(\ref{Eq:BlowdownLiquid1stLawWOVap}), we take
%the derivative and group terms to obtain
%%
%\begin{equation}
%   \left( c_\ell T_\ell  + P_g \nu_\ell\right)\dot{m}_\ell +
%    m_\ell c_\ell \dot{T}_\ell  =  \dot{Q}_\ell - \dot{Q}_v -  c_\ell T_\ell \dot{m}_e\label{Eq:CV2}
%\end{equation}
%%
%Similarly for the wall region, we arrive at
%%
%\begin{equation}
%     m_w c_w \dot{T}_w = \dot{Q}_w - \dot{Q}_\ell - \dot{Q}_g \label{Eq:CV3}
%\end{equation}
%
%Equations \ref{Eq:CV1} -\ref{Eq:CV3} and
%Eq.\ref{Eq:BlowdownMassConWOVap} can be written in matrix form as
%follows.
%%
%\begin{equation}
%   \left(\begin{array}{ccccccc}
%   A_{11} & 0 & A_{13} & 0 \\
%    A_{21} & A_{22} & 0  & 0 \\
%    0 & 0 & 0  & A_{34} \\
%    A_{41} &  & 0  & 0 \\
%   \end{array}\right)
%   %
%   \left(\begin{array}{c}
%    \dot{m}_\ell \\
%    \dot{T}_\ell  \\
%    \dot{T}_g \\
%   \dot{T}_w  \\
%   \end{array}\right) =
%   %
%   \left(\begin{array}{c}
%    b_1\\
%    b_2  \\
%    b_3 \\
%    b_4  \\
%   \end{array}\right)
%\end{equation}
%%
%where
%%
%\begin{eqnarray}
%   A_{11}& = &  - P_g \nu_\ell \\
%   %
%   A_{13}& = &  m_g c_g\\
%   %
%   A_{21} & = & c_\ell T_\ell + P_g \nu_\ell \\
%   %
%   A_{22} & = & m_\ell c_\ell\\
%   %
%   A_{34} & = & m_w c_w\\
%   %
%   A_{41} & = & 1\\
%   %
%   b_1 & = & \dot{Q}_v + \dot{Q}_g \\
%   %
%   b_2 & = & \dot{Q}_\ell - \dot{Q}_v -  c_\ell T_\ell\dot{m}_e\\
%   %
%   b_3 & = & \dot{Q}_w -\dot{Q}_\ell - \dot{Q}_g\\
%   %
%   b_4 & = & -\dot{m}_e\\
%\end{eqnarray}
%%
%
%Solving the system of equations yields
%%
%\begin{eqnarray}
%    \dot{m}_\ell &=& -\dot{m}_e\\
%    %
%    \dot{T}_\ell &=& \frac{1}{m_\ell c_\ell}\left( \dot{Q}_\ell - \dot{Q}_v  + P_g \nu_\ell \dot{m}_e\right)\\
%    %
%    \dot{T}_g &=&  \frac{1}{m_g c_g}\left( \dot{Q}_v + \dot{Q}_g - P_g \nu_\ell \dot{m}_e\right)\\
%    %
%    \dot{T}_w &=&  \frac{1}{m_w c_w}\left(\dot{Q}_w - \dot{Q}_\ell - \dot{Q}_g \right)\\
%\end{eqnarray}

\subsubsection{Blowdown Tank}

The blowdown tank model is significantly more complex than the
pressurant tank model due to the presence of liquid fuel and fuel
vapor contained in the tank ullage.   In this section, we develop a
state space model for a blow down tank.  We choose the state
variables to be the liquid mass and temperature, $m_\ell$ and
$T_\ell$, the gas temperature $T_g$, and tank wall temperature
$T_w$.


In Fig.\ref{fig:BlowDownTank} we see an illustration of a blow down
tank. We divide the tank into three control volumes: the gas region,
the liquid region, and the tank wall region. Mass flow occurs where
the pressurant gas exits the tank and at the boundary between the
liquid and gas in the form of evaporation. Heat transfer occurs
between all three control volumes as well as with the surroundings.
In summary, the physical processes modelled for a blow down tank are
%
\begin{compactenum}
    \item Vapor pressure is a function of liquid temperature.
    \item Liquid density is a function of liquid temperature.
    \item Heat transfer between the liquid and gas.
    \item Heat transfer between the tank wall and gas.
    \item Heat transfer between the tank wall and liquid.
    \item Heat transfer between the surroundings and tank.
    wall.
\end{compactenum}
%
The assumptions made in the tank model are
%
\begin{compactenum}
    \item Pressurant does not dissolve in liquid ($m_g = C$).
    \item Vapor and gas temperatures are equal.
    \item Vapor and gas volumes are equal.
\end{compactenum}
%
\begin{figure}[ht]
\centerline{
    \begin{picture}(100,510)
    \special{psfile= ./Images/BlowDownTankWVap.eps hoffset= -120 voffset= 170
    hscale=55 vscale=55}
    \makebox(80,525){$\dot{m}_e$,$h_l$}
    \makebox(-80,970){1.Gas/Vapor}
    \makebox(-90,940){$m_g, m_v, P_g, P_v, T_g$}
    \makebox(-190,905){$\dot{Q}_v$}
    \makebox(-140,905){$\dot{m}_v, h_{v}$}
    \makebox(10,905){$\dot{Q}_g$}
    \makebox(-135,820){2.Liquid}
    \makebox(-145,790){$m_\ell, T_\ell$}
    \makebox(-235,740){$\dot{Q}_\ell$}
    \makebox(-335,990){3.Tank Wall}
    \makebox(-335,970){$m_w, T_w$}
    \makebox(-265,1030){$\dot{Q}_w$}
    \end{picture}}\vskip -3.65 in  \caption{ Blow Down Tank Diagram} \label{fig:BlowDownTank}
\end{figure}

Assume we are given $m_g$, the tank diameter $D$, and hence know the
total tank volume $V_t$, and we know the physical constants
associated with the liquid and gas ($R_g$, $c_g$, $\nu_g$, $c_\ell$,
$\nu_\ell(T_\ell)$ and $P_v(T_\ell))$. We choose the state variables
$m_\ell$, $T_\ell$, $T_g$, and $T_w$, all other tank properties can
be calculated from these state variables using the following
equations:
%
\begin{eqnarray}
    V_\ell & = &  \nu_\ell(T_\ell)m_\ell \label{Eq:BlowDownVell}\\
   %
    V_g & = &  V_t - V_\ell\\
   %
   P_g & = &  \frac{m_g R_g T_g}{V_g}\\
   %
   P_v & = &  P_v(T_\ell)\\
   %
   m_v & = & \frac{P_v V_g}{R_v T_g}\\
   %
   P_t & = &  P_v + P_g \label{Eq:BlowDownPt}
\end{eqnarray}

%
To determine the state equations governing $m_\ell$, $T_\ell$,
$T_g$, and $T_w$ we apply the 1st law of thermodynamics and the law
of conservation of mass. The 1st Law applied to the gas control
volume is
%
\begin{equation}
   \frac{d}{dt}\left( m_v u_v + m_g u_g \right) = \dot{Q}_v + \dot{Q}_g - P_t \dot{V}_g +
    \dot{m}_v h_{v} \label{Eq:BlowdownGas1stLaw}
\end{equation}
%
The 1st Law applied to the liquid control volume is
%
\begin{equation}
   \frac{d}{dt}\left( m_\ell u_\ell \right) = \dot{Q}_\ell - \dot{Q}_v +
   P_t \dot{V}_g - \dot{m}_v h_{lg} - \dot{m}_e h_\ell \label{Eq:BlowdownLiquid1stLaw}
\end{equation}
%
The 1st Law applied to the wall control volume yields
%
\begin{equation}
     \frac{d}{dt}\left( m_w u_w \right) = \dot{Q}_w - \dot{Q}_\ell - \dot{Q}_g \label{Eq:BlowdownWall1stLaw}
\end{equation}
%
and finally from conservation of mass:
%
\begin{equation}
    \dot{m}_\ell = -\dot{m}_e - \dot{m}_v \label{Eq:BlowdownMassCon}
\end{equation}
%
we also know that
%
\begin{equation}
    \dot{m}_v = \frac{P_v \dot{V}_g}{R_v T_g} - \frac{P_v V_g \dot{T}_g}{R_v T_g^2} \label{Eq:BlowDownGasLaw}
\end{equation}
%
where we assume that
%
\begin{equation}
   \dot{P}_v \approx 0
\end{equation}

Equations (\ref{Eq:BlowdownGas1stLaw}) - (\ref{Eq:BlowDownGasLaw})
are five equations in five unknowns ($m_v$, $m_\ell$, $T_\ell$,
$T_g$, and $T_w$). Our approach is to use
Eq.~(\ref{Eq:BlowdownMassCon}) to eliminate $\dot{m}_v$ terms.  The
result is a system of four equations in four unknowns using
Eqs.~(\ref{Eq:BlowdownGas1stLaw}), (\ref{Eq:BlowdownLiquid1stLaw}),
(\ref{Eq:BlowdownWall1stLaw}), and (\ref{Eq:BlowDownGasLaw}).  The
result we seek is four decoupled ordinary differential equations for
$m_\ell$, $T_\ell$, $T_g$, and $T_w$.


Let's continue with Eq.~(\ref{Eq:BlowdownGas1stLaw}).  We need to
rewrite the equation in terms of $\dot{m}_\ell$ and $\dot{T}_g$
($\dot{T_w}$ and $\dot{T}_\ell$ don't appear explicitly).  Expanding
the derivatives assuming $\dot{m}_g = 0$ yields
%
\begin{equation}
   \dot{m}_v c_v T_g + m_v c_v \dot{T}_g +  m_g c_g \dot{T}_g = \dot{Q}_v + \dot{Q}_g - P_t \dot{V}_g + \dot{m}_v h_{v}
\end{equation}
%
Now, substituting $\dot{m}_v = -\dot{m}_\ell - \dot{m}_e$ and noting
that $\dot{V}_g = -\nu_l \dot{m}_\ell$ if we assume
%
 \[\dot{\nu}_\ell = \frac{d \nu_\ell}{dT_\ell }\dot{T}_\ell
\approx 0
\]
%
we arrive at
%
\begin{equation}
   \begin{split}
      \left( T_g R_v - P_t \nu_\ell  \right) \dot{m}_\ell + \left( m_v c_v + m_g c_g\right)\dot{T}_g = \\
      \dot{Q}_v + \dot{Q}_g - \dot{m}_e T_g R_v   \label{Eq:BlowdownWVaporEq1}
   \end{split}
\end{equation}
%

Now continuing with Eq.~(\ref{Eq:BlowdownLiquid1stLaw}) expanding
the derivatives and making similar substitutions as we made
previously we obtain
%
\begin{equation}
\begin{split}
    \dot{m}_\ell c_\ell T_\ell + m_\ell c_\ell \dot{T}_\ell = \dot{Q}_\ell - \dot{Q}_v +
    P_t(-\nu_\ell \dot{m}_\ell) - \\ (-\dot{m}_\ell - \dot{m}_e)h_{v} - \dot{m}_e c_\ell T_\ell
\end{split}
\end{equation}
%
Grouping terms we obtain
%
\begin{equation}
\begin{split}
    ( c_\ell T_\ell  + P_t v_l - h_{v})\dot{m}_\ell + ( m_\ell c_\ell )\dot{T}_\ell = \\
    \dot{Q}_\ell - \dot{Q}_v + \dot{m}_e (h_v - c_\ell T_\ell) \label{Eq:BlowdownWVaporEq2}
\end{split}
\end{equation}

For the wall region, described by Eq.~(\ref{Eq:BlowdownWall1stLaw}),
we arrive at
\begin{equation}
     (m_w c_w) \dot{T}_w = \dot{Q}_w - \dot{Q}_\ell - \dot{Q}_g \label{Eq:BlowdownWVaporEq3}
\end{equation}

Finally, by eliminating $\dot{m}_v$ in the Gas Law shown in
Eq.~(\ref{Eq:BlowDownGasLaw}) we obtain
%
\begin{equation}
    -\dot{m}_\ell - \dot{m}_e =   \frac{P_v (-\nu_\ell \dot{m}_\ell)}{R_v T_g} - \frac{P_v V_g \dot{T}_g}{R_v T_g^2}
\end{equation}
%
Grouping terms yields the result
\begin{equation}
    \left(  1 - \frac{P_v\nu_\ell }{R_v T_g}  \right) \dot{m}_\ell - \frac{P_v V_g }{R_v T_g^2}\dot{T}_g  =  - \dot{m}_e \label{Eq:BlowdownWVaporEq4}
\end{equation}

Equations (\ref{Eq:BlowdownWVaporEq1}),
(\ref{Eq:BlowdownWVaporEq2}), (\ref{Eq:BlowdownWVaporEq3}), and
(\ref{Eq:BlowdownWVaporEq4}) are four coupled ordinary differential
equations that can be decoupled by casting them in matrix form as
follows:
%
\begin{equation}
   \left(\begin{array}{ccccccc}
   A_{11} & 0 & A_{13}  & 0 \\
    A_{21} & A_{22} & 0  & 0 \\
    0 & 0 & 0  & A_{34} \\
    A_{41} &  & A_{43}  & 0 \\
   \end{array}\right)
   %
   \left(\begin{array}{c}
    \dot{m}_\ell \\
    \dot{T}_\ell  \\
    \dot{T}_g \\
   \dot{T}_w  \\
   \end{array}\right) =
   %
   \left(\begin{array}{c}
    b_1\\
    b_2  \\
    b_3 \\
    b_4  \\
   \end{array}\right)
\end{equation}
%
where
%
\begin{eqnarray}
   A_{11}& = & T_g R_v - P_t \nu_\ell \label{Eq:BlowDownA11}\\
   %
   A_{13}& = & m_v c_v + m_g c_g\\
   %
   A_{21} & = & c_\ell T_\ell + P_t v_l - h_{v}\\
   %
   A_{22} & = & m_\ell c_\ell\\
   %
   A_{34} & = & m_w c_w\\
   %
   A_{41} & = & 1  - \nu_\ell/\nu_v\\
   %
   A_{43} & = & - m_v/T_g\\
   %
   b_1 & = & \dot{Q}_v + \dot{Q}_g - \dot{m}_e T_g R_v \label{Eq:BlowDownb1}\\
   %
   b_2 & = & \dot{Q}_\ell - \dot{Q}_v + \dot{m}_e (h_v - c_\ell T_\ell) \\
   %
   b_3 & = & \dot{Q}_w -\dot{Q}_\ell - \dot{Q}_g\\
   %
   b_4 & = & -\dot{m}_e \label{Eq:BlowDownb4}
\end{eqnarray}
%


The solution to the equations is
%
\begin{eqnarray}
  \dot{m}_\ell &=& \frac{A_{43} b_{1}-A_{13} b_{4}}{A_{11} A_{43}-A_{41} A_{13}} \label{Eq:BlowDownmdotDiffEq}\\
  %
  %\dot{T}_\ell &=& \frac{-A_{21}A_{43}b_1+ (A_{11}A_{43}-A_{41}A_{13})b_2+A_{21}A_{13}b_4}{A_{22}\left( A_{11}A_{43}-A_{41}A_{13}\right)}\\
  \dot{T}_\ell &=& \frac{1}{A_{22}}\left( b_2 -  A_{21} \frac{A_{43} b_{1}-A_{13} b_{4}}{A_{11} A_{43}-A_{41} A_{13}}\right)\\
  %
  \dot{T}_g &=& \frac{A_{11} b_{4} - A_{41} b_{1}}{A_{11} A_{43}-A_{41} A_{13}}\\
  %
  \dot{T}_w &=& \frac{b_3}{A_{34}} \label{Eq:BlowDownTwdotDiffEq}
\end{eqnarray}
%
For the adiabatic model we set all heat transfer rates, $\dot{Q}$,
to zero in Eqs.~(\ref{Eq:BlowDownb1})-(\ref{Eq:BlowDownb4}) and so
there are only two state variables as $\dot{T}_w = 0$ and so $T_w =$
constant.

Now let's develop equations for an isothermal model of a blow down
tank.  In the isothermal model, we assume $T_\ell = T_g = T_w = T$.
The only state variable that requires a differential equation is
$m_\ell$.  Because $T_g$, $T_\ell$, and hence, $P_v$ are constant,
we know that
%
\begin{equation}
    \dot{m}_v = \frac{P_v \dot{V}_g}{R_v T_g}
\end{equation}
%
Subtituting this result into Eq.(\ref{Eq:BlowdownMassCon}) and
solving for $\dot{m}_\ell$ we obtain.
%
\begin{equation}
    \dot{m}_\ell = -\frac{\dot{m}_e}{\left( 1 - \displaystyle\frac{P_v \nu_\ell}{R_v T}\right)}
\end{equation}


In summary for the heat transfer model for a blow down tank, we
choose $m_\ell$, $T_\ell$, $T_g$, and $T_w$ are state variables.
Eqs.~(\ref{Eq:BlowDownVell})-(\ref{Eq:BlowDownPt}) are used to
calculate the remaining tank properties, and
Eqs.~(\ref{Eq:BlowDownmdotDiffEq})-(\ref{Eq:BlowDownTwdotDiffEq})
are used to model the tank states as functions of time.

For the all three models, heat transfer, adiabatic, and isothermal,
knowing the state variables $m_\ell$, $T_\ell$, $T_g$, and $T_w$ we
compute the remaining tank properties using
%
\begin{eqnarray}
    V_\ell & = &  \nu_\ell(T_\ell)m_\ell  \nonumber \\
   %
    V_g & = &  V_t - V_\ell\nonumber \\
   %
   P_g & = &  \frac{m_g R_g T_g}{V_g}\nonumber \\
   %
   P_v & = &  P_v(T_\ell)\nonumber \\
   %
   m_v & = & \frac{P_v V_g}{R_v T_g}\nonumber \\
   %
   P_t & = &  P_v + P_g \nonumber
\end{eqnarray}

%
The models differ in the number of state variables and in the state
rate equations.  A summary is presented below.

\noindent\textit{Blow Down Tank: Heat Transfer}

\noindent State Variables:  $m_\ell$, $T_\ell$, $T_g$, $T_w$
%
\begin{eqnarray}
  \dot{m}_\ell &=& \frac{A_{43} b_{1}-A_{13} b_{4}}{A_{11} A_{43}-A_{41} A_{13}}\nonumber \\
  %
  %\dot{T}_\ell &=& \frac{-A_{21}A_{43}b_1+ (A_{11}A_{43}-A_{41}A_{13})b_2+A_{21}A_{13}b_4}{A_{22}\left( A_{11}A_{43}-A_{41}A_{13}\right)}\\
  \dot{T}_\ell &=& \frac{1}{A_{22}}\left( b_2 -  A_{21} \frac{A_{43} b_{1}-A_{13} b_{4}}{A_{11} A_{43}-A_{41}
  A_{13}}\right) \nonumber \\
  %
  \dot{T}_g &=& \frac{A_{11} b_{4} - A_{41} b_{1}}{A_{11} A_{43}-A_{41}
  A_{13}} \nonumber \\
  %
  \dot{T}_w &=& \frac{b_3}{A_{34}} \nonumber
\end{eqnarray}
%
where $A_{ij}$ and $b_i$ are given by
Eqs.~(\ref{Eq:BlowDownA11})-(\ref{Eq:BlowDownb4}).

\noindent\textit{Blow Down Tank:  Adiabtic}

\noindent State Variables:  $m_\ell$, $T_\ell$, $T_g$

%
\begin{eqnarray}
  \dot{m}_\ell &=& \frac{A_{43} b_{1}-A_{13} b_{4}}{A_{11} A_{43}-A_{41} A_{13}}\nonumber \\
  %
  %\dot{T}_\ell &=& \frac{-A_{21}A_{43}b_1+ (A_{11}A_{43}-A_{41}A_{13})b_2+A_{21}A_{13}b_4}{A_{22}\left( A_{11}A_{43}-A_{41}A_{13}\right)}\\
  \dot{T}_\ell &=& \frac{1}{A_{22}}\left( b_2 -  A_{21} \frac{A_{43} b_{1}-A_{13} b_{4}}{A_{11} A_{43}-A_{41}
  A_{13}}\right) \nonumber \\
  %
  \dot{T}_g &=& \frac{A_{11} b_{4} - A_{41} b_{1}}{A_{11} A_{43}-A_{41}
  A_{13}} \nonumber
\end{eqnarray}
%
where $A_{ij}$ and $b_i$ are given by
Eqs.~(\ref{Eq:BlowDownA11})-(\ref{Eq:BlowDownb4}).  Note that all
heat flow rates, $\dot{Q}$, are set to zero.

\noindent\textit{Blow Down Tank:  Isothermal}

\noindent State Variables:  $m_\ell$
%
\begin{equation}
    \dot{m}_\ell = -\frac{\dot{m}_e}{\left( 1 - \displaystyle\frac{P_v \nu_\ell}{R_v
    T}\right)} \nonumber
\end{equation}

\subsubsection{Pressure Regulated Tank}

The pressure regulated fuel tank model is the most complex tank
model supported in GMAT.  The model complexity is due to the
presence of liquid fuel and fuel vapor contained in the tank ullage,
and due to mass and energy transfer from the pressurant tank to the
ullage of the regulated fuel tank. In this section, we develop a
state space model for a pressure regulated tank. Note, to model a
pressure regulated tank using a heat transfer or adiabitic model, we
must simultaneously solve the equations associated with the
pressurant tank.  For the regulated tank model, we choose the state
variables to be the liquid mass and temperature, $m_\ell$ and
$T_\ell$, the gas temperature $T_g$,  tank wall temperature $T_w$,
and the incoming pressurant gas mass $m_i$.


In Fig.\ref{fig:PressureRegulatedTank} we see an illustration of a
pressure regulated tank. Like the blow down tank model, we divide
the tank into three control volumes: the gas region, the liquid
region, and the tank wall region. Mass flow occurs where the
pressurant gas exits the tank, at the boundary between the liquid
and gas in the form of evaporation, and from the pressurant tank to
the ullage of the regulated tank.  Heat transfer occurs between all
three control volumes as well as with the surroundings. Hence, the
physical processes modelled for a blow down tank are the same as
those listed for the blow down tank, with the added process of mass
flow from the pressurant tank.
%
\begin{figure}[ht]
\centerline{
    \begin{picture}(100,510)
    \special{psfile= ./Images/PressureRegulatedTank.eps hoffset= -120 voffset= 170
    hscale=55 vscale=55}
    \makebox(80,525){$\dot{m}_e$,$h_l$}
    \makebox(-80,970){1.Gas/Vapor}
    \makebox(-90,940){$m_g, m_v, P_g, P_v, T_g$}
    \makebox(-190,905){$\dot{Q}_v$}
    \makebox(-140,905){$\dot{m}_v, h_{lg}$}
    \makebox(10,905){$\dot{Q}_g$}
    \makebox(-135,820){2.Liquid}
    \makebox(-145,790){$m_\ell, T_\ell$}
    \makebox(-235,740){$\dot{Q}_\ell$}
    \makebox(-335,990){3.Tank Wall}
    \makebox(-335,970){$m_w, T_w$}
    \makebox(-265,1030){$\dot{Q}_w$}
    \makebox(-045,1030){$\dot{m}_i$, $h_i$}
    \end{picture}}\vskip -3.65 in  \caption{ Pressure Regulated Tank Diagram} \label{fig:PressureRegulatedTank}
\end{figure}

The derivation of the state equations for a pressure regulated tank
follows naturally from the derivation of the blow down tank.  The
only control volume that differs between the two models is the
gas/vapor control volume.  Applying the 1st Law of thermodynamics to
the gas/vapor control volume of the pressure regulated tank gives us
%
\begin{equation}
   \frac{d}{dt}\left( m_v u_v + m_g u_g \right) = \dot{Q}_v + \dot{Q}_g - P_t \dot{V}_g +
    \dot{m}_v h_v + \dot{m}_p h_p\label{Eq:RegulatedGas1stLaw}
\end{equation}
%
Taking the time derivative of the gas law for the gas contained in
the tank ullage yields
%
\begin{equation}
   \dot{m}_g = \frac{P_g \dot{V}_g}{R_g
   T} - \frac{P_g V_g \dot{T}_g}{R_g T_g^2} \label{Eq:RegulatedGasLawDeriv}
\end{equation}
%

Equations (\ref{Eq:RegulatedGas1stLaw}) and
(\ref{Eq:RegulatedGasLawDeriv}), together with equations
(\ref{Eq:BlowdownWVaporEq2}), (\ref{Eq:BlowdownWVaporEq3}), and
(\ref{Eq:BlowdownWVaporEq4}) are a system of 5 equations in 5
unknowns which can be decoupled using simple linear algebra.
However, first we must expand Eqs.~(\ref{Eq:RegulatedGas1stLaw}) and
(\ref{Eq:RegulatedGasLawDeriv}) and write them in terms of the state
rate derivatives.  Expanding Eq.~(\ref{Eq:RegulatedGas1stLaw}) we
arrive at
\begin{equation}
   \begin{split}
   \left( R_v T_g - P_t \nu_\ell \right)\dot{m}_\ell + \left( m_v c_v + m_g c_g \right)
   \dot{T}_g + \\ \left(c_g Tg - h_p \right)\dot{m}_g = \dot{Q}_v +
   \dot{Q}_g - \dot{m}_e R_v T_g
   \end{split}
\end{equation}
%
Similarly, for Eq.~(\ref{Eq:RegulatedGasLawDeriv}) we obtain
%
\begin{equation}
    \frac{\nu_\ell}{\nu_g}\dot{m}_\ell + \frac{m_g}{T_g}\dot{T}_g + \dot{m}_g = 0
\end{equation}
%

To integrate the state equations we must decouple the equations and
this is easily done by casting the equations in matrix form and
solving the system of equations.  We can write the equations is
state space for as follows
%
\begin{equation}
   \left(\begin{array}{ccccccc}
   A_{11} & 0 & A_{13}  & 0 & A_{15}\\
    A_{21} & A_{22} & 0  & 0 & 0 \\
    0 & 0 & 0  & A_{34} & 0\\
    A_{41} & 0 & A_{43}  & 0 & 0\\
    A_{51} & 0 & A_{43}  & 0 &  A_{55}\\
   \end{array}\right)
   %
   \left(\begin{array}{c}
    \dot{m}_\ell \\
    \dot{T}_\ell  \\
    \dot{T}_g \\
   \dot{T}_w  \\
   \dot{m}_g
   \end{array}\right) =
   %
   \left(\begin{array}{c}
    b_1\\
    b_2  \\
    b_3 \\
    b_4  \\
    0
   \end{array}\right)
\end{equation}
%
where the coefficients $A_{ij}$ and $b_i$ are given by
%
\begin{eqnarray}
   A_{11}& = & T_g R_v - P_t \nu_\ell \label{Eq:RegulatedA11}\\
   %
   A_{13}& = & m_v c_v + m_g c_g\\
   %
   A_{15} & = & c_g T_g - h_p\\
   %
   A_{21} & = & c_\ell T_\ell + P_t v_l - h_{lg}\\
   %
   A_{22} & = & m_\ell c_\ell\\
   %
   A_{34} & = & m_w c_w\\
   %
   A_{41} & = & 1  - \nu_\ell/\nu_v\\
   %
   A_{43} & = & - m_v/T_g\\
   %
   A_{51} & = & \nu_\ell/\nu_g\\
   %
   A_{53} & = & m_g/T_g \\
   %
   A_{55} & = & 1 \\
   %
   b_1 & = & \dot{Q}_v + \dot{Q}_g - \dot{m}_e R_v T_g \label{Eq:Regulatedb1}\\
   %
   b_2 & = & \dot{Q}_\ell - \dot{Q}_v + \dot{m}_e (h_{lg} - c_\ell T_\ell)\\
   %
   b_3 & = & \dot{Q}_w -\dot{Q}_\ell - \dot{Q}_g\\
   %
   b_4 & = & -\dot{m}_e \label{Eq:Regulatedb4}
\end{eqnarray}

\begin{eqnarray}
   \dot{m}_\ell &=& \frac{A_{55} A_{43} b_1 - b_4 A_{13} A_{55} + b_4 A_{15} A_{53}}{D}\\
   \dot{T}_\ell &=& \frac{b_2 - A_{21}\dot{m}_\ell}{A_{22}}\\
   %
   \dot{T}_g  &=& \frac{-A_{41} A_{55} b_1 + b_4 A_{11} A_{55} - b_4 A_{51}
    A_{15}}{D}\\
   %
   \dot{T}_w &=& \frac{b_3}{A_{34}}\\
   %
   \dot{m}_g &=& \frac{-b_1 A_{51} A_{43} + b_1 A_{41} A_{53} - b_4 A_{11} A_{53} + b_4 A_{51} A_{13}}{D}
\end{eqnarray}
%
where
%
\begin{equation}
      D = A_{55} A_{43} A_{11} - A_{43} A_{51} A_{15} + A_{41} A_{15} A_{53} - A_{41} A_{13} A_{55}
\end{equation}

For the adiabatic model we set all heat transfer rates, $\dot{Q}$,
to zero in  Eqs.~(\ref{Eq:Regulatedb1})-(\ref{Eq:Regulatedb4}). For
the adiabatic model there is only four state variables as $\dot{T}_w
= 0$ and so $T_w =$ constant.

Now let's develop equations for an isothermal model of a pressure
regulated tank. In the isothermal model, we assume $T_\ell = T_g =
T_w = T$. The only state variables that require differential
equations are $m_\ell$ and $m_g$. Because $T_g =$ constant, and
hence, $P_v =$ constant, we know that
%
\begin{equation}
    \dot{m}_\ell = -\frac{\dot{m}_e}{\left( 1 - \displaystyle\frac{P_v \nu_\ell}{R_v
    T}\right)}
\end{equation}
%
Similarly, for $m_g$ we obtain
%
\begin{equation}
   \dot{m}_g = \frac{\nu_\ell}{\nu_g}\frac{\dot{m}_e}{\left( 1 - \displaystyle\frac{P_v \nu_\ell}{R_v
    T}\right)}
\end{equation}

\subsubsection{Heat Transfer}

Heat transfer models are from Ring\cite{Ring:64} and
Incropera\cite{Incropera:06} and Pitts\cite{Pitts:98}

\begin{equation}
    \dot{Q} = h A \Delta T
\end{equation}

\begin{equation}
    \frac{h L}{k} = Nu = c(Gr_L Pr)^a
\end{equation}
%
so
%
\begin{equation}
       h = \frac{k c}{L}(Gr_L Pr)^a
\end{equation}
\begin{table}[ht]
\centering \caption{ Dimensionless Heat Transfer Terms
\cite{Incropera:06}}
\begin{tabular}{p{.65 in} p{.950 in} p{1.4 in}}
  \hline\hline
  % after \\: \hline or \cline{col1-col2} \cline{col3-col4} ...
  Parameter & Definition & Interpretation \\
  \hline
  Grashof number ($Gr_L$) & \hspace{ 1 in} $\displaystyle\frac{g \beta(T_s - T_\infty)L^3}{\nu^2}$  & Measure of the ratio of buoyancy forces to viscous forces.\\
  %
  & \\
  %
  Prandtl number ($Pr$) & \hspace{ 1 in} $\displaystyle\frac{c \mu}{k} = \frac{\mu}{\alpha}$  & Ratio of the momentum and thermal diffusivities.\\
  %
  & \\
  %
  Nusselt number ($Nu_L$) & \hspace{ 1 in} $\displaystyle\frac{h L}{k_f}$ & Ratio of convection to pure conduction heat transfer.  \\
  %
  & \\
  %
  Reynolds number ($Re_L$) & \hspace{ 1 in} $\displaystyle\frac{V L}{\nu}$ & Ratio of inertia and viscous forces.  \\
  \hline
\end{tabular}
\end{table}

\subsubsection{ Physiochemical Properties }

Hydrazine Properties \cite{HydraBook}
\[
c = 3084 \mbox{J/kg/K}
\]
\[
\rho \mbox{ (kg/m}^3) = 1025.6 - 0.87415 \mbox{ }T \mbox{ }(^o\mbox{C})- 0.45283e^{-3}\mbox{ }T^2 \mbox{ }(^o\mbox{C})
\]
%
\[
\rho \mbox{ (kg/m}^3) = 1230.6 - 0.62677 \mbox{ }T \mbox{ }(^o\mbox{K})- 0.45283e^{-3}\mbox{ }T^2 \mbox{ }(^o\mbox{K})
\]


Some models are from \cite{Ricciardi:87}

\begin{eqnarray}
  \rho_\ell(T_\ell) & = &K_1 + K_2 T_\ell + K_3 T_\ell^2
  \mbox{  (kg/m$^3$)}\\
  %
  \frac{d \rho_\ell}{dT_\ell } &  = & K_2 + 2K_3 T_\ell
  \mbox{  (kg/m$^3$)}
\end{eqnarray}
%
\begin{eqnarray}
  P_v & =  &\displaystyle C_1 e^{(C_2 + C_3 T_\ell + C_4 T_\ell^2)} \mbox{
  (N/m$^2$)}\\
  %
  \frac{d P_v}{d T_\ell}& = &\displaystyle C_1 ln{(C_2 + C_3 T_\ell C_4 T_\ell^2)}\left(C_3 + 2C_4T_\ell\right)
\end{eqnarray}
%
\begin{equation}
    m_d = P_g m_\ell^\alpha\left( \frac{T_\ell}{294}\right)^2
\end{equation}
%
\begin{equation}\begin{split}
    \frac{d m_d}{dt} =  m_\ell^\alpha\left(
    \frac{T_\ell}{294}\right)^2\dot{P}_g + \alpha P_g m_\ell^{(\alpha - 1)}\left(
    \frac{T_\ell}{294}\right)^2 \dot{m}_\ell \\ + 2 P_g m_\ell^\alpha\left(
    \frac{T_\ell}{294}\right)\dot{T}_\ell
    \end{split}
\end{equation}
%

\begin{table} \centering
\caption{Constants for Density Equations}
\begin{tabular}{lcc}
  \hline
  % after \\: \hline or \cline{col1-col2} \cline{col3-col4} ...
  Const. & \mbox{N}$_2$\mbox{0}$_4$ & CH$_3$N$_2$H$_3$ \\
  \hline \hline
  $K_1$ (kg/m$^3$) & 2066 & 1105.3 \\
  $K_2$ (kg/m$^3$/K )& -1.979 & -0.9395 \\
  $K_3$ (kg/m$^3$/K$^2) $ & -4.826e-4 & 0 \\
  \hline
\end{tabular}
\end{table}

\begin{table} \centering
\caption{Constants for Vapor Pressure Equations}
\begin{tabular}{lcc}
  \hline
  % after \\: \hline or \cline{col1-col2} \cline{col3-col4} ...
  Const. & \mbox{N}$_2$\mbox{0}$_4$ & CH$_3$N$_2$H$_3$ \\
  \hline \hline
  $C_1$ (kg/m$^3$) & 6895 & 6895\\
  $C_2$ (kg/m$^3$/K )& 18.6742 & 12.4293 \\
  $C_3$ (kg/m$^3$/K$^2) $ & -5369.57 & 2543.37 \\
  $C_4$ (kg/m$^3$/K$^2) $ & 194721 & 0 \\
  \hline
\end{tabular}
\end{table}

\begin{table} \centering
\caption{Constants for Dissolved Pressurant Equations}
\begin{tabular}{lcc}
  \hline
  % after \\: \hline or \cline{col1-col2} \cline{col3-col4} ...
  Const. & \mbox{N}$_2$\mbox{0}$_4$ & CH$_3$N$_2$H$_3$ \\
  \hline \hline
  $\alpha$ & 3.335e-11 & 2.059e-11 \\
  \hline
\end{tabular}
\end{table}


\subsection{Mass Properties}

\subsubsection{Spherical Tank}
\clearpage
\begin{figure}[ht]
\centerline{
    \begin{picture}(110,440)
    \special{psfile= ./Images/PartiallyFilledTank.eps hoffset= -110 voffset=
    90
    hscale=55 vscale=55}
    \makebox(-47,665){$h$}
    \makebox(135,620){$r$}
    \makebox(0,695){$\hat{y}$}
    \makebox(-165,850){$\hat{z}$}
    \makebox(-165,515){($\hat{x}$ is out of the page)}
    \end{picture}}\vskip -3.65 in  \caption{ Geometry For Mass Properties of Partially Filled Spherical Tank} \label{fig:PartiallyFilledTank}
\end{figure}

\begin{equation}
    V = \frac{1}{3}\pi \left( 3r - h \right)h^2
\end{equation}
%
\begin{equation}
    cg_z = \displaystyle\frac{-\displaystyle\frac{3}{4}h^2 + 3hr - 3r^2}{3r - h}
\end{equation}
%
\begin{equation}
    cg_x = cg_y = 0
\end{equation}
%
\begin{equation}
   I_{zz} = \frac{\pi \rho}{2}\left(\frac{1}{5}\left( h - r\right)^5 - \frac{2}{3}r^2\left( h - r\right)^3  + r^4(h-r) + \frac{8}{15}r^5 \right)
\end{equation}
%
\begin{equation}
   I_{xx} = \frac{\pi \rho}{2}\left(-\frac{3}{10}\left( h - r\right)^5 + \frac{1}{3}r^2\left( h - r\right)^3  + \frac{1}{2}r^4(h-r) + \frac{8}{15}r^5 \right)
\end{equation}
%
\begin{equation}
   I_{yy} = I_{xx}
\end{equation}
%
\begin{equation}
    \mathbf{I}' = \mathbf{R}_{bt}^T
    \left(\begin{array}{ccc}
         I_{xx} & 0 & 0 \\
         0 & I_{yy} & 0\\
         0 & 0 & I_{zz}\\
    \end{array}\right)
     \mathbf{R}_{bt} \
\end{equation}
%
Estey \cite{Estey:83} gives appoximate equations for the area of
different portions for a partially filled sphere. The area of the
spherical shell in contact with the gaseous region is given by
%
\begin{equation}
   A_g = 4.0675 \cdot V^{2/3}\left(\frac{V_g}{V}\right)^{0.62376}
\end{equation}
%
The area of the boundary between the liquid and the gaseous region
is given by
%
\begin{equation}
   A_b = A_g - 3.4211 \cdot V^{2/3}\left(\frac{V_g}{V}\right)^{1.24752}
\end{equation}
%
Finally, the area of the spherical shell in contact with the liquid
is given by
%
\begin{equation}
   A_l = \pi D^2 - A_g
\end{equation}
%

%\input{FlowModelling}

\section{Environment Models}


\subsection{Ephemerides}

\subsubsection{Analytic Ephemeris Model}

\begin{itemize}
     \item For a new body, the user must input the central body by
     choosing from the 9 Planets or the sun.
     %
     \item The user must provide the epoch.
     %
     \item  The user must provide the keplerian elements, in the
     central body centered, MJ2000Eq axis system.
     %
     \item  The user can provide a $\mu$ value for use in the
     solution of the equations of motion.
     %
\end{itemize}

The body should store the users original input for the state and
epoch, and the state and epoch calculated at the last request for
state information. Then, when the next request is made for state
information, the epoch and state from the last request are used as
the input state for next calculation.

\subsection{Atmospheric Density}

\begin{equation}
28K_p + 0.03e^{K_p} = A_p + 100\left(1 - e^{(-0.08A_p)}\right)
\end{equation}

\subsubsection{Jacchia Roberts}

\subsubsection{MSISE-90}

A. E. Hedin, Extension of the MSIS Thermospheric Model into the
Middle and Lower Atmosphere, J. Geophys. Res. 96, 1159, 1991.

Discuss observed vs. adjusted for F10.7 values, also URSI Series D

For testing http://nssdc.gsfc.nasa.gov/space/model/models/msis.html

http://www.agu.org/journals/ja/ja0212/2002JA009430/ \\
go to auxillary material on the left side menu and open the
tables-datasets.doc

Other useful models http://nssdc.gsfc.nasa.gov/space/model/

\subsubsection{Exponential Atmosphere}


%\section{Appendix 1:  Derivation of the Orbital Equations of
%Motion}
%
%
%
%\section{Notation}
%
%
%%\begin{tabbing}
%%12345678 \= Reynolds number based on length $s$ \kill
%%$\mathbf{r}$        \> Position vector \\
%%$\mathbf{v}$        \> Velocity vector \\
%%$m$              \> Number of spacecraft in formation \\
%%$n_k$              \> Number of maneuvers along $k^{th}$ trajectory \\
%%$t$                 \> Time \\
%%$\boldsymbol\Phi$   \> State transition matrix \\
%%$\mathbf{A}$   \> Upper left 3x3 partition of $\boldsymbol\Phi$ \\
%%$\mathbf{B}$   \> Upper right 3x3 partition of $\boldsymbol\Phi$ \\
%%$\mathbf{C}$   \> Lower left 3x3 partition of $\boldsymbol\Phi$\\
%%$\mathbf{D}$   \> Lower right 3x3 partition of $\boldsymbol\Phi$
%%\\
%%$\Delta\mathbf{v}_{jk}$   \> $j^{th}$ impulsive maneuver on $k^{th}$ trajectory\\
%%$\Delta v_{jk}$   \> Magnitude of $j^{th}$  maneuver on $k^{th}$ trajectory\\
%%$\mathcal{P}_{ok}$  \>  Initial trajectory of $k^{th}$ spacecraft\\
%%$\mathcal{P}_{fk}$  \>  Final trajectory of $k^{th}$ spacecraft\\
%%$N$                  \>  Number of Boundary Value Problems\\
%%$\mathbf{X}$     \>  Vector of independent variables\\
%%$\mathbf{C}$     \>  Vector of constants\\
%%\end{tabbing}
%%
%%\subsection{Subscripts}
%%\begin{tabbing}
%%12345678 \= \kill
%%$i$   \> Maneuver location index , ($2 \leq i \leq n_k-1$) \\
%%$j$   \> Maneuver epoch index, ($1 \leq j \leq n_k$) \\
%%$k$   \> Trajectory index , ($1 \leq k \leq m$) \\
%%$o$   \> Initial conditions \\
%%$f$   \> Final conditions \\
%%$I$  \>  Inertial Frame
%%\end{tabbing}
%%
%%\subsection{Superscripts}
%%\begin{tabbing}
%%12345678  \= \kill
%%$r$       \> Position solution \\
%%$v$       \> Velocity solution \\
%%$+$       \> Post-maneuver condition\\
%%$-$       \> Pre-maneuver condition
%%\end{tabbing}
%
%\section{Spacecraft Equations of Motion}
%
%
%%
%\begin{figure}[!]
%\centerline{
%\begin{picture}(100,500)
%\special{psfile= NBodyDiagram.eps hoffset= -135 voffset= -45
%hscale=85 vscale=85} \makebox(-20,585){$\hat{\mathbf{x}}_{I}$}
%\makebox(270,700){$\hat{\mathbf{y}}_{I}$}
%\makebox(-330,770){$\mathbf{r}_{s}$}
%\makebox(-330,877){$\mathbf{r}$}
%\makebox(-270,900){$\mathbf{r}_k$}
%\makebox(-390,964){$\mathbf{r}_{kj}$}
%\makebox(-500,814){$\mathbf{r}_{j}$} \makebox(-500,980){Central
%Body} \makebox(-320,995){$k^{th}$ Body}
%\end{picture}}\vskip -4.0 in  \caption{ N Body Illustration} \label{fig:NBody}
%\end{figure}
%
%\section{General Form}
%
%
%
%In general, the equations of motion can be written as
%%
%\begin{equation}
%  \frac{d}{dt}\left( m_s \displaystyle\frac{d \mathbf{r}_s}{dt}\right) = \sum
%  \mathbf{F}_k
%\end{equation}
%%
%where $\mathbf{r}_s$ is the spacecraft's position with respect to
%an inertial frame, and $m_s$ is the spacecraft mass.  Expanding
%the left hand side we get
%%
%\begin{equation}
%  \dot{m}_s\displaystyle\frac{d \mathbf{r}_s}{dt} + m_s \frac{d^2
%  \mathbf{r}_s}{dt^2}= \mathbf{F}_G + \mathbf{F}_T + \mathbf{F}_D
%  + \mathbf{F}_S + \mathbf{F}_O\label{Eq:EOMLHSExp}
%\end{equation}
%%
%where $\mathbf{F}_g$ is the force due to gravity, $\mathbf{F}_T$
%is the force due to thrust, $\mathbf{F}_D$ is the force due to
%drag, $\mathbf{F}_S$ is the force due to solar radiation pressure,
%and $\mathbf{F}_O$ are other forces. We usually want the equations
%of motion of the spacecraft expressed with respect to a central
%body. So, we note that
%%
%\begin{equation}
%     \mathbf{r}_s = \mathbf{r}_j + \mathbf{r}
%\end{equation}
%%
%Taking the first and second time derivatives gives us
%%
%%\begin{equation}
%%     \frac{d\mathbf{r}_s}{dt} = \frac{d\mathbf{r}_j}{dt} +
%%     \frac{d\mathbf{r}}{dt} \label{Eq:InertDeriv1}
%%\end{equation}
%%
%\begin{equation}
%     \frac{d^2\mathbf{r}_s}{dt^2} = \frac{d^2\mathbf{r}_j}{dt^2} +
%     \frac{d^2\mathbf{r}}{dt^2} \label{Eq:InertDeriv2}
%\end{equation}
%%
%We can substitute Eq.~(\ref{Eq:InertDeriv2}) into
%Eq.~(\ref{Eq:EOMLHSExp}) to get
%%
%\begin{equation}
%       \dot{m}_s\displaystyle\frac{d \mathbf{r}_s}{dt} + m_s \left( \frac{d^2\mathbf{r}_j}{dt^2} +
%     \frac{d^2\mathbf{r}}{dt^2}\right)= \mathbf{F}_G + \mathbf{F}_T + \mathbf{F}_D
%  + \mathbf{F}_S + \mathbf{F}_O \label{Eq:EOMAllExpanded}
%\end{equation}
%%
%The term that contains $\dot{m}_s$ appears to be problematic.
%However, if $\dot{m}_s \neq 0$, then we have a force due to
%thrust, $\mathbf{F}_T$, acting on the spacecraft.  In an inertial
%frame, this thrust is written as
%%
%\begin{equation}
%     \mathbf{F}_T = \dot{m}_s \left(\frac{d \mathbf{r}_s}{dt} +
%     \mathbf{v}_e\right)\label{Eq:InertialFt}
%\end{equation}
%%
%where $\mathbf{v}_e$ is the velocity of the exhaust with respect
%to the spacecraft.  Substituting Eq.~(\ref{Eq:InertialFt}) into
%Eq.~(\ref{Eq:EOMAllExpanded}) we arrive at.
%%
%\begin{equation}
%          \frac{d^2\mathbf{r}}{dt^2} =
%      \underbrace{\frac{\mathbf{F}_G}{m_s}-\frac{d^2\mathbf{r}_j}{dt^2}}_{gravity terms} +
%      \frac{\mathbf{F}_D}{m_s}
%  + \frac{\mathbf{F}_S}{m_s} + \frac{\mathbf{F}_O}{m_s}  +
%  \frac{\dot{m}_s}{m_s}\mathbf{v}_e\label{Eq:EOMGeneral}
%\end{equation}
%
%\subsection{Gravitational Acceleration}
%
%\begin{itemize}
%     \item  $\mathbf{r}_{j}$ = Position vector of central body
%     in inertial frame
%     %
%     \item  $ m_s $ =  Spacecraft mass
%     %
%     \item  $\mathbf{r}_{s}$ =  Spacecraft position w/r/t inertial
%     frame
%     %
%     \item $\mathbf{r}$ = Spacecraft position vector w/r/t central
%     body
%     %
%     \item $\mathbf{r}_{kj}$ = Position vector from central body to  $k^{th}$ body
%     %
%     \item $\mathbf{r}_sk$ = Vector from the $k^{th}$ body to
%     the spacecraft
%     %
%     \item $\mathbf{F}_k$ = Force of $k^{th}$ body on s/c
%     %
%     \item $n_b$ = Number of secondary bodies
%\end{itemize}
%
%Let's take a look at the gravitational terms in
%Eq.~(\ref{Eq:EOMGeneral})
%%
%\begin{equation}
%          \frac{d^2\mathbf{r}}{dt^2}
%          =\frac{\mathbf{F}_G}{m_s}-\frac{d^2\mathbf{r}_j}{dt^2}\label{Eq:GravEOM}
%\end{equation}
%%
%$\mathbf{F}_g$ is the force on the spacecraft due to gravity.  It
%is useful to start by assuming that the Earth and spacecraft are
%point masses.  Then we know from Newton's Law of Gravitation that
%%
%\begin{equation}
%     \mathbf{F}_G = -\frac{G m_s m_j}{r^3}\mathbf{r}
%\end{equation}
%%
%and
%%
%\begin{equation}
%    \frac{d^2\mathbf{r}_j}{dt^2} = \frac{G m_s }{r^3}\mathbf{r}
%\end{equation}
%%
%Substituting these equations into Eq.~(\ref{Eq:GravEOM}) yields
%%
%\begin{equation}
%   \frac{d^2\mathbf{r}}{dt^2} = -\frac{G m_s }{r^3}\mathbf{r} -
%   \frac{G m_j }{r^3}\mathbf{r} = -\frac{G \left(m_s +m_j
%     \right)}{r^3}\mathbf{r}
%\end{equation}
%%
%This is the well known two-body orbital equation of motion
%%
%\begin{equation}
%        \frac{d^2\mathbf{r}}{dt^2} = -\frac{\mu}{r^3}\mathbf{r}
%        \label{Eq:TwoBodyEOM}
%\end{equation}
%%
%where $\mu = G \left(m_s +m_j \right)$
%
%Now lets consider the case when there are other gravitational
%bodies in the system and that all of the gravitational bodies are
%non-spherical.  However, we'll assume the mass distribution of the
%spacecraft is negligible.  In this case we can write the forces on
%the spacecraft that are in addition to the central body force as
%%
%\begin{equation}
%   \frac{ \mathbf{F}_G}{m_s} = \nabla \phi_{sj}^o + \sum_{\stackrel{k=1}{k \neq j}}^{n_b} \nabla \left(\phi_{sk}^s + \phi_{sk}^o
%    \right)
%\end{equation}
%%
%where $\nabla$ is the gradient operator, $\phi$ a bodies
%gravitational potential, a superscript ``$s$" denotes the
%spherical portion of gravitational potential, and a superscript
%``$o$" denotes the oblate portion the gravitational potential.
%From physics we know that
%%
%\begin{equation}
%    \frac{d^2 \mathbf{r}}{dt^2} = \nabla \phi
%\end{equation}
%%
%where we choose to define the potential $\phi$ according to
%%
%\begin{equation}
%     \phi = \frac{\mu}{r}
%\end{equation}
%%
%Then, for the spherical terms, we know that
%%
%\begin{equation}
%    \nabla \phi_{sk}^s = -\frac{G m_k }{r_{sk}^3}\mathbf{r}_{sk} =
%    \frac{G m_k }{r_{ks}^3}\mathbf{r}_{ks}
%\end{equation}
%%
%%
%\begin{equation}
%    \frac{\mathbf{F}_G}{m_s} =  \nabla \phi_{sj}^o + \sum_{\stackrel{k=1}{k \neq j}}^{n_b}\left(\frac{G m_k }{r_{ks}^3}\mathbf{r}_{ks}   + \nabla
%    \phi_{k}^o\right)      \label{Eq:FG}
%\end{equation}
%
%
%We can apply Newton's laws to the central body to arrive at
%%
%\begin{equation}
%     \frac{d^2\mathbf{r}_j}{dt^2} = \frac{1}{m_j}\sum_{\stackrel{k=1}{k \neq j}}^{n_b}\nabla\left(
%     \phi_{kj}^s + \phi_{kj}^o
%     \right) = \sum_{\stackrel{k=1}{k \neq j}}^{n_b}\left(-
%     \frac{G m_k}{r_{kj}^3}\mathbf{r}_{kj} +
%     \frac{\nabla\phi_{kj}^o}{m_j}
%     \right) \label{Eq:CBEOM}
%\end{equation}
%%
%Eqs.~(\ref{Eq:FG}) and(\ref{Eq:CBEOM}) represent the forces in
%addition to the two-body point mass force in the equations of
%motion.  Augmenting Eq.~(\ref{Eq:TwoBodyEOM}) with
%Eqs.~(\ref{Eq:FG}) and(\ref{Eq:CBEOM}) we arrive at
%%
%\begin{equation}\begin{split}
%    \frac{d^2\mathbf{r}}{dt^2}
%    = &-\frac{\mu}{r^3}\mathbf{r} +
%    %
%    \nabla \phi_{sj}^o + \sum_{\stackrel{k=1}{k \neq j}}^{n_b}\left(\frac{G m_k }{r_{ks}^3}\mathbf{r}_{ks}   + \nabla
%    \phi_{ks}^o\right)\\
%     &+
%     %
%    \sum_{\stackrel{k=1}{k \neq j}}^{n_b}\left(-
%     \frac{G m_k}{r_{kj}^3}\mathbf{r}_{kj} +
%     \nabla\phi_{kj}^o
%     \right)
%\end{split}\end{equation}
%%
%Grouping similar terms we arrive at
%%
%\begin{equation}\begin{split}
%    \frac{d^2\mathbf{r}}{dt^2}
%    =  &-\frac{\mu}{r^3}\mathbf{r} +  \nabla \phi_{sj}^o +
%    %
%    G\sum_{\stackrel{k=1}{k \neq j}}^{n_b}m_k \left(\frac{\mathbf{r}_{ks}}{r_{ks}^3} -
%     \frac{\mathbf{r}_{kj} }{r_{kj}^3}   \right)\\
%     %
%   & + \sum_{\stackrel{k=1}{k \neq j}}^{n_b}\left( \nabla
%    \phi_{ks}^o +
%     \nabla\phi_{kj}^o
%     \right)
%\end{split} \end{equation}
%
%
%%\section{Spacecraft Equations of Motion}
%%
%%
%%%
%%\begin{figure}[!]
%%\centerline{
%%\begin{picture}(100,500)
%%\special{psfile= NBodyDiagram.eps hoffset= -155 voffset= -45
%%hscale=85 vscale=85} \makebox(-20,585){$\hat{\mathbf{x}}_{I}$}
%%\makebox(270,700){$\hat{\mathbf{y}}_{I}$}
%%\makebox(-330,770){$\mathbf{r}_{s}$}
%%\makebox(-330,877){$\mathbf{r}$}
%%\makebox(-270,900){$\mathbf{r}_k$}
%%\makebox(-390,964){$\mathbf{r}_{kj}$}
%%\makebox(-500,814){$\mathbf{r}_{j}$} \makebox(-500,980){Central
%%Body} \makebox(-320,995){$k^{th}$ Body}
%%\end{picture}}\vskip -4.0 in  \caption{ N Body Illustration} \label{fig:NBody}
%%\end{figure}
%%
%%\section{General Form}
%%
%%
%%
%%In general, the equations of motion can be written as
%%%
%%\begin{equation}
%%  \frac{d}{dt}\left( m_s \displaystyle\frac{d \mathbf{r}_s}{dt}\right) = \sum
%%  \mathbf{F}_k
%%\end{equation}
%%%
%%where $\mathbf{r}_s$ is the spacecraft's position with respect to
%%an inertial frame, and $m_s$ is the spacecraft mass.  Expanding
%%the left hand side we get
%%%
%%\begin{equation}
%%  \dot{m}_s\displaystyle\frac{d \mathbf{r}_s}{dt} + m_s \frac{d^2
%%  \mathbf{r}_s}{dt^2}= \mathbf{F}_G + \mathbf{F}_T + \mathbf{F}_D
%%  + \mathbf{F}_S + \mathbf{F}_O\label{Eq:EOMLHSExp}
%%\end{equation}
%%%
%%where $\mathbf{F}_g$ is the force due to gravity, $\mathbf{F}_T$
%%is the force due to thrust, $\mathbf{F}_D$ is the force due to
%%drag, $\mathbf{F}_S$ is the force due to solar radiation pressure,
%%and $\mathbf{F}_O$ are other forces. We usually want the equations
%%of motion of the spacecraft expressed with respect to a central
%%body. So, we note that
%%%
%%\begin{equation}
%%     \mathbf{r}_s = \mathbf{r}_j + \mathbf{r}
%%\end{equation}
%%%
%%Taking the first and second time derivatives gives us
%%%
%%%\begin{equation}
%%%     \frac{d\mathbf{r}_s}{dt} = \frac{d\mathbf{r}_j}{dt} +
%%%     \frac{d\mathbf{r}}{dt} \label{Eq:InertDeriv1}
%%%\end{equation}
%%%
%%\begin{equation}
%%     \frac{d^2\mathbf{r}_s}{dt^2} = \frac{d^2\mathbf{r}_j}{dt^2} +
%%     \frac{d^2\mathbf{r}}{dt^2} \label{Eq:InertDeriv2}
%%\end{equation}
%%%
%%We can substitute Eq.~(\ref{Eq:InertDeriv2}) into
%%Eq.~(\ref{Eq:EOMLHSExp}) to get
%%%
%%\begin{equation}
%%       \dot{m}_s\displaystyle\frac{d \mathbf{r}_s}{dt} + m_s \left( \frac{d^2\mathbf{r}_j}{dt^2} +
%%     \frac{d^2\mathbf{r}}{dt^2}\right)= \mathbf{F}_G + \mathbf{F}_T + \mathbf{F}_D
%%  + \mathbf{F}_S + \mathbf{F}_O \label{Eq:EOMAllExpanded}
%%\end{equation}
%%%
%%The term that contains $\dot{m}_s$ appears to be problematic.
%%However, if $\dot{m}_s \neq 0$, then we have a force due to
%%thrust, $\mathbf{F}_T$, acting on the spacecraft.  In an inertial
%%frame, this thrust is written as
%%%
%%\begin{equation}
%%     \mathbf{F}_T = \dot{m}_s \left(\frac{d \mathbf{r}_s}{dt} +
%%     \mathbf{v}_e\right)\label{Eq:InertialFt}
%%\end{equation}
%%%
%%where $\mathbf{v}_e$ is the velocity of the exhaust with respect
%%to the spacecraft.  Substituting Eq.~(\ref{Eq:InertialFt}) into
%%Eq.~(\ref{Eq:EOMAllExpanded}) we arrive at.
%%%
%%\begin{equation}
%%          \frac{d^2\mathbf{r}}{dt^2} =
%%      \underbrace{\frac{\mathbf{F}_G}{m_s}-\frac{d^2\mathbf{r}_j}{dt^2}}_{gravity terms} +
%%      \frac{\mathbf{F}_D}{m_s}
%%  + \frac{\mathbf{F}_S}{m_s} + \frac{\mathbf{F}_O}{m_s}  +
%%  \frac{\dot{m}_s}{m_s}\mathbf{v}_e\label{Eq:EOMGeneral}
%%\end{equation}
%%
%%\subsection{Gravitational Acceleration}
%%
%%\begin{itemize}
%%     \item  $\mathbf{r}_{j}$ = Position vector of central body
%%     in inertial frame
%%     %
%%     \item  $ m_s $ =  Spacecraft mass
%%     %
%%     \item  $\mathbf{r}_{s}$ =  Spacecraft position w/r/t inertial
%%     frame
%%     %
%%     \item $\mathbf{r}$ = Spacecraft position vector w/r/t central
%%     body
%%     %
%%     \item $\mathbf{r}_{kj}$ = Position vector from central body to  $k^{th}$ body
%%     %
%%     \item $\mathbf{r}_{sk}$ = Vector from the $k^{th}$ body to
%%     the spacecraft
%%     %
%%     \item $\mathbf{F}_k$ = Force of $k^{th}$ body on s/c
%%     %
%%     \item $n_b$ = Number of secondary bodies
%%\end{itemize}
%%
%%Let's take a look at the gravitational terms in
%%Eq.~(\ref{Eq:EOMGeneral})
%%%
%%\begin{equation}
%%          \frac{d^2\mathbf{r}}{dt^2}
%%          =\frac{\mathbf{F}_G}{m_s}-\frac{d^2\mathbf{r}_j}{dt^2}\label{Eq:GravEOM}
%%\end{equation}
%%%
%%$\mathbf{F}_g$ is the force on the spacecraft due to gravity.  It
%%is useful to start by assuming that the Earth and spacecraft are
%%point masses.  Then we know from Newton's Law of Gravitation that
%%%
%%\begin{equation}
%%     \mathbf{F}_G = -\frac{G m_s m_j}{r^3}\mathbf{r}
%%\end{equation}
%%%
%%and
%%%
%%\begin{equation}
%%    \frac{d^2\mathbf{r}_j}{dt^2} = \frac{G m_s }{r^3}\mathbf{r}
%%\end{equation}
%%%
%%Substituting these equations into Eq.~(\ref{Eq:GravEOM}) yields
%%%
%%\begin{equation}
%%   \frac{d^2\mathbf{r}}{dt^2} = -\frac{G m_s }{r^3}\mathbf{r} -
%%   \frac{G m_j }{r^3}\mathbf{r} = -\frac{G \left(m_s +m_j
%%     \right)}{r^3}\mathbf{r}
%%\end{equation}
%%%
%%This is the well known two-body orbital equation of motion
%%%
%%\begin{equation}
%%        \frac{d^2\mathbf{r}}{dt^2} = -\frac{\mu}{r^3}\mathbf{r}
%%        \label{Eq:TwoBodyEOM}
%%\end{equation}
%%%
%%where $\mu = G \left(m_s +m_j \right)$
%%
%%Now lets consider the case when there are other gravitational
%%bodies in the system and that all of the gravitational bodies are
%%non-spherical.  However, we'll assume the mass distribution of the
%%spacecraft is negligible.  In this case we can write the forces on
%%the spacecraft that are in addition to the central body force as
%%%
%%\begin{equation}
%%   \frac{ \mathbf{F}_G}{m_s} = \nabla \phi_{sj}^o + \sum_{\stackrel{k=1}{k \neq j}}^{n_b} \nabla \left(\phi_{sk}^s + \phi_{sk}^o
%%    \right)
%%\end{equation}
%%%
%%where $\nabla$ is the gradient operator, $\phi$ a bodies
%%gravitational potential, a superscript ``$s$" denotes the
%%spherical portion of gravitational potential, and a superscript
%%``$o$" denotes the oblate portion the gravitational potential.
%%From physics we know that
%%%
%%\begin{equation}
%%    \frac{d^2 \mathbf{r}}{dt^2} = \nabla \phi
%%\end{equation}
%%%
%%where we choose to define the potential $\phi$ according to
%%%
%%\begin{equation}
%%     \phi = \frac{\mu}{r}
%%\end{equation}
%%%
%%Then, for the spherical terms, we know that
%%%
%%\begin{equation}
%%    \nabla \phi_{sk}^s = -\frac{G m_k }{r_{sk}^3}\mathbf{r}_{sk} =
%%    \frac{G m_k }{r_{ks}^3}\mathbf{r}_{ks}
%%\end{equation}
%%%
%%%
%%\begin{equation}
%%    \frac{\mathbf{F}_G}{m_s} =  \nabla \phi_{sj}^o + \sum_{\stackrel{k=1}{k \neq j}}^{n_b}\left(\frac{G m_k }{r_{ks}^3}\mathbf{r}_{ks}   + \nabla
%%    \phi_{k}^o\right)      \label{Eq:FG}
%%\end{equation}
%%
%%
%%We can apply Newton's laws to the central body to arrive at
%%%
%%\begin{equation}\begin{split}
%%     \frac{d^2\mathbf{r}_j}{dt^2} = \frac{1}{m_j}\sum_{\stackrel{k=1}{k \neq j}}^{n_b}\nabla\left(
%%     \phi_{kj}^s + \phi_{kj}^o
%%     \right) = \\\sum_{\stackrel{k=1}{k \neq j}}^{n_b}\left(-
%%     \frac{G m_k}{r_{kj}^3}\mathbf{r}_{kj} +
%%     \frac{\nabla\phi_{kj}^o}{m_j}
%%     \right) \label{Eq:CBEOM}
%%     \end{split}
%%\end{equation}
%%%
%%Eqs.~(\ref{Eq:FG}) and(\ref{Eq:CBEOM}) represent the forces in
%%addition to the two-body point mass force in the equations of
%%motion.  Augmenting Eq.~(\ref{Eq:TwoBodyEOM}) with
%%Eqs.~(\ref{Eq:FG}) and(\ref{Eq:CBEOM}) we arrive at
%%%
%%\begin{equation}\begin{split}
%%    \frac{d^2\mathbf{r}}{dt^2}
%%    = -\frac{\mu}{r^3}\mathbf{r} +
%%    %
%%    \nabla \phi_{sj}^o + \sum_{\stackrel{k=1}{k \neq j}}^{n_b}\left(\frac{G m_k }{r_{ks}^3}\mathbf{r}_{ks}   + \nabla
%%    \phi_{ks}^o\right) +\\
%%     %
%%    \sum_{\stackrel{k=1}{k \neq j}}^{n_b}\left(-
%%     \frac{G m_k}{r_{kj}^3}\mathbf{r}_{kj} +
%%     \nabla\phi_{kj}^o
%%     \right)\end{split}
%%\end{equation}
%%%
%%Grouping similar terms we arrive at
%%%%
%%%\begin{equation}\begin{split}
%%%    \frac{d^2\mathbf{r}}{dt^2}
%%%    =  -\frac{\mu}{r^3}\mathbf{r} +  \nabla \phi_{sj}^o +
%%%    %
%%%    G\sum_{\stackrel{k=1}{k \neq j}}^{n_b}m_k \left(\frac{\mathbf{r}_{ks}}{r_{ks}^3} -
%%%     \frac{\mathbf{r}_{kj} }{r_{kj}^3}   \right)
%%%     %
%%%   + \\\sum_{\stackrel{k=1}{k \neq j}}^{n_b}\left( \nabla
%%%    \phi_{ks}^o +
%%%     \nabla\phi_{kj}^o
%%%     \right)\end{split}
%%%\end{equation}
%%
%%
%%
%% Let's
%%return to equation Eq.~(\ref{Eq:a_cb}). We need to calculate
%%$\mba_{cb}$ in cartesian coordinates.  However, we know the
%%potential in spherical coordinates.  So, we need to use the chain
%%rule to take the derivative and we see that
%%%
%%\begin{equation}\begin{split}
%%   \{\mathbf{a}_{cb}\}_F = \nabla U(r,\phi,\lambda)  = \frac{\partial U(r,\phi,\lambda)}{\partial \mbr_F}
%%   =\\\frac{\partial U}{\partial r}\frac{\partial r}{\partial \mbr_F} +
%%   \frac{\partial U}{\partial \phi}\frac{\partial \phi}{\partial
%%   \mbr_F}+ \frac{\partial U}{\partial \lambda}\frac{\partial \lambda}{\partial \mbr_F}
%%   \label{Eq:a_cb2}\end{split}
%%\end{equation}
%%%
%%The six partial derivatives in Eq.~(\ref{Eq:a_cb2}) are given in the
%%GTDS\cite{GTDS} math spec as
%%%
%%\begin{equation}\begin{split}
%%     \frac{\partial U}{\partial r} = &-\frac{\mu}{r^2}\sum_{\ell =
%%     2}^{\infty}\sum_{m=0}^{\ell}\left(
%%     \frac{R_{\otimes}}{r}\right)^\ell(\ell + 1)P_{\ell m} \left[sin{\phi}  \right]\\
%%    %
%%      &\left( C_{\ell m}\cos{m \lambda}+ S_{\ell m}\sin{m \lambda}  \right)
%%\end{split}\end{equation}
%%%
%%\begin{equation}\begin{split}
%%     \frac{\partial U}{\partial \phi} = &\frac{\mu}{r}\sum_{\ell =
%%     2}^{\infty}\sum_{m=0}^{\ell}\left(
%%     \frac{R_{\otimes}}{r}\right)^\ell\left( P_{\ell, m+1} \left[sin{\phi}  \right] - m \tan{\phi}P_{\ell m}\left[\sin{\phi}\right]\right)\\ &
%%    %
%%      \left( C_{\ell m}\cos{m \lambda}+ S_{\ell m}\sin{m \lambda}
%%     \right)
%%\end{split}\end{equation}
%%\begin{equation}\begin{split}
%%     \frac{\partial U}{\partial \lambda} = & \frac{\mu}{r}\sum_{\ell =
%%     2}^{\infty}\sum_{m=0}^{\ell}\left(
%%     \frac{R_{\otimes}}{r}\right)^\ell m  P_{\ell, m} \left[sin{\phi}  \right]
%%     \\
%%     %
%%     &  \left( S_{\ell m}\cos{m \lambda}+ C_{\ell m}\sin{m \lambda}
%%     \right)
%%\end{split}\end{equation}
%%
