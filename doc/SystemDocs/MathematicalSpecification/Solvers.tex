\chapter{Solvers}  \label{Ch:Solvers}

\section{Differential Correction}

\section{Broyden's Method}

\section{Newton's Method}

\section{Matlab fmincon}

The user first creates a solver and names it.  An example is

 \st{ Create fminconOptimizer SPQfmincon}

 The user creates an optimization sequence by issuing an optimize
 command, followed by the name of the optimizer to use

 \st{Optimize SQPfmincon}

 \st{EndOptimize}

 \section{The Vary Command}
The user defines the independent variables by the vary command,

\begin{table}[htb]
\caption{ Available Commands in an fmincon Loop }
\begin{tabular}{p{1.5 in} p{1.5 in}}
   \hline
   Value & Command\\
   \hline \hline
     $X_i$ & \st{Vary} \\
    Upper Bound on $X_i$ & \st{Vary}  \\
   Lower Bound on $X_i$ & \st{Vary} \\
   Nondimensionalization Factor 1 & \st{Vary} \\
   Nondimensionalization Factor 2 & \st{Vary} \\
   Nonlinear constraint function& \st{NonLinearConstraint} \\
   Linear constraint function & \st{LinearConstraint} \\
   Cost Function & \st{OptimizerName.Cost = }\\
   \hline
 \end{tabular}
 \label{Table:EventFunction_dValues}
\end{table}
