\chapter{Coordinate Systems} \label{Ch:CoordinateSystems}

%\section{Temp}
%
%From IAU-76 theory, we can covert from a vector in Earth fixed IAU-76, denoted $\mathbf{r}_{F_{76}}$, to
%inertial IAU-76/FK5, denoted $\mathbf{r}_{I_{76}}$ using the following expression.
%%
%\begin{equation}
%     \mathbf{r}_{F_{76}}  = \mathbf{W}(t) \mathbf{R}(t) \mathbf{N}(\Psi,\epsilon)\mathbf{P}(t)  \mathbf{r}_{I_{76}}
%\end{equation}
%%
%where $\mathbf{W}(t)$,$\mathbf{R}(t)$, $\mathbf{N}(\Psi,\epsilon)$, and $\mathbf{P}(t)$ are respectively polar motion, sidereal time, nutation, and precession.  The IERS provides corrections to $\Psi$ and $\epsilon$ to enable software codes that use 1976-based computer implementations to transform from ITRF to GCRF using
%%
%\begin{equation}
%     \mathbf{r}_{ITRF}  = \mathbf{W}(t) \mathbf{R}(t) \mathbf{N}(\Psi + \delta\Psi,\epsilon + \delta\epsilon)\mathbf{P}(t)    \mathbf{r}_{I_{GCRF}}
%\end{equation}
%%
%If the fixed frames of the two theories are equivalent, then we can equate the two previous expressions to arrive at
%%
%\begin{equation}
%     \mathbf{r}_{I_{76}}   = \mathbf{P}^T(t) \mathbf{N}^T(\Psi ,\epsilon)\mathbf{N}(\Psi + \delta\Psi,\epsilon + \delta \epsilon)\mathbf{P}  (t)  \mathbf{r}_{I_{GCRF}}
%\end{equation}
%%
%I suspect this is wrong because
%%
%\begin{equation}
%     \mathbf{r}_{ITRF}  \neq \mathbf{r}_{F_{76}}
%\end{equation}
%
%There are numerous coordinate systems used in space mission
%analysis, that when used appropriately can greatly simplify the work
%and yield insight that is not obvious otherwise.  Some examples are
%equatorial and ecliptic systems, and rotating coordinate systems
%based on the relative motion of two bodies such as the Earth and
%Moon.  GMAT has the capability to calculate many parameters in
%different coordinate systems, and these parameters can then be used
%in plots, reports, solvers, control flow statements and stopping
%conditions to name a few.
%
%In this chapter we investigate how GMAT performs coordinate system
%transformations, and how different coordinate systems are defined.
%We begin by defining some notation.  Next, we look at how to
%transform a vector and its first derivative from one coordinate
%system to another when the coordinate systems are translating and
%rotating with respect to one another.  Finally, we look at each
%coordinate system defined in GMAT and discuss how to find its
%rotation matrix and the first derivative of the rotation matrix to
%rotate to the J2000 coordinate system.

\section{General Coordinate System \\ Transformations } \index{Coordinate systems!general transformations}

GMAT has the capability to take a position and velocity vector in
one coordinate system, and convert them to another coordinate system
that may be both translating and rotating with respect to the
original system.  In this section we derive the equations governing
coordinate system transformations and describe the algorithm GMAT
uses to transform position and velocity vectors.

We start by defining some notation.  In Fig.~\ref{fig:Frames}, we
see an illustration of a point ``$p$" and two
coordinate systems $\mathcal{F}_\mathcal{O}$ and $\mathcal{F}_\mathcal{F}$.
Define the the position of $p$
expressed in $\mathcal{F}_\mathcal{O}$  as
%
$\mathbf{r}_p^\mathcal{O}$.
%
Define the position of point $p$ with respect to
frame $\mathcal{F}_\mathcal{F}$ as
%
$\mathbf{r}^\mathcal{F}_p$
%
\textit{It is important to
note that using this notation,} $\|\mathbf{r}_p^\mathcal{O} \| \neq \|\mathbf{r}_p^F\|$, \textit{
because the transformation defined in the notation contains a translation from the origin
of $\mathcal{F}_\mathcal{O}$ to the origin of $\mathcal{F}_\mathcal{\mcF}$ as well as a coordinate rotation.}
$\mathbf{r}_{f/i}$ is the vector from the origin of $\mathcal{F}_i$
to origin of $\mathcal{F}_f$.  Define the rotation matrix that rotates
from $\mathcal{F}_\mcI$ to $\mathcal{F}_\mcF$ as $\mathbf{R}^{\mcF/\mcI}$.
Finally, let's define the  angular velocity $\mbox{\boldmath{$\omega
$}}_{f/i}$ as the angular velocity of $\mathcal{F}_\mcI$ with respect to
$\mathcal{F}_\mcF$.  To simplify the notation, we assume that a vector
is expressed in the frame denoted by the superscript.  If need to define a
point ``p" with respect to $\mathcal{F}_\mathcal{O}$, but express this result in
$\mathcal{F}_\mathcal{F}$
we use the notation $[\mathbf{r}_p^\mathcal{O}]^\mcF$
curly brackets.  In summary, we have
%
\begin{tabbing}
\index{Coordinate systems!nomenclature}
%
12345678 \= Reynoldsnumber based on length $s$ \kill
$\mathbf{r}^{\mathcal{O}}_{p}$        \> Position of point $p$ w/r/t frame $\mathcal{F}_\mathcal{O}$
                               expressed in  $\mathcal{F}_\mathcal{O}$ \\
%
$[\mathbf{r}^{\mathcal{O}}_{p} ]^\mcF$      \> Position of point $p$ w/r/t frame $\mathcal{F}_\mathcal{O}$
                                   expressed in $\mathcal{F}_\mcF$   \\
%
$\mathbf{r}_{f/o}^\mcF$        \> Position vector from origin of $\mathcal{F}_\mathcal{O}$ to origin
                         of $\mathcal{F}_\mcF$ expressed in $\mathcal{F}_\mcF$  \\
%
$\mathbf{R}^{\mcF/\mathcal{O}}$      \> Rotation matrix from frame $\mathcal{F}_\mathcal{O}$ to $\mathcal{F}_\mcF$\\
%
$\mbox{\boldmath{$\omega $}}_{f/o}^\mathcal{O}$   \>  Angular velocity of frame
$\mathcal{F}_\mathcal{O}$ w/r/t $\mathcal{F}_\mcF$,   expressed in frame
$\mathcal{F}_\mathcal{O}$ \\
%
$\mbox{\boldmath{$\omega $}}_{f/o}^\mcF$   \>  Angular velocity of
frame $\mathcal{F}_\mathcal{O}$ w/r/t $\mathcal{F}_\mcF$,  expressed in
frame
$\mathcal{F}_\mcF$ \\
\end{tabbing}


\begin{figure*}[htb]
 \centerline{
\begin{picture}(100,230)
\special{psfile= Images/Frames.eps hoffset= -160 voffset= -200
hscale=60 vscale=60}
\makebox(-140,15){$\mathcal{F}_{\mathcal{O}}$}
\makebox(175,140){$\mathcal{F}_{\mcF}$}
\makebox(-300,240){$\mathbf{r}_{o}^\mathcal{O}$}
\makebox(-265,180){$\mathbf{r}_{i/f}$}
\makebox(-165,310){$\mathbf{r}_{o}^\mcF$}
\makebox(-220,430){$o$}
\end{picture}}
\caption{ Illustration of a Translating and Rotating Coordinate
System } \label{fig:Frames}
\end{figure*}


From inspection of Figure \ref{fig:Frames} we can write
%
\begin{equation}
      \mathbf{r}_{o}^\mcF = \underbrace{\mathbf{R}^{\mcF/\mcI}\mathbf{r}_{o}^\mcI}_{Rot.} +
     \underbrace{  \mathbf{r}_{i/f}^\mcF}_{Trans.}
     \label{Eq:PosTransform}
\end{equation}
%
Equation (\ref{Eq:PosTransform}) is the equation used to convert a
vector known in frame $\mathcal{F}_i$ to a vector in frame
$\mathcal{F}_f$, where both a rotation and a translation are
required.  The first term in Eq. (\ref{Eq:PosTransform}) is the
term that performs the rotation portion of the transformation.
Here, $\mathbf{r}_{i}$ is the position vector w/r/t to
$\mathcal{F}_i$ and is expressed in $\mathcal{F}_i$.
$\mathbf{R}_{fi}$ is the rotation matrix that rotates from
$\mathcal{F}_i$ to $\mathcal{F}_f$.   $\mathbf{r}_{if}$ is the
vector that goes from the origin of $\mathcal{F}_f$ to the origin
of $\mathcal{F}_i$, and is expressed in $\mathcal{F}_f$.

We also need to be able to determine the time rate of change of a
vector in frame $\mathcal{F}_f$ if we know the time rate of change
of the vector in $\mathcal{F}_i$.  To determine the equation that
describes the transformation, we must take the derivative of
Eq.~(\ref{Eq:PosTransform}) with respect to time.
\begin{equation}
     \frac{d\mathbf{r}_{f}}{dt} = \frac{d\mathbf{R}_{fi}\mathbf{r}_{i}}{dt}
     +\frac{d \mathbf{r}_{if}}{dt}
\end{equation}
%
Let's use a single dot above a variable to denote the first
derivative of that variable with respect to time.  Then, we can
expand this to obtain
%
\begin{equation}
     \dot{\mathbf{r}}_{f} = \dot{\mathbf{R}}_{fi}\mathbf{r}_{i} +
     \mathbf{R}_{fi}\dot{\mathbf{r}_{i}}
     +\dot{ \mathbf{r}}_{if}  \label{Eq:WithRdot}
\end{equation}
%
In Eq.~(\ref{Eq:WithRdot}) we see a term that contains the time
derivative of the rotation matrix from $\mathcal{F}_i$ to
$\mathcal{F}_f$.  We can write the time derivative of
$\mathbf{R}_{fi}$ as
%
\begin{equation}
     \dot{\mathbf{R}}_{fi} = \mathbf{R}_{fi}
     \mbox{\boldmath{$\omega $}}^{\mbox{x}}_{fi} =  \{\mbox{\boldmath{$\omega
     $}}^{\mbox{x}}_{fi}\}_f\mathbf{R}_{fi}\label{Eq:dRdt}
\end{equation}
%
where $\mbox{\boldmath{$\omega $}}_{fi}$ is the angular velocity of
$\mathcal{F}_i$ with respect to $\mathcal{F}_f$ expressed in
$\mathcal{F}_i$.  The skew symmetric matrix, $\boldsymbol \omega
^x$, is defined as
\begin{equation}
\boldsymbol \omega ^x= \left( \begin{array}{ccc}
  0 & -\omega_z  & \omega_y \\
  \omega_z & 0 & -\omega_x \\
  -\omega_y & \omega_x & 0 \\
\end{array} \right)
\end{equation}
%
In summary, using Eq.~(\ref{Eq:dRdt}) to transform a derivative
vector from $\mathcal{F}_i$  to $\mathcal{F}_f$ we can use any of
the following three equations:
%
\begin{equation}
     \dot{\mathbf{r}}_{f}  = \underbrace{\mathbf{R}_{fi}
     \mbox{\boldmath{$\omega $}}^{\mbox{x}}_{fi}\mathbf{r}_{i} + \mathbf{R}_{fi}\dot{\mathbf{r}}_{i}}_{Rot.}
     + \underbrace{\dot{\mathbf{r}}_{if}}_{Trans.} \label{Eq:Transform1}
\end{equation}
%
\begin{equation}
    \dot{\mathbf{r}}_{f}  = \underbrace{\{\mbox{\boldmath{$\omega $}}^{\mbox{x}}_{fi}\}_f\mathbf{R}_{fi}\mathbf{r}_{i} + \mathbf{R}_{fi}\dot{\mathbf{r}}_{i}}_{Rot.}
     + \underbrace{\dot{\mathbf{r}}_{if}}_{Trans.} \label{Eq:Transform2}
\end{equation}
%
\begin{equation}
     \dot{\mathbf{r}}_{f} = \underbrace{\dot{\mathbf{R}}_{fi}\mathbf{r}_{i} + \mathbf{R}_{fi}\dot{\mathbf{r}}_{i}
     }_{Rot.}+  \underbrace{\dot{\mathbf{r}}_{if}}_{Trans.}\label{Eq:Transform3}
\end{equation}
%
We choose between Eqs.~(\ref{Eq:Transform1}), (\ref{Eq:Transform2}),
or (\ref{Eq:Transform3}) depending on the type of information we
have, and which frame is most convenient to express the angular
velocity $\boldsymbol \omega_{fi}$ in.  In general, we know
$\mathbf{r}_{i}$ and $\dot{\mathbf{r}}_{i}$.  To perform the
transformation we need to determine $\mathbf{R}$,
$\dot{\mathbf{R}}$, and $\dot{\mathbf{r}}_{if}$ and these quantities
depend on $\mathcal{F}_i$ and $\mathcal{F}_f$.

One of the difficulties in implementing coordinate system
transformations in GMAT is that we often can't calculate
$\mathbf{R}_{fi}$ and $\dot{\mathbf{R}}_{fi}$ directly.  For
example, it is nontrivial to directly calculate the rotation matrix
from the Earth fixed frame to the Moon fixed frame.  Hence, we need
to choose a convenient intermediate coordinate system. We choose the
axis system defined by Earth's mean equinox and mean equator at the
J2000 epoch, denoted $\mathcal{F}_{J_{2k}}$, as the intermediate
reference frame for all transformations that require an intermediate
transformation. This choice is motivated by the fact that most of
the data needed to calculate $\mathbf{R}$ and $\dot{\mathbf{R}}$ is
given so that it is fast and convenient to calculate
$\mathbf{R}_{J_{2k},i}$, and $\dot{\mathbf{R}}_{J_{2k},i}$.

The steps taken to perform a general coordinate transformation in
GMAT are described below and illustrated in
Fig.~\ref{fig:RotSequence}. \index{Coordinate systems!transformation
algorithm} We start with a vector and its first derivative known in
frame $\mathcal{F}_i$, and wish to determine the vector and its
first derivative with respect to frame $\mathcal{F}_f$.  However, we
assume that the transformation to go directly from $\mathcal{F}_i$
to $\mathcal{F}_f$ is not known.

The first step in the process is to perform a rotation from
$\mathcal{F}_i$ to $\mathcal{F}_{J_{2k}}$. We define this
intermediate system as $\mathcal{F}_1$.  No translation is performed
in step one. Using only the rotation portions of from
Eqs.~(\ref{Eq:PosTransform}) and (\ref{Eq:Transform3}) we see that
%
\begin{equation}
          \left\{\mathbf{r}_i\right\}_{1} = \mathbf{R}_{J_{2k},i}
          \mathbf{r}_{i} \label{Eq:PosRot1}
\end{equation}
%
\begin{equation}
          \left\{\dot{\mathbf{r}}_i\right\}_1 = \dot{\mathbf{R}}_{J_{2k},i}\mathbf{r}_{i} +
          \mathbf{R}_{J_{2k},i}\dot{\mathbf{r}}_{i} \label{Eq:VelRot1}
\end{equation}
%
The second step is to perform a translation from the origin of
$\mathcal{F}_i$ to the origin of $\mathcal{F}_f$.  We define this
second intermediate system as $\mathcal{F}_2$. $\mathcal{F}_2$ has
the same origin as $\mathcal{F}_f$ but has the same axes as
$\mathcal{F}_{J_{2k}}$.  From inspection of
Fig.\ref{fig:RotSequence} we can see that
%
\begin{equation}
     \left\{\mathbf{r}_i\right\}_{1}  = \left\{\mathbf{r}_{Ri}\right\}_{J_{2k}}+ \left\{\mathbf{r}_{fR}\right\}_{J_{2k}} +
     \left\{\mathbf{r}_f\right\}_{2}
\end{equation}
%
Solving for $\mathbf{r}_f$ we obtain
%
\begin{equation}
     \left\{\mathbf{r}_f\right\}_{2} =  \left\{\mathbf{r}_i\right\}_{1} -
      \left\{\mathbf{r}_{Ri}\right\}_{J_{2k}} - \left\{
      \mathbf{r}_{fR}\right\}_{J_{2k}}\label{Eq:Translater}
\end{equation}
%
where $\left\{\mathbf{r}_{Ri}\right\}_{J_{2k}}$ is the vector from
the origin of $\mathcal{F}_i$ to the origin of $\mathcal{F}_R$
expressed in $\mathcal{F}_{J{2k}}$.  Similarly
$\left\{\mathbf{r}_{fR}\right\}_{J_{2k}}$ is the vector from the
origin of $\mathcal{F}_R$ to the origin of $\mathcal{F}_f$
expressed in $\mathcal{F}_{J{2k}}$. Because the vector $ \left\{
\mathbf{r}_f\right\}_2$ is expressed in an inertial system we can
we can take the derivative of Eq~.(\ref{Eq:Translater}) to obtain
%
\begin{equation}
      \left\{\mathbf{v}_f\right\}_{2} =  \left\{\mathbf{v}_i\right\}_{1} - \left\{ \mathbf{v}_{Ri} \right\}_{J_{2k}} - \left\{\mathbf{v}_{fR}\right\}_{J_{2k}}
\end{equation}
%
where $\left\{\mathbf{v}_{Ri}\right\}_{J_{2k}}$ is the velocity of
the origin of $\mathcal{F}_R$ w/r/t the origin of $\mathcal{F}_i$.
Similarly, $\left\{\mathbf{v}_{fR}\right\}_{J_{2k}}$ is the
velocity of the origin of $\mathcal{F}_f$ w/r/t the origin of
$\mathcal{F}_R$. Finally, we perform a rotation from
$\mathcal{F}_{J_{2k}}$ to $\mathcal{F}_f$ about the origin of
$\mathcal{F}_f$ to obtain the desired quantities.
%
\begin{equation}
    \mathbf{r}_f = \mathbf{R}_{f,J_{2k}}
    \left\{\mathbf{v}_f\right\}_{2} \label{Eq:PosRot2}
\end{equation}
%
\begin{equation}
  \dot{\mathbf{r}}_f = \dot{\mathbf{R}}_{f,J_{2k}}\left\{\mathbf{r}_f\right\}_{2} + \mathbf{R}_{f,J_{2k}}\left\{\mathbf{v}_f\right\}_{2}
   \label{Eq:VelRot2}
 \end{equation}
%
\begin{figure*}[htb]
 \centerline{
\begin{picture}(100,320)
\special{psfile= Images/RotSequence.eps hoffset= -200 voffset= -160
hscale=80 vscale=80}
%
\makebox(-175,315){$\mathcal{F}_i$}
\makebox(350,295){$\mathcal{F}_f$}
\makebox(-140,410){$\mathcal{F}_2$}
\makebox(-440,220){$\mathcal{J}_{2k}$}
\makebox(-830,460){$\mathcal{F}_1$}
\makebox(-730,580){$\mathbf{r}_{i}$}
\makebox(-490,560){$\mathbf{r}_{f}$}
\makebox(-740,290){$\mathbf{r}_{Ri}$}
\makebox(-510,290){$\mathbf{r}_{fR}$}
\end{picture}}\vspace{ -1 in} \caption{ General Coordinate System Transformation Approach in GMAT} \label{fig:RotSequence}
\index{Coordinate systems!transformation algorithm}
\end{figure*}
%


\section{Pseudo-Rotating Coordinate\\ Systems}\index{Psuedo-Rotating
Coordinate Systems} \label{Sec:PseudoRotating}

In mission analysis, sometimes it is useful to consider a rotating
coordinate system to be inertial at a given instant in time. In this
case, we ignore the effects of rotation on the velocity.  Let's call
systems where we neglect the rotational effects on velocity
 pseudo-rotating coordinate systems.

To perform transformations to a pseudo-rotating coordinate system,
the equations to convert a  position vector do not change and are
given by Eqs.~(\ref{Eq:PosRot1}) and (\ref{Eq:PosRot2}).  However,
the velocity conversion equations change because we neglect the
terms that contain $\dot{\mathbf{R}}$.  For pseudo-rotating
coordinate systems the velocity transformations shown in
Eqs.~(\ref{Eq:VelRot1}) and (\ref{Eq:VelRot2}) become
%
\begin{equation}
          \left\{\frac{d\mathbf{r}_i}{dt}\right\}_1 =  \mathbf{R}_{J_{2k},i}\frac{d\mathbf{r}_{i}}{dt}
\end{equation}
%
and
%
\begin{equation}
  \frac{d\mathbf{r}_f}{dt} =  \mathbf{R}_{f,J_{2k}}\left\{\mathbf{v}_f\right\}_{2}
\end{equation}
%

To perform the transformations describe in the last few sections, we
need to be able to calculate the rotation matrix between any
coordinate system and $\mathcal{F}_{J_{2k}}$, and the derivative of
the rotation matrix. In the following sections we calculate these
matrices for the systems used in GMAT.  We assume that we want the
transformation from some generic frame $\mathcal{F}_i$ to
$\mathcal{F}_{J_{2k}}$ which is the Earth Mean J2000 Equatorial
(MJ2000Eq) system.  The rotation matrix from $\mathcal{F}_{J_{2k}}$
to $\mathcal{F}_i$ can be found by the simple relationship.
%
\begin{equation}
      \mathbf{R}_{i,J_{2k}} = \mathbf{R}_{J_{2k},i}^{-1} = \mathbf{R}_{J_{2k},i}^{T}
\end{equation}
%
and
%
\begin{equation}
      \dot{\mathbf{R}}_{if} = \dot{\mathbf{R}}_{J_{2k},i}^{-1} = \dot{\mathbf{R}}_{J_{2k},i}^{T}
\end{equation}




\clearpage

\section{ITRF and ICRF}

The computation for the ICRF and ITRF tranformations in GMAT are from Kaplan\cite{Kaplan:05} and Capolla\cite{Coppola:etal:05} \emph{et.al.} and employ the IAU 2000A nutation IAU 2006 precession models.  The transformation is performed using three intermediate rotations as follows:
%
\begin{equation}
     \mathbf{R} = \mathbf{C}^T\mathbf{R}_3(-\theta)\mathbf{W}
\end{equation}
%
where $\mathbf{W}$ is the polar motion matrix, $\mathbf{R}_3$ is a 3 rotation through the Earth rotation angle $\theta$ (see Section \ref{sec:BasicRotationMatrices}), and
$\mathbf{C}$ is a single matrix that captures precession, nutation, and frame bias.
The time derivative of the rotation matrix assumes that the only significant time variation of the rotation matrix is due to the Earth's spin and is computed using
%
\begin{equation}
     \mathbf{R} = \mathbf{C}^T\mathbf{R}_3(-\theta)\boldsymbol{\omega}_E^x\mathbf{W}
\end{equation}
%
where
%
\begin{equation}
   \boldsymbol{\omega}_E^x =
   \left(\begin{array} {ccc}
     0 & -\omega_e & 0 \\
     \omega_e & 0 & 0 \\
     0 & 0 & 0
   \end{array}\right)
\end{equation}
%
and $\omega_e$ is computed using Eq.~(\ref{Eq:EarthAngularVelocity}).

$\mathbf{W}$ is computed from
Eq.~6.15 in Kaplan\cite{Kaplan:05} as shown below.
%
\begin{equation}
   \mathbf{W} = \mathbf{R}_3(-s')\mathbf{R}_2(x_p)\mathbf{R}_1(y_p)
\end{equation}
%
The variable $s'$ is computed from
%
\begin{equation}
    s' =  (-47 \mu\mbox{as}) T_{TT}
\end{equation}
%
where $T_{TT}$ is given by Eq.~(\ref{Eq:T_TTComputation})and $x_p$ and $y_p$ are interpolated using third order lagrange interpolation from Earth Orientation Parameters (EOP) files provided by the IERS. The Earth Rotation angle is computed from
%
\begin{equation}
    \theta = 2\pi(0.7790572732640 + 1.00273781191135448(JD_{UT1} - 2451545.0))
\end{equation}
%
where $JD_{UT1}$ is the Julian date expressed in UT1 using Eq.~(\ref{Eq:UT1ToUTC}).

$\mathbf{C}^T$ is computed using
%
\begin{equation}
    \mathbf{C}^T = \left(\begin{array}{ccc}
    1 - b X^2 & -b X Y & X\\
    -b X Y & 1-b Y^2 & Y\\
    -X & -Y & 1 - b(X^2 + Y^2)
    \end{array}\right) \mathbf{R}_{3}(s)
\end{equation}
%
where
%
\begin{equation}
    b = \frac{1}{1 + \sqrt{1 - X^2 - Y^2}}
\end{equation}
%
The variables $X$, $Y$, are respectively the $x$-component and $y$-component of the Celestial Intermediate
Pole unit vector (CIP), and $s$ is called the Celestial Intermediate Origin (CIO) locator. The computation of $X$, $Y$, and $s$ requires evaluating series expansions with thousands of terms and many trigonometric function evaluations and is computationally expensive.  Vallado \emph{et. al.} show that it is possible to precompute $X$, $Y$, and $s$ at one day intervals and interpolate the values using ninth order Lagrange interpolation and provide values that are accurate to within the error of the theory itself.  The interpolation of $X$, $Y$, and $s$ is over two orders of magnitude faster than the series evaluation.  Interpolation of data at 1 day intervals is possible because all physical affects with a period of two days or less are not included in the theory for $X$, $Y$,
$s$ and are accounted for in the daily observations provided by the IERS in the EOP files\cite{Kaplan:05}.  GMAT interpolates tabulated values $X$, $Y$, $s$ created using the IAU SOFA routines.  $s$ is computed using the routine iauS06a.c which uses the 2000A nutation model and 2006 precession model.  $X$ and $Y$ are computed using the IAU SOFA routine called iauXy06 which also uses the the 2000A nutation model and 2006 precession model.


\section{Transformation from ICRT to MJ2000Eq}

With the inclusion of ICRF-based systems, GMAT supports two types of axis systems: (1) axis systems that are
based on IAU-1976 theory (called MJ2000Eq in GMAT), and (2) axis systems based on IAU-2000 theory (called ICRF in GMAT).  Rotation from ICRF to MJ2000Eq is performed using the Frame Bias matrix, $\mathbf{B}$, given by Kaplan\cite{Kaplan:05}.
%%
\begin{equation}
   \mathbf{B} = \left(
   \begin{array}{ccc}
       1 - 0.5(d\alpha_0^2 + \xi_0^2) & d \alpha_0 & -\xi_0\\
       -d \alpha_0 - \eta_0 \xi_0 & 1 - 0.5(d \alpha_0^2 + \eta_0^2) & -\eta_0\\
      \xi_0 - \eta_0 d \alpha_0 & \eta_0+\zeta_0 d \alpha_0 & 1 - 0.5(\eta_0^2 + \xi_0^2)
   \end{array}
   \right)
\end{equation}
%
where $d\alpha_0 = -14.6 \mbox{ mas}$, $\xi_0 = -16.6170 \mbox{ mas}$, and $\eta_0 = -6.8192 \mbox{ mas}$.

The rotation between any two axes, defind here as $\mathcal{A}_1$ and $\mathcal{A}_2$, are performed in three steps:
%
\begin{enumerate}
   \item Convert from $\mathcal{A}_1$ to the base axes for $\mathcal{A}_1$.
   \item If $\mathcal{A}_1$ and $\mathcal{A}_2$ have different base axis systems, use $\mathbf{B}$ to convert from $\mathcal{A}_1$ to the base axes of $\mathcal{A}_2$
   \item Convert from the base axes for $\mathcal{A}_2$ to $\mathcal{A}_2$
\end{enumerate}


\section{The $\mathcal{F}_{J_{2k}}$ Inertial System and FK5 Reduction}

It is well know that Newton's laws must be applied in an inertial
system.  The struggle to determine a truly inertial system has
continued since Newton's time.  In reality, the best we can do is
approximate a truly inertial system  in which to apply Newton's
laws. In GMAT that system is the FK5 system, here called
$\mathcal{F}_{J_{2k}}$.  The $\mathcal{F}_{J_{2k}}$ system is
referenced to the Earth's equator and the Earth's orbit about the
sun.  Because neither of these two planes are fixed in space, we
must pick an epoch and define an inertial system based on the
geometry at that epoch.  This epoch is commonly chosen as the J2000
epoch.  In this section, we present the definition of the
$\mathcal{F}_{J_{2k}}$ system, and discuss the transformation from
$\mathcal{F}_{J_{2k}}$  to the Earth Fixed system.  This
transformation is called FK5 reduction.  We begin with a conceptual
discussion of how the Earth's spin axis moves with respect to
inertial space.  We conclude this section with a presentation of the
mathematical theory of FK5 reduction.



\subsection{Overview of FK5 Reduction} \index{FK5 Reduction!overview}
\index{FK5 Reduction!Earth fixed system}

The inertial system most commonly used in astrodynamics as of this
writing is the FK5 system.  We call this system
$\mathcal{F}_{J_{2k}}$.  The $\mathcal{F}_{J_{2k}}$ system is used
for many calculations in GMAT.  The two most important are for
integrating equations of motion, and as an intermediate system for
coordinate system transformation. $\mathcal{F}_{J_{2k}}$ is used
throughout the astrodynamics community as the coordinate system to
represent time varying data such as planetary ephemerides and
planetary pole and prime meridian locations.

The rigorous mathematical definition of $\mathcal{F}_{J_{2k}}$ is
complex. So, let's start with a simple qualitative explanation.
The nominal $z$-axis of $\mathcal{F}_{J_{2k}}$ is normal to the
Earth's equatorial plane.  The nominal $x$-axis points along the
line formed by the intersection of the Earth's equatorial plane
and the ecliptic plane, in the direction of Aries.  The nominal
$y$-axis completes the right-handed system. Both the equatorial
and ecliptic planes move slowly with respect to inertial space.
The rigorous definition of FK5 uses the mean planes of the
ecliptic and equator, at the J2000 epoch. We'll take a closer look
at the mathematical definitions of the mean ecliptic and equator
below.


FK5 reduction is the transformation that rotates a vector
expressed in the $\mathcal{F}_{J_{2k}}$ system to the Earth Fixed
coordinate system. To perform this transformation obviously
requires an understanding of how the Earth's orientation changes
with respect to $\mathcal{F}_{J_{2k}}$.  The time varying
orientation of the Earth is complex and is due to complicated
interactions between the Earth and it's external environment and
complicated internal dynamics. In fact, the dynamic orientation of
the Earth is so complicated that we can't model it completely and
FK5 reduction is a combination of dynamics models and empirical
observations that are updated daily.

We describe the orientation of the Earth using three types of
motion.  The first type, including precession and nutation,
describes how the Earth's principal moment of inertia moves with
respect to inertial space\cite{seidelmann}.  The motion is
illustrated in Fig. \ref{fig:FK5FigOne}. The principal moment of
inertia is defined as the Celestial Ephemeris Pole\index{Celestial
Ephemeris Pole}, and due to the fact that Earth's mass distribution
changes with time, the Celestial Ephemeris Pole is not constant with
respect to the Earth's surface. Precession is the coning motion that
the Celestial Ephemeris Pole makes around the ecliptic north pole.
The other principal component of the motion of the Celestial
Ephemeris Pole is commonly called nutation and is the oscillation in
the angle between the Celestial Ephemeris Pole and the north
ecliptic pole. The theory of Precession and Nutation come from
dynamics models of the Earth's motion.  The second type of motion is
called sidereal time, and represents a rotation about the Celestial
Ephemeris Pole. The sidereal time model is a combination of theory
ad observation. The third motion is that of the Earth's
instantaneous spin axis with respect to the Earth's surface.  As,
we'll see below, the Earth's spin axis is not constant with respect
to the Earth's crust and it's motion is called Polar Motion.  A
portion of polar motion is due to complicated dynamics, and a
portion is due to unmodelled errors in nutation. Polar motion is
determined from observation. Now that we've had a brief introduction
to precession, nutation, sidereal time, and polar motion, let's look
at each in more detail.

\subsubsection{Precession}\index{FK5 Reduction!precession}

As we mentioned above, precession is the coning motion of the
Celestial Ephemeris Pole about the ecliptic north pole and is
illustrated in Fig \ref{fig:FK5FigOne}. The motion is caused by two
primary effects. The first is the motion of the ecliptic plane due
to the gravitational effects of the Sun, Moon, and planets on the
Earth's orbit, and is called planetary precession. If the Earth's
equator were fixed in inertial space, the effects of planetary
precession would cause a precession of the equinox of about 12" per
century and a decrease in the obliquity of the ecliptic of about 47"
per century \cite{seidelmann}.  The second cause of precession is
due to the gravitational attraction of the Sun and Moon on the
irregular mass distribution of the Earth. This causes a change in
the orientation of the Earth's equatorial plane with respect to
inertial space with a smooth, long-period motion with a period of
26,000 years.  The combined effects of planetary and lunisolar
precession are called general precession and account for the secular
and long period motion of the Celestial Ephemeris Pole (the short
period motion is called nutation).  \index{FK5 Reduction!secular
motion} \index{FK5 Reduction!long period motion}The secular and long
period motion is often used to define a mean equator and equinox,
because it does not contain the short period motion that is modelled
using nutation. Precession is modelled using three cubic equations
that are shown in Section %\ref{Sec:Precession}

\subsubsection{Nutation}\index{FK5 Reduction!nutation}

Nutation is the most complex motion in FK5 reduction.   According to
Seidelmann\cite{seidelmann}, nutation is ``the short period motion
of the Earth's rotation axis with respect to a space-fixed
coordinate system."  Nutation is actually a superposition of motions
with many different periods, the longest of which is 18.6 years and
is associated with the regression of the node of the Moon's orbit.
There are nutation effects due to the gravitational torque of the
Sun, Moon, and planets on the Earth's irregular mass distribution.
There are also Nutation effects due to the fact that Earth is not a
rigid body.  Nutation motion has an amplitude of about 9" and is
usually represented as the sum of two components one in longitude
and one in obliquity.

\clearpage

Nutation is modelled by separating the free and forced motion of the
Earth. The forcing terms are due to torques from the Sun, Moon, and
planets.   The free terms are determined by observation because they
are beyond our current modelling abilities. The resulting theory is
a series expansion that contains coefficients and is a function of
the location of the Sun, Moon, and planets. Nutation is intimately
connected with polar motion, and in fact, as we'll see in a later
section, errors in nutation modelling are captured in polar motion
measurements.
%
\begin{figure}[h]
    \begin{picture}(100,200)
       \special{psfile= Images/NutPrec.eps hoffset= 0 voffset=
       0
       hscale=10 vscale=10}
    \end{picture}
    \makebox(-180,90){$\Upsilon$} \makebox(-200,90){}
    \caption{Inertial Motion of Earth's Spin Axis}
    \label{fig:FK5FigOne}
\end{figure}
%
Nutation is modelled by separating the free and forced motion of the
Earth. The forcing terms are due to torques from the Sun, Moon, and
planets.   The free terms are determined by observation because they
are beyond our current modelling abilities.   The resulting theory
is a series expansion that contains coefficients and is a function
of the location of the Sun, Moon, and planets.  Nutation is
intimately connected with polar motion, and in fact, as we'll see in
a later section, errors in nutation modelling are captured in polar
motion measurements.

\clearpage

\subsubsection{Sidereal Time}\index{FK5 Reduction!polar motion}

%-------------------------------------------------------------------------
%------------------------------------------------------------------------
\begin{figure*}\begin{picture}(100,250)\begin{picture}(100,250)
\makebox(500,250){$\mathbf{R}_{IF} =
\mathbf{PREC}^T\mathbf{NUT}^T\mathbf{ST}^T\mathbf{PM}^T =
\underbrace{\mathbf{R}_3^T(-\zeta)\mathbf{R}_2^T(\Theta)\mathbf{R}_3^T(-z)}_{\mathbf{PREC}^T}
%
\underbrace{\mathbf{R}_1^T(\bar{\epsilon})\mathbf{R}_3^T(-\Delta
\Psi)\mathbf{R}_1^T({-\epsilon})}_{\mathbf{NUT}^T}
%
\underbrace{\mathbf{R}_3^T(\theta_{AST})}_{\mathbf{ST}^T}
%
\underbrace{\mathbf{R}_1^T(-y_p)\mathbf{R}^T_2(-x_{p})}_{\mathbf{PM}^T}$}
%%
\makebox(-510,160){$\mathbf{R}_{FI} = \mathbf{PM}\hspace{.05
in}\mathbf{ST}\hspace{.05 in}\mathbf{NUT}\hspace{.05
in}\mathbf{PREC} =
\overbrace{\mathbf{R}_2(-x_{p})\mathbf{R}_1(-y_p)}^{\mathbf{PM}}
%
\overbrace{\mathbf{R}_3(\theta_{AST})}^{\mathbf{ST}}
%
\overbrace{\mathbf{R}_1({-\epsilon})\mathbf{R}_3(-\Delta
\Psi)\mathbf{R}_1(\bar{\epsilon})}^{\mathbf{NUT}}
%
\overbrace{\mathbf{R}_3(-z)\mathbf{R}_2(\Theta)\mathbf{R}_3(-\zeta)}^{\mathbf{PREC}}$}
%
\makebox(-783,00){FK5} \makebox(-772,18){MODEq}
\makebox(-780,40){MODEc} \makebox(-790,60){TODEc}
\makebox(-796,77){TODEq}\makebox(-810,98){PEF}
\makebox(-810,120){ITRF}
%
\makebox(-820,300){FK5} \makebox(-810,320){MODEq}
\makebox(-817,340){MODEc} \makebox(-826,360){TODEc}
\makebox(-832,380){TODEq}\makebox(-846,400){PEF}
\makebox(-847,420){ITRF}
%
\drawline[10](-400,0)(-55,0)\drawline[10](-55,0)(-55,65)
\drawline[10](-400,10)(-158,10)\drawline[10](-158,10)(-158,65)
\drawline[10](-400,20)(-185,20)\drawline[10](-185,20)(-185,65)
\drawline[10](-400,30)(-230,30)\drawline[10](-230,30)(-230,65)
\drawline[10](-400,40)(-260,40)\drawline[10](-260,40)(-260,65)
\drawline[10](-400,50)(-310,50)\drawline[10](-310,50)(-310,65)
\drawline[10](-400,60)(-383,60)\drawline[10](-383,60)(-383,65)
%
\drawline[10](-398,150)(-385,150)\drawline[10](-385,150)(-385,140)
\drawline[10](-398,160)(-285,160)\drawline[10](-285,160)(-285,140)
\drawline[10](-398,170)(-258,170)\drawline[10](-258,170)(-258,140)
\drawline[10](-398,180)(-210,180)\drawline[10](-210,180)(-210,140)
\drawline[10](-398,190)(-173,190)\drawline[10](-173,190)(-173,140)
\drawline[10](-398,200)(-130,200)\drawline[10](-130,200)(-130,140)
\drawline[10](-398,210)(-40,210)\drawline[10](-40,210)(-40,140)
%
\end{picture}\end{picture}\caption{Intermediate Transformations and Coordinate Systems in FK5 Reduction}
\end{figure*}
%-------------------------------------------------------------------------
%------------------------------------------------------------------------

\subsection{Precession Calculations} \index{FK5 Reduction!precession}\label{Sec:PrecessionTheory}

\begin{equation}
   JD_{TDB} \approx JD_{TT}
\end{equation}
%
\begin{equation}
   T_{TDB} = \frac{JD_{TDB} - 2,451,545.0}{36525}
\end{equation}
%
\begin{equation}
    \zeta = 2306.2181^{"} T_{TDB} +0.30188 T_{TDB}^2 + 0.017998
    T_{TDB}^3
\end{equation}
%
\begin{equation}
    \Theta = 2004.3109^{"} T_{TDB} -0.42665 T_{TDB}^2 - 0.041833
    T_{TDB}^3
\end{equation}
%
\begin{equation}
    z = 2306.2181^{"}T_{TDB} + 1.09468 T_{TDB}^2 + 0.018203
    T_{TDB}^3
\end{equation}
%
\begin{equation}
     \mathbf{P} =
     \mathbf{R}_3(-z)\mathbf{R}_2(\Theta)\mathbf{R}_3(-\zeta)
\end{equation}

\subsection{Nutation Calculations} \index{FK5 Reduction!nutation} \label{Sec:NutationTheory}

GMAT has the ability to use either the 1980 IAU Theory of Nutation,
or the IERS 1996 Theory of Nutation.  There are some calculations
that are common to both, so let's look at them first.  The mean
obliquity of the ecliptic at the J2000 epoch, $\bar{\epsilon}$, is
given by

\begin{equation}
    \begin{split}
    \bar{\epsilon} = & 84381.448 - 46.8150T_{TDB} - 0.00059T_{TDB}^2 \\ & +
    0.001813T_{TDB}^3
    \end{split}
\end{equation}

As we mentioned previously, Earth's nutation is caused by the
combined gravitational effect of the Moon and Sun.  So, we would
expect to see the time dependent location of the Moon and Sun appear
in the equations for Earth nutation.  The theories of nutation
described below take into account of the Moon and Sun position by
modelling mean anomalies of the Moon and Sun, $l$ and $l'$
respectively, the mean argument of latitude of the Moon, $F$,  the
difference between the mean longitude of the Sun and Moon, $D$, and
the mean longitude of the ascending node of the Moon's orbit,
$\Omega$. The equations used to determine these values as a function
of $T_{TDB}$ are:
%
\begin{equation}
      \begin{split}
      l = & 134.96340251^{\circ} + (1717915923.2178 T_{TDB} + \\ & 31.8792 T_{TDB}^2 + 0.051635 T_{TDB}^3 - 0.00024470
      T_{TDB}^4)^{"}
      \end{split}
\end{equation}
%
\begin{equation}
      \begin{split}
     l' = & 357.52910918^{\circ} + ( 129596581.0481T_{TDB}- \\ & 0.5532
      T_{TDB}^2 - 0.000136T_{TDB}^3 - 0.00001149T_{TDB}^4)^{"}
      \end{split}
\end{equation}
%
\begin{equation}
      \begin{split}
     F = & 93.27209062^{\circ} + (1739527262.8478T_{TDB}
     - \\ &
       12.7512T_{TDB}^2 + 0.001037T_{TDB}^3 + 0.00000417T_{TDB}^4)^{"}
      \end{split}
\end{equation}
%
\begin{equation}
      \begin{split}
      D = & 297.85019547^{\circ} + (1602961601.2090T_{TDB} -
      \\ &
       6.3706T_{TDB}^2 + 0.006593T_{TDB}^3 - 0.00003169T_{TDB}^4)^{"}
      \end{split}
\end{equation}
%
\begin{equation}
      \begin{split}
      \Omega  = & 125.04455501^{\circ} + (  -6962890.2665T_{TDB}
      + \\ &
      7.4722T_{TDB}^2 + 0.007702T_{TDB}^3 - 0.00005939T_{TDB}^4)^{"}
      \end{split}
\end{equation}

\subsubsection{ 1980 Nutation Theory } \index{FK5 Reduction!1980 nutation theory}


\begin{equation}
     \Delta \Psi_{1980} = \sum_{i=1}^{106}\left( A_i + B_i T_{TDB}
     \right) \sin{a_p}
\end{equation}
%
\begin{equation}
     \Delta \epsilon_{1980} = \sum_{i=1}^{106}\left( C_i + D_i T_{TDB}
     \right) \cos{a_p}
\end{equation}
%
\begin{equation}
     a_p = a_{1i} l + a_{2i} l' + a_{3i} F+
     a_{4i} D + a_{5i}\Omega
\end{equation}
%
\begin{equation}
     \mathbf{N} = \mathbf{R}_1(-\epsilon) \mathbf{R}_3(-\Delta
     \Psi)\mathbf{R}_1(\bar \epsilon)
\end{equation}

\subsubsection{ 1996 Nutation Theory }\index{FK5 Reduction!1996 nutation
theory}

The 1996 theory of nutation published by the IERS is a higher
fidelity model of Earth nutation.  There are two primary differences
between the 1908 IAU theory and the 1996 IERS theory.  The first
difference is the 1996 theory uses a 263 term series expansion for
the effects of the Earth and Moon.  The 1980 theory uses a 106 term
series.  The second difference is that the 1996 theory has a second
series expansion to account for the effects of nutation caused by
the more massive planets.  The planetary series expansion is a 118
term series.  Let's begin with the equations for the Earth and
Moon's effect on Earth nutation, according to the 1996 IERS theory:
%
\begin{equation}
     \Delta \Psi_{1996} = \sum_{i=1}^{263}\left( A_i + B_i T_{TDB}
     \right) \sin{a_p} +
     E_i\cos{a_p}
\end{equation}
%
\begin{equation}
     \Delta \epsilon_{1996} = \sum_{i=1}^{263}\left( C_i + D_i T_{TDB}
     \right) \cos{a_p} + F_i\sin{a_p}
\end{equation}
%
\begin{equation}
     a_p = a_{1i} M_\circ + a_{2i} M_\odot + a_{3i}u_{M\circ} +
     a_{4i}D_{\odot} + a_{5i}\Omega_{\circ}
\end{equation}
%

To calculate the planetary effects on nutation, we begin by
calculating the mean heliocentric longitude of the planets.  Only
the effects of Venus(V), Mars(M),  Jupiter{J), and Saturn(S) are
included in the theory. We require the Earth's (E) mean longitude
also.  The mean longitudes are calculated using:
%
\begin{eqnarray}
    \lambda_{V} &=& 181.979800853^\circ + 58,517.8156748T_{TDB} \nonumber \\
    %
    \lambda_{E} &=& 100.466448494^\circ + 35,999.3728521T_{TDB} \nonumber \\
    %
    \lambda_{M} &=& 355.433274605^\circ + 19,140.299314T_{TDB} \nonumber \\
    %
    \lambda_{J} &=& 34.351483900^\circ + 3,034.90567464T_{TDB} \nonumber \\
    %
    \lambda_{S} &=& 50.0774713998^\circ + 1,222.11379404T_{TDB} \nonumber
\end{eqnarray}
%
The general precession in longitude, $p_a$, is calculated using
%
\begin{equation}
    p_a = 1.39697137214^\circ T_{TDB} + 0.0003086T_{TDB}^2 \nonumber
\end{equation}
%
Finally, the planetary terms are calculated using:
%
\begin{equation}
     \Delta \Psi_{pl} = \sum_{i=1}^{118}\left( A_i + B_i T_{TDB}
     \right) \sin{a_{pl}}
\end{equation}
%
\begin{equation}
     \Delta \epsilon_{pl} = \sum_{i=1}^{118}\left( C_i + D_i T_{TDB}
     \right) \cos{a_{pl}}
\end{equation}
%
\begin{eqnarray}
     a_{pl} = &a_{1i} \lambda_V + a_{2i} \lambda_E + a_{3i} \lambda_M + a_{4i}\lambda_J + a_{5i} \lambda_S + \nonumber\\
     &a_{6i} p_a + a_{7i}D  + a_{8i}F +
    a_{9i} l +  a_{10i}\Omega
\end{eqnarray}
%
\begin{equation}
     \Delta\Psi = \Delta\Psi_{1996} +
     \underbrace{\Delta\Psi_{pl}}_{optional}
\end{equation}
%
\begin{equation}
     \Delta\epsilon = \Delta\epsilon_{1996} +
     \underbrace{\Delta\epsilon_{pl}}_{optional}
\end{equation}
%
\begin{equation}
    \epsilon = \bar{\epsilon} + \Delta\epsilon
\end{equation}
 In GMAT, the planetary terms are optional.  If the user has
selected to include the planetary terms, the
\begin{equation}
     \mathbf{N} = \mathbf{R}_1(-\epsilon) \mathbf{R}_3(-\Delta
     \Psi)\mathbf{R}_1(\bar \epsilon)
\end{equation}

\subsection{Sidereal Time Calculations} \index{FK5 Reduction!sidereal time}

To calculate the sidereal time of the Earth, we need the current
time which is then used to determine the Greenwich Mean Sidereal
Time (GMST) and the equation of the equinoxes.  GMST is calculated
using:
%
\begin{equation}
     \begin{split}
        \theta_{GMST} =  & 1.00965822615e6 + 4.746600277219299e10T_{UT1}\\ & + 1.396560T_{UT1}^2 +
        9.3e-5T_{UT1}^3  \hspace{.15 in} \mbox{(arcseconds)}
     \end{split}
\end{equation}
%
The calculation of the equation of the equinoxes is dependent upon
the time. If the Julian date falls after 2450449.5, then we use
%
\begin{equation}
   EQ_{equinox} = \Delta \Psi \cos{\epsilon} +
   0.00264^{"}\sin{\Omega} + 0.000063^{"}\sin{2\Omega}
\end{equation}
%
If the Julian date falls on or before 2450449.5 we use
\begin{equation}
   EQ_{equinox} = \Delta \Psi \cos{\epsilon}
\end{equation}

%
\begin{equation}
   \theta_{AST} = \theta_{GMST} + EQ_{equinox}
\end{equation}
%
\begin{equation}
   \mathbf{ST} = \mathbf{R}_3(\theta_{AST} )
\end{equation}

\subsection{Polar Motion Calculations} \index{FK5 Reduction!polar motion}

\begin{equation}
     \mathbf{PM} = \mathbf{R}_2 (-x_p) \mathbf{R}_1(-y_p)
\end{equation}


\section{ Deriving $\mathbf{R}_{J_{2k},i}$ and $\dot{\mathbf{R}}_{J_{2k},i}$ for Various
Coordinate Systems }

In GMAT, there are numerous coordinate systems that can be used to
define variables, stopping conditions, spacecraft states, and other
quantities.  Some examples include the Earth centered mean ecliptic
of J2000 system, the Earth-fixed system, the Mars equator system,
and the Earth-Moon rotating system.

In the following subsections, we determine how
$\mathbf{R}_{J_{2k},i}$ and $\dot{\mathbf{R}}_{J_{2k},i}$ are
calculated in GMAT for all of the coordinate systems available in
GMAT.  Let's begin by looking at coordinate systems defined by the
equator of a celestial body.

%-------------------------------------------------------------------------------------
%-------------------------------------------------------------------------------------
\subsection{Equator System}  \label{Sec:Equator} \index{Coordinate systems!body
equator}
%-------------------------------------------------------------------------------------
%-------------------------------------------------------------------------------------

The Equator axis system has the following nominal configuration:
%
\begin{itemize}
\item $x$-axis:  Along the line formed by the intersection of the bodies equator and the ecliptic plane.
%
\item $y$-axis:  Completes the right-handed set.
%
\item $z$-axis:  Normal to the  equatorial plane.
\end{itemize}
%
The Equator system in GMAT is a true equator of date axis system.
The equatorial coordinate system is defined only for celestial
bodies. For a particular body, the equatorial system is defined by
the bodies equatorial plane and its intersection with the ecliptic
plane, at the current epoch.   The Earth and Moon have highly
accurate models for their equatorial systems and and are treated at
the end of this section. For the remaining bodies in the solar
system, the equatorial coordinate system is calculated in GMAT using
data published by the International Astronomical Union
(IAU)\cite{Seidelmann:etal:02}. The IAU publishes data that gives
the spin axis direction and prime meridian location of all the
planets and major moons as a function of time.  For the Earth, GMAT
uses FK5 reduction for the Equator system.  For the Moon, GMAT can
use either the IAU data, or Euler angles provided in the JPL DE405
files.

Let's look more closely at the data provided by the IAU. Figure
\ref{fig:IAUEqDef} contains an illustration of the three variables,
$\alpha_o$, $\delta_o$, and $W$, that are used to define a body's
spin axis and prime meridian location w/r/t MJ2000Eq. $\alpha_o$ and
$\delta_o$ are used to define a body's spin axis direction. $W$ is
the body's sidereal time.  The equations for $\alpha_o$, $\delta_o$,
and $W$ for the nine planets and the Earth's moon are found in
Tables \ref{Table:PlanetsPolesMeridians} and
\ref{Table:LunaPoleMeridian}.
%
\begin{figure*}[htb]
 \centerline{
\begin{picture}(100,260)
\special{psfile = Images/IAUEqDef.ps hoffset= -195 voffset= -200
hscale=80 vscale=80} \makebox(55,470){$(\alpha_o,\delta_o)$}
\makebox(265,460){North pole of planet} \makebox(-520,260){$W$}
\makebox(-500,220){$90^{\circ} - \delta_o$}
\makebox(-680,395){$\Upsilon$}
\makebox(-722,315){$90^{\circ}+\alpha_o$} \makebox(-312,355){Prime
Meridian} \makebox(-342,165){Equator of $\mathcal{F}_{J_{2k}}$}
%
\end{picture}}\vspace{ -1 in} \caption{ IAU Definition of Pole and Meridian Locations for Planets
and Moons} \label{fig:IAUEqDef} \index{Coordinate
systems!Equatorial}
\end{figure*}
%
From inspection of Fig. \ref{fig:IAUEqDef} we see that
%
\begin{equation}
     \mathbf{R}_{J_{2k},i} = \mathbf{R}_3^{T}(90^{\circ} + \alpha_o)
     \mathbf{R}_1^{T}(90^{\circ}- \delta_o) \label{Eq:EquatorR}
\end{equation}
%
$\alpha_o$ and $\delta_o$ vary slowly with time, so we can assume
the derivative of $\mathbf{R}_{Ii}$ for the Equator system is the
zero matrix.
%
\begin{equation}
  \dot{\mathbf{R}}_{J_{2k},i} =   \begin{pmatrix}
     0.0 & 0.0 & 0.0\\
     0.0 & 0.0 & 0.0\\
     0.0 & 0.0 & 0.0
     \end{pmatrix}
\end{equation}

If the user chooses to use the DE405 files to determine the Moon's
orientation, then GMAT gets a set of Euler angles and rates from the
DE405 files.  We then use the following equations to determine
$\mathbf{R}_{J_{2k},i}$ and $\dot{\mathbf{R}}_{J_{2k},i}$.
%
\begin{eqnarray}
   \mathbf{R}_{J_{2k},i} &=&
   \mathbf{R}_3(\theta_1)^T\mathbf{R}_1(\theta_2)^T\\
   %
   \dot{\mathbf{R}}_{J_{2k},i} &=&
   \mathbf{R}_3(\theta_1)^T\dot{\mathbf{R}}_1^T(\theta_2) + \dot{\mathbf{R}}_3^T(\theta_1)\mathbf{R}_1^T(\theta_2)
\end{eqnarray}
%
where
%
\begin{equation}
   \dot{\mathbf{R}}_1(\theta_2) = \begin{pmatrix}
     0.0 & 0.0 & 0.0\\
     0.0 & -\dot{\theta}_2\sin{\theta_2} & \dot{\theta}_2\cos{\theta_2}\\
     0.0 & -\dot{\theta}_2\cos{\theta_2} & -\dot{\theta}_2\sin{\theta_2}
     \end{pmatrix}
\end{equation}
%
and
%
\begin{equation}
   \dot{\mathbf{R}}_3(\theta_1) = \begin{pmatrix}
     -\dot{\theta}_1\sin{\theta_1} & \dot{\theta}_1\cos{\theta_1} & 0.0\\
     -\dot{\theta}_1\cos{\theta_1} & -\dot{\theta}_1\sin{\theta_1} & 0.0 \\
     0.0                           & 0.0                          &
     0.0
     \end{pmatrix}
\end{equation}

Finally, for the Earth, the Equator axis system a true of date
equator system and is calculated using the algorithm described in
Sec. \ref{Sec:TODEq}.
%-------------------------------------------------------------------------------------
%-------------------------------------------------------------------------------------
\subsection{MJ2000 Ecliptic (MJ2000Ec) } \label{Sec:MJ2000Ec} \index{Coordinate systems!mean J2000
ecliptic}
%-------------------------------------------------------------------------------------
%-------------------------------------------------------------------------------------

The MJ2000 Ecliptic axis system is defined as follows:
%
\begin{itemize}
\item $x$-axis:  Along the line formed by the intersection of the Earth's mean
                 equator and the mean ecliptic plane, at the J2000
                 epoch.  The axis points in the direction of the
                 first point of Aries.
%
\item $y$-axis:  Completes the right-handed set.
%
\item $z$-axis:  Normal to the mean ecliptic plane at the J2000 Epoch.
\end{itemize}
%
The matrix to rotate from MJ2000 Ecliptic (MJ2000Ec)  to MJ2000
Equatorial  (MJ2000Eq) is a rotation about the $x$-axis through the
obliquity of the ecliptic at the J2000 epoch which is
23.439291$^\circ$:
%
\begin{equation}
  \mathbf{R} =   \begin{pmatrix}
     1.0 & 0.0 & 0.0\\
     0.0 & 0.91748206 & -0.397777156\\
     0.0 & 0.39777716 & 0.9174820621
     \end{pmatrix} \label{Eq:Ec2Eq}
\end{equation}
%
GMAT uses more significant digits than included here.  The rotation
matrix is constant by definition so its time derivative is
identically the zero matrix.
%
\begin{equation}
  \dot{\mathbf{R}} =   \begin{pmatrix}
     0.0 & 0.0 & 0.0\\
     0.0 & 0.0 & 0.0\\
     0.0 & 0.0 & 0.0
     \end{pmatrix}
\end{equation}

%-------------------------------------------------------------------------------------
%-------------------------------------------------------------------------------------
\subsection{True of Epoch Equator (TOEEq) }
\label{Sec:TOEEq} \index{Coordinate systems!true of epoch equator}
%-------------------------------------------------------------------------------------
%-------------------------------------------------------------------------------------

The True of Epoch Equator axis system is defined as follows:
%
\begin{itemize}
\item $x$-axis:  Along the true equinox at the chosen epoch.
                 The axis points in the direction of the
                 first point of Aries.
%
\item $y$-axis:  Completes the right-handed set.
%
\item $z$-axis:  Normal to the Earth's true equatorial plane at the chosen Epoch.
\end{itemize}
%
The TOEEq axis system is an intermediate system in FK5 reduction.
$\mathbf{R}_{Ii}$ and $ \dot{\mathbf{R}}_{Ii}$ for the TOEEq system
are calculated using the following equations
%
\begin{equation}
      \mathbf{R}_{Ii} = \mathbf{N}^T(t_o)\mathbf{P}^T(t_o)
\end{equation}
%
where $t_o$ is the epoch defined in the coordinate system
description provided by the user in the epoch field.  Hence, $t_o$
is a constant value for the TOEEq system.  For a given $t_o$, the
matrices associated with the TOEEq system only need to be evaluated
once and can be reused later when necessary. $\mathbf{P}(t_o)$ and
$\mathbf{N}(t_o)$ are part of the FK5 reduction algorithm and are
explained in detail in Sec. \ref{Sec:NutationTheory} and
\ref{Sec:PrecessionTheory} . Because $t_o$ is fixed for a TOEEq
system, the derivative of $\mathbf{R}_{Ii}$ is identically equal to
the zero matrix.
%
\begin{equation}
  \dot{\mathbf{R}} =   \begin{pmatrix}
     0.0 & 0.0 & 0.0\\
     0.0 & 0.0 & 0.0\\
     0.0 & 0.0 & 0.0
     \end{pmatrix}
\end{equation}


%-------------------------------------------------------------------------------------
%-------------------------------------------------------------------------------------
\subsection{Mean of Epoch Equator (MOEEq)}
\label{Sec:MOEEq} \index{Coordinate systems!mean of epoch equator}
%-------------------------------------------------------------------------------------
%-------------------------------------------------------------------------------------

The Mean of Epoch Equator axis system is defined as follows:
%
\begin{itemize}
\item $x$-axis:  Along the mean equinox at the chosen epoch.
                 The axis points in the direction of the
                 first point of Aries.
%
\item $y$-axis:  Completes the right-handed set.
%
\item $z$-axis:  Normal to the Earth's mean equatorial plane at the chosen Epoch.
\end{itemize}
%
The MOEEq is an intermediate system in FK5 reduction and
$\mathbf{R}_{Ii}$ and $ \dot{\mathbf{R}}_{Ii}$ for the MOEEq system
can be calculated using the following equations
%
\begin{equation}
      \mathbf{R}_{Ii} = \mathbf{P}^T(t_o)
\end{equation}
%
where $t_o$ is the epoch defined in the coordinate system
description provided by the user in the epoch field.  Hence $t_o$ is
a constant value for the MOEEq system.  For a given $t_o$, the
matrices associated with the MOEEq system only need to be evaluated
once and can be reused later when necessary. $\mathbf{P}(t_o)$ is
described in Sec. \ref{Sec:PrecessionTheory}. Because $t_o$ is fixed
for a MOEEq system, the derivative of $\mathbf{R}_{Ii}$ is the zero
matrix.
%
\begin{equation}
  \dot{\mathbf{R}} =   \begin{pmatrix}
     0.0 & 0.0 & 0.0\\
     0.0 & 0.0 & 0.0\\
     0.0 & 0.0 & 0.0
     \end{pmatrix}
\end{equation}

%-------------------------------------------------------------------------------------
%-------------------------------------------------------------------------------------
\subsection{True of Date Equator (TODEq)}
\label{Sec:TODEq} \index{Coordinate systems!true of date equator}
%-------------------------------------------------------------------------------------
%-------------------------------------------------------------------------------------

$\mathbf{R}_{Ii}$ and $ \dot{\mathbf{R}}_{Ii}$ for the TODEq
system can be calculated using the following equations
Vallado\cite{vallado2} Fig. 3-29).
%
\begin{equation}
      \mathbf{R}_{Ii} = \mathbf{N}^T(t_o)\mathbf{P}^T(t_o)
\end{equation}
%
where $t_o$ is the epoch.  Unlike the TOEEq sytem, for the TODEq
system $t_o$ is a variable and usually comes from the epoch of the
object whose state we wish to convert.  $\mathbf{P}(t_o)$ and
$\mathbf{N}(t_o)$ are part of the FK5 reduction algorithm and can
be found in Vallado pgs. 214 - 219.  Because $t_o$ is not fixed
for a TODEq system the derivative of $\mathbf{R}_{Ii}$ is
non-zero. However, we will assume its derivative is negligibly
small so
   that
\begin{equation}
  \dot{\mathbf{R}} =   \begin{pmatrix}
     0.0 & 0.0 & 0.0\\
     0.0 & 0.0 & 0.0\\
     0.0 & 0.0 & 0.0
     \end{pmatrix}
\end{equation}

%-------------------------------------------------------------------------------------
%-------------------------------------------------------------------------------------
\subsection{Mean of Date Equator (MODEq)}
\label{Sec:MODEq} \index{Coordinate systems!mean of date equator}
%-------------------------------------------------------------------------------------
%-------------------------------------------------------------------------------------

$\mathbf{R}_{Ii}$ and $ \dot{\mathbf{R}}_{Ii}$ for the MODEq
system can be calculated using the following equations
%
\begin{equation}
      \mathbf{R}_{Ii} = \mathbf{P}^T(t_o)
\end{equation}
%
where $t_o$ is the epoch.  Unlike the MOEEq sytem, for the MODEq
system $t_o$ is a variable and usually comes from the epoch of the
object whose state we wish to convert.  $\mathbf{P}(t_o)$ and
$\mathbf{N}(t_o)$ are part of the FK5 reduction algorithm and can
be found in Vallado\cite{vallado2} pgs. 214 - 219.  Because $t_o$
is not fixed for a  MODEq system, the derivative of
$\mathbf{R}_{Ii}$ is non-zero. However, we will assume its
derivative is negligibly small so that
   %
\begin{equation}
  \dot{\mathbf{R}} =   \begin{pmatrix}
     0.0 & 0.0 & 0.0\\
     0.0 & 0.0 & 0.0\\
     0.0 & 0.0 & 0.0
     \end{pmatrix}
\end{equation}


%-------------------------------------------------------------------------------------
%-------------------------------------------------------------------------------------
\subsection{Mean of Date Ecliptic (MODEc)}
\label{Sec:MODEc} \index{Coordinate systems!mean of epoch equator}
%-------------------------------------------------------------------------------------
%-------------------------------------------------------------------------------------

$\mathbf{R}_{Ii}$ and $ \dot{\mathbf{R}}_{Ii}$ for the MODEc
system can be calculated using the following equations
%
\begin{equation}
      \mathbf{R}_{Ii} = \mathbf{P}^T(t_o)\mathbf{R}_1^T(\bar{\epsilon})
\end{equation}
%
where $t_o$ is the epoch.  For the MODEc system $t_o$ is a
variable and usually comes from the epoch of the object whose
state we wish to convert.   $\mathbf{P}(t_o)$ comes from the FK5
reduction algorithm and can be found in Vallado\cite{vallado2}
pgs. 214 - 219.  $\bar{\epsilon}$ is given by
Vallado\cite{vallado2}, Eq. (3-52).  For a more complete
discussion, you can also refer to Seidelmann\cite{seidelmann},
pgs. 114 - 115.   Because $t_o$ is not fixed for a MODEc system,
the derivative of $\mathbf{R}_{Ii}$ is non-zero. However, we will
assume its derivative is negligibly small so that
%
\begin{equation}
  \dot{\mathbf{R}} =   \begin{pmatrix}
     0.0 & 0.0 & 0.0\\
     0.0 & 0.0 & 0.0\\
     0.0 & 0.0 & 0.0
     \end{pmatrix}
\end{equation}

%-------------------------------------------------------------------------------------
\subsection{True of Date Ecliptic (TODEc)}
\label{Sec:TODEc} \index{Coordinate systems!true of date equator}
%-------------------------------------------------------------------------------------
%-------------------------------------------------------------------------------------

$\mathbf{R}_{Ii}$ and $ \dot{\mathbf{R}}_{Ii}$ for the TODEc
system can be calculated using the following equations
%
\begin{equation}
      \mathbf{R}_{Ii} = \mathbf{P}^T(t_o)\mathbf{R}_1^T(\bar{\epsilon})\mathbf{R}_3^T(-\Delta \Psi)
\end{equation}
%
where $t_o$ is the epoch.  Unlike the TOEEc sytem, for the TODEc
system $t_o$ is a variable and usually comes from the epoch of the
object whose state we wish to convert.  $\mathbf{P}(t_o)$is part
of the FK5 reduction algorithm and can be found in Vallado pgs.
214 - 219.   $\bar{\epsilon}$ is given by Vallado\cite{vallado2},
Eq. (3-52).  $\Delta \Psi$ is given by Eq.(3-62) in
Vallado\cite{vallado2}.  For a more complete discussion, you can
also refer to Seidelmann\cite{seidelmann}, pgs. 114 - 115. Because
$t_o$ is not fixed for a MODEq system, the derivative of
$\mathbf{R}_{Ii}$ is non-zero. However, we will assume its
derivative is negligibly small so that
%
\begin{equation}
  \dot{\mathbf{R}} =   \begin{pmatrix}
     0.0 & 0.0 & 0.0\\
     0.0 & 0.0 & 0.0\\
     0.0 & 0.0 & 0.0
     \end{pmatrix}
\end{equation}

%-------------------------------------------------------------------------------------
%-------------------------------------------------------------------------------------
\subsection{Topocentric Horizon}  \label{Sec:TopocentricHorizon} \index{Coordinate systems! Topocentric Horizon}
%-------------------------------------------------------------------------------------
%-------------------------------------------------------------------------------------


The Topocentric coordinate system has its origin at a ground station
with the following axes definitions
\begin{itemize}
\item $x$-axis:  Completes the right handed set and points south in
the local horizon system.
%
\item $y$-axis:  Points due East.
%
\item $z$-axis:  Normal to the local horizon.  The local horizon is
defined by the selection of the \st{HorizonReference} which is
either \st{Sphere} or \st{Ellipsoid}.
\end{itemize}

The calculation of the Topocentric to MJ2000Eq systems is different
for the different ground system state representations.  Regardless of
how the user inputs the state of a ground station, the Topocentric
axes are always defined with respect to the reference shape of the central body.  GMAT uses
the appropriate transformation to convert the user input to the cartesian location
of the ground station expressed in the local body-fixed frame.
This cartesian representation is used to calculate the topocentric axes.

Define the axes of the topocentric coordinate
system, expressed in the body fixed system as $\hat{\mathbf{x}}$,
$\hat{\mathbf{y}}$, and $\hat{\mathbf{z}}$.  Define, $x_F$, $y_F$, and $z_F$ as the location of the station in the local body-fixed coordinate system.
%If the \st{StateType} is \st{Spherical} and the \st{HorizonReference} is \st{Sphere} then calculate the body fixed location of the station using.
%%
%\begin{eqnarray}
%     x_F &=& (R_b + h)\cos{\phi_{sp}}\cos{\lambda} \\
%     y_F &=& (R_b + h)\cos{\phi_{sp}}\sin{\lambda} \\
%     z_F &=& (R_b + h)\sin{\lambda}
%\end{eqnarray}
%%
%where $R_b$ is the body's equatorial radius, $h$ is the height above the reference sphere,  $\phi_{sp}$ is the latitude w/r/t to the reference sphere, and $\lambda$ is the longitude.
%If the \st{StateType} is \st{Cartesian}, or \st{Spherical} with the \st{HorizonReference = Sphere} then calculate the latitude w/r/t to the reference ellipsoid as follows:
%%
%\begin{equation}
%     r_{xy} = \sqrt{ x_F^2 + y_F^2 }
%\end{equation}
%%
Calculate the geocentric latitude to use as an initial guess to find
the geodetic latitude
%
\begin{equation}
     \phi_{gd}  \approx \mbox{atan2}(z_F,  r_{xy}  );
\end{equation}
%
The eccentricity of the body is calculated from the flatting $f$ using
%
\begin{equation}
    e^2 = 2f-f^2
\end{equation}
%

\noindent Set $\delta = 1.0$ to initialize the loop, then,

\noindent While ( $\delta > 10^{-11}$ )
%
\begin{eqnarray}
   \phi' & = & \phi_{gd}\\
   C & = & \frac{R} { \sqrt{1 - e^2\sin^2{\phi}}    }\\
   \phi_{gd} & = & \mbox{atan} \left( \frac{z_F + C
   e^2\sin{\phi}}{r_{xy}}
   \right)\\
   \delta & = & | \phi_{gd} - \phi' |
\end{eqnarray}
%
EndWhile\\


The longitude  of the station location, $\lambda$,  is calculated
using

\begin{equation}
    \lambda = \mbox{atan2}(y_F,x_F)
\end{equation}

%
Finally,
%
\begin{equation}
     \hat{\mathbf{z}} = \left(
     \begin{array}{cc}
          \cos{\phi_{gd}}\cos{\lambda}\\
          \cos{\phi_{gd}}\sin{\lambda}\\
          \sin{\phi_{gd}}
     \end{array}
     \right) \label{Eq:GeodeLatToz}
\end{equation}

\begin{equation}
     \hat{\mathbf{y}} = \hat{\mathbf{k}}
     \times \hat{\mathbf{z}}
\end{equation}
%
where
%
\begin{equation}
     \hat{\mathbf{k}} = [0 \hspace{.05 in} 0 \hspace{.05 in} 1]^T
\end{equation}
%
\begin{equation}
     \hat{\mathbf{x}} = \hat{\mathbf{y}}
     \times \hat{\mathbf{z}}
\end{equation}
%
\begin{equation}
     \mathbf{R}_{FT} = [\hat{\mathbf{x}} \hspace{.05 in} \hat{\mathbf{y}}\hspace{.05 in} \hat{\mathbf{z}}]
\end{equation}

The last step is to determine the rotation matrix from the
topocenric system to the inertial system by using the rotation
matrix from body fixed to inertial, $\mathbf{R}_{IF}$, at the
desired epoch, $t$.  We determine the body fixed rotation matrix
using the algorithm in Sec. 3.1.9.  Once we have evaluated
$\mathbf{R}_{IF}$ and $\dot{\mathbf{R}}_{IF}$ we calculate
$\mathbf{R}_{IT}$ and $\dot{\mathbf{R}}_{IT}$ using
%
\begin{eqnarray}
      \mathbf{R}_{IT} &=& \mathbf{R}_{IF}
      \mathbf{R}_{FT}\\
%
      \dot{\mathbf{R}}_{IT} &=& \dot{\mathbf{R}}_{IF}
      \mathbf{R}_{FT}\label{Eq:TopoRdot}
\end{eqnarray}



%-------------------------------------------------------------------------------------
%-------------------------------------------------------------------------------------
\subsection{Celestial Body Fixed}  \label{Sec:Fixed} \index{Coordinate systems!body fixed}
%-------------------------------------------------------------------------------------
%-------------------------------------------------------------------------------------

The body fixed coordinate system is referenced to the body equator
and the prime meridian of the body.  The body fixed system for
Earth is found  by using FK5 reduction to the ITRF system as
described by Vallado.  The ITRF system is the earth fixed system.

Vallado denotes the four rotation sequences required to transform
from the ITRF to the FK5 system as [PM], the polar motion, [ST],
the sidereal time, [NUT], the nutation, and [PREC], the
precession. GMAT calculates these four rotation matrices as
described in Vallado.  The rotation matrix from ITRF to FK5 can be
written as follows.
%
\begin{equation}
     \mathbf{R}_{Ii}  = \mathbf{P}^T\mathbf{N}^T\mathbf{ST}^T\mathbf{PM}^T
     \label{Eq:RFK5}
\end{equation}
%
GMAT assumes that the only intermediate rotation that has a
significant time derivative is the sidereal time, [ST].  So, we can
write
%
\begin{equation}
     \dot{\mathbf{R}}_{Ii}  = \mathbf{P}^T\mathbf{N}^T\dot{\mathbf{ST}}^T\mathbf{PM}^T \label{Eq:RdotFK5}
\end{equation}
%
where $\dot{\mathbf{ST}}$ is given by
%
\begin{equation}
    \dot{\mathbf{ST}} =   \begin{pmatrix}
     -\omega_e\sin{\theta_{AST}}  & \omega_e\cos{\theta_{AST}} & 0.0\\
     -\omega_e\cos{\theta_{AST}} & -\omega_e\sin{\theta_{AST}} & 0.0\\
     0.0 & 0.0 & 0.0
     \end{pmatrix}
\end{equation}
%
and $\omega_e$ is given by
%
\begin{equation}
    \omega_e = 7.29211514670698e^{-5} \left( 1 - \frac{LOD}{86400}  \right) \label{Eq:EarthAngularVelocity}
\end{equation}
%
Note that the 2nd edition of Vallado\cite{vallado2} has
inconsistencies in Eqs.~(\ref{Eq:RFK5}) and (\ref{Eq:RdotFK5}) and
they are discussed in the errata to the 2nd edition.  We have
modified equations Eqs.~(\ref{Eq:RFK5}) and (\ref{Eq:RdotFK5})
according to the errata.

For bodies other than the earth,   the IAU gives the spin axis
direction as a function of time with respect to the MJ2000Eq
system and rotation of the prime meridian in the MJ2000Eq system.
This data for all of the planets and many moons can be found in
``Report of the IAU/IAG Working Group on Cartographic Coordinates
and Rotational Elements of the Planets and Satellites: 2000"
Seidelmann\cite{Seidelmann:etal:02} \emph{et.al.} Figure 1 in this
document explains the three variables, $\alpha_o$, $\delta_o$, and
$W$,  that are used to define the body spin axis and prime
meridian location w/r/t J2000.  The values of $\alpha_o$,
$\delta_o$, and $W$ for the nine planets and the Earth's moon are
found on pgs. 8 and 9.

Using the notation found in the reference we can write
%
\begin{equation}
     \mathbf{R}_{Ii} = \mathbf{R}_3^{T}(90^{\circ} + \alpha_o)
     \mathbf{R}_1^{T}(90^{\circ}- \delta_o)\mathbf{R}_3^T(W)
\end{equation}
%

For the derivative we assume that
%
\begin{equation}
    \frac{d}{dt}\left( \mathbf{R}_3^{T}(90^{\circ} + \alpha_o) \right)  =  \begin{pmatrix}
     0.0 & 0.0 & 0.0\\
     0.0 & 0.0 & 0.0\\
     0.0 & 0.0 & 0.0
     \end{pmatrix}
\end{equation}
%
and
\begin{equation}
    \frac{d}{dt}\left( \mathbf{R}_1^{T}(90^{\circ}- \delta_o) \right)  =  \begin{pmatrix}
     0.0 & 0.0 & 0.0\\
     0.0 & 0.0 & 0.0\\
     0.0 & 0.0 & 0.0
     \end{pmatrix}
\end{equation}
%
\begin{equation}
    \frac{d}{dt}\left( \mathbf{R}_3^{T}(W) \right)  =  \begin{pmatrix}
     -\dot{W}\mbox{sin}(W) & -\dot{W}\mbox{cos}(W) & 0.0\\
     \dot{W}\mbox{cos}(W) & -\dot{W}\mbox{sin}(W) & 0.0\\
     0.0 & 0.0 & 0.0
     \end{pmatrix}
\end{equation}
%
where $\dot{W}$ is the time derivative of W for the given body.
Note, Seidelmann\cite{Seidelmann:etal:02} does not provide the
values for $\dot{W}$ so we include them in Table
\ref{Table:PlanetsPolesMeridians}.
%
In summary
%
\begin{equation}
      \dot{\mathbf{R}} = \mathbf{R}_3^{T}(90^{\circ} + \alpha_o)
     \mathbf{R}_1^{T}(90^{\circ}- \delta_o)\frac{d}{dt}\mathbf{R}_3^{T}(W)
\end{equation}
%



%-------------------------------------------------------------------------------------
%-------------------------------------------------------------------------------------
\subsection{Body Inertial}  \label{Sec:BodyInertial} \index{Coordinate systems!body inertial}
%-------------------------------------------------------------------------------------
%-------------------------------------------------------------------------------------


The \st{BodyInertial} axis system is an inertial system based on the
equator of the celestial body chosen as the origin of the system.
The origin of a \st{BodyInertial} system must be a celestial body,
and cannot be a spacecraft, libration point etc. The axes are
defined as follows (except for Earth):
%
\begin{itemize}
\item $x$-axis:  Along the line formed by the intersection of the
bodies equator and the $x$-$y$ plane of the FK5 system, at the J2000
epoch.
%
\item $y$-axis:  Completes the right-handed set.
%
\item $z$-axis:  Along the bodies instantaneous spin axis direction at the
J2000 epoch.
\end{itemize}
%

For Earth, the \st{BodyInertial} axis system is the FK5 system.  For
all other bodies, the \st{BodyInertial} axis system is based upon
the bodies equator and spin axis at the J2000 epoch. So,
\st{BodyInertial} is essentially a true-of-epoch system referenced
to the chosen central body.  The body orientation at the J2000 epoch
is calculated from the IAU Data in Seidelmann
\cite{Seidelmann:etal:02} for the Sun, Mercury, Venus, Mars,
Jupiter, Saturn, Uranus, Neptune, and Pluto.  For the Moon, the
orientation at the J2000 epoch comes from the DE405 files.  Because
the \st{BodyInertial} system is an inertial system, the derivative
of the rotation matrix is always zero:
%
\begin{equation}
   \dot{\mathbf{R}}_{Ii}  = \begin{pmatrix}
     0.0 & 0.0 & 0.0\\
     0.0 & 0.0 & 0.0\\
     0.0 & 0.0 & 0.0
     \end{pmatrix}
\end{equation}
%
The rotation matrix, $\mathbf{R}_{Ii}$, is different for each
celestial body.  We begin by calculating the angles $\alpha$ and
$\delta$ used to define the bodies orientation with respect to the
FK5 system:
%
\begin{equation}
    \alpha = \alpha_o(T_{J2000})
\end{equation}
%
\begin{equation}
    \delta = \delta_o(T_{J2000})
\end{equation}
%
Where $T_{J2000} = 2451544.9999999990686$ TDB and the equations for
$\alpha_o$ and $\delta_o$ are given by Seidelmann
\cite{Seidelmann:etal:02} and reproduced in Table
\ref{Table:PlanetsPolesMeridians}. Finally, the rotation matrix is
calculated using
%
\begin{equation}
     \mathbf{R}_{J_{2k},i} = \mathbf{R}_3^{T}(90^{\circ} + \alpha)
     \mathbf{R}_1^{T}(90^{\circ}- \delta) \label{Eq:BodyEquatorR}
\end{equation}
%
The result is a rotation matrix, that is time invariant, for each
celestial body.%  The individual matrices are shown below.
%
%\noindent For the Sun
%%
%\begin{eqnarray}
%   R_{11} & = &  0.9606338208497771 \nonumber \\
%   R_{21} & = &  0.2778176780544363 \nonumber \\
%   R_{31} & = &  0 \nonumber \\
%   R_{12} & = &  -0.2494239059807128 \nonumber \\
%   R_{22} & = &  0.8624542595398803 \nonumber \\
%   R_{32} & = &  0.4404093156676427 \nonumber \\
%   R_{13} & = &  0.1223534934723278 \nonumber \\
%   R_{23} & = &  -0.4230720836476433 \nonumber \\
%   R_{33} & = &  0.8977971010607901
%\end{eqnarray}
%%
%For Mercury
%%
%\begin{eqnarray}
%   R_{11} & = &  0.9815938660446788 \nonumber \\
%   R_{21} & = &  0.1909803187332696 \nonumber \\
%   R_{31} & = &  0 \nonumber \\
%   R_{12} & = &  -0.1677571842642237 \nonumber \\
%   R_{22} & = &  0.8622324234816705 \nonumber \\
%   R_{32} & = &  0.4779254910806334 \nonumber \\
%   R_{13} & = &  0.09127436261733374 \nonumber \\
%   R_{23} & = &  -0.4691287304711406 \nonumber \\
%   R_{33} & = &  0.8784003785150228
%\end{eqnarray}
%%
%For Venus
%%
% \begin{eqnarray}
%   R_{11} & = &  0.9988399975085459 \nonumber \\
%   R_{21} & = &  0.04815245972043356 \nonumber \\
%   R_{31} & = &  0 \nonumber \\
%   R_{12} & = &  -0.04437694044018298 \nonumber \\
%   R_{22} & = &  0.9205233405740161 \nonumber \\
%   R_{32} & = &  0.3881590738545506 \nonumber \\
%   R_{13} & = &  0.01869081416890205 \nonumber \\
%   R_{23} & = &  -0.3877088083617988 \nonumber \\
%   R_{33} & = &  0.9215923900425705
%\end{eqnarray}
%%
%For the Earth
%%
%\begin{eqnarray}
%    R_{11} & = & 1 \nonumber \\
%    R_{12} & = & 0 \nonumber \\
%    R_{13} & = & 0 \nonumber \\
%    R_{21} & = & 0 \nonumber \\
%    R_{22} & = & 1 \nonumber \\
%    R_{23} & = & 0 \nonumber \\
%    R_{31} & = & 0 \nonumber \\
%    R_{32} & = & 0 \nonumber \\
%    R_{33} & = & 1
%\end{eqnarray}
%%
%For the Moon
%%
%\begin{eqnarray}
%    R_{11} & = & 0.998496505205088 \nonumber \\
%    R_{21} & = & -0.0548154092680678 \nonumber \\
%    R_{31} & = & 0 \nonumber \\
%    R_{12} & = & 0.0499357293985327 \nonumber \\
%    R_{22} & = & 0.909610125238044 \nonumber \\
%    R_{32} & = & 0.412451018902689 \nonumber \\
%    R_{13} & = & -0.0226086714041825 \nonumber \\
%    R_{23} & = & -0.411830900942613 \nonumber \\
%    R_{33} & = & 0.91097977859343
%\end{eqnarray}
%%
%For Mars
%%
%\begin{eqnarray}
%   R_{11} & = &  0.6732521982472343 \nonumber \\
%   R_{21} & = &  0.7394129276360177 \nonumber \\
%   R_{31} & = &  0 \nonumber \\
%   R_{12} & = &  -0.5896387605430038 \nonumber \\
%   R_{22} & = &  0.5368794307891334 \nonumber \\
%   R_{32} & = &  0.6033958972853944 \nonumber \\
%   R_{13} & = &  0.4461587269353554 \nonumber \\
%   R_{23} & = &  -0.4062376142607542 \nonumber \\
%   R_{33} & = &  0.7974417791532832
%\end{eqnarray}
%%
%For Jupiter
%%
%\begin{eqnarray}
%   R_{11} & = &  0.9994209020316729 \nonumber \\
%   R_{21} & = &  -0.03402735050216735 \nonumber \\
%   R_{31} & = &  0 \nonumber \\
%   R_{12} & = &  0.0307100286015568 \nonumber \\
%   R_{22} & = &  0.9019874904580493 \nonumber \\
%   R_{32} & = &  0.430668621100356 \nonumber \\
%   R_{13} & = &  -0.01465451212046692 \nonumber \\
%   R_{23} & = &  -0.4304192217768545 \nonumber \\
%   R_{33} & = &  0.9025101322420253
%\end{eqnarray}
%%
%For Saturn
%%
%\begin{eqnarray}
%   R_{11} & = &  -0.6506284356468798 \nonumber \\
%   R_{21} & = &  0.7593962330217962 \nonumber \\
%   R_{31} & = &  0 \nonumber \\
%   R_{12} & = &  -0.7545700815904118 \nonumber \\
%   R_{22} & = &  -0.6464935305479939 \nonumber \\
%   R_{32} & = &  0.1125615694996715 \nonumber \\
%   R_{13} & = &  0.08547883186107168 \nonumber \\
%   R_{23} & = &  0.07323575787752883 \nonumber \\
%   R_{33} & = &  0.9936447519469775
%\end{eqnarray}
%%
%For Uranus
%%
%\begin{eqnarray}
%   R_{11} & = &  0.9755767334083795 \nonumber \\
%   R_{21} & = &  -0.2196589111150187 \nonumber \\
%   R_{31} & = &  0 \nonumber \\
%   R_{12} & = &  -0.05749969269512644 \nonumber \\
%   R_{22} & = &  -0.2553748540714755 \nonumber \\
%   R_{32} & = &  0.9651308042166816 \nonumber \\
%   R_{13} & = &  -0.2119995815377986 \nonumber \\
%   R_{23} & = &  -0.9415591572895125 \nonumber \\
%   R_{33} & = &  -0.2617680858165513
%\end{eqnarray}
%%
%For Neptune
%%
%\begin{eqnarray}
%   R_{11} & = &  0.8717809455009272 \nonumber \\
%   R_{21} & = &  0.4898958900230839 \nonumber \\
%   R_{31} & = &  0 \nonumber \\
%   R_{12} & = &  -0.3337976487639454 \nonumber \\
%   R_{22} & = &  0.5940005535292547 \nonumber \\
%   R_{32} & = &  0.7319443094160927 \nonumber \\
%   R_{13} & = &  0.3585765089087282 \nonumber \\
%   R_{23} & = &  -0.6380951021167846 \nonumber \\
%   R_{33} & = &  0.6813644604126335
%\end{eqnarray}
%%
%For Pluto
%%
%\begin{eqnarray}
%   R_{11} & = &  0.7311155947298647 \nonumber \\
%   R_{21} & = &  0.6822536091093958 \nonumber \\
%   R_{31} & = &  0 \nonumber \\
%   R_{12} & = &  -0.1077863335431382 \nonumber \\
%   R_{22} & = &  0.1155058299435205 \nonumber \\
%   R_{32} & = &  0.9874413955017208 \nonumber \\
%   R_{13} & = &  0.6736854558650673 \nonumber \\
%   R_{23} & = &  -0.7219338031331282 \nonumber \\
%   R_{33} & = &  0.1579857286263988
%\end{eqnarray}
%%

\subsection{Object Referenced}\index{Coordinate systems!object
referenced}

An object referenced system is a coordinate system whose axes are
defined by the motion of one object with respect to another
object. GMAT allows the user to define many different types of
Object Referenced system.
%
\begin{figure*}[htb]
 \centerline{
\begin{picture}(100,290)
\special{psfile = Images/ObjectRefCS.eps hoffset= -180 voffset= -180
hscale=70 vscale=70} \makebox(-60,170){}
\makebox(173,368){$\mathbf{r}$} \makebox(-115,488){$\mathbf{v}$}
 \makebox(-175,478){$\mathbf{n}$}\makebox(-245,215){Location of Primary}
 \makebox(-20,335){Location of Secondary}
  \makebox(-270,155){}
\end{picture}}\vspace{ -1 in} \caption{ Diagram of an Object Referenced Coordinate System} \label{fig:ObjectRefCS}
\end{figure*}
%
In Fig. \ref{fig:ObjectRefCS} we see a diagram that defines the
directions a user can choose from in creating an Object Referenced
coordinate system.  There are six directions.  One is the relative
position, denoted here by $\mathbf{r}$, of the secondary object
with respect to the primary object, expressed in an inertial
frame. The second is the the relative velocity, denoted here by
$\mathbf{v}$, of the secondary object with respect to the primary,
expressed in an inertial frame. The third direction is the vector
normal to the direction of motion which is denoted by $\mathbf{n}$
and is calculated from $\mathbf{n} = \mathbf{r} \times
\mathbf{v}$.  The remaining three directions are the negative of
of the first three.

In GMAT, a user can use the directions described above to define
an Object Referenced coordinate system.  In doing so, the user can
choose two of the available directions, and define which of the
three axes, $x$, $y$, and $z$, they desire the direction to be
aligned with.  Given this information, GMAT automatically
constructs an orthogonal coordinate system.  For example, if user
chooses the $x$-axis to be in the direction of $\mathbf{r}$ and
the $z$-axis to be in the direction of $\mathbf{n}$, the GMAT
completes the right-handed set by setting the $y$-axis in the
direction of $\mathbf{n} \times \mathbf{r}$.  Obviously there are
some choices that not yield an orthogonal system, or that yield a
left handed system.  GMAT does not allow the user to select these
pairs of axes and throws an error message.

In general, given the unit vectors that define the axes system of
the Object Referenced system, but expressed in the inertial frame,
GMAT uses the following equations to determine $\mathbf{R}_{Ii}$
and $ \dot{\mathbf{R}}_{Ii}$.
%
\begin{equation}
     \mathbf{R}_{Ii} = \left[\hspace{.05 in} \hat{\mathbf{x}} \hspace{.2 in} \hat{\mathbf{y}} \hspace{.2 in} \hat{\mathbf{z}} \hspace{.05 in} \right]
\end{equation}
%
\begin{equation}
     \dot{\mathbf{R}}_{Ii} = \left[\hspace{.05 in} \dot{\hat{\mathbf{x}}} \hspace{.2 in}
     \dot{\hat{\mathbf{y}}} \hspace{.2 in} \dot{\hat{\mathbf{z}}} \hspace{.05 in} \right]
\end{equation}
%
%\begin{equation}
%         \mathbf{R}^{Ii} = \begin{pmatrix}
%               \hat{x}_1 & \hat{y}_1  & \hat{z}_1  \\
%               \hat{x}_2 & \hat{y}_2  & \hat{z}_2  \\
%               \hat{x}_3 & \hat{y}_3  & \hat{z}_3  \\
%               \label{Eq:RObjectReferenced}
%     \end{pmatrix}
%\end{equation}
%and
%\begin{equation}
%         \dot{\mathbf{R}}^{Ii} = \begin{pmatrix}
%               \dot{\hat{x}}_1 & \dot{\hat{y}}_1  & \dot{\hat{z}}_1  \\
%               \dot{\hat{x}}_2 & \dot{\hat{y}}_2  & \dot{\hat{z}}_2  \\
%               \dot{\hat{x}}_3 & \dot{\hat{y}}_3  & \dot{\hat{z}}_3
%               \\\label{Eq:RdotObjectReferenced}
%     \end{pmatrix}
%\end{equation}
%%
%where
%%
%\begin{equation}
%     \hat{\mathbf{x}} = \left[ \hat{x}_1 \hspace{.2 in}  \hat{x}_2 \hspace{.2 in}  \hat{x}_3
%     \right]^T \label{Eq:xhatdef}
%\end{equation}
%%
%\begin{equation}
%     \dot{\hat{\mathbf{x}}} = \left[ \dot{\hat{x}}_1 \hspace{.2 in}  \dot{\hat{x}}_2 \hspace{.2 in}  \dot{\hat{x}}_3
%     \right]^T \label{Eq:yhatdef}
%\end{equation}
%and similarly for $\hat{\mathbf{y}}$, $\dot{\hat{\mathbf{y}}}$,
%$\hat{\mathbf{z}}$ and $\dot{\hat{\mathbf{z}}}$.

Recall that the user chooses two axes to  an Object Referenced
system among the following choices:   $\hat{\mathbf{r}}$,
$\hat{\mathbf{v}}$, $\hat{\mathbf{n}}$,  $-\hat{\mathbf{r}}$,
$-\hat{\mathbf{v}}$, and $-\hat{\mathbf{n}}$.  In general, one of
the axes chosen by the user must be either $\hat{\mathbf{n}}$, or
$-\hat{\mathbf{n}}$.


If the user defines the $x$-axis and $y$-axis then GMAT determines
the z axis using
%
\begin{equation}
    \hat{\mathbf{z}} =  \hat{\mathbf{x}}\times\hat{\mathbf{y}}
\end{equation}
%
and
%
\begin{equation}
    \dot{\hat{\mathbf{z}}} =  \dot{\hat{\mathbf{x}}}
    \times\hat{\mathbf{y}}+  \hat{\mathbf{x}} \times
    \dot{\hat{\mathbf{y}}}
\end{equation}
%
If the user defines the $y$-axis and $z$-axis, then GMAT determines
the $x$ axis using
%
\begin{equation}
    \hat{\mathbf{x}} =  \hat{\mathbf{y}}\times\hat{\mathbf{z}}
\end{equation}
%
and
%
\begin{equation}
    \dot{\hat{\mathbf{x}}} =  \dot{\hat{\mathbf{y}}}
    \times\hat{\mathbf{z}}+  \hat{\mathbf{y}} \times
    \dot{\hat{\mathbf{z}}}
\end{equation}
%
And finally, if the user defines the $x$-axis and $z$-axis then GMAT
determines the $y$ axis using
%
\begin{equation}
    \hat{\mathbf{y}} =  \hat{\mathbf{z}}\times\hat{\mathbf{x}}
\end{equation}
%
and
%
\begin{equation}
    \dot{\hat{\mathbf{y}}} =  \dot{\hat{\mathbf{z}}}
    \times\hat{\mathbf{x}}+  \hat{\mathbf{z}} \times
    \dot{\hat{\mathbf{x}}}
\end{equation}
%


Depending on the users choice of axes for an Object Referenced
coordinate system, GMAT will need to calculate $\hat{\mathbf{r}}$,
$\hat{\mathbf{v}}$, $\hat{\mathbf{n}}$, $\dot{\hat{\mathbf{r}}}$,
$\dot{\hat{\mathbf{v}}}$, and $\dot{\hat{\mathbf{n}}}$.  These are
given by:
%
\begin{equation}
\hat{\mathbf{r}} = \frac{\mathbf{r}}{\| \mathbf{r}
     \|} = \frac{\mathbf{r}}{r}
\end{equation}
%
\begin{equation}
     \hat{\mathbf{v}} = \frac{\mathbf{v}}{\| \mathbf{v}
     \|} = \frac{\mathbf{v}}{v}
\end{equation}
%
\begin{equation}
     \hat{\mathbf{n}} = \frac{\mathbf{r}\times \mathbf{v}}{ \| \mathbf{r}\times \mathbf{v}
     \|}
\end{equation}
%
\begin{equation}
    \dot{\hat{\mathbf{r}}} = \frac{\mathbf{v}}{r}  -
     \frac{\hat{\mathbf{r}}}{r}
     \left(\hat{\mathbf{r}} \cdot
     \mathbf{v} \right)
\end{equation}
%
\begin{equation}
    \dot{\hat{\mathbf{v}}} = \frac{\mathbf{a}}{v} -
    \frac{\hat{\mathbf{v}}}{v}\left(\hat{\mathbf{v}}\cdot\mathbf{a}\right)
\end{equation}
%
\begin{equation}
     \dot{\hat{\mathbf{n}}} = \frac{\mathbf{r} \times \mathbf{a}}{n} - \frac{\hat{\mathbf{n}}}{n} \left(  \mathbf{r}\times\mathbf{a} \cdot \hat{\mathbf{n}} \right)
     \label{Eq:VBN_Ndot}
\end{equation}



%---------------------------------------------------------------------------------
%---------------------------------------------------------------------------------
\subsection{Geocentric Solar
Ecliptic (GSE)  }
%---------------------------------------------------------------------------------
%---------------------------------------------------------------------------------

%  Reference for this: http://sscweb.gsfc.nasa.gov/users_guide/Appendix_C.html
The Geocentric Solar Ecliptic system is a time varying axis system
often used to describe and analyze the Earth's magnetic field. The
coordinate system is defined such that
%
\begin{equation}
     \hat{\mathbf{x}} = \frac{\mathbf{r}_{sun}}{\| \mathbf{r}_{sun} \|}
\end{equation}
%
where $\mathbf{r}_{sun}$ is the vector from the Earth to the Sun
in the MJ2000Eq axis system.   The $z$-axis is defined to be the
ecliptic pole. To ensure we have an orthogonal system, we
calculate $\hat{\mathbf{z}}$ using
%
\begin{equation}
     \hat{\mathbf{z}} = \frac{\mathbf{r}_{sun} \times \mathbf{v}_{sun}}{ \|  \mathbf{r}_{sun} \times \mathbf{v}_{sun}   \|}
\end{equation}


% The unit vector pointing in the direction of the
%ecliptic pole, but expressed in the MJ2000Eq system, is simply the
%third column of rotation matrix from MJ2000Ec to MJ2000Eq given in
%Eq.~(\ref{Eq:Ec2Eq}). So,
%%
%\begin{equation}
%  \hat{\mathbf{z}} =   \begin{pmatrix}
%     0.0\\
%     -0.397777155914121383\\
%     0.917482062076895741
%     \end{pmatrix}
%\end{equation}
%
Finally, the $y$-axis completes the right-handed set
%
\begin{equation}
    \hat{\mathbf{y}} = \hat{\mathbf{z}} \times \hat{\mathbf{x}}
\end{equation}
%
We can construct the rotation matrix that goes from the GSE axis
system to the MJ2000Eq axis system as
%
\begin{equation}
     \mathbf{R}_{Ii} = \left[\hspace{.05 in} \hat{\mathbf{x}} \hspace{.2 in} \hat{\mathbf{y}} \hspace{.2 in} \hat{\mathbf{z}} \hspace{.05 in} \right]
\end{equation}
%
We also need to compute the derivative of the rotation matrix.  We
start by computing
%
\begin{equation}
     \frac{d\hat{\mathbf{x}}}{dt} =      \displaystyle\frac{\mathbf{v}_{sun}}{r_{sun}}  -\displaystyle
     \hat{\mathbf{x}}
     \left(\hat{\mathbf{x}}  \cdot
     \frac{\mathbf{v}_{sun}}{r_{sun}} \right)
\end{equation}
%
where $\mathbf{v}_{sun}$ is the velocity of the Sun with respect
to the Earth in the MJ2000Eq system.  We can approximate the
derivative of the $z$ axis using
%
\begin{equation}
     \frac{d\hat{\mathbf{z}}}{dt} \approx      \mathbf{0}
\end{equation}
%
\begin{equation}
     \frac{d\hat{\mathbf{y}}}{dt} =     \hat{\mathbf{z}} \times  \frac{d\hat{\mathbf{x}}}{dt}
\end{equation}
%
\begin{equation}
     \dot{\mathbf{R}}_{Ii} = \left[\hspace{.05 in} \frac{d\hat{\mathbf{x}}}{dt} \hspace{.2 in}
     \frac{d\hat{\mathbf{y}}}{dt} \hspace{.2 in} \frac{d\hat{\mathbf{z}}}{dt} \hspace{.05 in} \right]
\end{equation}


\subsection{ Geocentric Solar
Magnetic (GSM)}


\begin{equation}
     \hat{\mathbf{x}} = \frac{\mathbf{r}_{sun}}{\| \mathbf{r}_{sun} \|}
\end{equation}
%

Let's define the spherical coordinates of the Earth's dipole in
the Earth fixed frame to be $\lambda_d$ and $\phi_d$.  The
location of the dipole actually changes with time.  Also, the
dipole does not actually pass through the center of the Earth.
However, GMAT currently assumes that the dipole direction is
constant, and passes directly through the center of the Earth.  If
this approximation is not sufficient for future studies, the model
will have to be updated.

\begin{equation}
     \lambda_d = 70.1^{\circ}\mbox{  W}
\end{equation}
%
\begin{equation}
     \phi_d = 78.6 ^{\circ} \mbox{  N}
\end{equation}
%(http://nssdc.gsfc.nasa.gov/planetary/factsheet/earthfact.html)
%Latitude/Longitude of dipole N: 78.6 degrees N/70.1 degrees W
%Dipole offset (planet center to dipole center) distance: 0.0725 Re
%Latitude/Longitude of offset vector: 18.3 degrees N/147.8 degrees
%
The dipole vector in the Earth Fixed system is simply
%
\begin{equation}
    \{\mathbf{r}_d\}_F =   \begin{pmatrix}
               \cos{\phi_d}\cos{(-\lambda_d)}\\
                \cos{\phi_d}\sin{(-\lambda_d)}\\
                \sin{\phi_d}
     \end{pmatrix}
\end{equation}


If $R_{IF}$ is the rotation matrix from the Earth Fixed frame to
MJ2000Eq at the current epoch, then we can write the vector that
describes the dipole direction in the inertial frame as
%
\begin{equation}
        \{\mathbf{r}_d\}_I =   R_{IF}\{\mathbf{r}_d\}_F
\end{equation}
%
Then, the $y$-axis is defined as
%
\begin{equation}
     \hat{\mathbf{y}} = \frac{\{\mathbf{r}_d\}_I \times \hat{\mathbf{x}}  }{\|  \{\mathbf{r}_d\}_I \times \hat{\mathbf{x}}   \|}
\end{equation}
%
the $z$-axis is defined as
%
\begin{equation}
     \hat{\mathbf{z}} = \hat{\mathbf{x}} \times \hat{\mathbf{y}}
\end{equation}
%
and
%
\begin{equation}
     \mathbf{R}_{Ii} = \left[\hspace{.05 in} \hat{\mathbf{x}} \hspace{.2 in} \hat{\mathbf{y}} \hspace{.2 in} \hat{\mathbf{z}} \hspace{.05 in} \right]
\end{equation}
%
To calculate the derivative of the rotation matrix, we know that
%
%
\begin{equation}
     \frac{d\hat{\mathbf{x}}}{dt} =      \displaystyle\frac{\mathbf{v}_{sun}}{r_{sun}}  -\displaystyle
     \hat{\mathbf{x}}
     \left(\hat{\mathbf{x}}  \cdot
     \frac{\mathbf{v}_{sun}}{r_{sun}} \right)
\end{equation}
%
Let's define
%
\begin{equation}
      \mathbf{y}  = (R_{IF}\{\mathbf{r}_d\}_F) \times
      \hat{\mathbf{x}}
\end{equation}
%
and
%
\begin{equation}
      y  = \|(R_{IF}\{\mathbf{r}_d\}_F) \times
      \hat{\mathbf{x}}\|
\end{equation}
%
then
%
\begin{equation}
      \frac{d\mathbf{y}}{dt}  = \dot{\mathbf{y}}=\left(\dot{R}_{IF}\{\mathbf{r}_d\}_F\right) \times
      \hat{\mathbf{x}} + \left(R_{IF}\{\mathbf{r}_d\}_F\right) \times
      \frac{\hat{\mathbf{x}}}{dt}
\end{equation}
%
Now we can write
%
\begin{equation}
     \frac{d\hat{\mathbf{y}}}{dt} = \dot{\hat{\mathbf{y}}} =
     \frac{\dot{\mathbf{y}}}{y} - \hat{\mathbf{y}}\left(\hat{\mathbf{y}} \cdot \frac{\dot{\mathbf{y}}}{y}  \right)
\end{equation}
%
Finally,
\begin{equation}
    \dot{\hat{\mathbf{z}}} = \dot{\hat{\mathbf{x}}} \times
    \hat{\mathbf{y}}+ \hat{\mathbf{x}} \times
    \dot{\hat{\mathbf{y}}}
\end{equation}
%
and
%
\begin{equation}
     \dot{\mathbf{R}}_{Ii} = \left[\hspace{.05 in} \frac{d\hat{\mathbf{x}}}{dt} \hspace{.2 in}
     \frac{d\hat{\mathbf{y}}}{dt} \hspace{.2 in} \frac{d\hat{\mathbf{z}}}{dt} \hspace{.05 in} \right]
\end{equation}


\subsection{ Body-Spin Sun Coordinates}

The body-spin Sun coordinate system is a celestial-body-based coordinate system defined as follows:
%
\begin{itemize}
   \item x points from the central body to the sun
   \item y completes the right-handed set
   \item z lies in the plane of the body's spin axis and the x axis
\end{itemize}
%
This system is similar to the GSM system with the following two differences:  (1)  The magnetic field vector is replaced with the body's spin axis, and (2) the system is based on the fixed frame of the central body and not always referenced to the Earth fixed system, and (3) x points in the opposite direction in the two systems.

Define the vector $\mathbf{r}_{sun}$ as the vector from the central body to the sun. Then,
%
\begin{equation}
     \hat{\mathbf{x}} = \frac{\mathbf{r}_{sun}}{\| \mathbf{r}_{sun} \|}
\end{equation}
%
Define the body's spin axis in the body fixed system as
%
\begin{equation}
    \mathbf{r}_s^I =   \begin{pmatrix}
               0\\
                0\\
                1
     \end{pmatrix}
\end{equation}

If $\mathbf{R}^{I/F}$ is the rotation matrix from the central body's fixed frame to
MJ2000Eq at the current epoch, then we can write the vector that
describes the spin axis in the inertial frame as
%
\begin{equation}
     \mathbf{r}_s^I =  \mathbf{R}^{I/F} \mathbf{r}_s^F
\end{equation}
%
Then, the $y$-axis is defined as
%
\begin{equation}
     \hat{\mathbf{y}} = \frac{\mathbf{r}_s^I \times \hat{\mathbf{x}}  }{\|  \mathbf{r}_s^I\times \hat{\mathbf{x}}   \|}
\end{equation}
%
the $z$-axis is defined as
%
\begin{equation}
     \hat{\mathbf{z}} = \hat{\mathbf{x}} \times \hat{\mathbf{y}}
\end{equation}
%
and
%
\begin{equation}
     \mathbf{R}_{Ii} = \left[\hspace{.05 in} \hat{\mathbf{x}} \hspace{.2 in} \hat{\mathbf{y}} \hspace{.2 in} \hat{\mathbf{z}} \hspace{.05 in} \right]
\end{equation}
%
To calculate the derivative of the rotation matrix, we know that
%
%
\begin{equation}
     \frac{d\hat{\mathbf{x}}}{dt} =      \displaystyle\frac{\mathbf{v}_{sun}}{r_{sun}}  -\displaystyle
     \hat{\mathbf{x}}
     \left(\hat{\mathbf{x}}  \cdot
     \frac{\mathbf{v}_{sun}}{r_{sun}} \right)
\end{equation}
%
and
%
\begin{equation}
      \frac{d\mathbf{y}}{dt}  = \dot{\mathbf{y}}=\left(\dot{\mathbf{R}}^{I/F}\mathbf{r}_s^F\right) \times
      \hat{\mathbf{x}} + \mathbf{r}_s^I \times
      \frac{d\hat{\mathbf{x}}}{dt}
\end{equation}
%
Now we can write
%
\begin{equation}
     \frac{d\hat{\mathbf{y}}}{dt} = \dot{\hat{\mathbf{y}}} =
     \frac{\dot{\mathbf{y}}}{y} - \hat{\mathbf{y}}\left(\hat{\mathbf{y}} \cdot \frac{\dot{\mathbf{y}}}{y}  \right)
\end{equation}
%
Finally,
\begin{equation}
    \dot{\hat{\mathbf{z}}} = \dot{\hat{\mathbf{x}}} \times
    \hat{\mathbf{y}}+ \hat{\mathbf{x}} \times
    \dot{\hat{\mathbf{y}}}
\end{equation}
%
and
%
\begin{equation}
     \dot{\mathbf{R}}_{Ii} = \left[\hspace{.05 in} \frac{d\hat{\mathbf{x}}}{dt} \hspace{.2 in}
     \frac{d\hat{\mathbf{y}}}{dt} \hspace{.2 in} \frac{d\hat{\mathbf{z}}}{dt} \hspace{.05 in} \right]
\end{equation}


\section{Appendix 1: Derivatives of ObjectReferenced Unit Vectors}

The derivations of the above quantities are shown below.  We start
by deriving two derivatives with respect to $\mathbf{n}$, where
$\mathbf{n}$ is given by:
\begin{equation}
    \mathbf{n} =
    \mathbf{r}\times\mathbf{v}
\end{equation}
%
We need to determine two derivatives of $\mathbf{n}$. First
%
\begin{equation}
    \frac{d \mathbf{n}}{dt} = \frac{d }{d
    t}\left(\mathbf{r}\times\mathbf{v}\right) =
    \underbrace{\frac{d \mathbf{r}}{d t} \times\mathbf{v}}_0 +  \mathbf{r} \times
    \frac{d\mathbf{v}}{\partial t}
\end{equation}
%
so we know that
%
\begin{equation}
    \boxed{\frac{d \mathbf{n}}{dt} =   \mathbf{r} \times
    \mathbf{a}}
\end{equation}
%
The next useful derivative is
%
\begin{equation}
    \frac{dn}{dt} = \frac{d  \| \mathbf{n} \|}{dt} = \frac{d }{
    d
    t}\left( \mathbf{n}^T \mathbf{n} \right)^{1/2} =
    \frac{\mathbf{n}^T}{n}\frac{d \mathbf{n}}{dt}
\end{equation}
%
So we can write
%
\begin{equation}
    \boxed{\frac{dn}{dt} =
    \frac{\mathbf{n}}{n}\cdot\left(\mathbf{r}\times\mathbf{a}\right)}
\end{equation}
%
The following two derivatives are also useful
\begin{equation}
    \frac{dr}{dt} = \frac{d  \| \mathbf{r} \|}{dt} =
    \frac{d}{dt}(\mathbf{r}^T\mathbf{r})^{1/2} = \frac{\mathbf{v}  \cdot \mathbf{r} }{r}
\end{equation}
%
\begin{equation}
     \boxed{\frac{dr}{dt} = \frac{\mathbf{v}  \cdot \mathbf{r}
     }{r}}
\end{equation}
%
\begin{equation}
     \frac{dv}{dt} = \frac{d \| \mathbf{v} \|}{dt} = \frac{d}{dt}(\mathbf{v}^T\mathbf{v})^{1/2} = \frac{\mathbf{v}  \cdot \mathbf{a} }{v}
\end{equation}
%
so we can write
\begin{equation}
     \boxed{\frac{dv}{dt} = \frac{\mathbf{v}  \cdot \mathbf{a}
     }{v}}
\end{equation}

\begin{equation}
     \boxed{\hat{\mathbf{v}} = \frac{\mathbf{v}}{\| \mathbf{v}
     \|}}
\end{equation}
%
\begin{equation}
\boxed{\hat{\mathbf{r}} = \frac{\mathbf{r}}{\| \mathbf{r}
     \|}}
\end{equation}
%
\begin{equation}
     \boxed{\hat{\mathbf{n}} = \frac{\mathbf{r}\times \mathbf{v}}{ \| \mathbf{r}\times \mathbf{v}
     \|}}
\end{equation}
%
The time derivatives are derived as follows.
%
\begin{equation}
    \dot{\hat{\mathbf{r}}} =
    \frac{\partial \hat{\mathbf{r}}}{ \partial t} = \frac{\partial }{ \partial
    t}\left( \mathbf{r} r^{-1}  \right) = \frac{\mathbf{v}}{r} -
    \frac{\mathbf{r}}{r^2}\left(\frac{\mathbf{r}\cdot\mathbf{v}}{r}\right)
\end{equation}
%
which can be rewritten as
%
\begin{equation}
    \boxed{\frac{d\hat{\mathbf{r}}}{dt} = \frac{\mathbf{v}}{r}  -
     \frac{\hat{\mathbf{r}}}{r}
     \left(\hat{\mathbf{r}} \cdot
     \mathbf{v} \right)}
\end{equation}
%
\begin{equation}
    \dot{\hat{\mathbf{v}}} =
    \frac{\partial \hat{\mathbf{v}}}{ \partial t} = \frac{\partial }{ \partial
    t}\left( \mathbf{v} v^{-1}  \right) = \frac{\mathbf{a}}{v} -
    \frac{\mathbf{v}}{v^2}\left(\frac{\mathbf{v}\cdot\mathbf{a}}{v}\right)
\end{equation}
%
which can be rewritten as
%
\begin{equation}
    \boxed{\dot{\hat{\mathbf{v}}} = \frac{\mathbf{a}}{v} -
    \frac{\hat{\mathbf{v}}}{v}\left(\hat{\mathbf{v}}\cdot\mathbf{a}\right)}
\end{equation}
%
Finally,
%
\begin{equation}
     \dot{\hat{\mathbf{n}}} = \frac{d}{dt}\left( \mathbf{n} n^{-1}
     \right)= \frac{\mathbf{r} \times \mathbf{a}}{n} - \frac{\mathbf{n}}{n^3} \left(  \mathbf{r}\times\mathbf{a} \cdot \mathbf{n} \right)
\end{equation}
%
\begin{equation}
     \boxed{\dot{\hat{\mathbf{n}}} = \frac{\mathbf{r} \times \mathbf{a}}{n} - \frac{\hat{\mathbf{n}}}{n} \left(  \mathbf{r}\times\mathbf{a} \cdot \hat{\mathbf{n}} \right)
     }
\end{equation}

\subsection{Basic Rotation Matrices} \label{sec:BasicRotationMatrices}

\begin{equation}
   \mathbf{R}_1 = \left(\begin{array}{ccc}
      1 & 0 & 0 \\
      0 & \cos{\theta} & \sin{\theta} \\
      0 & -\sin{\theta} & \cos{\theta}
   \end{array}\right)
\end{equation}

\begin{equation}
   \mathbf{R}_2 = \left(\begin{array}{ccc}
      \cos{\theta}  & 0 & -\sin{\theta} \\
      0 & 1 & 0 \\
      \sin{\theta} & 0 & \cos{\theta}
   \end{array} \right)
\end{equation}


\begin{equation}
   \mathbf{R}_3 = \left(\begin{array}{ccc}
       \cos{\theta} & \sin{\theta} & 0 \\
      -\sin{\theta} &  \cos{\theta} & 0 \\
      0 & 0 & 1
   \end{array}\right)
\end{equation}



\begin{table*} \caption{Recommended Values for Pole and Prime Meridian Locations of the Sun and
Planets\cite{Seidelmann:etal:02}} \index{Planets!pole locations}
\index{Planets!prime meridian locations}\centering
\begin{tabular}{p{.5 in} p{4.5 in} }
  \hline\hline
  % after \\: \hline or \cline{col1-col2} \cline{col3-col4} ...
   Name & Values  \\
  \hline
  Sun & $\alpha_o = 286.13^{\circ}$ \hspace{.2 in} (deg)\\
       & $\delta_o = 63.87^{\circ}$ \hspace{.2 in} (deg)\\
       &$W = 84.10^{\circ}+ 14.1844000^{\circ}d$  \hspace{.2 in} (deg)\\
       &$\dot{W} =14.1844000^{\circ}$  \hspace{.2 in} (deg/s)\\
       & \\
 %------------------------------------------------------
  Mercury & $\alpha_o = 281.01 - 0.033T$\\
       & $\delta_o = 61.45 - 0.005T$\\
       & $W = 329.548 + 6.1385025d$\\
       &$\dot{W} =6.1385025$\\
       & \\
 %------------------------------------------------------
   Venus & $\alpha_o = 272.76$\\
       & $\delta_o = 67.16$\\
       & $W = 160.20 - 1.4813688d$\\
       &$\dot{W} = - 1.4813688$\\
       & \\
 %------------------------------------------------------
     Earth& $\alpha_o = 0.00 - 0.641T$\\
       & $\delta_o =  90.00 - 0.557T$ \\
       & $W = 190.147 + 360.9856235d$\\
       &$\dot{W} = 360.9856235$\\
       & \emph{Earth Data is included for completeness only, GMAT uses FK5 reduction for the Earth}\\
       & \\
 %------------------------------------------------------
   Mars & $\alpha_o = 317.68143 - 0.1061T$\\
       & $\delta_o = 52.88650 - 0.0609T$\\
       & $W = 176.630 + 350.89198226d$\\
       &$\dot{W} = 350.89198226$\\
       & \\
 %------------------------------------------------------
    Jupiter & $\alpha_o = 268.05 - 0.009T$\\
       & $\delta_o = 64.49 + 0.003T$\\
       & $W = 284.95 + 870.5366420d$\\
       &$\dot{W} = 870.5366420$\\
       & \\
 %------------------------------------------------------
    Saturn & $\alpha_o = 40.589 - 0.036T$\\
       & $\delta_o = 83.537 - 0.004T$\\
       & $W =  38.90 + 810.7939024d$\\
       &$\dot{W} = 810.7939024$\\
       & \\
 %------------------------------------------------------
    Uranus & $\alpha_o = 257.311$\\
       & $\delta_o = -15.175$\\
       & $W = 203.81-501.1600928d$\\
       &$\dot{W} = -501.1600928$\\
       & \\
 %------------------------------------------------------
     Neptune & $\alpha_o = 299.36 + 0.70\sin{N}$\\
       & $\delta_o = 43.46 - 0.51\cos{N}$\\
       & $W = 253.18 + 536.3128492d - 0.48 \sin{N}$\\
       &$\dot{W} = 536.3128492- 0.48\dot{N} \cos{N}$\\
       &$N = 357.85 + 52.316T$\\
       &$\dot{N} = 6.0551\times 10^{-4}$ \hspace{.2in}(deg/day)\\
       & \\
 %------------------------------------------------------
     Pluto & $\alpha_o = 313.02$\\
       & $\delta_o = 9.09$\\
       & $W = 236.77 - 56.3623195d$\\
       &$\dot{W} = - 56.3623195$\\
 %------------------------------------------------------
  \hline\hline \label{Table:PlanetsPolesMeridians}
\end{tabular}
\end{table*}

\begin{table*} \caption{Recommended Values for Pole and Prime Meridian Locations of Luna \cite{Seidelmann:etal:02}} \index{Luna!pole locations}
\index{Luna!prime meridian locations}\centering
\begin{tabular}{p{.5 in} p{4.5 in} }
  \hline\hline
  % after \\: \hline or \cline{col1-col2} \cline{col3-col4} ...
   Name & Values  \\
  \hline
 %------------------------------------------------------
  Luna & \[
\begin{array}{lllll}
               \alpha_o  = &269.9949 &+ 0.0031T             &- 3.8787\sin{E1} &- 0.1204\sin{E2}  \\
                        & \mbox{}  &+ 0.0700\sin{E3}      &- 0.0172\sin{E4} &+0.0072\sin{E6}\\
                        & \mbox{}  &- 0.0052\sin{E10}      &- 0.0043\sin{E13}
\end{array}
\]
\\
 %------------------------------------------------------
& \[
\begin{array}{lllll}
               \delta_o  = &66.5392 &+ 0.0130T             &+ 1.5419\cos{E1} &+0.0239\cos{E2}  \\
                        & \mbox{  }  &- 0.0278\cos{E3}      &+0.0068\cos{E4} &-0.00292\cos{E6}\\
                        & \mbox{  }  &+0.0009\cos{E7}    &+.0008\cos{E10} &-0.0009\cos{E13}
\end{array}
\]
\\
 %------------------------------------------------------
& \[
\begin{array}{lllll}
               W  =     &38.3213     &+ 13.17635815d     &-1.4\times 10^{-12}d^2 &+3.5610\sin{E1}  \\
                        & \mbox{  }  &+ 0.1208\sin{E2}  &-0.0642\sin{E3}        &+0.0158\sin{E4}\\
                        & \mbox{  }  &+ 0.0252\sin{E5}  &-0.0066\sin{E6}        &-0.0047\sin{E7}\\
                        & \mbox{  }  &- 0.0046\sin{E8}  &+0.0028\sin{E9}        &+0.0052\sin{E10}\\
                        & \mbox{  }  &+ 0.0040\sin{E11}  &+0.0019\sin{E12}        &-0.0044\sin{E13}\\
\end{array}
\]
\\
 %------------------------------------------------------
& \[
\begin{array}{llllll}
               \dot{W}  =     &+ 13.17635815     &-2.8\times 10^{-12}d   &-.18870\cos{E1}  \\
                        &-.01280\cos{E2}  &-.835\cos{E3}        &+.211\cos{E4}\\
                        &+.0248\cos{E5}  &-.17\cos{E6}        &-.061\cos{E7}\\
                        &-.0015\cos{E8}  &+.0049\cos{E9}        &-.00083\cos{E10}\\
                        &+.00001\cos{E11}  &+.00031\cos{E12}        &-.057\cos{E13}\\
\end{array}
\]
\\
% %------------------------------------------------------
where & \[
\begin{array}{lllllll}
                        &E1 = 125.045 - 0.0529921d     &E2 = 250.089 - 0.1059842d \\
                        &E3 = 260.008 + 13.0120009d    &E4 = 176.625 + 13.3407154d \\
                        &E5 = 357.529 + 0.9856003d     &E6 = 311.589 + 26.4057084d   \\
                        &E7 = 134.963 + 13.0649930d    &E8 = 276.617 + 0.3287146d \\
                        &E9 = 34.226 + 1.7484877d      &E10 = 15.134 - 0.1589763d \\
                        &E11 = 119.743 + 0.0036096d    &E12 = 239.961 + 0.1643573d   \\
                        &E13 = 25.053 + 12.9590088d \\
\end{array}
\]
\\
  \hline\hline \label{Table:LunaPoleMeridian}
\end{tabular}
\end{table*}

\clearpage
\begin{figure*}[htb]
\nopagebreak
 \large\centerline{
\begin{picture}(100,430)
\special{psfile= Images/CSFlowChart.eps hoffset= -250 voffset= -300
hscale=90 vscale=90}
%------------------------------------------------------------
\makebox(65,770){$\mathcal{F}_i$, \hspace{.01 in} $\mathcal{F}_f$,
\hspace{.01 in} $\mathbf{r}_i$, \hspace{.01 in} $\mathbf{v}_i$}
%
\put(-75,312){$\mathcal{F}_1.Axes = \mbox{FK5}$}
%
\put(-30,270){\small{True}}
%
\put(35,318){\small{False}}
%
\put(105,326){Calculate}
%
\put(125,310){$\mathbf{R}_{Ii}$}
%
\put(125,294){$\dot{\mathbf{R}}_{Ii}$}
%
\put(90,350){\Large{First Rotation}}
%
%
\put(80,170){$\{ \mathbf{r}_i \}_1 = \mathbf{R}_{Ii}
\mathbf{r}_i$}
%
\put(75,150){ $\{ \mathbf{v}_i \}_1 = \dot{ \mathbf{R} }_{Ii}
\mathbf{r}_i + \mathbf{R}_{Ii} \mathbf{v}_i$}
%
\put(-60,240){$\{ \mathbf{r}_i \}_1 = \mathbf{r}_i$}
%
\put(-65,220){ $\{ \mathbf{v}_i \}_1 = \mathbf{v}_i$}
%
\put(-100,150){ $\mathcal{F}_i.Origin = \mathcal{F}_f.Origin$ }
%
\put(-35,105){ \small{True} }
%
\put(-138,155){ \small{False} }
%
\put(-245,190){ \Large{Translation} }
%
\put(-235,163){ Calculate }
%
\put(-217,149){ $\mathbf{r}_{if}$ }
%
\put(-217,135){ $\mathbf{v}_{if}$ }
%
\put(-260,-3){ $\{\mathbf{r_f}\}_2 = \{\mathbf{r}_i\}_1 +
\mathbf{r}_{if}$ }
%
\put(-261,-24){ $\{\mathbf{v}_f\}_2 = \{\mathbf{v}_i\}_1 +
\mathbf{v}_{if}$ }
%
\put(-76,75){ $\{\mathbf{r}_f\}_2 = \{\mathbf{r}_i\}_1$ }
%
\put(-77,55){ $\{\mathbf{v}_f\}_2 = \{\mathbf{v}_i\}_1$ }
%
\put(-81,-19){ $\mathcal{F}_f.Axes = \mbox{FK5}$ }
%
\put(-70,-93){ $\mathbf{r}_f = \{\mathbf{r}_f\}_2$ }
%
\put(-70,-112){ $\mathbf{v}_f = \{\mathbf{v}_f\}_2$ }
%
\put(-55,-170){$\mathbf{r}_f$ ,  $\mathbf{v}_f$ }
%
\put(-31,-60){\small{True} }
%
\put(30,-12){\small{False} }
%
\put(82,22){\Large{Second Rotation} }
%
\put(106,-2){Calculate}
%
\put(120,-17){$\mathbf{R}_{fI}$}
%
\put(120,-33){$\dot{\mathbf{R}}_{fI}$}
%
\put(72,-159){$\mathbf{r}_f = \mathbf{R}_{fI} \{ \mathbf{r}_f
\}_2$}
%
\put(70,-180){$\mathbf{v}_f = \dot{ \mathbf{R} }_{fI}
 \{\mathbf{r}_f \}_2+ \mathbf{R}_{fI} \{ \mathbf{v}_f \}_2$ }
%
\put(-200,-220){ * See Sec. \ref{Sec:PseudoRotating} for Treatment
of Pseudo-Rotating Coordinate Systems}
%--------------------------------------------------------------------------------------
\end{picture}}\vspace{ 3.0 in} \caption{ Coordinate System Transformation Algorithm }
\index{Coordinate systems!transformation algorithm}
\end{figure*}

