\section{Spacecraft Model}

\subsection{RF Hardware Models}

The RF Hardware models in GMAT include transmitters, receiver, and transponders.  For each type of RF Hardware there are models for signal transmission/reception feasibility, received frequency, and transmitted frequency to name a few.  In the sections below we present these models for each type of RF Hardware.

\subsubsection{Receiver}

The receiver model supports a frequency model and a reception feasibility model.  The frequency model allows the user to specify the frequency capabilities of the receiver using different methods.  The feasibility model determines whether an RF signal sensed at the receiver can feasibly detected given the receivers capability as defined by the frequency model.
  
When the frequency model \st{CenterAndBandwidth} is selected, the feasibility test is as follows.  Define the frequency transmitted by the originating transmitter as $F_t$, the receiver's center frequency as $F_C$, the bandwidth as $\Delta F$, and the range rate between transmitter and receiver as $\dot{\rho}$ (which is positive when the distance $\rho$ is increasing).  The frequency at the receiver is calculated using
%
\begin{equation}
     F_r = F_t(1 - \frac{\dot{\rho}}{c})
\end{equation}
%
where $c$ is the speed of light.  Define the upper limit of the receiver as $F_u = F_c + \Delta F / 2$ and the lower limit as $F_l = F_c - \Delta F / 2$ .  If
%
\begin{equation}
       F_l \leq F_r \leq F_u
\end{equation}
%
then the received frequency is received by the receiver model.  If the above test fails, the receiver does not receive the signal.

\subsubsection{Transmitter}



\subsubsection{Transponder}


\subsubsection{Antenna}

\subsection{Thruster Models}

GMAT supports several thruster models.  The thruster models employ
physics and empirical data provided by the thruster manufacturer to
model thrust and mass flow rate used in orbit and attitude equations
of motion.   The thrust magnitude and $I_{sp}$ are assumed to be
functions of thruster inlet flow conditions including  pressure,
temperature, and for bi-propellant thrusters, the oxidizer to fuel
ratio.

In the following subsections we present models for thrust magnitude
and mass flow rates for several thruster types.  All thrusters have
a location and orientation.  The location is described in the
spacecraft body system.  The orientation can be described with
respect to any coordinate system known to GMAT. Let's define the
rotation matrix from the thruster frame $\mathcal{F}_T$ to Earth's
MJ2000 Equator as $\mathbf{R}_T$. Then, the thrust used in the orbit
equations of motion is
%
\begin{equation}
    \mathbf{F}_T = F_T \mathbf{R}_T \hat{\mathbf{T}}
\end{equation}
%
where $F_T$ is the thrust magnitude and is thruster dependent, and
$\hat{\mathbf{T}}$

Now let's look at how to calculate the thrust magnitude for a
mono-propellant chemical thruster.

\subsubsection{Mono-Propellant Chemical Thruster}

temperature.  The specific form of Eqs.~(\ref{Eq:MonoPropThrust})
and (\ref{Eq:MonoPropIsp}) are determined by fitting test data to
approximate thrust and $I_{sp}$ as function $T_i$ and $P_i$.  The
user can supply this relationship via a scripte
%
\begin{figure}[h!]
\centerline{
\begin{picture}(100,480)
\special{psfile= ./Images/MonoPropThruster.eps hoffset= -20 voffset= 250
hscale=25 vscale=25}
\makebox(100,540){$\dot{m}_e$,$P_e$,$v_e$}
%
\makebox(-105,710){$\dot{m}_c$,$T_c$,$P_c$}
%
\makebox(-105,920){$m_f$,$T_f$,$P_f$}
%
\makebox(-255,880){Catalyst Bed}
\end{picture}}\vskip -3.75 in  \caption{ Mono-Prop Thruster Diagram} \label{fig:MonoPropThruster}
\end{figure}

We assume thruster data is given as a function of thruster inlet
properties (as opposed to thrust chamber properties), and thrust
magnitude and $I_{sp}$ are modelled using
%
\begin{equation}
    F_T = f(P_i,T_i)\label{Eq:MonoPropThrust}
\end{equation}
%
\begin{equation}
    I_{sp} = f(P_i,T_i)\label{Eq:MonoPropIsp}
\end{equation}
%
where $P_i$ and $T_i$ are the thruster inlet pressure and
temperature.  The specific form of Eqs.~(\ref{Eq:MonoPropThrust})
and (\ref{Eq:MonoPropIsp}) are determined by fitting test data to
approximate thrust and $I_{sp}$ as function $T_i$ and $P_i$.  The
user can supply this relationship via a scripted equation or by
providing a function name.  After calculating $F_T$ and $I_{sp}$, we
calculate the mass flow rate using
%
\begin{equation}
   \dot{m}_e = \frac{F_T}{I_{sp}}
\end{equation}

\subsubsection{Bi-Propellant Chemical Thruster}

\begin{figure}[h!]
\centerline{
    \begin{picture}(100,470)
    \special{psfile= ./Images/BiPropThruster.eps hoffset= -20 voffset= 260
    hscale=25 vscale=25}
    \makebox(80,545){$\dot{m}_e$,}\makebox(-47,545){$P_e$,}
    \makebox(-30,545){$v_e$}
    \makebox(-70,730){$\dot{m}_c$,$m_r$,$T_c$,$P_c$}
    \makebox(-30,938){$m_f$,$T_f$,$P_f$}
    \makebox(-140,938){$m_o$,$T_o$,$P_o$}
    \end{picture}}\vskip -3.75 in  \caption{ Bi-Prop Thruster Diagram} \label{fig:BiPropThruster}
\end{figure}

\begin{equation}
    m_r = \frac{\dot{m}_o}{\dot{m}_f}
\end{equation}
%
\begin{equation}
    \dot{m}_c = \dot{m}_o + \dot{m}_f
\end{equation}
%
\begin{equation}
    T'=  \frac{\dot{m}_o T_o + \dot{m}_f T_f}{\dot{m}_o + \dot{m}_f}
\end{equation}
%
\begin{equation}
    P' =  \frac{\dot{m}_o P_o + \dot{m}_f P_f}{\dot{m}_o + \dot{m}_f}
\end{equation}
%
\begin{equation}
   F = f(P_c,T_c,of)
\end{equation}
%
\begin{equation}
   I_{sp} = f(P_c,T_c,of)
\end{equation}
%
\begin{equation}
   \dot{m}_e = \frac{\mathbf{F}_T}{I_{sp}}
\end{equation}

\subsubsection{Thruster Pulse Modelling}

\begin{equation}
T'=\frac{T(t)}{T_{max}} = \left\{\begin{array}{ll}
             \displaystyle\frac{t^2(t - t_{si})^2}{t_{si}^4} &
             \mbox{$t \leq t_{si}$}\\
             1 & \mbox{$t_{si}<t<t_{sf}$}\\
             \displaystyle\frac{(t - 2 t_{sf} + t_{f})^2(t - t_f)^2}{(t_f - t_{sf})^4}
             & \mbox{$t_{sf} \leq t \leq t_f$}\\
             \end{array}\label{eq:quartic}
      \right.
\end{equation}
%
\begin{figure}[ht]
\centerline{
    \begin{picture}(100,385)
    \special{psfile= ./Images/ThrustPulseProfile.eps hoffset= -150 voffset= 60
    hscale=65 vscale=65}
    \end{picture}}\vskip -3.75 in  \caption{ Sample Thrust Pulse Profile } \label{fig:ThrustPulseProfile}
\end{figure}
%
%
The time to the thrust centroid, $t_c$, is calculated using
%
\begin{equation}
     t_c = \displaystyle\frac{\displaystyle\int_0^{t_{f}} t \mbox{ }T'(t) dt}{\displaystyle\int_0^{t_{f}} T'(t)dt}
\end{equation}
%
performing the integral yields
%
\begin{equation}
    t_c = \frac{-4 t_{si}^2  + 4 t_{sf}^2  + 6 t_{sf} t_f + 5 t_f^2 }{-14 t_{si} + 14 t_{sf} + 16 t_f}
\end{equation}

\subsubsection{Thruster Hot Fire Test Data \\ and Thruster Models}

Thruster hot fire test data is used to develop empirical models that
describe thruster performance as a function of inlet conditions such
as fuel pressure and temperature.  In this section we'll discuss how
the empirical models are developed and discuss how the empirical
models are consistent with the physical models.  First we present
the physics model  for a thruster test stand experiment.  Next we
present what is measured during a thrust stand test, and show how
the measurements are used in combination with the physics model to
generate a model of thruster performance over  a given range of
thruster inlet conditions.

In Fig.~\ref{fig:ThrustStand} we see an illustration of a simple
thrust test setup.  The thruster is mounted to a rigid surface.
%
\begin{figure}[h!]
\centerline{
\begin{picture}(100,400)
\special{psfile= ./Images/ThrusterOnStand.eps hoffset= -35 voffset= 220
hscale=25 vscale=25}
\makebox(-120,700){$\dot{m}$}\makebox(-120,675){$T_i$}
\makebox(-125,650){$P_i$}\makebox(225,705){$\dot{m}$}\makebox(-190,680){$c^*=
I_{sp}g$}\makebox(-225,655){$P_e$}\makebox(-225,530){$F_T$}
\end{picture}}\vskip -3.5 in  \caption{ Thrust Stand Illustration} \label{fig:ThrustStand}
\end{figure}
%
The force due to thrust, $F_T$, can be written as
%
\begin{equation}
    F_T = \dot{m}v_e - (P_e - P_a)A_e
\end{equation}
%
where
\begin{tabbing}
    12345678 \= Reynolds number based on length $s$ \kill
    $\dot{m}$    \>  mass flow rate, kg/s \\
    $T_i$        \>  fuel inlet temperature, K$^{\circ}$\\
    $P_i$        \>  fuel inlet pressure, Pa\\
    $c^*$        \>  characteristic velcocity, m/s \\
    $I_sp$       \>  specific impulse, s \\
    $g_o$        \>  9.801 m/s$^2$ \\
    $P_e$        \>  nozzle exit pressure, Pa\\
    $A_e$        \>  nozzle exit area, m$^2$\\
    $F_T$        \>  force due to thrust, N\\
\end{tabbing}
%
we can rewrite this as
\begin{equation}
    F_T = \dot{m}\left(v_e - \frac{(P_e - P_a)A_e}{\dot{m}}\right)  \label{Eq:Fvsmv}
\end{equation}
%
From this equation we define the characteristic velocity, $c^*$,
using
%
\begin{equation}
    c^* = \left(v_e - \frac{(P_e - P_a)A_e}{\dot{m}}\right)
\end{equation}
%
In practice, $v_e$ or $P_e$ are not measured.  We'll assume the
tests are performed in a vacuum so $P_a = 0$.  To understand how we
relate the measurements to physical model, let's define a new
quantity $I_{sp}$, where
%
\begin{equation}
    I_{sp} = \frac{F_T}{\dot{m} g_o}
\end{equation}
%
we can rewrite this as
%
\begin{equation}
    F_T = \dot{m} I_{sp} g_o \label{Eq:IspVer2}
\end{equation}
%
Comparing Eq.s~(\ref{Eq:Fvsmv}) and (\ref{Eq:IspVer2}) we see that
%
\begin{equation}
     c^* = I_{sp} g_o = v_e - \frac{(P_e - P_a)A_e}{\dot{m}}
     \label{Eq:CstarIsp}
\end{equation}
%
Equation (\ref{Eq:CstarIsp}) shows that $I_{sp}$ is a measure of the
effective (characteristic) exhaust velocity.  $I_{sp}$ contains
information on the energy stored in the fuel and how that energy
translates to exit velocity.   When $I_{sp}$ is calculated from
measured thrust data, the $I_{sp}$ contains a correction for exhaust
velocity for force due to pressure $(P_e - P_a)A_e$.

The characteristic velocity, and hence $I_{sp}$, depend on the type
of fuel and the inlet temperature and pressure of the fuel.
Experimental data determines how
%
\begin{equation}
    I_{sp}(T_i,P_i) = \left.\frac{F_T}{\dot{m}g_o}\right|_{(T_i,P_i)}
\end{equation}
%
\begin{tabbing}
    12345678     \= Reynolds number based on length $s$ \kill
    $\dot{m}$    \>  Known      \\
    $T_i$        \>  Known      \\
    $P_i$        \>  Known      \\
    $g_o$        \>  Known      \\
    $F_T$        \>  Measured   \\
    $I_sp$       \>  Calculated \\
\end{tabbing}



\subsection{Tank Models}

Accurately modelling tank pressure changes is essential for accurate
maneuver modelling and reconstruction.  The following sections
discuss three types of tank models:  pressurant tank, regulated fuel
tank, and blowdown fuel tank.  For each tank there are three models:
isothermal, heat transfer, and adiabatic.

The models used in GMAT are based on work by Estey\cite{Estey:83} ,
Hearn\cite{Hearn:01} and Moran\cite{Moran}.  For each tank, we
select a set of state variables that when defined, allow us to
determine all remaining properties of the tank.  For the state
variables, we provide differential equations that describe how the
state variables change with respect to time.  The number of state
variables varies between different tanks, and with the model type
such as isothermal and heat transfer.

For each of the three tanks, we develop a heat transfer model, an
adiabatic model, and an isothermal model.  The heat transfer model
is derived using the laws of conservation of energy and the
conservation of mass. An adiabatic model is provided by setting the
heat transfer rates to zero in the heat transfer model. The
isothermal model for each tank is developed separately. Each ot
these models is useful for certain classes of maneuvers.  Isothermal
models are useful for maneuvers with low mass flow rates, adiabatic
models are useful for short maneuvers with high mass flow rates.
Heat transfer models are useful for long maneuvers with large mass
flow rates.

When developing heat transfer models, we'll assume that specific
internal energy is given by
%
\begin{equation}
    u = c T
\end{equation}
%
specific enthalpy for a liquid is given by
%
\begin{equation}
    h_\ell = c_\ell T_\ell
\end{equation}
%
and specific enthalpy  for a gas is given by
%
\begin{equation}
    h_g = T_g( c_g + R_g)
\end{equation}
%
The notation used in tank model development is shown below.  After
presenting notation, we present the dynamics model for a pressurant
tank.

 \noindent\textit{Nomenclature}\vspace{-.1 in}
\begin{tabbing}
    -----------------------     \= Reynolds number based on length $s$ \kill
    $A_g,A_\ell,A_w$    \> = Heat transfer area     \\
    $c_v, c_g$        \>  = Specific heat at constant volume     \\
    $D$        \>  = Tank diameter     \\
    $d$        \>  = Liquid surface diameter    \\
    $Gr$        \>  = Grashof number  \\
    $h_\ell, h_v$        \>  = Enthalpy     \\
    $m_g,m_\ell,m_w,m_v$    \> = Mass   \\
    $P_g,P_v,P_t$    \> = Pressure   \\
    $R_v,R_g$    \> = Gas constant   \\
    $T_g,T_\ell,T_w,T_v,T_a$    \> = Temperature   \\
    $u_g,u_\ell,u_w,u_v$    \> = Specific internal energy   \\
    $V_g,V_\ell,V_t$    \> = Volume   \\
    $\dot{W}$    \> = Work rate   \\
    $\dot{Q}_g,\dot{Q}_v,\dot{Q}_l,\dot{Q}_w$    \> = Heat transfer rate     \\
    $\nu_l,\nu_g,\nu_v$    \> = Specific volume  \\
\end{tabbing}
%
\noindent\textit{Subscripts}\vspace{-.1 in}
\begin{tabbing}
    -----------------------     \= Reynolds number based on length $s$ \kill
    $a$    \> = Ambient    \\
    $g$    \> = Pressurant gas    \\
    $\ell$    \> = Propellant liquid  \\
    $t$    \> = Total    \\
    $v$    \> = Propellant vapor \\
    $w$    \> = Tank wall      \\
    $e$    \> = Exit-flow      \\
    $i$    \> = In-flow
\end{tabbing}


\subsubsection{Pressurant Tank}

The pressurant tank model is the simplest of the tank models
primarily due to the fact that there is only one substance, the
pressurant gas, contained in the tank.  In this section, we develop
a state space model for pressurant tank dynamics.  We choose the
state variables to be pressurant gas mass and temperature, $m_g$ and
$T_g$ respectively, and tank wall temperature $T_w$.


In Fig.\ref{fig:PressurantTank} we see an illustration of a
pressurant tank.  We divide the tank into two control volumes: the
gas region and the tank wall region.  The only mass flow in the
system occurs where pressurant gas exits the tank.  Heat transfer
occurs between the gas and the wall, and the wall and the ambient
surroundings.
%
\begin{figure}[ht]
\centerline{
    \begin{picture}(110,440)
    \special{psfile= ./Images/PressurantTank.eps hoffset= -120 voffset= 115
    hscale=55 vscale=55}
    \makebox(90,525){$\dot{m}_e$,$h_g$}
    \makebox(-105,790){1.Gas}
    \makebox(-105,765){$m_g$, $P_g$, $T_g$}
    \makebox(-165,680){$\dot{Q}_g$}
    \makebox(-295,760){$\dot{Q}_w$}
    \makebox(-260,865){2.Tank Wall}
    \makebox(-270,840){$m_w$,  $T_w$}
    \end{picture}}\vskip -3.65 in  \caption{ Pressurant Tank Diagram} \label{fig:PressurantTank}
\end{figure}
%

Knowing the volume of the tank and the state variables  $m_g$,
$T_g$, and $T_w$, we calculate pressure from one of the following
two equations of state:
%
\begin{equation}
   P_g = \frac{m_g R_g T_g}{V_g}
\end{equation}
%
or from the Beattie-Bridgeman Eq.
%
\begin{equation}
   P_g = \frac{R_g T_g}{V_g} + \frac{a_g}{V_g^2} + \frac{b_g}{V_g^3}
\end{equation}
%

The state variables $m_g$, $T_g$, and $T_w$ are described by
ordinary differential equations found by applying the first law of
thermodynamics and the conservation of mass.  The 1st Law applied to
the gas control volume yields
%
\begin{equation}
   \frac{d}{dt}\left(m_g u_g \right) =  \dot{Q}_g - \dot{m}_e h_g
  \label{Eq:PressurantGas1stLaw}
\end{equation}
%
The 1st Law applied to the wall control volume  yields
%
\begin{equation}
     \frac{d}{dt}\left( m_w u_w \right) = \dot{Q}_w -  \dot{Q}_g \label{Eq:PressurantWall1stLaw}
\end{equation}
%
and finally from conservation of mass we obtain
%
\begin{equation}
    \dot{m}_g = -\dot{m}_e \label{Eq:PressurantMassCon}
\end{equation}
%
For these equations to be useful for numerical integration, we need
to expand the derivatives, and if necessary, decouple the equations
(as we'll see, for the pressurant tank, the equations are not
coupled).

Expanding the terms in Eq.~(\ref{Eq:PressurantGas1stLaw}) we have
%
\begin{equation}
    \dot{m}_g c_g T_g + m_g c_g \dot{T}_g = \dot{Q}_g - \dot{m}_e T_g \left( c_g + R_g\right)
\end{equation}
%
Similarly, expanding Eq.~(\ref{Eq:PressurantWall1stLaw}) we obtain
%
\begin{equation}
    m_w c_w \dot{T}_w = \dot{Q}_w -  \dot{Q}_g
\end{equation}
%
Solving the system of equations yields the following differential
equations of state for the pressurant tank heat transfer model.
%
\begin{eqnarray}
    \dot{m}_g &=& -\dot{m}_e\\
    %
    \dot{T}_g &=& \frac{1}{m_g c_g} \left( \dot{Q}_g - T_g R_g  \dot{m}_e  \right)\\
    %
    \dot{T}_w &=& \frac{1}{m_w c_w} \left( \dot{Q}_w - \dot{Q}_g  \right)
\end{eqnarray}
%
The adiabatic model is obtained by setting the terms $\dot{Q}_g$ and
$\dot{Q}_w$ to zero in the above equations.  (Note for the adiabatic
model there are only two state variables, $m_g$ and $T_g$, as the
wall temperature $T_w$ is removed from the system of equations.)
Similarly, the isothermal model is obtained by setting $\dot{T}_g$
and $\dot{T}_w$ to zero.  So, for the isothermal model there is only
one state variable $m_g$.

In summary, for the pressurant tank, all models calculate the tank
pressure using
%
\[
   P_g = \frac{m_g R_g T_g}{V_g}
\]
%
 then the specific equations for the heat transfer, adiabatic, and
isothermal models, are as follows

\noindent\textit{Pressurant Tank:  Heat Transfer}

\noindent State Variables:  $m_g$, $T_g$, $T_w$
%
\begin{eqnarray}
    \dot{m}_g &=& -\dot{m}_e \nonumber\\
    %
    \dot{T}_g &=& \frac{1}{m_g c_g} \left( \dot{Q}_g - T_g R_g  \dot{m}_e  \right)\nonumber\\
    %
    \dot{T}_w &=& \frac{1}{m_w c_w} \left( \dot{Q}_w - \dot{Q}_w  \right)\nonumber
\end{eqnarray}
%
\textit{Pressurant Tank:  Adiabtic}

\noindent State Variables:  $m_g$, $T_g$
%
\begin{eqnarray}
    \dot{m}_g &=& -\dot{m}_e \nonumber\\
    %
    \dot{T}_g &=& \dot{m}_e\frac{T_g R_g   }{m_g c_g} \nonumber
\end{eqnarray}
%
\noindent\textit{Pressurant Tank:  Isothermal}

\noindent State Variables:  $m_g$
%
\begin{equation}
    \dot{m}_g = -\dot{m}_e
\end{equation}

Now let's look at a model for a fuel tank operating in blow down
mode.

%\subsubsection{Blow-Down Tank w/o Vapor Pressure}
%
%
%\begin{figure}[ht]
%\centerline{
%    \begin{picture}(100,510)
%    \special{psfile= ./Images/BlowDownTank.eps hoffset= -120 voffset= 170
%    hscale=55 vscale=55}
%    \makebox(80,525){$\dot{m}_e$,$h_l$}
%    \makebox(-80,970){1.Gas}
%    \makebox(-90,940){$m_g, P_g, T_g$}
%    \makebox(-190,905){$\dot{Q}_v$}
%    %\makebox(-140,905){$\dot{m}_v, h_v$}
%    \makebox(10,905){$\dot{Q}_g$}
%    \makebox(-135,820){2.Liquid}
%    \makebox(-145,790){$m_\ell, T_\ell$}
%    \makebox(-235,740){$\dot{Q}_\ell$}
%    \makebox(-335,990){3.Tank Wall}
%    \makebox(-335,970){$m_w, T_w$}
%    \makebox(-277,1020){$\dot{Q}_w$}
%    \end{picture}}\vskip -3.65 in  \caption{ Bi-Prop Thruster Diagram} \label{fig:BlowDownTankWOVap}
%\end{figure}
%
%\textit{Assumptions}
%\begin{itemize}
%    \item Vapor pressure is zero.
%    \item Liquid density is constant.
%    \item Gas mass is constant.
%\end{itemize}
%%
%Assume we are given $m_g$, the tank diameter $D$, and hence know the
%total tank volume $V_t$, and we know the physical constants
%associated with the liquid and gas ($R_g$,$c_g$,$\nu_g$,$c_\ell$,
%$\nu_\ell$). We choose the state variables $m_\ell$, $T_\ell$,
%$T_g$, and $T_w$, all other tank properties can be calculated from
%these state variables using the following equations:
%%
%\begin{eqnarray}
%    V_\ell & = &  \nu_\ell m_\ell\\
%   %
%    V_g & = &  V_t - V_\ell\\
%   %
%   P_g & = &  \frac{m_g R_g T_g}{V_g}\\
%\end{eqnarray}
%%
%
%We require differential equations that describe the time rate of
%change of the state variables $m_\ell$, $T_\ell$, $T_g$, and $T_w$.
%The differential equations are found by applying the 1st law of
%thermodynamics and conservation of mass to the three control volumes
%illustrated in Fig. \ref{fig:BlowDownTankWOVap}. The 1st Law applied
%to the gas control volume yields
%%
%\begin{equation}
%   \frac{d}{dt}\left(m_g u_g \right) = \dot{Q}_v + \dot{Q}_g - P_g \dot{V}_g
%  \label{Eq:Blowdown1stLawWOVap}
%\end{equation}
%%
%The 1st Law applied to the liquid control volume yields
%%
%\begin{equation}
%   \frac{d}{dt}\left( m_\ell u_\ell \right) = \dot{Q}_\ell - \dot{Q}_v +
%   P_g \dot{V}_g   - \dot{m}_e h_\ell \label{Eq:BlowdownLiquid1stLawWOVap}
%\end{equation}
%%
%The 1st Law applied to the wall control volume  yields
%%
%\begin{equation}
%     \frac{d}{dt}\left( m_w u_w \right) = \dot{Q}_w - \dot{Q}_\ell - \dot{Q}_g
%\end{equation}
%%
%and finally from conservation of mass we obtain
%%
%\begin{equation}
%    \dot{m}_\ell = -\dot{m}_e \label{Eq:BlowdownMassConWOVap}
%\end{equation}
%
%
%The equations above give us four ordinary differential equations
%that allow us to solve for the tank states as a function of time.
%For numerical integration, we need to decouple these equations.
%
%Let's continue with Eq.~(\ref{Eq:Blowdown1stLawWOVap}).   Taking the
%derivative assuming $\dot{m}_g = 0$ and noting that $\dot{V}_g = -
%\nu_\ell \dot{m}_\ell $ yields
%%
%\begin{equation}
%     m_g c_g \dot{T}_g = \dot{Q}_v + \dot{Q}_g + P_g\nu_\ell \dot{m}_\ell
%\end{equation}
%%
%Gathering all state terms on the left hand side yields
%%
%\begin{equation}
%       -P_g\nu_\ell \dot{m}_\ell  + m_g c_g \dot{T}_g = \dot{Q}_v + \dot{Q}_g \label{Eq:CV1}
%\end{equation}
%
%
%Continuing with Eq.~(\ref{Eq:BlowdownLiquid1stLawWOVap}), we take
%the derivative and group terms to obtain
%%
%\begin{equation}
%   \left( c_\ell T_\ell  + P_g \nu_\ell\right)\dot{m}_\ell +
%    m_\ell c_\ell \dot{T}_\ell  =  \dot{Q}_\ell - \dot{Q}_v -  c_\ell T_\ell \dot{m}_e\label{Eq:CV2}
%\end{equation}
%%
%Similarly for the wall region, we arrive at
%%
%\begin{equation}
%     m_w c_w \dot{T}_w = \dot{Q}_w - \dot{Q}_\ell - \dot{Q}_g \label{Eq:CV3}
%\end{equation}
%
%Equations \ref{Eq:CV1} -\ref{Eq:CV3} and
%Eq.\ref{Eq:BlowdownMassConWOVap} can be written in matrix form as
%follows.
%%
%\begin{equation}
%   \left(\begin{array}{ccccccc}
%   A_{11} & 0 & A_{13} & 0 \\
%    A_{21} & A_{22} & 0  & 0 \\
%    0 & 0 & 0  & A_{34} \\
%    A_{41} &  & 0  & 0 \\
%   \end{array}\right)
%   %
%   \left(\begin{array}{c}
%    \dot{m}_\ell \\
%    \dot{T}_\ell  \\
%    \dot{T}_g \\
%   \dot{T}_w  \\
%   \end{array}\right) =
%   %
%   \left(\begin{array}{c}
%    b_1\\
%    b_2  \\
%    b_3 \\
%    b_4  \\
%   \end{array}\right)
%\end{equation}
%%
%where
%%
%\begin{eqnarray}
%   A_{11}& = &  - P_g \nu_\ell \\
%   %
%   A_{13}& = &  m_g c_g\\
%   %
%   A_{21} & = & c_\ell T_\ell + P_g \nu_\ell \\
%   %
%   A_{22} & = & m_\ell c_\ell\\
%   %
%   A_{34} & = & m_w c_w\\
%   %
%   A_{41} & = & 1\\
%   %
%   b_1 & = & \dot{Q}_v + \dot{Q}_g \\
%   %
%   b_2 & = & \dot{Q}_\ell - \dot{Q}_v -  c_\ell T_\ell\dot{m}_e\\
%   %
%   b_3 & = & \dot{Q}_w -\dot{Q}_\ell - \dot{Q}_g\\
%   %
%   b_4 & = & -\dot{m}_e\\
%\end{eqnarray}
%%
%
%Solving the system of equations yields
%%
%\begin{eqnarray}
%    \dot{m}_\ell &=& -\dot{m}_e\\
%    %
%    \dot{T}_\ell &=& \frac{1}{m_\ell c_\ell}\left( \dot{Q}_\ell - \dot{Q}_v  + P_g \nu_\ell \dot{m}_e\right)\\
%    %
%    \dot{T}_g &=&  \frac{1}{m_g c_g}\left( \dot{Q}_v + \dot{Q}_g - P_g \nu_\ell \dot{m}_e\right)\\
%    %
%    \dot{T}_w &=&  \frac{1}{m_w c_w}\left(\dot{Q}_w - \dot{Q}_\ell - \dot{Q}_g \right)\\
%\end{eqnarray}

\subsubsection{Blowdown Tank}

The blowdown tank model is significantly more complex than the
pressurant tank model due to the presence of liquid fuel and fuel
vapor contained in the tank ullage.   In this section, we develop a
state space model for a blow down tank.  We choose the state
variables to be the liquid mass and temperature, $m_\ell$ and
$T_\ell$, the gas temperature $T_g$, and tank wall temperature
$T_w$.


In Fig.\ref{fig:BlowDownTank} we see an illustration of a blow down
tank. We divide the tank into three control volumes: the gas region,
the liquid region, and the tank wall region. Mass flow occurs where
the pressurant gas exits the tank and at the boundary between the
liquid and gas in the form of evaporation. Heat transfer occurs
between all three control volumes as well as with the surroundings.
In summary, the physical processes modelled for a blow down tank are
%
\begin{compactenum}
    \item Vapor pressure is a function of liquid temperature.
    \item Liquid density is a function of liquid temperature.
    \item Heat transfer between the liquid and gas.
    \item Heat transfer between the tank wall and gas.
    \item Heat transfer between the tank wall and liquid.
    \item Heat transfer between the surroundings and tank.
    wall.
\end{compactenum}
%
The assumptions made in the tank model are
%
\begin{compactenum}
    \item Pressurant does not dissolve in liquid ($m_g = C$).
    \item Vapor and gas temperatures are equal.
    \item Vapor and gas volumes are equal.
\end{compactenum}
%
\begin{figure}[ht]
\centerline{
    \begin{picture}(100,510)
    \special{psfile= ./Images/BlowDownTankWVap.eps hoffset= -120 voffset= 170
    hscale=55 vscale=55}
    \makebox(80,525){$\dot{m}_e$,$h_l$}
    \makebox(-80,970){1.Gas/Vapor}
    \makebox(-90,940){$m_g, m_v, P_g, P_v, T_g$}
    \makebox(-190,905){$\dot{Q}_v$}
    \makebox(-140,905){$\dot{m}_v, h_{v}$}
    \makebox(10,905){$\dot{Q}_g$}
    \makebox(-135,820){2.Liquid}
    \makebox(-145,790){$m_\ell, T_\ell$}
    \makebox(-235,740){$\dot{Q}_\ell$}
    \makebox(-335,990){3.Tank Wall}
    \makebox(-335,970){$m_w, T_w$}
    \makebox(-265,1030){$\dot{Q}_w$}
    \end{picture}}\vskip -3.65 in  \caption{ Blow Down Tank Diagram} \label{fig:BlowDownTank}
\end{figure}

Assume we are given $m_g$, the tank diameter $D$, and hence know the
total tank volume $V_t$, and we know the physical constants
associated with the liquid and gas ($R_g$, $c_g$, $\nu_g$, $c_\ell$,
$\nu_\ell(T_\ell)$ and $P_v(T_\ell))$. We choose the state variables
$m_\ell$, $T_\ell$, $T_g$, and $T_w$, all other tank properties can
be calculated from these state variables using the following
equations:
%
\begin{eqnarray}
    V_\ell & = &  \nu_\ell(T_\ell)m_\ell \label{Eq:BlowDownVell}\\
   %
    V_g & = &  V_t - V_\ell\\
   %
   P_g & = &  \frac{m_g R_g T_g}{V_g}\\
   %
   P_v & = &  P_v(T_\ell)\\
   %
   m_v & = & \frac{P_v V_g}{R_v T_g}\\
   %
   P_t & = &  P_v + P_g \label{Eq:BlowDownPt}
\end{eqnarray}

%
To determine the state equations governing $m_\ell$, $T_\ell$,
$T_g$, and $T_w$ we apply the 1st law of thermodynamics and the law
of conservation of mass. The 1st Law applied to the gas control
volume is
%
\begin{equation}
   \frac{d}{dt}\left( m_v u_v + m_g u_g \right) = \dot{Q}_v + \dot{Q}_g - P_t \dot{V}_g +
    \dot{m}_v h_{v} \label{Eq:BlowdownGas1stLaw}
\end{equation}
%
The 1st Law applied to the liquid control volume is
%
\begin{equation}
   \frac{d}{dt}\left( m_\ell u_\ell \right) = \dot{Q}_\ell - \dot{Q}_v +
   P_t \dot{V}_g - \dot{m}_v h_{lg} - \dot{m}_e h_\ell \label{Eq:BlowdownLiquid1stLaw}
\end{equation}
%
The 1st Law applied to the wall control volume yields
%
\begin{equation}
     \frac{d}{dt}\left( m_w u_w \right) = \dot{Q}_w - \dot{Q}_\ell - \dot{Q}_g \label{Eq:BlowdownWall1stLaw}
\end{equation}
%
and finally from conservation of mass:
%
\begin{equation}
    \dot{m}_\ell = -\dot{m}_e - \dot{m}_v \label{Eq:BlowdownMassCon}
\end{equation}
%
we also know that
%
\begin{equation}
    \dot{m}_v = \frac{P_v \dot{V}_g}{R_v T_g} - \frac{P_v V_g \dot{T}_g}{R_v T_g^2} \label{Eq:BlowDownGasLaw}
\end{equation}
%
where we assume that
%
\begin{equation}
   \dot{P}_v \approx 0
\end{equation}

Equations (\ref{Eq:BlowdownGas1stLaw}) - (\ref{Eq:BlowDownGasLaw})
are five equations in five unknowns ($m_v$, $m_\ell$, $T_\ell$,
$T_g$, and $T_w$). Our approach is to use
Eq.~(\ref{Eq:BlowdownMassCon}) to eliminate $\dot{m}_v$ terms.  The
result is a system of four equations in four unknowns using
Eqs.~(\ref{Eq:BlowdownGas1stLaw}), (\ref{Eq:BlowdownLiquid1stLaw}),
(\ref{Eq:BlowdownWall1stLaw}), and (\ref{Eq:BlowDownGasLaw}).  The
result we seek is four decoupled ordinary differential equations for
$m_\ell$, $T_\ell$, $T_g$, and $T_w$.


Let's continue with Eq.~(\ref{Eq:BlowdownGas1stLaw}).  We need to
rewrite the equation in terms of $\dot{m}_\ell$ and $\dot{T}_g$
($\dot{T_w}$ and $\dot{T}_\ell$ don't appear explicitly).  Expanding
the derivatives assuming $\dot{m}_g = 0$ yields
%
\begin{equation}
   \dot{m}_v c_v T_g + m_v c_v \dot{T}_g +  m_g c_g \dot{T}_g = \dot{Q}_v + \dot{Q}_g - P_t \dot{V}_g + \dot{m}_v h_{v}
\end{equation}
%
Now, substituting $\dot{m}_v = -\dot{m}_\ell - \dot{m}_e$ and noting
that $\dot{V}_g = -\nu_l \dot{m}_\ell$ if we assume
%
 \[\dot{\nu}_\ell = \frac{d \nu_\ell}{dT_\ell }\dot{T}_\ell
\approx 0
\]
%
we arrive at
%
\begin{equation}
   \begin{split}
      \left( T_g R_v - P_t \nu_\ell  \right) \dot{m}_\ell + \left( m_v c_v + m_g c_g\right)\dot{T}_g = \\
      \dot{Q}_v + \dot{Q}_g - \dot{m}_e T_g R_v   \label{Eq:BlowdownWVaporEq1}
   \end{split}
\end{equation}
%

Now continuing with Eq.~(\ref{Eq:BlowdownLiquid1stLaw}) expanding
the derivatives and making similar substitutions as we made
previously we obtain
%
\begin{equation}
\begin{split}
    \dot{m}_\ell c_\ell T_\ell + m_\ell c_\ell \dot{T}_\ell = \dot{Q}_\ell - \dot{Q}_v +
    P_t(-\nu_\ell \dot{m}_\ell) - \\ (-\dot{m}_\ell - \dot{m}_e)h_{v} - \dot{m}_e c_\ell T_\ell
\end{split}
\end{equation}
%
Grouping terms we obtain
%
\begin{equation}
\begin{split}
    ( c_\ell T_\ell  + P_t v_l - h_{v})\dot{m}_\ell + ( m_\ell c_\ell )\dot{T}_\ell = \\
    \dot{Q}_\ell - \dot{Q}_v + \dot{m}_e (h_v - c_\ell T_\ell) \label{Eq:BlowdownWVaporEq2}
\end{split}
\end{equation}

For the wall region, described by Eq.~(\ref{Eq:BlowdownWall1stLaw}),
we arrive at
\begin{equation}
     (m_w c_w) \dot{T}_w = \dot{Q}_w - \dot{Q}_\ell - \dot{Q}_g \label{Eq:BlowdownWVaporEq3}
\end{equation}

Finally, by eliminating $\dot{m}_v$ in the Gas Law shown in
Eq.~(\ref{Eq:BlowDownGasLaw}) we obtain
%
\begin{equation}
    -\dot{m}_\ell - \dot{m}_e =   \frac{P_v (-\nu_\ell \dot{m}_\ell)}{R_v T_g} - \frac{P_v V_g \dot{T}_g}{R_v T_g^2}
\end{equation}
%
Grouping terms yields the result
\begin{equation}
    \left(  1 - \frac{P_v\nu_\ell }{R_v T_g}  \right) \dot{m}_\ell - \frac{P_v V_g }{R_v T_g^2}\dot{T}_g  =  - \dot{m}_e \label{Eq:BlowdownWVaporEq4}
\end{equation}

Equations (\ref{Eq:BlowdownWVaporEq1}),
(\ref{Eq:BlowdownWVaporEq2}), (\ref{Eq:BlowdownWVaporEq3}), and
(\ref{Eq:BlowdownWVaporEq4}) are four coupled ordinary differential
equations that can be decoupled by casting them in matrix form as
follows:
%
\begin{equation}
   \left(\begin{array}{ccccccc}
   A_{11} & 0 & A_{13}  & 0 \\
    A_{21} & A_{22} & 0  & 0 \\
    0 & 0 & 0  & A_{34} \\
    A_{41} &  & A_{43}  & 0 \\
   \end{array}\right)
   %
   \left(\begin{array}{c}
    \dot{m}_\ell \\
    \dot{T}_\ell  \\
    \dot{T}_g \\
   \dot{T}_w  \\
   \end{array}\right) =
   %
   \left(\begin{array}{c}
    b_1\\
    b_2  \\
    b_3 \\
    b_4  \\
   \end{array}\right)
\end{equation}
%
where
%
\begin{eqnarray}
   A_{11}& = & T_g R_v - P_t \nu_\ell \label{Eq:BlowDownA11}\\
   %
   A_{13}& = & m_v c_v + m_g c_g\\
   %
   A_{21} & = & c_\ell T_\ell + P_t v_l - h_{v}\\
   %
   A_{22} & = & m_\ell c_\ell\\
   %
   A_{34} & = & m_w c_w\\
   %
   A_{41} & = & 1  - \nu_\ell/\nu_v\\
   %
   A_{43} & = & - m_v/T_g\\
   %
   b_1 & = & \dot{Q}_v + \dot{Q}_g - \dot{m}_e T_g R_v \label{Eq:BlowDownb1}\\
   %
   b_2 & = & \dot{Q}_\ell - \dot{Q}_v + \dot{m}_e (h_v - c_\ell T_\ell) \\
   %
   b_3 & = & \dot{Q}_w -\dot{Q}_\ell - \dot{Q}_g\\
   %
   b_4 & = & -\dot{m}_e \label{Eq:BlowDownb4}
\end{eqnarray}
%


The solution to the equations is
%
\begin{eqnarray}
  \dot{m}_\ell &=& \frac{A_{43} b_{1}-A_{13} b_{4}}{A_{11} A_{43}-A_{41} A_{13}} \label{Eq:BlowDownmdotDiffEq}\\
  %
  %\dot{T}_\ell &=& \frac{-A_{21}A_{43}b_1+ (A_{11}A_{43}-A_{41}A_{13})b_2+A_{21}A_{13}b_4}{A_{22}\left( A_{11}A_{43}-A_{41}A_{13}\right)}\\
  \dot{T}_\ell &=& \frac{1}{A_{22}}\left( b_2 -  A_{21} \frac{A_{43} b_{1}-A_{13} b_{4}}{A_{11} A_{43}-A_{41} A_{13}}\right)\\
  %
  \dot{T}_g &=& \frac{A_{11} b_{4} - A_{41} b_{1}}{A_{11} A_{43}-A_{41} A_{13}}\\
  %
  \dot{T}_w &=& \frac{b_3}{A_{34}} \label{Eq:BlowDownTwdotDiffEq}
\end{eqnarray}
%
For the adiabatic model we set all heat transfer rates, $\dot{Q}$,
to zero in Eqs.~(\ref{Eq:BlowDownb1})-(\ref{Eq:BlowDownb4}) and so
there are only two state variables as $\dot{T}_w = 0$ and so $T_w =$
constant.

Now let's develop equations for an isothermal model of a blow down
tank.  In the isothermal model, we assume $T_\ell = T_g = T_w = T$.
The only state variable that requires a differential equation is
$m_\ell$.  Because $T_g$, $T_\ell$, and hence, $P_v$ are constant,
we know that
%
\begin{equation}
    \dot{m}_v = \frac{P_v \dot{V}_g}{R_v T_g}
\end{equation}
%
Subtituting this result into Eq.(\ref{Eq:BlowdownMassCon}) and
solving for $\dot{m}_\ell$ we obtain.
%
\begin{equation}
    \dot{m}_\ell = -\frac{\dot{m}_e}{\left( 1 - \displaystyle\frac{P_v \nu_\ell}{R_v T}\right)}
\end{equation}


In summary for the heat transfer model for a blow down tank, we
choose $m_\ell$, $T_\ell$, $T_g$, and $T_w$ are state variables.
Eqs.~(\ref{Eq:BlowDownVell})-(\ref{Eq:BlowDownPt}) are used to
calculate the remaining tank properties, and
Eqs.~(\ref{Eq:BlowDownmdotDiffEq})-(\ref{Eq:BlowDownTwdotDiffEq})
are used to model the tank states as functions of time.

For the all three models, heat transfer, adiabatic, and isothermal,
knowing the state variables $m_\ell$, $T_\ell$, $T_g$, and $T_w$ we
compute the remaining tank properties using
%
\begin{eqnarray}
    V_\ell & = &  \nu_\ell(T_\ell)m_\ell  \nonumber \\
   %
    V_g & = &  V_t - V_\ell\nonumber \\
   %
   P_g & = &  \frac{m_g R_g T_g}{V_g}\nonumber \\
   %
   P_v & = &  P_v(T_\ell)\nonumber \\
   %
   m_v & = & \frac{P_v V_g}{R_v T_g}\nonumber \\
   %
   P_t & = &  P_v + P_g \nonumber
\end{eqnarray}

%
The models differ in the number of state variables and in the state
rate equations.  A summary is presented below.

\noindent\textit{Blow Down Tank: Heat Transfer}

\noindent State Variables:  $m_\ell$, $T_\ell$, $T_g$, $T_w$
%
\begin{eqnarray}
  \dot{m}_\ell &=& \frac{A_{43} b_{1}-A_{13} b_{4}}{A_{11} A_{43}-A_{41} A_{13}}\nonumber \\
  %
  %\dot{T}_\ell &=& \frac{-A_{21}A_{43}b_1+ (A_{11}A_{43}-A_{41}A_{13})b_2+A_{21}A_{13}b_4}{A_{22}\left( A_{11}A_{43}-A_{41}A_{13}\right)}\\
  \dot{T}_\ell &=& \frac{1}{A_{22}}\left( b_2 -  A_{21} \frac{A_{43} b_{1}-A_{13} b_{4}}{A_{11} A_{43}-A_{41}
  A_{13}}\right) \nonumber \\
  %
  \dot{T}_g &=& \frac{A_{11} b_{4} - A_{41} b_{1}}{A_{11} A_{43}-A_{41}
  A_{13}} \nonumber \\
  %
  \dot{T}_w &=& \frac{b_3}{A_{34}} \nonumber
\end{eqnarray}
%
where $A_{ij}$ and $b_i$ are given by
Eqs.~(\ref{Eq:BlowDownA11})-(\ref{Eq:BlowDownb4}).

\noindent\textit{Blow Down Tank:  Adiabtic}

\noindent State Variables:  $m_\ell$, $T_\ell$, $T_g$

%
\begin{eqnarray}
  \dot{m}_\ell &=& \frac{A_{43} b_{1}-A_{13} b_{4}}{A_{11} A_{43}-A_{41} A_{13}}\nonumber \\
  %
  %\dot{T}_\ell &=& \frac{-A_{21}A_{43}b_1+ (A_{11}A_{43}-A_{41}A_{13})b_2+A_{21}A_{13}b_4}{A_{22}\left( A_{11}A_{43}-A_{41}A_{13}\right)}\\
  \dot{T}_\ell &=& \frac{1}{A_{22}}\left( b_2 -  A_{21} \frac{A_{43} b_{1}-A_{13} b_{4}}{A_{11} A_{43}-A_{41}
  A_{13}}\right) \nonumber \\
  %
  \dot{T}_g &=& \frac{A_{11} b_{4} - A_{41} b_{1}}{A_{11} A_{43}-A_{41}
  A_{13}} \nonumber
\end{eqnarray}
%
where $A_{ij}$ and $b_i$ are given by
Eqs.~(\ref{Eq:BlowDownA11})-(\ref{Eq:BlowDownb4}).  Note that all
heat flow rates, $\dot{Q}$, are set to zero.

\noindent\textit{Blow Down Tank:  Isothermal}

\noindent State Variables:  $m_\ell$
%
\begin{equation}
    \dot{m}_\ell = -\frac{\dot{m}_e}{\left( 1 - \displaystyle\frac{P_v \nu_\ell}{R_v
    T}\right)} \nonumber
\end{equation}

\subsubsection{Pressure Regulated Tank}

The pressure regulated fuel tank model is the most complex tank
model supported in GMAT.  The model complexity is due to the
presence of liquid fuel and fuel vapor contained in the tank ullage,
and due to mass and energy transfer from the pressurant tank to the
ullage of the regulated fuel tank. In this section, we develop a
state space model for a pressure regulated tank. Note, to model a
pressure regulated tank using a heat transfer or adiabitic model, we
must simultaneously solve the equations associated with the
pressurant tank.  For the regulated tank model, we choose the state
variables to be the liquid mass and temperature, $m_\ell$ and
$T_\ell$, the gas temperature $T_g$,  tank wall temperature $T_w$,
and the incoming pressurant gas mass $m_i$.


In Fig.\ref{fig:PressureRegulatedTank} we see an illustration of a
pressure regulated tank. Like the blow down tank model, we divide
the tank into three control volumes: the gas region, the liquid
region, and the tank wall region. Mass flow occurs where the
pressurant gas exits the tank, at the boundary between the liquid
and gas in the form of evaporation, and from the pressurant tank to
the ullage of the regulated tank.  Heat transfer occurs between all
three control volumes as well as with the surroundings. Hence, the
physical processes modelled for a blow down tank are the same as
those listed for the blow down tank, with the added process of mass
flow from the pressurant tank.
%
\begin{figure}[ht]
\centerline{
    \begin{picture}(100,510)
    \special{psfile= ./Images/PressureRegulatedTank.eps hoffset= -120 voffset= 170
    hscale=55 vscale=55}
    \makebox(80,525){$\dot{m}_e$,$h_l$}
    \makebox(-80,970){1.Gas/Vapor}
    \makebox(-90,940){$m_g, m_v, P_g, P_v, T_g$}
    \makebox(-190,905){$\dot{Q}_v$}
    \makebox(-140,905){$\dot{m}_v, h_{lg}$}
    \makebox(10,905){$\dot{Q}_g$}
    \makebox(-135,820){2.Liquid}
    \makebox(-145,790){$m_\ell, T_\ell$}
    \makebox(-235,740){$\dot{Q}_\ell$}
    \makebox(-335,990){3.Tank Wall}
    \makebox(-335,970){$m_w, T_w$}
    \makebox(-265,1030){$\dot{Q}_w$}
    \makebox(-045,1030){$\dot{m}_i$, $h_i$}
    \end{picture}}\vskip -3.65 in  \caption{ Pressure Regulated Tank Diagram} \label{fig:PressureRegulatedTank}
\end{figure}

The derivation of the state equations for a pressure regulated tank
follows naturally from the derivation of the blow down tank.  The
only control volume that differs between the two models is the
gas/vapor control volume.  Applying the 1st Law of thermodynamics to
the gas/vapor control volume of the pressure regulated tank gives us
%
\begin{equation}
   \frac{d}{dt}\left( m_v u_v + m_g u_g \right) = \dot{Q}_v + \dot{Q}_g - P_t \dot{V}_g +
    \dot{m}_v h_v + \dot{m}_p h_p\label{Eq:RegulatedGas1stLaw}
\end{equation}
%
Taking the time derivative of the gas law for the gas contained in
the tank ullage yields
%
\begin{equation}
   \dot{m}_g = \frac{P_g \dot{V}_g}{R_g
   T} - \frac{P_g V_g \dot{T}_g}{R_g T_g^2} \label{Eq:RegulatedGasLawDeriv}
\end{equation}
%

Equations (\ref{Eq:RegulatedGas1stLaw}) and
(\ref{Eq:RegulatedGasLawDeriv}), together with equations
(\ref{Eq:BlowdownWVaporEq2}), (\ref{Eq:BlowdownWVaporEq3}), and
(\ref{Eq:BlowdownWVaporEq4}) are a system of 5 equations in 5
unknowns which can be decoupled using simple linear algebra.
However, first we must expand Eqs.~(\ref{Eq:RegulatedGas1stLaw}) and
(\ref{Eq:RegulatedGasLawDeriv}) and write them in terms of the state
rate derivatives.  Expanding Eq.~(\ref{Eq:RegulatedGas1stLaw}) we
arrive at
\begin{equation}
   \begin{split}
   \left( R_v T_g - P_t \nu_\ell \right)\dot{m}_\ell + \left( m_v c_v + m_g c_g \right)
   \dot{T}_g + \\ \left(c_g Tg - h_p \right)\dot{m}_g = \dot{Q}_v +
   \dot{Q}_g - \dot{m}_e R_v T_g
   \end{split}
\end{equation}
%
Similarly, for Eq.~(\ref{Eq:RegulatedGasLawDeriv}) we obtain
%
\begin{equation}
    \frac{\nu_\ell}{\nu_g}\dot{m}_\ell + \frac{m_g}{T_g}\dot{T}_g + \dot{m}_g = 0
\end{equation}
%

To integrate the state equations we must decouple the equations and
this is easily done by casting the equations in matrix form and
solving the system of equations.  We can write the equations is
state space for as follows
%
\begin{equation}
   \left(\begin{array}{ccccccc}
   A_{11} & 0 & A_{13}  & 0 & A_{15}\\
    A_{21} & A_{22} & 0  & 0 & 0 \\
    0 & 0 & 0  & A_{34} & 0\\
    A_{41} & 0 & A_{43}  & 0 & 0\\
    A_{51} & 0 & A_{43}  & 0 &  A_{55}\\
   \end{array}\right)
   %
   \left(\begin{array}{c}
    \dot{m}_\ell \\
    \dot{T}_\ell  \\
    \dot{T}_g \\
   \dot{T}_w  \\
   \dot{m}_g
   \end{array}\right) =
   %
   \left(\begin{array}{c}
    b_1\\
    b_2  \\
    b_3 \\
    b_4  \\
    0
   \end{array}\right)
\end{equation}
%
where the coefficients $A_{ij}$ and $b_i$ are given by
%
\begin{eqnarray}
   A_{11}& = & T_g R_v - P_t \nu_\ell \label{Eq:RegulatedA11}\\
   %
   A_{13}& = & m_v c_v + m_g c_g\\
   %
   A_{15} & = & c_g T_g - h_p\\
   %
   A_{21} & = & c_\ell T_\ell + P_t v_l - h_{lg}\\
   %
   A_{22} & = & m_\ell c_\ell\\
   %
   A_{34} & = & m_w c_w\\
   %
   A_{41} & = & 1  - \nu_\ell/\nu_v\\
   %
   A_{43} & = & - m_v/T_g\\
   %
   A_{51} & = & \nu_\ell/\nu_g\\
   %
   A_{53} & = & m_g/T_g \\
   %
   A_{55} & = & 1 \\
   %
   b_1 & = & \dot{Q}_v + \dot{Q}_g - \dot{m}_e R_v T_g \label{Eq:Regulatedb1}\\
   %
   b_2 & = & \dot{Q}_\ell - \dot{Q}_v + \dot{m}_e (h_{lg} - c_\ell T_\ell)\\
   %
   b_3 & = & \dot{Q}_w -\dot{Q}_\ell - \dot{Q}_g\\
   %
   b_4 & = & -\dot{m}_e \label{Eq:Regulatedb4}
\end{eqnarray}

\begin{eqnarray}
   \dot{m}_\ell &=& \frac{A_{55} A_{43} b_1 - b_4 A_{13} A_{55} + b_4 A_{15} A_{53}}{D}\\
   \dot{T}_\ell &=& \frac{b_2 - A_{21}\dot{m}_\ell}{A_{22}}\\
   %
   \dot{T}_g  &=& \frac{-A_{41} A_{55} b_1 + b_4 A_{11} A_{55} - b_4 A_{51}
    A_{15}}{D}\\
   %
   \dot{T}_w &=& \frac{b_3}{A_{34}}\\
   %
   \dot{m}_g &=& \frac{-b_1 A_{51} A_{43} + b_1 A_{41} A_{53} - b_4 A_{11} A_{53} + b_4 A_{51} A_{13}}{D}
\end{eqnarray}
%
where
%
\begin{equation}
      D = A_{55} A_{43} A_{11} - A_{43} A_{51} A_{15} + A_{41} A_{15} A_{53} - A_{41} A_{13} A_{55}
\end{equation}

For the adiabatic model we set all heat transfer rates, $\dot{Q}$,
to zero in  Eqs.~(\ref{Eq:Regulatedb1})-(\ref{Eq:Regulatedb4}). For
the adiabatic model there is only four state variables as $\dot{T}_w
= 0$ and so $T_w =$ constant.

Now let's develop equations for an isothermal model of a pressure
regulated tank. In the isothermal model, we assume $T_\ell = T_g =
T_w = T$. The only state variables that require differential
equations are $m_\ell$ and $m_g$. Because $T_g =$ constant, and
hence, $P_v =$ constant, we know that
%
\begin{equation}
    \dot{m}_\ell = -\frac{\dot{m}_e}{\left( 1 - \displaystyle\frac{P_v \nu_\ell}{R_v
    T}\right)}
\end{equation}
%
Similarly, for $m_g$ we obtain
%
\begin{equation}
   \dot{m}_g = \frac{\nu_\ell}{\nu_g}\frac{\dot{m}_e}{\left( 1 - \displaystyle\frac{P_v \nu_\ell}{R_v
    T}\right)}
\end{equation}

\subsubsection{Heat Transfer}

Heat transfer models are from Ring\cite{Ring:64} and
Incropera\cite{Incropera:06} and Pitts\cite{Pitts:98}

\begin{equation}
    \dot{Q} = h A \Delta T
\end{equation}

\begin{equation}
    \frac{h L}{k} = Nu = c(Gr_L Pr)^a
\end{equation}
%
so
%
\begin{equation}
       h = \frac{k c}{L}(Gr_L Pr)^a
\end{equation}
\begin{table}[ht]
\centering \caption{ Dimensionless Heat Transfer Terms
\cite{Incropera:06}}
\begin{tabular}{p{.65 in} p{.950 in} p{1.4 in}}
  \hline\hline
  % after \\: \hline or \cline{col1-col2} \cline{col3-col4} ...
  Parameter & Definition & Interpretation \\
  \hline
  Grashof number ($Gr_L$) & \hspace{ 1 in} $\displaystyle\frac{g \beta(T_s - T_\infty)L^3}{\nu^2}$  & Measure of the ratio of buoyancy forces to viscous forces.\\
  %
  & \\
  %
  Prandtl number ($Pr$) & \hspace{ 1 in} $\displaystyle\frac{c \mu}{k} = \frac{\mu}{\alpha}$  & Ratio of the momentum and thermal diffusivities.\\
  %
  & \\
  %
  Nusselt number ($Nu_L$) & \hspace{ 1 in} $\displaystyle\frac{h L}{k_f}$ & Ratio of convection to pure conduction heat transfer.  \\
  %
  & \\
  %
  Reynolds number ($Re_L$) & \hspace{ 1 in} $\displaystyle\frac{V L}{\nu}$ & Ratio of inertia and viscous forces.  \\
  \hline
\end{tabular}
\end{table}

\subsubsection{ Physiochemical Properties }

Hydrazine Properties \cite{HydraBook}
\[
c = 3084 \mbox{J/kg/K}
\]
\[
\rho \mbox{ (kg/m}^3) = 1025.6 - 0.87415 \mbox{ }T \mbox{ }(^o\mbox{C})- 0.45283e^{-3}\mbox{ }T^2 \mbox{ }(^o\mbox{C})
\]
%
\[
\rho \mbox{ (kg/m}^3) = 1230.6 - 0.62677 \mbox{ }T \mbox{ }(^o\mbox{K})- 0.45283e^{-3}\mbox{ }T^2 \mbox{ }(^o\mbox{K})
\]


Some models are from \cite{Ricciardi:87}

\begin{eqnarray}
  \rho_\ell(T_\ell) & = &K_1 + K_2 T_\ell + K_3 T_\ell^2
  \mbox{  (kg/m$^3$)}\\
  %
  \frac{d \rho_\ell}{dT_\ell } &  = & K_2 + 2K_3 T_\ell
  \mbox{  (kg/m$^3$)}
\end{eqnarray}
%
\begin{eqnarray}
  P_v & =  &\displaystyle C_1 e^{(C_2 + C_3 T_\ell + C_4 T_\ell^2)} \mbox{
  (N/m$^2$)}\\
  %
  \frac{d P_v}{d T_\ell}& = &\displaystyle C_1 ln{(C_2 + C_3 T_\ell C_4 T_\ell^2)}\left(C_3 + 2C_4T_\ell\right)
\end{eqnarray}
%
\begin{equation}
    m_d = P_g m_\ell^\alpha\left( \frac{T_\ell}{294}\right)^2
\end{equation}
%
\begin{equation}\begin{split}
    \frac{d m_d}{dt} =  m_\ell^\alpha\left(
    \frac{T_\ell}{294}\right)^2\dot{P}_g + \alpha P_g m_\ell^{(\alpha - 1)}\left(
    \frac{T_\ell}{294}\right)^2 \dot{m}_\ell \\ + 2 P_g m_\ell^\alpha\left(
    \frac{T_\ell}{294}\right)\dot{T}_\ell
    \end{split}
\end{equation}
%

\begin{table} \centering
\caption{Constants for Density Equations}
\begin{tabular}{lcc}
  \hline
  % after \\: \hline or \cline{col1-col2} \cline{col3-col4} ...
  Const. & \mbox{N}$_2$\mbox{0}$_4$ & CH$_3$N$_2$H$_3$ \\
  \hline \hline
  $K_1$ (kg/m$^3$) & 2066 & 1105.3 \\
  $K_2$ (kg/m$^3$/K )& -1.979 & -0.9395 \\
  $K_3$ (kg/m$^3$/K$^2) $ & -4.826e-4 & 0 \\
  \hline
\end{tabular}
\end{table}

\begin{table} \centering
\caption{Constants for Vapor Pressure Equations}
\begin{tabular}{lcc}
  \hline
  % after \\: \hline or \cline{col1-col2} \cline{col3-col4} ...
  Const. & \mbox{N}$_2$\mbox{0}$_4$ & CH$_3$N$_2$H$_3$ \\
  \hline \hline
  $C_1$ (kg/m$^3$) & 6895 & 6895\\
  $C_2$ (kg/m$^3$/K )& 18.6742 & 12.4293 \\
  $C_3$ (kg/m$^3$/K$^2) $ & -5369.57 & 2543.37 \\
  $C_4$ (kg/m$^3$/K$^2) $ & 194721 & 0 \\
  \hline
\end{tabular}
\end{table}

\begin{table} \centering
\caption{Constants for Dissolved Pressurant Equations}
\begin{tabular}{lcc}
  \hline
  % after \\: \hline or \cline{col1-col2} \cline{col3-col4} ...
  Const. & \mbox{N}$_2$\mbox{0}$_4$ & CH$_3$N$_2$H$_3$ \\
  \hline \hline
  $\alpha$ & 3.335e-11 & 2.059e-11 \\
  \hline
\end{tabular}
\end{table}


\subsection{Mass Properties}

\subsubsection{Spherical Tank}
\clearpage
\begin{figure}[ht]
\centerline{
    \begin{picture}(110,440)
    \special{psfile= ./Images/PartiallyFilledTank.eps hoffset= -110 voffset=
    90
    hscale=55 vscale=55}
    \makebox(-47,665){$h$}
    \makebox(135,620){$r$}
    \makebox(0,695){$\hat{y}$}
    \makebox(-165,850){$\hat{z}$}
    \makebox(-165,515){($\hat{x}$ is out of the page)}
    \end{picture}}\vskip -3.65 in  \caption{ Geometry For Mass Properties of Partially Filled Spherical Tank} \label{fig:PartiallyFilledTank}
\end{figure}

\begin{equation}
    V = \frac{1}{3}\pi \left( 3r - h \right)h^2
\end{equation}
%
\begin{equation}
    cg_z = \displaystyle\frac{-\displaystyle\frac{3}{4}h^2 + 3hr - 3r^2}{3r - h}
\end{equation}
%
\begin{equation}
    cg_x = cg_y = 0
\end{equation}
%
\begin{equation}
   I_{zz} = \frac{\pi \rho}{2}\left(\frac{1}{5}\left( h - r\right)^5 - \frac{2}{3}r^2\left( h - r\right)^3  + r^4(h-r) + \frac{8}{15}r^5 \right)
\end{equation}
%
\begin{equation}
   I_{xx} = \frac{\pi \rho}{2}\left(-\frac{3}{10}\left( h - r\right)^5 + \frac{1}{3}r^2\left( h - r\right)^3  + \frac{1}{2}r^4(h-r) + \frac{8}{15}r^5 \right)
\end{equation}
%
\begin{equation}
   I_{yy} = I_{xx}
\end{equation}
%
\begin{equation}
    \mathbf{I}' = \mathbf{R}_{bt}^T
    \left(\begin{array}{ccc}
         I_{xx} & 0 & 0 \\
         0 & I_{yy} & 0\\
         0 & 0 & I_{zz}\\
    \end{array}\right)
     \mathbf{R}_{bt} \
\end{equation}
%
Estey \cite{Estey:83} gives appoximate equations for the area of
different portions for a partially filled sphere. The area of the
spherical shell in contact with the gaseous region is given by
%
\begin{equation}
   A_g = 4.0675 \cdot V^{2/3}\left(\frac{V_g}{V}\right)^{0.62376}
\end{equation}
%
The area of the boundary between the liquid and the gaseous region
is given by
%
\begin{equation}
   A_b = A_g - 3.4211 \cdot V^{2/3}\left(\frac{V_g}{V}\right)^{1.24752}
\end{equation}
%
Finally, the area of the spherical shell in contact with the liquid
is given by
%
\begin{equation}
   A_l = \pi D^2 - A_g
\end{equation}
%

%\input{FlowModelling}
