\chapter{Low Thrust Optimization Using Orthogonal Collocation} \label{Ch:OrthogonalCollocation} 

\section{Nomenclature}

\begin{tabbing}
12345678 \= Reynolds number based on length $s$ \kill\\
$t$  \>   Time in dimensional units \\
$\tau$  \>  Time in nondimensional units $(-1 \leq \tau \leq 1)$\\
$t_I$  \>   Initial time \\
$t_F$  \>  Final time\\
$\mathbf{r}(t)$        \> Position vector in continous time form\\
$\mathbf{v}(t)$        \> Velocity vector in continous time form\\
$m(t)$        \> Mass in continous time form\\
$\mathbf{x}(t)$   \>  State Vector in continous time form\\
$\mathbf{r}_k$        \> Position vector in discrete time form at $k^{th}$ point\\
$\mathbf{v}_k$        \> Velocity vector in discrete time form at $k^{th}$ point\\
$m_k$        \> Mass in discrete time form at $k^{th}$ point\\
$\mathbf{x}_k$   \>  State Vector in discrete time form at $k^{th}$ point\\
$n_x$              \> Number of elements in state vector\\
$n_u$              \> Number of elements in control vector\\
$N$              \> Number of collocation points\\
$n_k$              \> Number of quadrature points \\
$L_i$              \> $i^{th}$ Lagrange polynomial \\
\end{tabbing}

\section{Dynamics Model}

\section{Continuous Time Problem}

\begin{equation}
     J = \Phi \left(t_I,\mbx(t_I),t_F,\mbx(t_F)\right) + \int_{t_I}^{t_F} \mathcal{L}\left( \mbx(t),\mbu(t),t;\mbq \right)dt
\end{equation}
%
subject to
%
\begin{equation}
   \frac{\partial \mathbf{x}}{\partial t} = \mathbf{f}\left( \mbx(t),\mbu(t),t;\mbq \right)
\end{equation}

\section{Discretization of State and Dynamics}

\begin{equation}
    \mbx(\tau_k) \approx \mbx_k =  \sum^{N}_{j = 1,j \neq i} \mbx_i L_i(\tau_k)
\end{equation}
%
where
%
\begin{equation}
    L_i = \prod^{N}_{j = 1,j \neq i} \frac{\tau - \tau_j}{\tau_i - \tau_j}
\end{equation}
%
%
\begin{equation}
   \dot{\mbx}(t_k) \approx \dot{\mbx}_k =  \sum^{N}_{j = 1,j \neq i} \mbx_i \dot{L}_i(\tau_k)  
\end{equation}

Define the matrix $\mathbf{X}$ as follows
%
\begin{equation}
     \mathbf{X} = \left[
     \begin{array}{ccc}
         \mbx_1^T \\
         \mbx_2^T \\
         \vdots \\
         \mbx_N^T
     \end{array}
     \right]
     %
     =
     %
     \left[
     \begin{array}{ccc}
         \mbr_1^T & \mbv_1^T & m_1 \\
         \mbr_2^T & \mbv_2^T & m_2\\
         \vdots \\
         \mbr_N^T & \mbv_N^T & m_N
     \end{array}
     \right]
     %
\end{equation}
%
Similarly, define the matrix $\mathbf{F}$ as 
%
\begin{equation}
   \mathbf{F} = \left[
     \begin{array}{ccc}
         \mathbf{f}_1^T \\
         \mathbf{f}_2^T \\
         \vdots \\
         \mathbf{f}_N^T
     \end{array}
     \right]
\end{equation}
%
and
%
\begin{equation}
   \mathbf{U} = \left[
     \begin{array}{ccc}
         \mathbf{u}_1^T \\
         \mathbf{u}_2^T \\
         \vdots \\
         \mathbf{u}_N^T
     \end{array}
     \right]
\end{equation}

then,
%
\begin{equation}
    \dot{\mathbf{X}} = \mathbf{D}\mathbf{X}
\end{equation}
%
\begin{equation}
 \mathbf{D}\mathbf{X} - \mathbf{F}(\mathbf{X},\mathbf{U},\mathbf{Q}) = \mathbf{0}
\end{equation}
%
\begin{equation}
    \dot{\mathbf{x}}_k^T = \mathbf{D}_{(k,:)}\mathbf{X}
\end{equation}
%

\begin{equation}
     \frac{\partial \dot{\mbx}_k}{\partial \mbx_{\ell}} = \left[ \mathbf{0}_{n_x,\ell-1} 
     \hspace{.1 in} \mathbf{D}_{(k,:)}^T \hspace{.1 in}  \mathbf{0}_{n_x,n_x - \ell+1} \right]
\end{equation}
%
\begin{equation}
     \frac{\partial \dot{\mbx}_k}{\partial \mathbf{u}_{\ell}} = \mathbf{0}_{n_x,n_u} 
\end{equation}


\begin{equation}
     \frac{\partial \mathbf{f}_k}{\partial \mathbf{x}_{\ell}} = 
     \left\{ \begin{array}{ccc}
    &\mathbf{A}(t_\ell) \hspace{.1 in} & k = \ell \\
    &\mathbf{0}_{n_x,n_x}  \hspace{.1 in} & k \neq \ell  \\
    \end{array} \right. 
\end{equation}
%
\begin{equation} 
     \frac{\partial \mathbf{f}_k}{\partial \mathbf{u}_{\ell}} = 
          \left\{ \begin{array}{ccc}
          & \displaystyle\frac{\partial \mbf(t_\ell)}{\partial \mbu} \hspace{.1 in} & k = \ell \\
          &\mathbf{0}_{n_x,n_u}  \hspace{.1 in} & k \neq \ell  \\
          \end{array} \right. 
\end{equation}

\section{Discretization of Cost Function}