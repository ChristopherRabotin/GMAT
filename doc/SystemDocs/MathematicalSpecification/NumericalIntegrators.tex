\chapter{Numerical Integrators} \label{Ch:NumericalIntegrators}

\section{Runge-Kutta Integrators} \index{Numerical
integrators!Runge-Kutta}

The Runge-Kutta integration scheme is a single step method used to
solve  differential equations for $n$ coupled variables of the
form

\[{{dr^i}\over{dt}} = f(t,r)\]

(The superscript in this discussion refers to the variables; hence
$f^i$ is the $i^{th}$ variable, and $r^{(n)}$ refers to all $n$
variables.) The method takes an integration step, $h$, by breaking
the  interval into several stages (usually of smaller size) and
calculating  estimates of the integration result at each stage.
The later stages use the  results of the earlier stages. The
cumulative effect of the integration is  an approximate total step
$\delta t$, accurate to a given order in the  series expansion of
the differential equation, for the state variables  $r_i(t+\delta
t)$ given the state $r_i(t)$.

The time increment for a given stage is given as a multiple $a_i$
of the  total time step desired; thus for the $i^{th}$ stage the
interval used  for the calculation is $a_i \delta t$; the estimate
of the integrated  state at this stage is given by

\[ k_i^{(n)} = \delta t f(t+a_i\delta t, r^{(n)}(t) + \sum_{j=1}^{i-1}b_{ij}k_j^{(n)}) \]

where $b_{ij}$ contains a set of coefficients specific to the
Runge-Kutta instance being calculated. Given the results of the
stage calculations, the  total integration step can be calculated
using another set of coefficients, $c_j$ and the formula

\[ r^{(n)}(t+\delta t) = r^{(n)}(t) + \sum_{j=1}^{stages}c_j k_j^{(n)} \]

The error control for these propagators is implemented by
comparing the  results of two different orders of integration. The
difference between the  two steps provides an estimate of the
accuracy of the step; a second set of  coefficients corresponding
to this second integration scheme can be used to obtain a solution

\[r'^{(n)}(t+\delta t) = r^{(n)}(t) + \sum_{j=1}^{stages}c_j^* k_j^{*(n)}\]

With care, the stage estimates $k_j$ and $k_j^*$ can be selected
so  that they are the same; in that case, the estimate of the
error in the  integration $\Delta^{(n)}$ can be written

\[ \Delta^{(n)} = \left| \sum_{j=1}^{stages}(c_j - c_j^*) k_j^{(n)} \right| \]

(The difference between the coefficients $c_j - c_j^*$ is the
array of  error estimate coefficients (ee) in this code.)

Once the estimated error has been calculated, the size of the
integration  step can be adapted to a size more appropriate to the
desired accuracy of the  integration. If the step results in a
solution that is not accurate enough, the step needs to be
recalculated with a smaller step size. Labeling the desired
accuracy $\alpha$ and the obtained accuracy $\epsilon$
(calculated, for instance, as the largest element of the array
$\Delta$), the new step used by the Runge-Kutta integrator is

\[\delta t_{new}= \sigma\delta t\left({{\alpha}\over{\epsilon}}\right)^{1/(m-1)}\]

where $m$ is the order of truncation of the series expansion of
the  differential equations being solved. The factor $\sigma$ is a
safety  factor incorporated into the calculation to avoid
unnecessary iteration over attempted steps. Common practice is to
set this factor to 0.9; that is the default value used in this
implementation.

Similarly, if the step taken does not result in the desired
accuracy, you may  want to increase the step size parameter for
the next integration step. The  new estimate for the desired
stepsize is given by

\[\delta t_{new}= \sigma\delta t\left({{\alpha}\over{\epsilon}}\right)^{1/(m)}\]

Sometimes you do not want to increase the stepsize in this manner;
for example,  you may want to keep the maximum step taken at some
fixed value. This  implementation provides a mechanism for
specifying a maximum allowed step.

Sometimes it is convenient to request steps of a specified size,
regardless  of the stepsize control algorithm or the calculation
of the \char`\"{}best step\char`\"{}  described above. This
implementation accomplishes that task by taking  multiple error
controlled steps is necessary to step across the requested
interval.

Both of these features are implemented using the boolean flags
described in  the base class for the integrators. See the
documentation for the {\bf Integrator} {\rm
(p.\,\pageref{classIntegrator})} class for more information about
these flags.



\subsection{Constructor \& Destructor Documentation}
\subsubsection{\setlength{\rightskip}{0pt plus 5cm}Runge\-Kutta::Runge\-Kutta (int {\em st}, int {\em order})}\label{classRungeKutta_a0}



Provides the greatest relative error in the state vector.

This method takes the state vector and calculates the error in
each  component. The error is then divided by the change in the
component. The function returns the largest of the resulting
relative errors.

Override this method if you want a different error estimate for
the stepsize  control. For example, we are using

\[error_i = \left({{\Delta_i(t+\delta t)}\over {r_i(t+\delta t) - r_i(t)}}\right)\]

Another popular approach is to divide the estimated error
$\Delta_i$ by  the norm of the corresponding 3-vector; for
instance, divide the error in x  by the magnitude of the
displacement in position for the step.


\section{Prince-Dormand Integrators}

\section{Adams Bashforth Moulton}  \index{Numerical
integrators!Adams-Bashforth-Moulton}

Implementation of the Adams-Bashford-Moulton Predictor-Corrector.

This code implements a fourth-order Adams-Bashford predictor /
Adams-Moulton corrector pair to integrate a set of first order
differential equations. The algorithm is found at  {\tt
http://chemical.caeds.eng.uml.edu/onlinec/white/math/s1/s1num/s1num.html}
or in Bate, Mueller and White, pp. 415-417.

The predictor step extrapolates the next state $r_{i+1}$ of the
variables using the the derivative information $(f)$ at the
current state and three  previous states of the variables, by
applying the equation

\[ r_{i+1}^{*j} = r_i^j + {{h}\over{24}}\left[55 f_n^j - 59 f_{n-1}^j + 37 f_{n-2}^j - 9 f_{n-3}^j \right] \]

The corrector uses derivative information evaluated for this
state, along  with the derivative information at the original
state and two preceding  states, to tune this state, giving the
final, corrected state:

\[ r_{i+1}^{j} = r_i^j + {{h}\over{24}}\left[9 f_{n+1}^{*j} + 19 f_{n}^j - 5 f_{n-1}^j + 1 f_{n-2}^j \right] \]

Bate, Mueller and White give the estimated accuracy of this
solution to be

\[ee = {{19}\over{270}} \left|r_{i+1}^{*j} - r_{i+1}^{j}\right|\]

Method used to fire the step refinement (the corrector phase).

\section{Bulirsch-Stoer} \index{Numerical integrators!Bulirsch-Stoer}

\section{Stopping Condition Algorithm} \index{Numerical integrators!Stopping condition algorithm}
\index{Stopping condition algorithm}

\section{Integrator Coefficients} \index{Numerical integrators!Coefficients}





\begin{table*}[htb]
\centering \vspace{0 pt}
\index{Numerical integrators!PD45
coefficients} \caption{Prince-Dormand 45 Coefficients }
\begin{tabular}{c|cccccccccccccc}
   $a_i$ & & & & $b_{ij}$  \\
   \hline\hline
   %  Line 1 ---------------------------------
   0  & 0  & \vspace{.1in} \\
   %
   %  Line 2 ---------------------------------
   $\frac{2}{9}$ & $\frac{2}{9}$ &  0  & \vspace{.1in} \\
   %
   %  Line 3 ---------------------------------
   $\frac{1}{3}$ &  $\frac{1}{12 }$ & $\frac{ 1 }{ 4 }$ & 0 & \vspace{.1in} \\
   %
   %  Line 4 ---------------------------------
   $\frac{5}{9}$  & $\frac{55}{324}$ & $-\frac{25}{108}$ & $\frac{50}{81}$ & 0 & \vspace{.1in} \\
   %
   %  Line 5 ---------------------------------
   $\frac{2}{3}$ & $\frac{83}{330}$ & $-\frac{13}{22}$ & $\frac{61}{66}$ &
   $\frac{9}{110}$ & 0 &\vspace{.1in} \\
   %
   %  Line 6 ---------------------------------
   1   &    $-\frac{19}{28}$ & $\frac{9}{4}$ & $\frac{1}{7}$ &
    $-\frac{27}{7}$ & $\frac{22}{7}$ & 0 & \vspace{.1in} \\
   %
   %  Line 7 ---------------------------------
   1 &    $\frac{19}{200}$ &  0 & $\frac{3}{5}$ & $-\frac{243}{400}$ &
    $\frac{33}{40}$ & $\frac{7}{80}$ & 0 & \vspace{.1in} \\ \hline
    \\
   %
   %  Line 8 ---------------------------------
   $c_j$ &  $\frac{19}{200}$ &  0 & $\frac{3}{5}$ & $-\frac{243}{400}$ &
    $\frac{33}{40}$ & $\frac{7}{80}$ & 0 & \vspace{.1in} \\
   %
   %  Line 9 ---------------------------------
   $e_j$ & $\frac{19}{200}$   - $\frac{431}{5000}$ &  0 &  $\frac{3}{5}$      - $\frac{333}{500}$ &
    $-\frac{243}{400}$  + $\frac{7857}{10000}$ &  $\frac{33}{40}$    - $\frac{957}{1000}$ &
    $\frac{7}{80}$     - $\frac{193}{2000}$ &  $\frac{1}{50}$  &    \vspace{.1in} \\
   %
  \hline\hline
\end{tabular}
\label{Table:PD45Coefficients}
\end{table*}




\begin{table*}[htb]
\centering \vspace{0 pt}
\index{Numerical integrators!PD56 coefficients}
\caption[Prince-Dormand 56 Coefficients]{Prince-Dormand 56 Coefficients (Warning:  There is an error in the original source for these and we have not found the correct coefficients yet!!) }
\begin{tabular}{c|cccccccccccccc}
   $a_i$ & & & & $b_{ij}$  \\
   \hline\hline
   %  Line 1 ---------------------------------
    0 &
    0 & \vspace{.1in} \\
   %  Line 2 ---------------------------------
    $\frac{1}{10}$ &
    $\frac{1}{10}$ &
    0 &  \vspace{.1in} \\
   %  Line 3 ---------------------------------
    $\frac{2}{9}$ &
    $-\frac{2}{81}$ &
    $\frac{20}{81}$ &
    0 &  \vspace{.1in} \\
   %  Line 4 ---------------------------------
    $\frac{3}{7}$ &
    $\frac{615}{1372}$ &
    $-\frac{270}{343}$ &
    $\frac{1053}{1372}$ &
    0 &  \vspace{.1in} \\
   %  Line 5 ---------------------------------
    $\frac{3}{5}$ &
    $\frac{3243}{5500}$ &
    $-\frac{54}{55}$ &
    $\frac{50949}{71500}$ &
    $\frac{4998}{17875}$ &
    0 &  \vspace{.1in} \\
   %  Line 6 ---------------------------------
    $\frac{4}{5}$ &
    $-\frac{26492}{37125}$ &
    $\frac{72}{55}$ &
    $\frac{2808}{23375}$ &
    $-\frac{24206}{37125}$ &
    $\frac{338}{495}$ &
    0 &  \vspace{.1in} \\
   %  Line 7 ---------------------------------
    1 &
    $\frac{5561}{2376}$ &
    $-\frac{35}{11}$ &
    $-\frac{24117}{31603}$ &
    $\frac{899983}{200772}$ &
    $-\frac{5225}{1836}$ &
    $\frac{3925}{4056}$ &
    0 &  \vspace{.1in} \\
   %  Line 8 ---------------------------------
    1 &
    $\frac{465467}{266112}$ &
    $-\frac{2945}{1232}$ &
    $\frac{10513573}{3212352}$ &
    $-\frac{5610201}{14158144}$ &
    $-\frac{424325}{205632}$ &
    $\frac{376225}{454272}$ &
    0 &
    0 & \\ \hline \\
   %  Line 9 ---------------------------------
    $c_j$ &$\frac{61}{864}$ &
    0 &
    $\frac{98415}{321776}$ &
    $\frac{16807}{146016}$ &
    $\frac{1375}{7344}$ &
    $\frac{1375}{5408}$ &
    $-\frac{37}{1120}$ &
    $\frac{1}{10}$ &  \vspace{.1in} \\
   %  Line 10 ---------------------------------
    $e_j$ & $\frac{61}{864}$   - $\frac{821}{10800}$ &
    0 &
    $\frac{98415}{321776}$   - $\frac{19683}{71825}$ &
    $\frac{16807}{146016}$  - $\frac{175273}{912600}$ &
    $\frac{1375}{7344}$   - $\frac{395}{3672}$ &
    $\frac{1375}{5408}$   - $\frac{785}{2704}$ &
    $-\frac{37}{1120}$   - $\frac{3}{50}$ &
    $\frac{1}{10}$ &  \vspace{.1in} \\
  \hline\hline
\end{tabular}
\label{Table:PD56Coefficients}
\end{table*}



\begin{table*}[htb]
\centering \vspace{0 pt} \caption{Runge-Kutta-Fehlberg 56
Coefficients }
\index{Numerical integrators!RKF56 coefficients}
\begin{tabular}{c|cccccccccccccc}
   $a_i$ & & & & $b_{ij}$  \\
   \hline\hline
   %
   %  Line 1 ---------------------------------
   \vspace{.1in} \\
    0 & 0 &  \vspace{.1in} \\
   %
   %  Line 2 ---------------------------------
    $\frac{1}{6}$ &
    $\frac{1}{6}$ &
    0 &  \vspace{.1in} \\
    %
    %  Line 3 ---------------------------------
    $\frac{4}{15}$ &
    $\frac{4}{75}$ &
    $\frac{16}{75}$ &
    0 & \vspace{.1in} \\
    %
    %  Line 4 ---------------------------------
    $\frac{2}{3}$ &
    $\frac{5}{6}$ &
    $-\frac{8}{3}$ &
    $\frac{5}{2}$ &
    0 & \vspace{.1in} \\
    %
    %  Line 5 ---------------------------------
    $\frac{4}{5}$ &
    $-\frac{8}{5}$ &
    $\frac{144}{25}$ &
    -4 &
    $\frac{16}{25}$ &
    0 & \vspace{.1in} \\
    %
    %  Line 6 ---------------------------------
    1 &
    $\frac{361}{320}$ &
    $-\frac{18}{5}$ &
    $\frac{407}{128}$ &
    $-\frac{11}{80}$ &
    $\frac{55}{128}$ &
    0 & \vspace{.1in} \\
    %
    %  Line 7 ---------------------------------
    0 &
    $-\frac{11}{640}$ &
    0 &
    $\frac{11}{256}$ &
    $-\frac{11}{160}$ &
    $\frac{11}{256}$ &
    0 &
    0 & \vspace{.1in} \\
    %
    %  Line 8 ---------------------------------
    1 &
    $\frac{93}{640}$ &
    $-\frac{18}{5}$ &
    $\frac{803}{256}$ &
    $-\frac{11}{160}$ &
    $\frac{99}{256}$ &
    0 &
    1 &
    0 & \vspace{.1in} \\ \hline \\
    %
    %  Line 9 ---------------------------------
    $c_j$ & $\frac{7}{1408}$ &
    0 &
    $\frac{1125}{2816}$ &
    $\frac{9}{32}$ &
    $\frac{125}{768}$ &
    0 &
    $\frac{5}{66}$ &
    $\frac{5}{66}$ & \vspace{.1in} \\
    %
    %  Line 10 ---------------------------------
    $e_j$ & $\frac{7}{1408}$   - $\frac{31}{384}$ &
    0 &
    0 &
    0 &
    0 &
    $-\frac{5}{66}$ &
    $\frac{5}{66}$ &
    $\frac{5}{66}$ & \vspace{.1in} \\
    %
  \hline\hline
\end{tabular}
\label{Table:RKF56Coefficients}
\end{table*}

%\clearpage

\begin{landscape}
\begin{table*}
\centering
\index{Numerical integrators!PD78 coefficients}
\caption{Prince-Dormand 78 Coefficients }
\begin{tabular}{c|cccccccccccccc}
   $a_i$ & & & & & & & $b_{ij}$ & & & & & \\
   \hline\hline
   %  Line 1 ---------------------------------
   0  & 0 \vspace{.1in} \\
   %
   %  Line 2 ---------------------------------
   $\frac{1}{18}$ & $\frac{1}{18}$ \vspace{.1in} \\
   %
   %  Line 3 ---------------------------------
   $\frac{1}{12}$ &  $\frac{1}{48 }$ & $\frac{ 1 }{ 16 }$ \vspace{.1in} \\
   %
   %  Line 4 ---------------------------------
   $\frac{1}{8}$  & $\frac{1}{ 32}$ & 0 & $\frac{3}{32}$ \vspace{.1in} \\
   %
   %  Line 5 ---------------------------------
   $\frac{5}{ 16}$ & $\frac{5}{16}$ & 0 & $-\frac{75}{64}$ & $\frac{75 }{64 }$\vspace{.1in} \\
   %
   %  Line 6 ---------------------------------
    $\frac{3}{8}$ & $\frac{ 3}{80}$  & 0 &  0 & $\frac{3}{16}$ & $\frac{3}{20}$ \vspace{.1in} \\
   %
   %  Line 7 ---------------------------------
    $\frac{59}{400}$ & $\frac{29443841}{614563906}$ & 0 & 0 & $\frac{77736538}{692538347}$
    & -$\frac{28693883}{1125000000}$ &$\frac{23124283}{1800000000}$  \vspace{.1in} \\
   %
   %  Line 8 ---------------------------------
   $\frac{93}{200}$ &     $\frac{16016141}{ 946692911}$ & 0 & 0 &  $\frac{61564180}{  158732637}$ &
    $\frac{22789713}{ 633445777}$ & $\frac{545815736}{  2771057229}$ &
    $-\frac{180193667}{1043307555}$ & \vspace{.1in} \\
   %
   %  Line 9 ---------------------------------
    $\frac{5490023248}{9719169821}$ &     $\frac{39632708}{  573591083}$ & 0 & 0 & $-\frac{433636366}{ 683701615}$
    & $-\frac{421739975}{ 2616292301}$ & $\frac{100302831}{ 723423059}$ &
    $\frac{790204164}{ 839813087}$ & $\frac{800635310}{ 3783071287}$ & \vspace{.1in} \\
   %
   %  Line 10 ---------------------------------
   $\frac{13}{20}$ &     $\frac{246121993}{1340847787}$ &  0 &  0 &  $-\frac{37695042795}{15268766246}$ &
    $-\frac{309121744}{1061227803}$ & $-\frac{12992083}{490766935}$ & $\frac{6005943493}{2108947869}$ &
    $\frac{393006217}{1396673457}$ & $\frac{123872331}{1001029789}$ & \vspace{.1in} \\
   %
   %  Line 11 ---------------------------------
    $\frac{1201146811}{1299019798}$ &     $-\frac{1028468189}{ 846180014}$ &  0 &  0 & $\frac{8478235783}{508512852}$ &
    $\frac{1311729495}{1432422823}$ &  $-\frac{10304129995}{1701304382}$ &
    $-\frac{48777925059}{3047939560}$ & $\frac{15336726248}{1032824649}$ &
    $-\frac{45442868181}{3398467696}$ & $\frac{3065993473}{597172653}$ & \vspace{.1in} \\
   %
   %  Line 12 ---------------------------------
    1 &     $\frac{185892177}{ 718116043}$ & 0 & 0 & $-\frac{3185094517}{667107341}$ &
    $-\frac{477755414}{1098053517}$ & $-\frac{703635378}{230739211}$ &
    $\frac{5731566787}{1027545527}$ & $\frac{5232866602}{850066563}$ &
    $-\frac{4093664535}{808688257}$ & $\frac{3962137247}{1805957418}$ &
    $\frac{65686358}{487910083}$ & \vspace{.1in} \\
   %
   %  Line 13 ---------------------------------
    1 &     $\frac{403863854}{491063109}$ & 0 & 0 & $-\frac{5068492393}{434740067}$ &
    $-\frac{411421997}{543043805}$ & $\frac{652783627}{914296604}$ &
    $\frac{11173962825}{925320556}$ & $-\frac{13158990841}{6184727034}$ &
    $\frac{3936647629}{1978049680}$ & $-\frac{160528059}{685178525}$ &
    $\frac{248638103}{1413531060}$ & 0 & \vspace{.1in}\\ \hline  \\
   %
   %  c_j coefficients -----------------------
   $c_j$ & $\frac{14005451 }{ 335480064}$ &  0 &  0 &  0 & 0 &
   $-\frac{59238493}{1068277825}$ & $\frac{ 181606767 }{758867731}$ &
   $\frac{561292985 }{ 797845732}$ & $-\frac{1041891430}{ 1371343529}$ &
   $\frac{760417239}{ 1151165299}$ & $\frac{118820643}{ 751138087}$ &
   $-\frac{528747749}{ 2220607170}$ & $\frac{1}{4}$  \vspace{.1in} \\
   %
   %  \hat{c}_j coefficients -----------------------
   $\hat{c}_j$ & $\frac{ 13451932 }{ 455176623 }$ & 0 & 0 & 0 & 0 & $-\frac{808719846}{976000145}$ &
   $\frac{1757004468}{5645159321}$ & $\frac{656045339}{265891186.0}$ & $-\frac{3867574721}{1518517206}$
   & $\frac{465885868}{322736535}$ & $\frac{53011238}{ 667516719}$
   &  $\frac{ 2.0 }{ 45.0}$ &  0 \\
   %
  \hline\hline
\end{tabular}
\label{Table:PD78Coefficients}
\end{table*}
\end{landscape}

