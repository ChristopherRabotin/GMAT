%--------------------------------------------------------------------
%-----------------------Begin Table Here-----------------------------
%----- (Comment the \clearpage code if the table doesn't break ------
%------ across multiple pages or is the first in a section) ---------
%--------------------------------------------------------------------
%\clearpage
\noindent \tablecaption{Propagate Command}
\tablefirsthead{\hline\hline}\label{Table:PropagateCommand}
\tablehead{\multicolumn{2}{c}{Table ~\ref{Table:PropagateCommand}:
Propagate Command \ldots continued}\\\\ \hline\hline}
\tabletail{\hline\hline} \tablelasttail{\hline\hline}
\begin{supertabular*}{6.5 in}{@{}p{1.0 in}@{\extracolsep{\fill}}p{5.0 in}@{}}
    \multicolumn{2}{c}{ScriptSyntax}\\
    \hline\\
    \multicolumn{2}{l}
    {\st{Propagate Mode BackProp \emph{PropagatorName}(SatList1,\{StopCondList1\}) \ldots}}\\
    \multicolumn{2}{l}{\st{BackProp\emph{PropagatorName}(SatListN,\{StopCondListN\})}}\\\\

    \hline\hline
    Option & Option Description\\
    \hline
    %------- New Item
    \st{BackProp} & Default: None. Options: [ Backwards or None ]:
    The \st{BackProp} option allows the user to set the flag to enable or disable backwards propagation for all
    spacecraft in the the \st{SatListN} option.
    The Backward Propagation GUI check box field stores all the data in \st{BackProp}. A check indicates backward
    propagation is enabled and no check indicates forward propagation. In the script, \st{BackProp} can be the word
    Backwards for backward propagation or blank for forward propagation. Units: N/A.
    \index{\st{Backwards Propagation}}\\\\
    %------- New Item
    \st{Mode} & Default: None.  Options: [ Synchronized or None ]:
    The \st{Mode} option allows the user to set the propagation mode for the propagator that will affect all of the
    spacecraft added to the \st{SatListN} option. For example, if synchronized is selected, all spacecraft are
    propagated at the same step size. The Propagate Mode GUI field stores all the data in \st{Mode}. In the script,
    Mode is left blank for the None option and the text of the other options available is used for their respective
    modes. Units: N/A. \index{\st{Propagation Mode}}\\\\
    %------- New Item
    \emph{PropagatorName} & Default: DefaultProp. Options: [ Default propagator or any user-defined propagator ]:
    The \emph{PropagatorName} option allows the user to select a user defined propagator to use in spacecraft and/or
    formation propagation. The Propagator GUI field stores all the data in \emph{PropagatorName}. Units: N/A.\\\\
    %------- New Item
    \st{SatListN} & Default: DefaultSC. Options: [ Any existing spacecraft or
    formations, not being propagated by another propagator in the same Propagate event.  Multiple spacecraft must be
    expressed in a comma delimited list format. ]: The \st{SatListN} option allows the user to enter all the
    satellites and/or formations they want to propagate using the \emph{PropagatorName} propagator settings.
    The Spacecraft List GUI field stores all the data in \st{SatListN}. Units: N/A.\\\\
    %------- New Item
    \st{StopCondListN} /Parameter & Default: DefaultSC.ElapsedSecs =. Options: [ Any single element user accessible
    spacecraft parameter followed by an equal sign ]. The
    \st{StopCondListN} option allows the user to enter all the parameters used for the propagator stopping condition.
    See the \st{StopCondListN}/Condition Option/Field for additional details to the \st{StopCondListN} option.
    Units: N/A. \\\\
    %------- New Item
    \st{StopCondListN} /Condition & Default: 8640.0. Options: [ Real Number, Array element, Variable, spacecraft
    parameter, or any user defined parameter ].
    The \st{StopCondListN} option allows the user to enter the propagator stopping condition's value for the
    \st{StopCondListN} Parameter field. Units: Dependant on the condition selected. \\\\

    \hline\hline
    \multicolumn{2}{c}{Script Examples}\\
    \hline
    \multicolumn{2}{l}{\% Single spacecraft propagation with one stopping condition}\\
    \multicolumn{2}{l}{\% Syntax \#1}\\
    \multicolumn{2}{l}{\st{Propagate DefaultProp(DefaultSC, \{DefaultSC.ElapsedSecs = 8640.0\});}}\\\\

    \multicolumn{2}{l}{\% Single spacecraft propagation with one stopping condition}\\
    \multicolumn{2}{l}{\% Syntax \#2}\\
    \multicolumn{2}{l}{\st{Propagate DefaultProp(DefaultSC) \{DefaultSC.ElapsedSecs = 8640.0\};}}\\\\

    \multicolumn{2}{l}{\% Single spacecraft propagation by one integration step}\\
    \multicolumn{2}{l}{\st{Propagate DefaultProp(DefaultSC);}}\\\\

    \multicolumn{2}{l}{\% Multiple spacecraft propagation by one integration step}\\
    \multicolumn{2}{l}{\st{Propagate DefaultProp(Sat1, Sat2, Sat3);}}\\\\

    \multicolumn{2}{l}{\% Single formation propagation by one integration step}\\
    \multicolumn{2}{l}{\st{Propagate DefaultProp(DefaultFormation);}}\\\\

    \multicolumn{2}{l}{\% Single spacecraft backwards propagation by one integration step}\\
    \multicolumn{2}{l}{\st{Propagate Backwards DefaultProp(DefaultSC);}}\\\\

    \multicolumn{2}{l}{\% Two spacecraft synchronized propagation with one stopping condition}\\
    \multicolumn{2}{l}{\st{Propagate Synchronized DefaultProp(Sat1, Sat2, \{DefaultSC.ElapsedSecs = 8640.0\});}}\\\\

    \multicolumn{2}{l}{\% Multiple spacecraft propagation with multiple stopping conditions and propagation settings}\\
    \multicolumn{2}{l}{\% Syntax \#1}\\
    \multicolumn{2}{l}{\st{Propagate Prop1(Sat1,Sat2, \{Sat1.ElapsedSecs = 8640.0, Sat2.MA = 90\}) \ldots}}\\
    \multicolumn{2}{l}{\st{Prop2(Sat3, \{Sat3.TA = 0.0\});}}\\\\

    \multicolumn{2}{l}{\% Multiple spacecraft propagation with multiple stopping conditions and propagation settings}\\
    \multicolumn{2}{l}{\% Syntax \#2}\\
    \multicolumn{2}{l}{\st{Propagate Prop1(Sat1,Sat2) \{Sat1.ElapsedSecs = 8640.0, Sat2.MA = 90\}} \ldots}\\
    \multicolumn{2}{l}{\st{Prop2(Sat3) \{Sat3.TA = 0.0\};}}\\
\end{supertabular*}\\
