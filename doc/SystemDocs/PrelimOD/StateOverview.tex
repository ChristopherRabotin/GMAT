\section{State Vector Overview}

GMAT requires two distinct types of state information when solving estimation problems.  The
estimators operate on an estimation state vector containing the elements that are estimated or
considered during the estimation process.  That process requires a mechanism that models the
evolution of elements of the estimation state vector -- and, potentially, other model elements --
over time. These elements are assembled into a propagation state vector designed to facilitate fast
numerical evolution from one epoch to another.

State vectors in GMAT provide the following capabilities:

\begin{itemize}
\item Specify the values of the represented parameters at an epoch
\item Provide an extendable structure for state data
\item Identify each element by type for the purposes of mapping to and from objects
\item Identify each element by name so that user friendly vector descriptions can be generated
\item Allow access to the raw state vector for direct manipulation to meet performance requirements
\item Provide mechanisms to manipulate the vector element by element or as collections of elements
\item Allow a handshake mechanism so that the system components that manipulate the state vector
can map the elements onto other system components
\item Make decomposition of the state vector into subvectors simple to manage in the code
\end{itemize}

These derived requirements drive the design for the class structure used to model all GMAT state
vectors.  Customization of the state vector for the specific needs of the other elements of GMAT is
performed using a state vector manager.  The state vector manager provides the interfaces required
by the propagation and estimation subsystems to synchronize the state data with GMAT's objects, and
with the numerical engines controlled by those subsystems.

The state manager base class contains the structures and basic interfaces used to provide these
features:

\begin{itemize}
\item Manage a state vector
\item Copy object data into and out of the state vector
\item Map vector elements to their mathematical models
\item Ensure data consistency for the state vector
\item Provides a generic interface used by subclasses to construct state manipulation components
like derivative models and estimation state manipulators
\end{itemize}




<OTHER TEXT STILL IN THE WORKS>

The following sections describe the classes, derived from GMAT's base state representation, that
are used to model the estimation state vector and the propagation state vector.