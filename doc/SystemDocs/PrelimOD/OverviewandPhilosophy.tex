
To solve OD problems in GMAT, a user will have to create and
configure many objects that model physical components involved in
the measurement process such as spacecraft, receivers, clocks,
transmitters and celestial objects to name a few.  A user will also
have to configure measurement models, provide truth data or define
simulation parameters, and configure numerical solvers to determine
the best state estimate. Below we discuss how the user will provide
these and other types of information to define and solve OD problems
in GMAT.  In a sense, this portion of the document is a  short
user's guide  written before developing the software. This approach
is useful in several ways: it defines how a user will interact with
GMAT and allows analysts to provide feedback on the interfaces
before the system is written, and it gives some insight into various
designs for the the underlying software architecture.

We begin this chapter with a high level view of OD from the
perspective of a user who needs to apply GMAT to orbit
determination. We provide a categorization of models and algorithms
that a user will have control over and explain the overall
philosophy of the user interface in GMAT.  The high level
categorization explains where in the GMAT script and GUI a user
would need to go to set various types of information such as
dynamics models, process noise parameters, and measurement models to
name a few.  Next, we provide a detailed view of the primary objects
and commands required for orbit determination in GMAT.  These
sections go into detail on what fields are on each objects, and what
commands would be employed for different types of analysis.

\section{Overview and Philosophy}

To uniquely define an orbit determination problem, a user must often
provide hundreds of pieces of information ranging from clock drift
parameters, process noise characteristics, spacecraft physical
properties, ground station properties, and atmospheric modelling
parameters to name just a few. This section explains how these
pieces of information will be organized in GMAT and the philosophy
behind the organization of the data.  The goal is to provide an
intuitive interface for orbit determination analysis and explain
where a user would go to set different types of data, and why the
choices have been made.

To explain the organization of models and data, we present three
levels of detail - high, intermediate, and low - into the
organization of data and models for orbit determination
applications.  The high level view groups all models into five
categories and explains how these groups interact in the estimation
process.  These high level categories are: measurement participants,
measurements, estimators, dynamics, and commands. The intermediate
level view contains more detail about what types of data are set on
each object and model. For example, in the intermediate level view,
you learn where clock bias information is located and where process
noise information is located.  The low level view describes the
actual field names and settings contained in the five categories.
For example, in the low level view, you learn the script syntax to
set process noise parameters, and the different process noise models
that are available.

Let's begin with the high level view.

\subsection{High Level View of System Components}

According to our proposed approach we organize the objects and
models to perform OD into four categories:  estimators, measurement
participants, measurement models, and dynamics models.  In this
section, we discuss each of these classes of objects in more detail
and describe the philosophy and functionality for each of the four
categories.

\begin{figure}[htbp!]
    \begin{center}
    \begin{picture}(270,150)
    \special{psfile=ODObjects.eps
    hscale= 85 vscale= 85 hoffset = -130 voffset = -440}
    \end{picture}
    \end{center}
    \vspace{0.2 in}
    \label{Fig:ODObjects}
    \caption{High Level View: Objects and Commands for OD Applications }
\end{figure}


Measurement Participants The first category of objects is
Measurement Participants which are defined as any physical object
that is part of a measurement process.  Examples include:

?   Spacecraft and sensors ?   Ground station and sensors ? Aircraft
and sensors ?   Celestial objects ?   Other, passive or active

For example,  an altimeter measurement for a LEO spacecraft has two
participants: Earth and the spacecraft.  For a GPS pseudorange
measurement, the participants include the GPS constellation and the
user spacecraft.   For a two-way range measurement between a ground
station and a spacecraft, both the ground station and the spacecraft
are participants in the measurement.

In GMAT, measurement participants are created and configured
separately from measurements.  For example, if the user requires a
Doppler measurement between say Hubble and Canberra, they first
configure a spacecraft to model Hubble, and then they configure a
ground station to model Canberra.  Once the Hubble and Canberra
objects are configured, the user creates a measurement object and
configures it to create a Doppler measurement between Hubble and
Canberra.


GPS Constellation Measurement Models A measurement model describes
how measurement participants interact to produce a measurement
quantity used in the OD process.  Once configured, a measurement
model provides the following information: observed measurement
values (whether they are read from a file or simulated),  the
computed (or expected) value of the measurement, and the measurement
partial derivatives.   In summary, Measurement Models provide:

?   Truth quantities ?   Observed ?   Simulated ?   Computed
(expected) quantities ?   Partial derivatives

\subsection{Intermediate Level View of System Components}

\begin{figure}[htbp!]
    \begin{center}
    \begin{picture}(270,350)
    \special{psfile=ODOrganization.eps
    hscale= 85 vscale= 85 hoffset = -130 voffset = -340}
    \end{picture}
    \end{center}
    \vspace{0.2 in}
    \label{Fig:ODOrganization}
    \caption{Intermediate Level View: Objects and Commands Data and Settings }
\end{figure}

%To configure a measurement model, the measurement participants must
%be created and appropriately configured.   Fields on the Measurement
%Model object allow the user to define many different types of
%measurements given the specified list of participants.   If observed
%measurements are available from a standard file format, the user
%configures the measurement Model to read data from the desired file.
%In this case, the user must set object Ids on the measuement
%participants to match the Ids on the measurement file.  GMAT uses
%the file format and measurement Ids to determine how computed values
%are to be calculated.
%


\verbatiminput{GSMeasurement.script}
\verbatiminput{BatchLeastSquares.script}
\verbatiminput{ExtendedKalmanFilter.script}

%Estimators
%
%The third component required for an estimation problem is a solver.
%The user will have many solvers to choose from including but not
%limited to
%
%?   Initial orbit determination ?   Batch (least squares, other) ?
%Filters (SRIF, EKF, UKF)
%
%In GMAT the solver is a relatively simple object compared to the
%measurement model and the measurement  participants.  The job of the
%solver is to query configured measurement models for the observed
%and computed values  and the partial derivatives  and use this
%information  to generate state estimates.  Hence,  the estimator in
%GMAT knows little about the details of the measurement model
%computations or the configurations of the participants.
%
%A sample script segment for a BatchLeastSquare he user will specify
%the SolveFor and Consider parameters on the estimator, along with
%what measurements they would like the solver to process.
%
%%Create BatchLeastSquares BLSE BLSI.Measurements = BLSI.Propagator =
%%BLSE.SolveFor = BLSE.Consider = BLSE.SolutionEpoch =
%%BLSE.AbsoluteConvergenceTol = BLSE.Propagator =
%%
%%Create ExtendedKalmanFilter EKF EKF.Measurements = EKF.Propagator =
%%EKF.SolveFor = EKF.Consider = EKF.SolutionEpoch = EKF.Propagator =
%%EKF.ProcessNoiseModel = EKF.Smoothing = Dynamics Models
%
%Dynamics Models ?   Participant dynamics ?   Variational equations ?
%Process Noise
%
%Commands and Application Control
%
%Application Control Modes ?   RunEstimator ?   SimulateData ?
%RunEstimatorSequence
